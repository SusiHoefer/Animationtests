\documentclass{article}
\usepackage[utf8]{inputenc}
\usepackage[T1]{fontenc}
\usepackage[ngerman]{babel}
\usepackage[hyphens]{url} 
\usepackage{hyperref}
\usepackage{listings}
\usepackage{soul}
\usepackage{xcolor} 
\usepackage{graphicx} 
\DeclareUnicodeCharacter{2060}{\nolinebreak} % might not be necessary for you
\DeclareRobustCommand{\hlgray}[1]{{\sethlcolor{lightgray}\hl{#1}}} 

                    \usepackage{fancyhdr}
        \pagestyle{fancy}
        \fancyhf{}
        \fancyhead[LE,RO]{}
        \fancyhead[RE,LO]{KONDE Weißbuch}
       \fancyfoot[LE,RO]{\thepage}
        \renewcommand{\headrulewidth}{2pt}
\begin{document}
\begin{titlepage}
\author{Hrsg. v. Helmut W. Klug unter Mitarbeit von \\
Selina Galka und  Elisabeth Steiner im HRSM Projekt \\
"Kompetenznetzwerk Digitale Edition". \\
URL: www.digitale-edition.at} 
\title{KONDE Weißbuch} 
\date{} 
\maketitle
\end{titlepage}
\newpage
\section*{Über das KONDE Weißbuch}
Ein Weißbuch ist eine Sammlung an Vorschlägen für ein bestimmtes Vorgehen \\
in definierten Anwendungsfällen. Der vorliegende Anwendungsfall ist die \\
Digitale Edition, das KONDE Weißbuch bietet Vorschläge und Lösungsansätze \\
für mögliche Fragestellungen oder Forschungsszenarien, die bei der \\
Konzeption und Erstellung von Digitalen Editionen auftreten können. Die \\
Weißbucheinträge sind prinzipiell einführend konzipiert, und wollen Fragen \\
rund um das Thema zu beantworten, gängige Lösungswege beschreiben sowie \\
einschlägige Forschungsliteratur, Software und Projekte als Hilfe zur \\
Selbsthilfe anbieten. Darüberhinaus sind die Artikel durchaus auch dafür \\
gedacht, Kontakte zu Experten und Institutionen herzustellen.\\

Das KONDE Weißbuch versammelt kurze Einträge zu möglichst diversen \\
Themen rund um den Forschungsbereich Digitale Edition. Die Einträge \\
können interessensgeleitet über eine intensive interne (und darüberhinaus \\
externe) Verlinkung erforscht werden, sie sind bestimmten Themen- \\
bereichen zugeordnet und über einen Stichwort- und Autorenindex er-\\
schlossen. Visuelle Zugänge bieten ein Prozessdiagramm zum Thema \\
"Digitalisierung" und eine Netzwerkvisualisierung der Abhängigkeiten\\
der einzelnen Weißbuchartikeln untereinander.\\

Jeder Eintrag zu einem Stichwort bietet neben einem kurzen erläuternden \\
Text Literaturhinweise, Softwarelösungen und Links zu Beispielprojekten. \\
Alle Einträge sind thematisch vernetzt.\\
\newpage
        \section*{ACDH-CH - Austrian Centre for Digital Humanities and Cultural Heritage} \emph{Ďurčo, Matej; matej.durco@oeaw.ac.at}\\
        
    Im Jahr 2015 wurde das \emph{Austrian Centre for Digital Humanities (ACDH)} der Österreichischen Akademie der Wissenschaften als Forschungsinstitut mit der erklärten Absicht eingerichtet, die Geisteswissenschaften durch die Anwendung digitaler Methoden und Werkzeuge in einem breiten Spektrum von Wissenschaftsbereichen zu fördern. Ende 2019 wurde das Institut einer umfassenden Umstrukturierung unterzogen, die zu drei grundlegenden Forschungsbereichen führte: \emph{Infrastructure and Services} (Infrastruktur und Service), \emph{Digital Humanities Research} (digitale geisteswissenschaftliche Forschung) und \emph{Cultural Heritage Research} (Kulturerbeforschung). Im Zuge dieser Reorganisation wurden dem Institut eine Reihe von Forschungsgruppen angegliedert.\\
            
        Seit 1. Jänner 2020 vereint das erweiterte \emph{Austrian Centre for Digital Humanities and Cultural Heritage} (ACDH-CH) zwei Schwerpunkte der Österreichischen Akademie der Wissenschaften in einem Institut, die (a) geisteswissenschaftliche Grundlagenforschung in Langzeitprojekten zur Erschließung und Erhaltung des kulturellen Erbes und (b) Forschung zu den methodischen und theoretischen Paradigmen der digitalen Dokumentation, Verarbeitung, Erforschung und Visualisierung der digitalen Geisteswissenschaften betreiben. Innerhalb des ACDH-CH sollen sich beide Säulen zunehmend gegenseitig befruchten und so zur Entwicklung einer gemeinsamen Arbeit am reichen Schatz des kulturellen Gedächtnisses Europas beitragen.\\
            
        Die Infrastruktureinheit bietet ein wachsendes Portfolio von Dienstleistungen an: Betrieb eines Repositoriums für digitale Ressourcen, Hosting und Veröffentlichung von Daten, Entwicklung von Software und Arbeit in einem eng geknüpften Netzwerk von spezialisierten Wissenszentren in ganz Europa durch eine beratende Funktion für die Forschungsgemeinschaft.\\
            
        In der DH-Forschung befasst sich das Team hauptsächlich mit text- und sprachbezogenen Fragen mit Schwerpunkt auf historischen Fragestellungen. Aktuelle Projekte decken ein breites Spektrum geisteswissenschaftlicher Bereiche ab und untersuchen technische Standards, Infrastrukturkomponenten, semantische Technologien und texttechnische Methoden. All diese Aktivitäten sind in die europäischen Infrastrukturkonsortien CLARIN-ERIC und DARIAH-EU eingebettet.\\
            
        Die CH-Säule des ACDH-CH konzentriert sich auf die langfristigen Forschungsaktivitäten der ÖAW. Die aktuellen CH-Forschungsgruppen befassen sich hauptsächlich mit der Erforschung von Sprache, Musik, Literatur und Biographien mit besonderem Schwerpunkt auf dem österreichischen Kontext.\\
            
        \subsection*{Projekte:}\href{https://www.oeaw.ac.at/de/acdh/about-acdh-ch/mission/}{ACDH-CH Mission Statement}\subsection*{Literatur:}\begin{itemize}\item Ďurčo, Matej; Mörth, Karlheinz: CLARIN-DARIAH. AT–Weaving the network. In: 9th Language Technologies Conference. Information Society–IS. Ljubljana: 2014.\item Kálmán, Tibor; Ďurčo, Matej; Fischer, Frank; Larrousse, Nicolas; Leone, Claudio; Mörth, Karlheinz; Thiel, Carsten: A landscape of data–working with digital resources within and beyond DARIAH. In: International Journal of Digital Humanities 1: 2019, S. 113–131.\end{itemize}\subsection*{Themen:}Institutionen\subsection*{Zitiervorschlag:}Ďurčo, Matej. 2021. ACDH-CH - Austrian Centre for Digital Humanities and Cultural Heritage. In: KONDE Weißbuch. Hrsg. v. Helmut W. Klug unter Mitarbeit von Selina Galka und Elisabeth Steiner im HRSM Projekt "Kompetenznetzwerk Digitale Edition". URL: https://gams.uni-graz.at/o:konde.1\newpage\section*{API} \emph{Schneider, Gerlinde; gerlinde.schneider@uni-graz.at}\\
        
    \emph{Application Programming Interfaces} (API) – im Deutschen als Programmierschnittstellen bezeichnet – ermöglichen es, auf eine definierte Weise auf maschinlesbare Daten zuzugreifen. Diese Schnittstellen machen eine Kommunikation/einen Datenaustausch von Maschinen untereinander möglich, ohne Rücksicht auf dahinter liegende Systeme und Technologien nehmen zu müssen. \\
            
        Von besonderer Relevanz sind sogenannte RESTFul APIs (REST steht für \emph{Representational State Transfer}). Diese setzen auf den Webstandards HTTP und HTML auf und erlauben es, auf Daten/Ressourcen kontrolliert über URLs zuzugreifen. Das ermöglicht es, Webanwendungen mit Daten aus verschiedenen, verteilten Ressourcen zu speisen. Gängige Formate zum Datenaustausch sind \href{http://gams.uni-graz.at/o:konde.215}{XML} und JSON sowie immer öfter auch das \href{http://gams.uni-graz.at/o:konde.131}{RDF}-basierte JSON-LD-Format. (Richardson/Amundsen 2013, S. 274f.)\\
            
        Auch im Bereich der \href{http://gams.uni-graz.at/o:konde.59}{Digitalen Edition} ist in den letzten Jahren eine zunehmende Nutzung von APIs feststellbar. Editionen sind dabei nicht mehr ausschließlich als statischer Text zu betrachten (Bleier/Klug 2018, S.VIf.), sondern werden immer mehr zu Anwendungen, die einerseits externe Daten konsumieren und zur Beantwortung von Forschungsfragen mit eigenen Daten verknüpfen (Witt 2018, S. 255 ff.) und andererseits selbst Daten zur Wieder- und Weiterverwendung bereitstellen. Im ersten Fall greifen Editionen beispielsweise als Clients auf Schnittstellen zu und integrieren die erhaltenen Daten beispielsweise auf Ebene der Benutzeroberfläche oder durch Anreicherung auf Datenebene.\\
            
        Für Digitale Editionen relevante APIs sind z. B. die APIs des \emph{\href{http://gams.uni-graz.at/o:konde.123}{International Image Interoperability Framework}} (IIIF), die einen definierten Zugriff auf Bildobjekte und -sammlungen anbieten, sowie die API des Webservices \emph{correspSearch}, die es ermöglicht, auf aggregierte Metadaten zu unterschiedlichsten Briefeditionen zuzugreifen. In diesem Bereich wurden in den letzten Jahren einige beispielhafte, unmittelbar für Editionen einsetzbare Client-Anwendungen entwickelt bzw. adaptiert (z. B. \emph{Mirador Open} und \emph{OpenSeadragon} für IIIF, \emph{csLink} für \emph{correspSearch}). In der klassischen Philologie Verbreitung findet \emph{Canonical Text Services} (CTS), ein Protokoll zur Identifizierung und Abfrage von Textpassagen über kanonische Referenzen. Zur Identifizierung und Anreicherung von editorischen Daten kommen außerdem häufig APIs mit universellerer Ausrichtung zur Anwendung. Hier zu nennen sind die Schnittstellen von Normdateien wie der \href{http://gams.uni-graz.at/o:konde.109}{GND} oder von themenbezogenen Datenbanken wie etwa \emph{\href{http://gams.uni-graz.at/o:konde.107}{GeoNames}} im geografischen Bereich. Auch allgemeine Wissensbasen wie \href{http://gams.uni-graz.at/o:konde.112}{Wikidata} und \emph{DBpedia} werden über deren Schnittstellen abgefragt und profitieren selbst immer öfter von Daten aus Digitalen Editionen, die über die entsprechenden APIs in den Datenpool zurückgespielt werden. \\
            
        \subsection*{Literatur:}\begin{itemize}\item Bleier, Roman; Klug, Helmut W.: Discussing Interfaces in Digital Scholarly Editing. In: Digital Scholarly Editions as Interfaces 12. Norderstedt: 2018, S. V-XV.\item Dumont, Stefan: correspSearch – Connecting Scholarly Editions of Letters. In: Journal of the Text Encoding Initiative: 2016.\item Richardson, Leonard; Amundsen, Mike: RESTful Web APIs. Beijing u.a.: 2013.\item Witt, Jeffrey C: Digital Scholarly Editions and API Consuming Applications. In: Digital Scholarly Editions as Interfaces 12. Norderstedt: 2018, S. 219–247.\end{itemize}\subsection*{Projekte:}\href{https://iiif.io}{International Image Interoperability Framework}, \href{https://projectmirador.org}{Mirador}, \href{https://openseadragon.github.io}{OpenSeadragon}, \href{https://correspsearch.net}{correspSearch}, \href{https://github.com/correspSearch/csLink}{csLink}, \href{http://cite-architecture.github.io/cts_spec/specification.html#cts}{Canonical Text Services}, \href{https://wiki.dbpedia.org}{DBpedia}, \href{https://json-ld.org/}{JSON-LD}\subsection*{Software:}\href{https://www.wikidata.org/wiki/Wikidata:Main_Page}{Wikidata}, \href{geonames.org}{Geonames}, \href{https://github.com/dbpedia-spotlight/dbpedia-spotlight/wiki}{DBpedia Spotlight}\subsection*{Verweise:}\href{https://gams.uni-graz.at/o:konde.123}{IIIF}, \href{https://gams.uni-graz.at/o:konde.107}{Geonames}, \href{https://gams.uni-graz.at/o:konde.109}{GND}, \href{https://gams.uni-graz.at/o:konde.112}{Wikidata}, \href{https://gams.uni-graz.at/o:konde.59}{Digitale Edition}, \href{https://gams.uni-graz.at/o:konde.87}{Bereitstellung von Forschungsdaten}\subsection*{Themen:}Archivierung, Software und Softwareentwicklung\subsection*{Zitiervorschlag:}Schneider, Gerlinde. 2021. API. In: KONDE Weißbuch. Hrsg. v. Helmut W. Klug unter Mitarbeit von Selina Galka und Elisabeth Steiner im HRSM Projekt "Kompetenznetzwerk Digitale Edition". URL: https://gams.uni-graz.at/o:konde.31\newpage\section*{Alternativen zur Textkodierung mit TEI} \emph{Hinkelmanns, Peter; peter.hinkelmanns@sbg.ac.at }\\
        
    Auf hierarchischen Graphen basieren zahlreiche Modelle der Textkodierung. Im
                  geisteswissenschaftlichen Bereich kann hier an erster Stelle das Modell der \emph{Text Encoding Initiative} (\href{http://gams.uni-graz.at/o:konde.178}{TEI}) (TEI: P5 Guidelines) genannt
                  werden, aber es folgen auch andere verbreitete Formate wie etwa \emph{WordprocessingML }(WordprocessingML 2017) oder die inzwischen abgelöste \emph{Extensible HyperText Markup Language} (XHTML) (XHTML
                     2000) diesem Ansatz. Gemein ist diesen Modellen, dass streng
                  hierarchisch Text von einer großen Einheit – etwa einem Absatz – hin zu
                  untergeordneten Einheiten – einem Satz – strukturiert wird.\\
            
        Diese strikte Hierarchisierung bedeutet, dass ein Überschneiden von Elementen
                  nicht möglich ist. Eine solche Überschneidung kann etwa ein über eine Seitengrenze
                  hinaus laufender Satz sein:\\
            
        \begin{verbatim}<page><s></page><page></s></page>\end{verbatim}Das kurze Beispiel ist ein Verstoß gegen die hierarchischen Regeln von \href{http://gams.uni-graz.at/o:konde.215}{XML}. Dieser Umstand ist wiederholt
                  diskutiert worden, etwa von Claus Huitfeldt (Huitfeldt 1994, S. 237),
                  Steven J. DeRose (DeRose 2004) und von Desmond Schmidt und Robert
                  Colomb. (Schmidt/Colomb 2009, S. 498–99) Trotz der TEI zugrunde
                  liegenden hierarchischen Struktur bietet TEI zahlreiche Möglichkeiten, wie leere
                  Elemente oder \emph{\href{http://gams.uni-graz.at/o:konde.171}{Stand-off-Markup}}, um auch solche überlappende Strukturen abzubilden. Übersichtlich hat dies
                  James Cummings dargestellt. (Cummings 2018)\\
            
        Eine Alternative zur Kodierung als hierarchischer Graph ist etwa die als
                  Variantengraph. Geprägt wurde diese Idee durch Desmond Schmidt und Robert Colomb
                     (Schmidt/Colomb 2009). Sie schlagen vor, Unterschiede zwischen
                  Varianten eines Textes als Graph darzustellen. So ergibt sich für jede Textversion
                  ein bestimmter Pfad durch den Graph. Umgesetzt wurde dies im Datenformat des
                  Textkollationierers \emph{CollateX }(Haentjens Dekker/Middell 2010–2019). Die Tokens des Textgraphs
                  bilden die Knoten des Graphs; die Kanten, welche die Knoten untereinander
                  verbinden, sind den einzelnen Textvarianten zugeordnet. \\
            
        Weitere Ansätze zur Textkodierung umfassen etwa: \\
            
        \begin{itemize}\item {Textformate: \href{http://gams.uni-graz.at/o:konde.184}{FtanML}}\item {Textformate: \href{http://gams.uni-graz.at/o:konde.185}{GrAF}}\item {Textformate: \href{http://gams.uni-graz.at/o:konde.186}{Kadmos}}\item {Textformate: \href{http://gams.uni-graz.at/o:konde.187}{LAF}}\item {Textformate: \href{http://gams.uni-graz.at/o:konde.188}{LMNL}}\item {Textformate: \href{http://gams.uni-graz.at/o:konde.189}{TAGML}}\item {Textformate: \href{http://gams.uni-graz.at/o:konde.190}{TexMECS} (GODDAG)}\item {Textformate: \href{http://gams.uni-graz.at/o:konde.191}{XStandoff}}\end{itemize}Eine Plattform für Editionsvorhaben, die Textvariation und Stemmata mittels
                  Graphen abbilden, ist \emph{Stemmaweb}. (Andrews/Mace
                     2013)\\
            
        \subsection*{Literatur:}\begin{itemize}\item Andrews, Tara; Macé, Caroline: Beyond the tree of texts: Building an empirical model of
                              scribal variation through graph analysis of texts and stemmata Beyond the tree of texts. In: Literary and Linguistic Computing 28: 2013, S. 504–521.\item Cummings, James: A World of Difference. Myths and misconceptions about
                              the TEI. In: DH2017. Montréal: 2017.\item DeRose, Steven J: Markup Overlap. A Review and a Horse. In: Extreme Markup Languages 2004 Proceedings. Montréal, Québec: 2004.\item Efer, Thomas: Graphdatenbanken für die textorientierten
                              e-Humanities. Leipzig: 2016. URL: \url{https://nbn-resolving.org/urn:nbn:de:bsz:15-qucosa-219122}.\item Documentation. URL: \url{https://collatex.net/doc/}\item Haentjens Dekker, Ronald; Bleeker, Elli; Buitendijk, Bram; Kulsdom, Astrid; Birnbaum, David J: TAGML: A markup language of many dimensions TAGML. In: Balisage: The Markup Conference 2018. Washington, DC: 2018.\item Huitfeldt, Claus: Multi-dimensional texts in a one-dimensional
                              medium. In: Computers and the Humanities 28: 1994, S. 235–241.\item Schmidt, Desmond; Colomb, Robert: A data structure for representing multi-version texts
                              online. In: International Journal of Human-Computer Studies 67: 2009, S. 497–514.\item Sperberg-McQueen, C. M.; Huitfeldt, Claus: GODDAG: A Data Structure for Overlapping
                              Hierarchies GODDAG. In: Digital Documents: Systems and Principles. Springer: 2004, S. 139–160.\item Structure of a WordprocessingML document (Open XML
                              SDK) WordprocessingML. URL: \url{https://docs.microsoft.com/en-us/office/open-xml/structure-of-a-wordprocessingml-document}\item TEI: P5 Guidelines TEI Guidelines. URL: \url{http://www.tei-c.org/Guidelines/P5/}\item XHTML™ 1.0 The Extensible HyperText Markup Language
                              (Second Edition). URL: \url{http://www.w3.org/TR/xhtml1}\end{itemize}\subsection*{Software:}\href{https://collatex.net}{CollateX}, \href{https://stemmaweb.net/}{The Stemmaweb
                           Project}\subsection*{Verweise:}\href{https://gams.uni-graz.at/o:konde.178}{TEI}, \href{https://gams.uni-graz.at/o:konde.184}{Textformate: FtanML}, \href{https://gams.uni-graz.at/o:konde.185}{Textformate: GrAF}, \href{https://gams.uni-graz.at/o:konde.186}{Textformate: Kadmos}, \href{https://gams.uni-graz.at/o:konde.187}{Textformate: LAF}, \href{https://gams.uni-graz.at/o:konde.188}{Textformate: LMNL}, \href{https://gams.uni-graz.at/o:konde.189}{Textformate: TAGML}, \href{https://gams.uni-graz.at/o:konde.190}{Textformate: TexMECS
                           (GODDAG)}, \href{https://gams.uni-graz.at/o:konde.191}{Textformate: XStandoff}\subsection*{Themen:}Annotation und Modellierung\subsection*{Zitiervorschlag:}Hinkelmanns, Peter. 2021. Alternativen zur Textkodierung mit TEI. In: KONDE Weißbuch. Hrsg. v. Helmut W. Klug unter Mitarbeit von Selina Galka und Elisabeth Steiner im HRSM Projekt "Kompetenznetzwerk Digitale Edition". URL: https://gams.uni-graz.at/o:konde.15\newpage\section*{Analysemethoden} \emph{Klug, Helmut W.; helmut.klug@uni-graz.at }\\
        
    Digitale Editionen beschränken sich in der Regel nicht ausschließlich auf die
                  reine Textpräsentation, sondern bieten zusätzliche Funktionen, mithilfe derer die
                  erstellten Daten analysiert werden können. Abhängig vom edierten Text und den
                  zugrunde liegenden Forschungsfragen stehen unterschiedliche Analysemethoden zur
                  Verfügung. Beim Versuch einer Gliederung könnte man zwischen der
                  Quellentextanalyse (\href{http://gams.uni-graz.at/o:konde.172}{Stemmatologie}, \href{http://gams.uni-graz.at/o:konde.105}{Collation},
                     \href{http://gams.uni-graz.at/o:konde.28}{Textgenese}, \href{http://gams.uni-graz.at/o:konde.113}{Lagenvisualisierung}, ...) oder ggf.
                  Analysen von Bilddigitalisaten (z. B. Layouterkennung, \href{http://gams.uni-graz.at/o:konde.149}{OCR}, \href{http://gams.uni-graz.at/o:konde.224}{HTR}), Audio- und Videodaten (z. B.
                  Texterkennung, Sentimentanalyse), Analysen semantischer Informationen (z. B. \href{http://gams.uni-graz.at/o:konde.54}{Datenvisualisierung}, \href{http://gams.uni-graz.at/o:konde.210}{Visualisierungstools},
                  Netzwerkanalyse, \href{http://gams.uni-graz.at/o:konde.74}{Dramenanalyse}),
                  inhaltlicher Erschließung mithilfe unterschiedlicher Suchstrategien (z. B. \href{http://gams.uni-graz.at/o:konde.82}{facettierte Suche}, \href{http://gams.uni-graz.at/o:konde.211}{Volltextsuche}) und Sprachanalysen
                  (z. B. \href{http://gams.uni-graz.at/o:konde.145}{NLP}, \href{http://gams.uni-graz.at/o:konde.141}{NER}, \href{http://gams.uni-graz.at/o:konde.194}{Textmining}, \href{http://gams.uni-graz.at/o:konde.71}{Distant
                     Reading}, Stilometrie) unterscheiden.\\
            
        \subsection*{Software:}\href{https://transkribus.eu/Transkribus/}{Transkribus}, \href{http://wlt.synat.pcss.pl/}{Virtual
                           Transcription Laboratory}, \href{https://collatex.net}{CollateX}, \href{http://donnevariorum.dh.tamu.edu/toolsandresources/collation-software/}{DV Coll}, \href{https://sites.google.com/a/ctsdh.luc.edu/hrit-intranet/documentation/tools}{HRIT Tools}, \href{http://juxtacommons.org}{Juxta-Commons}, \href{https://manuscriptdesk.uantwerpen.be/md/Main_Page}{Manuscript
                           Desk}, \href{http://www.fragmentarytexts.org/2012/11/simpletct-simple-text-comparison-tool/}{Simpletct}, \href{https://sourceforge.net/projects/tei-comparator/}{TEI
                           Comparator}, \href{https://code.activestate.com/ppm/Text-TEI-Collate/}{Text-TEI-collate}, \href{http://www.tustep.uni-tuebingen.de/}{TUSTEP}, \href{http://v-machine.org/}{Versioning Machine}, \href{https://stemmaweb.net/}{The Stemmaweb
                           Project}, \href{http://www.teitok.org/index.php?action=about}{TEITOK}, \href{https://d3js.org}{D3js}, \href{https://nodegoat.net/}{Node Goat}, \href{https://public.tableau.com/s/}{Tableau}, \href{https://stemmaweb.net/}{The Stemmaweb
                           Project}, \href{https://www.canva.com/}{Canva}, \href{http://www.chronozoomproject.org/}{ChronoZoom}, \href{https://geobrowser.de.dariah.eu/}{Dariah
                           Geobrowser}, \href{https://neatline.org/}{Neatline}, \href{https://www.overviewdocs.com/}{Overview}, \href{http://hdlab.stanford.edu/palladio/}{Palladio}, \href{https://rawgraphs.io/}{RAW (3D)}, \href{https://storymap.knightlab.com/}{Storymap}, \href{http://timeline.knightlab.com/}{TimelineJS}, \href{https://timelinestoryteller.com/app/}{TimelineStoryteller}, \href{https://github.com/leoba/VisColl}{Viscoll}, \href{https://visone.info/}{Visione}, \href{https://voyant-tools.org/}{Voyant}, \href{http://www.fon.hum.uva.nl/praat/}{Praat}, \href{http://corpus-tools.org/annis/}{ANNIS}, \href{http://www.teitok.org/index.php?action=about}{TEITOK}, \href{http://www.tustep.uni-tuebingen.de/}{TUSTEP}, \href{https://www.elastic.co/de/}{elasticsearch}, \href{http://oxygenxml.com/}{Oxygen}, \href{http://lucene.apache.org/solr/}{Solr}, \href{https://textgrid.de/}{TextGrid}, \href{https://transkribus.eu/Transkribus/}{Transkribus}, \href{https://www.clarin.eu/content/services}{CLARIN-mediated NLP-services}, \href{https://enrich.acdh.oeaw.ac.at}{enrich/stanbol
                           (ACDH-OeAW)}, \href{http://corpus-tools.org/pepper/}{SaltNPepper}, \href{https://code.google.com/archive/p/topic-modeling-tool/}{topic-modelling-tool}, \href{https://weblicht.sfs.uni-tuebingen.de/weblicht/}{weblicht}, \href{http://www.teitok.org/index.php?action=about}{TEITOK}, \href{http://opennlp.apache.org/}{Apache
                           OPENNLP}, \href{http://ucrel.lancs.ac.uk/claws/}{CLAWS
                           POS-Tagger for English}, \href{http://art.uniroma2.it/external/chaosproject/}{Chaos}, \href{http://www.languagecomputer.com/}{CiceroLight}, \href{https://github.com/dbpedia-spotlight/dbpedia-spotlight/wiki}{DBpedia Spotlight}, \href{http://www.talp.upc.edu/}{FreeLing}, \href{https://www.digitisation.eu}{IMPACT Tools and
                           Data}, \href{https://www.nltk.org/}{Natural Language Toolkit
                           (nltk)}, \href{http://cltk.org/}{Classical Language Toolkit
                           (cltk)}, \href{https://spacy.io/}{spacy }, \href{https://github.com/zalandoresearch/flair}{flair}, \href{https://www.ims.uni-stuttgart.de/forschung/ressourcen/werkzeuge/german-ner/}{German NER}, \href{https://github.com/tudarmstadt-lt/GermaNER}{GermaNER}, \href{https://www.cis.uni-muenchen.de/~schmid/tools/TreeTagger/}{TreeTagger}, \href{https://www.cis.uni-muenchen.de/~schmid/tools/RNNTagger/}{RNNTagger}, \href{https://github.com/tsproisl/SoMeWeTa}{SoMeWeTa}, \href{https://github.com/ee-2/SurrogateGeneration}{Surrogate Generation}, \href{http://neuroner.com/}{NeuroNER}\subsection*{Verweise:}\href{https://gams.uni-graz.at/o:konde.145}{NLP}, \href{https://gams.uni-graz.at/o:konde.141}{NER}, \href{https://gams.uni-graz.at/o:konde.194}{Textmining}, \href{https://gams.uni-graz.at/o:konde.71}{Distant Reading}, \href{https://gams.uni-graz.at/o:konde.54}{Datenvisualisierung}, \href{https://gams.uni-graz.at/o:konde.210}{Visaualisierungstools}, \href{https://gams.uni-graz.at/o:konde.74}{Dramennetzwerkanalyse}, \href{https://gams.uni-graz.at/o:konde.82}{facettierte Suche}, \href{https://gams.uni-graz.at/o:konde.211}{Volltextsuche}, \href{https://gams.uni-graz.at/o:konde.172}{Stemmatologie}, \href{https://gams.uni-graz.at/o:konde.105}{collation}, \href{https://gams.uni-graz.at/o:konde.28}{Textgenese}, \href{https://gams.uni-graz.at/o:konde.113}{Lagenvisualisierung}, \href{https://gams.uni-graz.at/o:konde.149}{OCR}, \href{https://gams.uni-graz.at/o:konde.224}{HTR}\subsection*{Themen:}Datenanalyse, Digitale Editionswissenschaft, Einführung\subsection*{Zitiervorschlag:}Klug, Helmut W. 2021. Analysemethoden. In: KONDE Weißbuch. Hrsg. v. Helmut W. Klug unter Mitarbeit von Selina Galka und Elisabeth Steiner im HRSM Projekt "Kompetenznetzwerk Digitale Edition". URL: https://gams.uni-graz.at/o:konde.16\newpage\section*{Annotation (Literaturwissenschaft: grundsätzlich)} \emph{Boelderl, Artur R.; artur.boelderl@aau.at / Fanta, Walter; walter.fanta@aau.at}\\
        
    Jedes Mal, wenn ein Ausdruck in einem Text erscheint, gewinnt er dort seine Bedeutung durch den Bezug zu einem anderen Text. Diese seine Iterabilität (zeichentheoretisch gesprochen: die allgemeine Wiederholbarkeit) als Möglichkeitsbedingung von Textualität überhaupt zeitigt im literaturwissenschaftlichen Bereich Konsequenzen, denen die digitale Online-Edition literarischer Textkorpora durch Annotation Rechnung zu tragen hat. Sie repräsentiert unter den gegenwärtigen technischen Bedingungen eine neuartige Form von \href{http://gams.uni-graz.at/o:konde.34}{Kommentar}, die von der Polysemie des literarischen Textes ausgeht, um sich in einer nicht einschränkenden, nicht urteilenden Art möglicher Verstehensweisen anzunehmen und diese im Rahmen der Kommentierung an die Leser zu vermitteln. \\
            
        Der althergebrachten editionswissenschaftlichen Forderung, dass der Kommentar dem Text den Weg bahnen solle, kann durch Annotation, d. h. Auszeichnung des Textes nach \href{http://gams.uni-graz.at/o:konde.215}{XML}/\href{http://gams.uni-graz.at/o:konde.178}{TEI}, Genüge getan werden, insofern sie erlaubt, jene textologisch bedingten Überschneidungen und Überlappungen der Grenzen zwischen Text und Kommentar zu markieren, ohne diese jedoch selbst zu übertreten. Das philologische Ziel, das Textkorpus als Resultat von und Beitrag zu einem größeren Ganzen im Sinne eines letztlich epochenspezifischen Diskurshorizonts erkennbar zu machen, wird dadurch erreicht, dass die Annotation sowohl seine diachronen wie synchronen Aspekte erfasst: diachron, insofern sie die textgenetische wie diskursgeschichtliche Perspektive verzeichnet, welcher sich die faktische Gestalt des Korpus, ob veröffentlicht oder nicht, ob ‘vollendet’ oder fragmentarisch, auf nicht-kausale bzw. nicht-deterministische Weise verdankt; synchron, insofern sie den den jeweiligen Zeitgenossen des Textkorpus mutmaßlich möglichen Verständnishorizont erschließen hilft, ohne den heutigen User auf diesen festzulegen. Die so verstandene Annotation, die die textuellen Dimensionen \href{http://gams.uni-graz.at/o:konde.28}{Textgenese} (in den drei Perspektivierungsgrößen \href{http://gams.uni-graz.at/o:konde.26}{Mikro-}, \href{http://gams.uni-graz.at/o:konde.24}{Meso-} und \href{http://gams.uni-graz.at/o:konde.23}{Makrogenese}), \href{http://gams.uni-graz.at/o:konde.21}{Intra-} und \href{http://gams.uni-graz.at/o:konde.20}{Intertextualität} sowie \href{http://gams.uni-graz.at/o:konde.19}{Interdiskursivität} umfasst und für die sich als Umsetzungsform eine \href{http://gams.uni-graz.at/o:konde.96}{Hybridedition} empfiehlt, stellt damit die Einlösung des unter traditionellen editorischen Bedingungen unmöglichen Desiderats sicher, dass die Kommentierung des Textes als Teil der digitalen Online-Edition gerade \textbf{nicht}  veralte. (Boelderl 2018, Fn.5 und Boelderl/Fanta 2020)\\
            
        \subsection*{Literatur:}\begin{itemize}\item Boelderl, Artur R.; Fanta, Walter: MUSIL ONLINE – Vorüberlegungen zur Kommentierung. In: Annotieren, Kommentieren, Erläutern. Aspekte des Medienwandels. Berlin, Boston: 2020, S. 147–157.\item Boelderl, Artur R.: Vom Livre irréalisé zum Texte hyperréalisé. In: Digitale Metamorphose: Digital Humanities und Editionswissenschaft: 2018.\end{itemize}Dieser Beitrag wurden im Kontext des FWF-Projekts "MUSIL ONLINE – interdiskursiver Kommentar" 
                  (P 30028-G24) verfasst.\subsection*{Software:}\href{http://oxygenxml.com/}{Oxygen}\subsection*{Verweise:}\href{https://gams.uni-graz.at/o:konde.19}{Annotation: Interdiskursivität}, \href{https://gams.uni-graz.at/o:konde.20}{Intertextualität (Fokus: Literaturwissenschaft)}, \href{https://gams.uni-graz.at/o:konde.21}{Intratextualität (Fokus: Literaturwissenschaft - Bsp. Musil)}, \href{https://gams.uni-graz.at/o:konde.28}{Textgenese}, \href{https://gams.uni-graz.at/o:konde.23}{Makrogenese (Fokus: Literaturwissenschaft - Bsp. Musil)}, \href{https://gams.uni-graz.at/o:konde.24}{Mesogenese (Fokus: Literaturwissenschaft - Bsp. Musil)}, \href{https://gams.uni-graz.at/o:konde.26}{Mikrogenese (Fokus: Literaturwissenschaft - Bsp. Musil)}, \href{https://gams.uni-graz.at/o:konde.96}{Hybridedition}, \href{https://gams.uni-graz.at/o:konde.29}{Annotationsstandards}, \href{https://gams.uni-graz.at/o:konde.30}{Annotationsumgebung}\subsection*{Projekte:}\href{http://musilonline.at}{Musil Online}\subsection*{Themen:}Annotation und Modellierung, Digitale Editionswissenschaft\subsection*{Zitiervorschlag:}Boelderl, Artur R.; Fanta, Walter. 2021. Annotation (Literaturwissenschaft: grundsätzlich). In: KONDE Weißbuch. Hrsg. v. Helmut W. Klug unter Mitarbeit von Selina Galka und Elisabeth Steiner im HRSM Projekt "Kompetenznetzwerk Digitale Edition". URL: https://gams.uni-graz.at/o:konde.17\newpage\section*{Annotationsstandards} \emph{Galka, Selina; selina.galka@uni-graz.at }\\
        
    Im Sinne der \href{http://gams.uni-graz.at/o:konde.6}{Digitalen Nachhaltigkeit} sollte bei der \href{http://gams.uni-graz.at/o:konde.17}{Annotation} von digitalen Objekten auf standardisierte Formate zurückgegriffen werden, um Daten besser austauschen oder zusammenführen zu können. Standardisierte Verfahren zur Textannotation sind am weitesten verbreitet und dokumentiert, aber auch für andere Objekte wie Musik, Bild oder Video werden Verfahren entwickelt und Standards etabliert. (Rapp 2017, S. 262)\\
            
        Besonders die Annotation von Text hat eine lange Geschichte und ist durch die Arbeit der \emph{Text Encoding Initiative} geprägt. Von der \href{http://gams.uni-graz.at/o:konde.178}{TEI} werden Richtlinien zur Kodierung von Text erstellt und etabliert, wobei darauf geachtet wird, dass die Annotationen sowohl für die Maschine als auch für den Menschen lesbar sind. (Rapp 2017, S. 262)\\
            
        Auch die Annotationen von Bildern sind üblicherweise textuell verfasst. Mit digitalen Werkzeugen können polygone Markierungen auf dem digitalen Bild angebracht werden, die über Pixelkoordinaten abgelegt und in XML abgespeichert werden können. (Jannidis/Kohle/Rehbein 2017, S. 264) Für zweidimensionale Vektorgrafiken wird häufig das SVG-Format, ein \href{http://gams.uni-graz.at/o:konde.215}{XML}-Standard, benutzt. Außerdem können Ontologien und Vokabularien wie \emph{Iconclass}, \href{http://gams.uni-graz.at/o:konde.108}{Getty}-\emph{Vocabularies} oder das \emph{\href{http://gams.uni-graz.at/o:konde.133}{CIDOC Conceptual Reference Model}} verwendet werden. (Rapp 2017, S. 263)\\
            
        Für gesprochene Sprache wird häufig der Standard EXMARaLDA (\emph{Extensible Markup Language for Discourse Annotation}) verwendet. Digitale musikalische Dokumente und Strukturen können z. B. nach den Richtlinien der MEI (\emph{Music Encoding Initiative}) oder mit \emph{MusicXML} annotiert werden (\href{http://gams.uni-graz.at/o:konde.139}{Digitale Musikedition}).\\
            
        Weitere Beispiele für Annotationsstandards:\\
            
        \begin{itemize}\item {KML (\emph{Keyhole Markup Language}), ein Standard zur Kodierung von Geodaten}\item {MathML (\emph{Mathematical Markup Language}), ein Standard zur Kodierung  von mathematischen Formeln}\item {CEI (\emph{Charters Encoding Initiative}), ein Standard zur Kodierung von mittelalterlichen Urkunden}\item {Metadatenstandards wie \href{http://gams.uni-graz.at/o:konde.128}{DCMI} oder \href{http://gams.uni-graz.at/o:konde.129}{METS}}\item {\href{http://gams.uni-graz.at/o:konde.131}{RDF} (\emph{Resource Description Framework}) für die Beschreibung von Ressourcen}\item {SKOS (\emph{Simple Knowledge Organization System}) für die Beschreibung von kontrollierten Vokabularien, Thesauri o. Ä.}\end{itemize}\subsection*{Literatur:}\begin{itemize}\item Rapp, Andrea: Manuelle und automatische Annotation. In: Digital Humanities. Eine Einführung. Stuttgart: 2017, S. 253–267.\item SKOS Simple Knowledge Organization System Reference SKOS. URL: \url{http://www.w3.org/TR/skos-reference}\item Dublin Core Metadata Initiative. URL: \url{https://dublincore.org}\item CIDOC CRM. URL: \url{http://www.cidoc-crm.org}\item Music Encoding Initiative. URL: \url{https://music-encoding.org}\item TEI: P5 Guidelines TEI Guidelines. URL: \url{http://www.tei-c.org/Guidelines/P5/}\item Charters Encoding Initiative. URL: \url{https://www.cei.lmu.de}\item Keyhole Markup Language. URL: \url{https://developers.google.com/kml}\end{itemize}\subsection*{Verweise:}\href{https://gams.uni-graz.at/o:konde.178}{TEI}, \href{https://gams.uni-graz.at/o:konde.108}{Getty}, \href{https://gams.uni-graz.at/o:konde.133}{CIDOC}, \href{https://gams.uni-graz.at/o:konde.128}{DCMI}, \href{https://gams.uni-graz.at/o:konde.129}{METS}, \href{https://gams.uni-graz.at/o:konde.131}{RDF}, \href{https://gams.uni-graz.at/o:konde.124}{Metadatenformate für Bilddateien}\subsection*{Themen:}Annotation und Modellierung, Einführung\subsection*{Zitiervorschlag:}Galka, Selina. 2021. Annotationsstandards. In: KONDE Weißbuch. Hrsg. v. Helmut W. Klug unter Mitarbeit von Selina Galka und Elisabeth Steiner im HRSM Projekt "Kompetenznetzwerk Digitale Edition". URL: https://gams.uni-graz.at/o:konde.29\newpage\section*{Annotationsumgebung} \emph{Galka, Selina; selina.galka@uni-graz.at }\\
        
    Digitales \href{http://gams.uni-graz.at/o:konde.17}{Annotieren} ist in den Digitalen Geisteswissenschaften ein zentraler Bereich und variiert je nach wissenschaftlichem Ziel und Forschungsgegenstand. (Kollatz u.a. 2017, S.15) Das schlägt sich auch in einzelnen fachspezifischen Annotationsumgebungen nieder, die unterschiedliche Funktionalitäten bieten können.\\
            
        Neben Annotationsumgebungen, die vor allem in der Korpuslinguistik genutzt werden, wie \emph{\href{http://gams.uni-graz.at/o:konde.212}{WebLicht}} oder \emph{ExMaralda}, gibt es beispielsweise Annotationsumgebungen zur Erkennung und Annotation von Handschriften, wie \emph{Transkribus}. Oft wird auch eine Verlinkung von Bild und Text ermöglicht (vgl. \emph{Textgrid}, \emph{Digital Mappa}). Annotationsumgebungen können on- oder offline verfügbar sein.\\
            
        Lordick u.a. (2016) definieren unter anderem folgende Anforderungen für Annotationswerkzeuge:\\
            
        \begin{itemize}\item {Organisation der Annotationsgegenstände in einem stabilen System}\item {Verfügbarkeit von Daten}\item {Berücksichtigung bestehender Austausch- und Beschreibungsformate}\item {Verknüpfbarkeit, Importier- und Exportierbarkeit von Inhalten}\item {Ein- und Ausblenden von unterschiedlichen Annotationsebenen}\item {\href{http://gams.uni-graz.at/o:konde.104}{Kollaboratives Arbeiten}, \href{http://gams.uni-graz.at/o:konde.14}{Versionierung}, Zugriffs- und Rechtemanagement von Akteuren}\item {Berücksichtigung von \href{http://gams.uni-graz.at/o:konde.9}{Lizenzen und Datenschutz}}\item {\href{http://gams.uni-graz.at/o:konde.205}{Usability} – leichte Erlern- und Bedienbarkeit}\end{itemize}Einige Ansätze und Werkzeuge sind bereits etabliert. Lordick u.a.  (2016) merken jedoch an, dass weitere Forschungsarbeit nötig ist, vor allem auch die Entwicklung von Werkzeugen für kollaborative Arbeit.\\
            
        \subsection*{Literatur:}\begin{itemize}\item Lordick, Harald; Becker, Rainer; Bender, Michael; Borek, Luise; Hastik, Canan; Kollatz, Thomas; Macha, Beata; Rapp, Andrea; Reiche, Ruth; Walkowski, Niels-Oliver: Digitale Annotationen in der geisteswissenschaftlichen Praxis. In: Bibliothek Forschung und Praxis 40: 2016.\item Kollatz, Thomas; Hegel, Philipp; Veentjer, Ubbo; Söring, Sibylle; Funk, Stefan E: Annotieren und Publizieren mit DARIAH-DE und TextGrid. In: DHd 2017. Digitale Nachhaltigkeit. Konferenzabstracts. 13. bis 18. Februar 2017. Universität Bern, S. 15–19.\end{itemize}\subsection*{Software:}\href{https://www.annotationstudio.org/}{Annotation Studio}, \href{https://github.com/gsbodine/crowd-ed}{Crowd-Ed}, \href{https://wiki.tei-c.org/index.php/CWRC-Writer}{CWRC-Writer}, \href{https://schoenberginstitute.org/dm-tools-for-digital-annotation-and-linking/}{Digital Mappa}, \href{http://www.bbaw.de/telota/software/ediarum}{Ediarum}, \href{https://github.com/faustedition/ext-imageannotation}{Faustedition Image Annotator }, \href{https://gate.ac.uk/download/}{GATE}, \href{https://sites.google.com/a/ctsdh.luc.edu/hrit-intranet/documentation/tools}{HRIT Tools}, \href{http://hyperimage.ws/de/}{Hyperimage}, \href{https://web.hypothes.is/}{hypothes.is}, \href{https://neatline.org/}{Neatline}, \href{https://nodegoat.net/}{Node Goat}, \href{http://oxygenxml.com/}{Oxygen}, \href{https://github.com/oxygenxml/TEI-Facsimile-Plugin}{Oxygen-TEI-Facsimile-Plugin}, \href{http://pybossa.com/}{PyBOSSA}, \href{http://corpus-tools.org/pepper/}{SaltNPepper}, \href{https://textgrid.de/}{TextGrid}, \href{https://www.textlab.org/about/}{TextLab}, \href{http://thepund.it/}{ThePundit}, \href{https://transkribus.eu/Transkribus/}{Transkribus}, \href{http://www.tustep.uni-tuebingen.de/}{TUSTEP}, \href{https://weblicht.sfs.uni-tuebingen.de/weblicht/}{weblicht}, \href{http://menus.nypl.org/}{What's On the Menu?}, \href{http://en.wikisource.org/wiki/Main_Page}{Wikisource}, \href{http://www.teitok.org/index.php?action=about}{TEITOK}, \href{https://github.com/dbpedia-spotlight/dbpedia-spotlight/wiki}{DBpedia Spotlight}, \href{https://www.digitisation.eu}{IMPACT Tools and Data}, \href{https://www.nltk.org/}{Natural Language Toolkit (nltk)}, \href{http://cltk.org/}{Classical Language Toolkit (cltk)}, \href{https://www.cis.uni-muenchen.de/~schmid/tools/TreeTagger/}{TreeTagger}, \href{https://www.cis.uni-muenchen.de/~schmid/tools/RNNTagger/}{RNNTagger}, \href{https://github.com/tsproisl/SoMeWeTa}{SoMeWeTa}, \href{https://catma.de/}{CATMA}, \href{https://exmaralda.org/de/}{EXMARaLDA Partitur-Editor}, \href{https://correspsearch.net/creator/index.xql?l=de}{CMIF Creator}, \href{https://www.edirom.de/edirom-projekt/index.html}{Edirom}, \href{https://github.com/acdh-oeaw/xsl-tokenizer}{xsl-tokenizer}, \href{http://www.txstep.de}{TXSTEP}\subsection*{Verweise:}\href{https://gams.uni-graz.at/o:konde.17}{Annotation (grundsätzlich)}, \href{https://gams.uni-graz.at/o:konde.12}{WebLicht}, \href{https://gams.uni-graz.at/o:konde.104}{Kollaboratives Arbeiten}, \href{https://gams.uni-graz.at/o:konde.14}{Versionierung}, \href{https://gams.uni-graz.at/o:konde.9}{Lizenzmodelle}, \href{https://gams.uni-graz.at/o:konde.205}{Usability}\subsection*{Themen:}Annotation und Modellierung\subsection*{Zitiervorschlag:}Galka, Selina. 2021. Annotationsumgebung. In: KONDE Weißbuch. Hrsg. v. Helmut W. Klug unter Mitarbeit von Selina Galka und Elisabeth Steiner im HRSM Projekt "Kompetenznetzwerk Digitale Edition". URL: https://gams.uni-graz.at/o:konde.30\newpage\section*{Apparat} \emph{Rieger, Lisa; lrieger@edu.aau.at / Klug, Helmut W.; helmut.klug@uni-graz.at / Bleier, Roman; roman.bleier@uni-graz.at }\\
        
    Unter einem Apparat im engeren Sinn versteht man in der Editionswissenschaft „die am Fuß der Seiten, im Anhang […], bei mehrbändigen Reihen auch zu e. Sonderband zusammengefaßten(!) \href{http://gams.uni-graz.at/o:konde.192}{textkrit.} Anmerkungen“ (Wilpert 2001, S. 41), im weiteren Sinn jedoch alles, „was nicht direkt als zusammenhängend gedruckter Edierter Text wiedergegeben ist“ (Scheibe 1988, S. 89). Abgesehen von seinen Hauptbestandteilen, der Beschreibung von Überlieferung und Textkonstitution sowie dem Variantenverzeichnis, kann er auch Informationen zur Entstehungs-, Text- und Wirkungsgeschichte, Register und Erläuterungen enthalten. (Scheibe 1988, S. 89) Neben dem \href{http://gams.uni-graz.at/o:konde.75}{Edierten Text} gilt er als wichtigster Bestandteil einer \href{http://gams.uni-graz.at/o:konde.93.}{historisch-kritischen Edition}(Träger 1986, S. 37)\\
            
        In der Geschichte der Editionswissenschaft wurden dem Apparat, je nach vorherrschender Auffassung, in ihrer Zahl und Aufgabenstellung unterschiedliche Funktionen zugesprochen. (Plachta 1996, S. 31–38) Heute zählt in historisch-kritischen Editionen zu seinen wichtigsten Aufgaben die detaillierte Dokumentation der \href{http://gams.uni-graz.at/o:konde.28}{Textgenese}, damit der Leser nicht auf die überlieferten Originaldokumente angewiesen ist. Der genaue Aufbau des Apparats wird jedoch sowohl von den Eigenschaften des zu edierenden Textes und der Arbeitsweise des Autors als auch vom mit der Edition verfolgten Ziel bestimmt: Auswahlapparat, Einblendungsapparat, Einzelstellenapparat, Emendationsapparat, Fußnotenapparat, Kolumnenapparat, Lemmaapparat, Positiver Apparat, Similienapparat, Stufenapparat, Synoptischer Apparat, Testimonienapparat, Variantenapparat, Zweitapparat (vgl. die jew. Stichwörter in Edlex).  \\
            
        Für die Gestaltung des Apparats im engeren Sinn gibt es folgende Möglichkeiten:\\
            
        1.	Einzelstellenapparat\textbf{:}  eignet sich für die Dokumentation von nicht komplexen Textabweichungen und Korrekturen und kann als ‘positiver’ oder ‘negativer’ Apparat vorkommen. Im positiven Apparat erfolgt die Angabe der jeweiligen Varianz inkl. einer Sigle für den Textträger bzw. Schreiber in einer eckigen Klammer direkt nach dem \href{http://gams.uni-graz.at/o:konde.115}{Lemma} aus dem edierten Text, im negativen Apparat wird sie ohne Lemma angeführt. Korrekturen in den Dokumenten können durch eine Zeichenkombination für den jeweiligen Korrekturvorgang ergänzt werden.\\
            
        2.	Einblendungsapparat: Korrekturvorgänge werden mit Hilfe einer Zeichenkombination und evtl. einer Sigle für Schreiber und Schreibmaterial direkt im fortlaufenden Text angegeben. Er eignet sich zur Beschreibung der Genese eines Textträgers oder mehrerer Textträger mit wenigen Abweichungen.\\
            
        3.	Treppenapparat: Korrekturvorgänge werden mit einer aufsteigenden Ziffern- oder Buchstabenfolge chronologisch geordnet, wobei die höhere Textstufe die ihr vorangehende aufhebt. Der Treppenapparat eignet sich für Textträger mit zahlreichen Anmerkungen und sowohl sofortigen als auch späteren Korrekturen.\\
            
        4.	\href{http://gams.uni-graz.at/o:konde.174}{Synoptischer Apparat}: Dieser führt zuerst die Grundform des Textes an und ergänzt diese anschließend um eine parallele Wiedergabe sämtlicher überlieferter Korrekturen, Varianten und Ergänzungen, die jeweils einen Verweis auf die betroffene Zeile des edierten Textes und eine Sigle für den Textträger erhalten. Der synoptische Apparat eignet sich für Arbeitsprozesse, die sich auf zahlreiche Textträger oder Textstufen verteilen. (Plachta 1996, S.  99–107)\\
            
        Der Umfang des Apparats bestimmt auch die Zuordnung zu bestimmten Editionstypen: Befolgt eine Edition zwar die wissenschaftlichen Prinzipien der Textkonstitution, kann in ihrem Apparat die Textgenese jedoch nicht dem Standard entsprechend wiedergeben, stellt sie nicht mehr eine ‘historisch-kritische’, sondern eine rein ‘kritische’ Edition dar. (Hagen 1988, S. 46)\\
            
        Die TEI beschreibt im Kapitel 12 der \emph{Guidelines}, wie Überlieferungsvarianten mithilfe von XML/TEI (u. a. <app>, <rdg>) modelliert werden können.  Im Rahmen digitaler Editionen ist z. B. eine Herausforderung, das herkömmliche Layout des Buches auf die Darstellung im Web zu übertragen oder die Kommentardaten mithilfe von Webtechnologien darzustellen. Ein Beispiel für die klassische Anzeige eines mehrstöckigen Apparates im Rahmen einer Digitalen Edition findet sich z. B. auf der Website \emph{Lyrik des deutschen Mittelalters} oder als interaktive Anzeige bei \emph{Saint Patrick’s Confessio} oder mit Hilfe der Editionssoftware EVT realisiert: \emph{Edizione Logica Avicennae}. \\
            
        \subsection*{Literatur:}\begin{itemize}\item Vorlesungsunterlagen: Einführung in die Editionsphilologie. Otto-Friedrichs-Universität Bamberg: 2009. URL: \url{https://www.uni-bamberg.de/fileadmin/uni/fakultaeten/split_professuren/literaturvermittlung/Materialien_NDL_I/ES_I_Materialien_Editionsphilologie_SoSe_2009.pdf}.\item Estis, Alexander: Glossar zur editionsphilologischen Fachterminologie. Albert-Ludwigs-Universität Freiburg: [Ohne Datum]. URL: \url{http://portal.uni-freiburg.de/germanistische-mediaevistik/studium/material/glossar}.\item Hagen, Waltraud: Von den Ausgabetypen Vom Umgang mit Editionen. In: Vom Umgang mit Editionen. Eine Einführung in Verfahrensweisen und Methoden der Textologie. Berlin: 1988, S. 31–54.\item Kraft, Herbert: Editionsphilologie. Zweite, neubearbeitete und erweitertete Auflage mit Beiträgen von Diana Schilling und Gert Vonhoff Editionsphilologie. Frankfurt am Main; Berlin; Bern; Bruxelles; New York; Oxford; Wien: 2001.\item Editionsphilologie: Apparat. URL: \url{https://literaturkritik.de/public/online_abo/lexikon-literaturwissenschaft-editionsphilologie-apparat,11,5,911}\item Plachta, Bodo: Editionswissenschaft. Eine Einführung in Methode und Praxis der Edition neuerer Texte Editionswissenschaft: 1997.\item Scheibe, Siegfried: Von den textkritischen und genetischen Apparaten Vom Umgang mit Editionen. In: Vom Umgang mit Editionen. Eine Einführung in Verfahrensweisen und Methoden der Textologie. Berlin: 1988, S. 85–159.\item Scheibe, Siegfried: Apparat Wörterbuch der Literaturwissenschaft. In: Wörterbuch der Literaturwissenschaft. Leipzig: 1986, S. 714.\item Wilpert, Gero von: Apparat Sachwörterbuch der Literatur. In: Sachwörterbuch der Literatur. 8., verbesserte und erweiterte Auflage. Stuttgart: 2001, S. 925.\item Editionslexikon. URL: \url{http://edlex.de/}\item P5: Guidelines for Electronic Text Encoding and Interchange. Ch. 12 Critical Apparatus. URL: \url{https://www.tei-c.org/release/doc/tei-p5-doc/en/html/TC.html}\item Apollon, Daniel; Béisle, Claire: The Digital Fate of the Critical Apparatus. In: Digital critical editions. Urbana, Chicago, Springfield: 2014.\end{itemize}\subsection*{Verweise:}\href{https://gams.uni-graz.at/o:konde.59}{Digitale Edition}, \href{https://gams.uni-graz.at/o:konde.75}{Editionstext}, \href{https://gams.uni-graz.at/o:konde.80}{Elemente digitaler Editionen}, \href{https://gams.uni-graz.at/o:konde.93}{historisch-kritische Edition}, \href{https://gams.uni-graz.at/o:konde.115}{Lemmatisierung}, \href{https://gams.uni-graz.at/o:konde.174}{Synopse}, \href{https://gams.uni-graz.at/o:konde.28}{Textgenese}, \href{https://gams.uni-graz.at/o:konde.192}{Textkritik und Kommentar}\subsection*{Software:}\href{http://evt.labcd.unipi.it/}{EVT}, \href{http://www.tustep.uni-tuebingen.de/}{TUSTEP}\subsection*{Projekte:}\href{http://www.ldm-digital.de}{Lyrik des deutschen Mittelalters}, \href{https://www.confessio.ie/etexts/confessio_latin#01}{Saint Patrick’s Confessio}, \href{http://evt.labcd.unipi.it/demo/evt2-beta2/avicenna/#/readingTxt?d=doc_1&p=C-112v&s=text-body-div&e=critical}{Edizione Logica Avicennae}, \href{https://edlex.de/}{Edlex}, \href{https://tei-c.org}{Text Encoding Initiative}\subsection*{Themen:}Digitale Editionswissenschaft\subsection*{Lexika}\begin{itemize}\item \href{https://edlex.de/index.php?title=Apparat}{Edlex: Editionslexikon}\item \href{https://wiki.helsinki.fi/display/stemmatology/Apparatus}{Parvum Lexicon Stemmatologicum}\item \href{https://lexiconse.uantwerpen.be/index.php/lexicon/critical-apparatus/}{Lexicon of Scholarly Editing}\end{itemize}\subsection*{Zitiervorschlag:}Rieger, Lisa; Klug, Helmut W.; Bleier, Roman. 2021. Apparat. In: KONDE Weißbuch. Hrsg. v. Helmut W. Klug unter Mitarbeit von Selina Galka und Elisabeth Steiner im HRSM Projekt "Kompetenznetzwerk Digitale Edition". URL: https://gams.uni-graz.at/o:konde.32\newpage\section*{Archivausgabe} \emph{Klug, Helmut W.; helmut.klug@uni-graz.at}\\
        
    Archivausgaben bereiten die historische Quelle mit einer \href{http://gams.uni-graz.at/o:konde.66}{diplomatischen Transkription} und dem parallelen Faksimile sowie zusätzlichen Sekundärinformationen (Überlieferung, Schrifterschließung, inhaltliche Erschließung und Kommentierung) auf, um – quasi als Archiv – den Benutzerinnen und Benutzern die Überlieferung in vollem Umfang und in hyperdiplomatischer Transkription zur Verfügung zu stellen: “Dem Namen steht dabei der Gedanke des abzubildenden Bestandes und der noch nicht erfolgten inhaltlichen Verarbeitung Pate, wie er mit dem klassischen Archiv als Dokumentenspeicher assoziiert wird.” (Sahle 2013, I, S. 218)\\
            
        Ziel ist dabei, die Quelle möglichst objektiv ohne \href{http://gams.uni-graz.at/o:konde.100}{Interpretation} der Befunde darzustellen. Kritikerinnen und Kritiker sprechen dieser Publikationsform den Charakter einer Edition ab. (Sahle 2013, I, S. 218–20) Digitale Editionen bieten sich schon aufgrund des Wegfallens aller Beschränkungen des Printmediums als Umsetzungsmethode für Archivausgaben an.\\
            
        \subsection*{Literatur:}\begin{itemize}\item Sahle, Patrick: Digitale Editionsformen. Zum Umgang mit der Überlieferung unter den Bedingungen des Medienwandels. Teil 1: Das typografische Erbe. Norderstedt: 2013.\item Shillingsburg, Peter L: Principles for Electronic Archives, Scholarly Editions, and Tutorial. In: The Literary text in the Digital Age. Ann Arbor: 1996, S. 23–35.\item Nutt-Kofoth, Rüdiger: Sichten - Perspektiven auf Text. In: Medienwandel / Medienwechsel in der Editionswissenschaft. Berlin/Boston: 2013, S. 19–29.\item Stadler, Peter: Die Grenzen meiner Textverarbeitung bedeuten die Grenzen meiner Edition. In: Medienwandel / Medienwechsel in der Editionswissenschaft. Berlin/Boston: 2013, S. 31–40.\end{itemize}\subsection*{Verweise:}\href{https://gams.uni-graz.at/o:konde.72}{documentary editing}, \href{https://gams.uni-graz.at/o:konde.65}{diplomatische Edition}, \href{https://gams.uni-graz.at/o:konde.66}{diplomatische Transkription}, \href{https://gams.uni-graz.at/o:konde.197}{Transkription}\subsection*{Projekte:}\href{http://www.nietzschesource.org/#eKGWB}{Nietzsche Source - Digitale kritische Gesamtausgabe}, \href{https://whitmanarchive.org}{The Walt Whitman Archive}, \href{http://vangoghletters.org/vg/}{Van Gogh Letters}\subsection*{Themen:}Digitale Editionswissenschaft\subsection*{Lexika}\begin{itemize}\item \href{https://edlex.de/index.php?title=Archivausgabe}{Edlex: Editionslexikon}\item \href{https://lexiconse.uantwerpen.be/index.php/lexicon/archival-collection-edition-2/}{Lexicon of Scholarly Editing}\end{itemize}\subsection*{Zitiervorschlag:}Klug, Helmut W. 2021. Archivausgabe. In: KONDE Weißbuch. Hrsg. v. Helmut W. Klug unter Mitarbeit von Selina Galka und Elisabeth Steiner im HRSM Projekt "Kompetenznetzwerk Digitale Edition". URL: https://gams.uni-graz.at/o:konde.33\newpage\section*{Audio- bzw. Musik- und Videoformate} \emph{Steiner, Christian; christian.steiner@uni-graz.at}\\
        
    Es gibt eine bereits seit Jahrzehnten geführte Diskussion über Audio- bzw. Musik- und Videoformate, die für die \href{http://gams.uni-graz.at/o:konde.6}{Langzeitarchivierung} (LZA) geeignet sind. Zu berücksichtigen sind folgende Aspekte:\\
            
        \begin{itemize}\item {Das Format sollte öffentlich und \href{http://gams.uni-graz.at/o:konde.152}{offen} dokumentiert werden.}\item {Das Format ist nicht-proprietär.}\item {Das Format ist weit verbreitet.}\item {Das Format ist selbstdokumentierend.}\item {Das Format kann mit leicht zugänglichen Werkzeugen geöffnet, gelesen und zugänglich gemacht werden.}\end{itemize}Eine digitale Audio- oder Videodatei besteht aus einem Container mit Quelldaten, die durch einen ‘Codec’ verarbeitet wurden. Ein Codec (\emph{coder-decoder, compressor-decompressor, compress-decompress}) wandelt das analoge Signal (von einem Mikrofon, einer Videokamera usw.) in die Einsen und Nullen einer digitalen Datei um. Ein Codec kann auch verwendet werden, um bereits in einem digitalen Format vorhandenes Material in einem anderen digitalen Format zu kodieren.\\
            
        Bei digitalen Audiodateien herrscht mittlerweile internationaler Konsensus über die einzusetzenden Formate, bei Videoformaten ist dieser Prozess noch nicht zur Gänze abgeschlossen:\\
            
        Audio (in absteigender Priorisierung)\\
            
        \begin{itemize}\item {\emph{Broadcast Wave }(BWF, WAV)}\item {\emph{Free Lossless Audio Codec} (FLAC)}\end{itemize}Video (keine Priorisierung vorhanden)\\
            
        \begin{itemize}\item {\emph{Digital Moving Picture Exchange Bitmap} (DPX)}\item {\emph{Audio Video Interleaved Format} (AVI)}\item {\emph{QuickTime File Format }(MOV)}\item {\emph{Windows Media Video 9 File Format }(WMV)}\item {MPEG 4}\item {\emph{Material Exchange Format} (MXF)}\end{itemize}\subsection*{Literatur:}\begin{itemize}\item National Archives: Frequently asked questions about Digital Audio and Video. URL: \url{https://www.archives.gov/records-mgmt/initiatives/dav-faq.html}\item Sustainability of Digital Formats: Planning for Library of Congress Collections Sustainability of Digital Formats. URL: \url{https://www.loc.gov/preservation/digital/formats/}\item Pau Saavedra i Bendito. Centre de Recerca i: ICA-PAAG. Short Guides. ICA – Photographic and Audiovisual Archives Group (PAAG): 2014. URL: \url{https://www.ica.org/sites/default/files/PAAG_guides_Digital_Video_Archive%20EN.pdf}.\item DFG-Praxisregeln "Digitalisierung", Deutsche Forschungsgemeinschaft: 2016. URL: \url{https://www.dfg.de/formulare/12_151/}.\end{itemize}\subsection*{Software:}\href{http://gams.uni-graz.at/archive/objects/o:gams.doku/methods/sdef:TEI/get?locale=de}{GAMS}, \href{https://duraspace.org/fedora/}{Fedora}, \href{https://www.nch.com.au/scribe/index.html}{Express Scribe}, \href{http://www.fon.hum.uva.nl/praat/}{Praat}, \href{https://iiif.io/}{iiif}, \href{https://www.anvil-software.org}{Anvil: Video Annotation Research Tool}, \href{https://github.com/digirati-co-uk/timeliner}{Variations Audio Timeliner}\subsection*{Verweise:}\href{https://gams.uni-graz.at/o:konde.6}{Digitale Nachhaltigkeit}, \href{https://gams.uni-graz.at/o:konde.152}{Open Access}, \href{https://gams.uni-graz.at/o:konde.95}{Hörspieledition}, \href{https://gams.uni-graz.at/o:konde.139}{Musikedition}, \href{https://gams.uni-graz.at/o:konde.85}{Filmedition}\subsection*{Themen:}Archivierung\subsection*{Zitiervorschlag:}Steiner, Christian. 2021. Audio- bzw. Musik- und Videoformate. In: KONDE Weißbuch. Hrsg. v. Helmut W. Klug unter Mitarbeit von Selina Galka und Elisabeth Steiner im HRSM Projekt "Kompetenznetzwerk Digitale Edition". URL: https://gams.uni-graz.at/o:konde.121\newpage\section*{Barrierefreies Webdesign} \emph{Galka, Selina; selina.galka@uni-graz.at }\\
        
    Unter barrierefreiem Webdesign versteht man die Gestaltung einer Website, so dass diese Benutzerinnen und Benutzern in größtmöglichem Ausmaß zur Verfügung steht, auch wenn die  Benutzerinnen und Benutzer Einschränkungen wie z. B. Seh-, Lern-, Hör-, Sprach-  oder Körperbehinderungen und Epilepsie unterliegen. (Jendryschick 2009, S. 90–91)\\
            
        Das W3C entwarf zu diesem Zwecke die \emph{Web Content Accessibility Guidelines} (WCAG) (WCAG 2018), wobei es sich um einen internationalen Standard zur barrierefreien Gestaltung von Webangeboten handelt. Die WCAG folgen den Prinzipien der Wahrnehmbarkeit, Bedienbarkeit, Verständlichkeit und Robustheit. Demnach sollen z. B. allen nicht-textlichen Inhalten Textalternativen beigefügt werden, die Inhalte sollten anpassbar, unterscheidbar, lesbar, verständlich und grundsätzlich per Tastatur zugänglich sein, auch sollte beispielsweise auf die Kompatibilität mit unterschiedlichen Zugangsgeräten geachtet werden. Wichtig ist außerdem die Überprüfung der Webseiten auf Konformität; diese kann z. B. mit online verfügbaren Validations-Services durchgeführt werden. (WCAG 2018)\\
            
        Die \emph{\href{http://gams.uni-graz.at/o:konde.205}{Usability}} spielt auch bei \href{http://gams.uni-graz.at/o:konde.59}{Digitalen Editionen} eine sehr große Rolle – die Edition muss den Benutzerinnen und Benutzern transparent und einfach zugänglich sein, um die Nutzung anzuregen. (del Turco 2011, Paragraph 6) Hier rücken z. B. gute hypertextuelle Funktionalitäten in den Fokus, der Umgang mit \emph{Special Characters}, Bildmanipulationstools oder erweiterte Suchfunktionalitäten. (del Turco 2011, Paragraph 24–29)\\
            
        \subsection*{Literatur:}\begin{itemize}\item Web Content Accessibility Guidelines (WCAG) Overview. URL: \url{https://www.w3.org/WAI/standards-guidelines/wcag/}\item Jendryschik, Michael: Einführung in XHTML, CSS und Webdesign. Standardkonforme, moderne und barrierefreie Websites erstellen. München: 2009.\item Radtke, Angie; Charlier, Michael: Barrierefreies Webdesign. Attraktive Websites zugänglich gestalten. München: 2008.\item Rosselli Del Turco, Roberto: After the Editing is Done. Designing a Graphic User Interface for Digital Editions. In: Digital Medievalist 7: 2011.\end{itemize}\subsection*{Verweise:}\href{https://gams.uni-graz.at/o:konde.206}{User-testing}, \href{https://gams.uni-graz.at/o:konde.207}{User-centered Design}, \href{https://gams.uni-graz.at/o:konde.205}{Usability}\subsection*{Projekte:}\href{https://validator.w3.org}{Markup Validation Service}\subsection*{Themen:}Interfaces\subsection*{Zitiervorschlag:}Galka, Selina. 2021. Barrierefreies Webdesign. In: KONDE Weißbuch. Hrsg. v. Helmut W. Klug unter Mitarbeit von Selina Galka und Elisabeth Steiner im HRSM Projekt "Kompetenznetzwerk Digitale Edition". URL: https://gams.uni-graz.at/o:konde.35\newpage\section*{Benutzerinnen und Benutzer Digitaler Editionen} \emph{Klug, Helmut W.; helmut.klug@uni-graz.at    }\\
        
    Die Frage nach Benutzerinnen und Benutzern von Editionen, nach deren Vorbildung und Aufgaben sowie nach dem Wechselspiel zwischen Editor, Edition und Benutzerinnen und Benutzern nahm mit dem schwindenden Einfluss des Lachmannschen Editionsparadigmas zu: Was kann man Benutzerinnen und Benutzern zumuten, was voraussetzen, was sollen Editorin und Editor und Edition an Information und Hilfestellung anbieten? \\
            
        So wird der Bezug zu Benutzerinnen und Benutzern u. a. auch in den \emph{DFG Förderkriterien für wissenschaftliche Editionen in der Literaturwissenschaft} mehrmals angesprochen. Außer Frage steht aber, dass allein schon Textauswahl und die Wahl eines bestimmten Editionstypus die Benutzergruppe bestimmen: Während historisch-kritische oder textgenetische Editionen eher auf ein wissenschaftlich vor- und ausgebildetes Publikum ausgerichtet sind und dessen aktive Mitarbeit für eine Rezeption Voraussetzung ist, sollen zweisprachige Texteditionen oder Studienausgaben Schülerinnen und Schüler, Studierende oder ‘die interessierten Laien’ ansprechen, da sie eine erhöhte Benutzbarkeit bieten. \\
            
        Die letztgenannte anonyme Gruppe möglicher Benutzerinnen und Benutzer wird in der Forschungsliteratur kontrovers diskutiert (vgl. Runow 2014, Henzel 2019), für die \href{http://gams.uni-graz.at/o:konde.59}{Digitale Edition}, die in der Regel \href{http://gams.uni-graz.at/o:konde.152}{frei verfügbar} im Internet publiziert wird, stellen Laiennutzer mitunter eine durchaus interessante Nutzergruppe dar, die vor allem im Zusammenhang mit den edierten Texten (Art der Texte, Inhalt der Texte etc.) zu- oder abnimmt. \\
            
        Zusätzlich zu den bereits für die Buchedition diskutierten Hürden (\href{http://gams.uni-graz.at/o:konde.32}{Variantenapparat}, textgenetische Informationen, editorische Eingriffe etc.) und Hilfsmitteln (\href{http://gams.uni-graz.at/o:konde.146}{Normalisierung}, \href{http://gams.uni-graz.at/o:konde.34}{Kommentar}, Übersetzung usw.) eröffnen sich für Benutzerinnen und Benutzer Digitaler Editionen durch das Medium Internet neue Probleme, aber auch neue Möglichkeiten der Rezeption. Zum einen ist der Umgang mit der Publikationsform Website im Vergleich mit der jahrhundertealten Buchtradition mitunter immer noch neu und ungewohnt, sodass das Auffinden von Informationen, das Lesen von Texten auf dem Bildschirm oder auch nur die Navigation durch die Seiten eine Herausforderung darstellen kann. Zum anderen setzt die Rezeption der neuen Publikationsform durchaus eine positive und aktive Herangehensweise der Benutzerinnen und Benutzer voraus. Generell sind Editorinnen und Editoren gefordert, die aktuellsten Erkenntnisse in Bezug auf ein \href{http://gams.uni-graz.at/o:konde.207}{benutzerfreundliches Webdesign} umzusetzen; Designfragen müssen sich aber letztendlich auch für die Digitale Edition am gewählten Editonstypus und den damit assoziierten Benutzerinnen und Benutzern orientieren, da unterschiedliche Editionsformen unterschiedlich komplexe Umsetzung erfordern. \\
            
        Fraglos überwiegen aber die Vorteile des digitalen Publizierens, da sich durch die digitale Umsetzung neue Möglichkeiten und Methoden ergeben, die Editionsdaten zu präsentieren: Das reicht von der Darstellung mehrfacher \href{http://gams.uni-graz.at/o:konde.174}{Textsynopsen} über das  Ein- und Ausblenden von Lesarten, \href{http://gams.uni-graz.at/o:konde.28}{textgenetischer} oder normalisierender Textstufen bis hin zu komplexen \href{http://gams.uni-graz.at/o:konde.54}{Visualisierungen}.\\
            
        Gerade im Zusammenhang mit der Digitalen Edition werden Benutzerinnen und Benutzer (oder besser: Userinnen und User) noch viel zu wenig berücksichtigt! Es gibt aber bereits Überlegungen zu Fragebögen (Henzel 2019, S. 76–79) und Auswertungen von Umfragen (Porter 2013 und 2016, Caria/Mathiak 2018), die einen überfälligen Richtungswechsel andeuten.\\
            
        \subsection*{Literatur:}\begin{itemize}\item Veit, Joachim: Musikedition 2.0: Das 'Aus' für den Edierten Notentext? In: editio 29: 2015, S. 70–84.\item Runow, Holger: Wem nützt was? Mediävistische Editionen (auch) vom Nutzer aus gedacht. In: editio. Internationales Jahrbuch für Editionswissenschaft 28: 2014, S. 50–57.\item Henzel, Katrin: Digitale genetische Editionen aus der Nutzerperspektive. In: Textgenese in der digitalen Edition. Berlin/Munich/Boston: 2019, S. 66–80.\item DFG: Informationen für Geistes- und Sozialwissenschaftler/innen: Förderkriterien für wissenschaftliche Editionen in der Literaturwissenschaft. DFG: 2015. URL: \url{https://www.dfg.de/download/pdf/foerderung/grundlagen_dfg_foerderung/informationen_fachwissenschaften/geisteswissenschaften/foerderkriterien_editionen_literaturwissenschaft.pdf}.\item Vom Nutzen der Editionen. Zur Bedeutung moderner Editorik für die Erforschung von Literatur- und Kulturgeschichte. Hrsg. von  und Thomas Bein. Berlin/Boston: 2015.\item Hofmeister, Wernfried: Beim Vorwort genommen. Historisch-kritischer Blick auf explizite Nutzwert-Reflexionen in Vorworten und sonstigen Selbsterläuterungen altgermanistischer Textausgaben auf Basis eines Grazer editionswissenschaftlichen Seminars. In: Editio 28: 2014, S. 68–81.\item Nutt-Kofoth, Rüdiger: Schreiben und Lesen. Für eine produktions- und rezeptionsorientierte Präsentation des Werktextes in der Edition. In: Text und Edition - Positionen und Perspektiven. Berlin: 2000, S. 165–202.\item Porter, Dot: Medievalists and the Scholarly Digital Edition. In: Scholarly Editing: The Annual of the Association for Documentary Editing 34: 2013.\item . In: "What is an edition anyway?" My Keynote for the Digital Scholarly Editions as Interfaces conference, University of Graz "What is an edition anyway?: 2016.\item Caria, Federico; Mathiak, Brigitte: A Hybrid Focus Group for the Evaluation of Digital Scholarly Editions of Literary Authors. In: Digital Scholarly Editions as Interfaces 12. Norderstedt: 2018, S. 267–285.\end{itemize}\subsection*{Verweise:}\href{https://gams.uni-graz.at/o:konde.98}{Interface}, \href{https://gams.uni-graz.at/o:konde.18}{Benutzerschnittstelle}, \href{https://gams.uni-graz.at/o:konde.35}{Barrierefreies Webdesign}, \href{https://gams.uni-graz.at/o:konde.54}{Datenvisualisierung}, \href{https://gams.uni-graz.at/o:konde.76}{Editionstypen}, \href{https://gams.uni-graz.at/o:konde.78}{Editor-testing}, \href{https://gams.uni-graz.at/o:konde.207}{User-centered Design}\subsection*{Projekte:}\href{http://www.ldm-digital.de}{Lyrik des deutschen Mittelalters}, \href{http://www.lokalbericht.ch}{http://www.lokalbericht.ch}, \href{http://digi.ub.uni-heidelberg.de/wgd/}{Welscher Gast Digital}\subsection*{Themen:}Digitale Editionswissenschaft, Einführung, Interfaces\subsection*{Zitiervorschlag:}Klug, Helmut W. 2021. Benutzerinnen und Benutzer Digitaler Editionen. In: KONDE Weißbuch. Hrsg. v. Helmut W. Klug unter Mitarbeit von Selina Galka und Elisabeth Steiner im HRSM Projekt "Kompetenznetzwerk Digitale Edition". URL: https://gams.uni-graz.at/o:konde.148\newpage\section*{Benutzerschnittstelle (Fokus: Literaturwissenschaft – Bsp. Musil)} \emph{Fanta, Walter; walter.fanta@aau.at / Boelderl, Artur R.;
                  artur.boelderl@aau.at}\\
        
    Mit Blick auf die Modellierung der Daten zur bestmöglichen Vermittlung derselben
                  an die Nutzerinnen und Nutzer auf einer nicht weiter vorab spezifizierten
                  Oberfläche (= Benutzerschnittstelle) wird das Prinzip \emph{annotation
                     first} empfohlen – es besagt, dass der prozedurale Zweck von
                  Textauszeichnungen hintangestellt wird. \href{http://gams.uni-graz.at/o:konde.17}{Annotationen} werden in erster Linie \textbf{nicht}  dazu
                  verwendet, um spezifische optische Lösungen für die Präsentation von
                  Textdokumenten in User-\href{http://gams.uni-graz.at/o:konde.98}{Interfaces}
                  zu generieren, sondern um eine langlebige, interoperable Nachnutzung der Daten zu
                  garantieren. Die Verschmelzung von Repräsentation und Präsentation in Digitalen
                  Editionen – wie etwa in der DVD-Edition der \emph{Klagenfurter
                     Ausgabe} – wird für überwunden erklärt, da das \href{http://gams.uni-graz.at/o:konde.126}{Markup} der Daten in dieser ungetrennten Ausgabe
                  über kurz oder lang unlesbar wird. Dagegen setzt sich \emph{Musilonline} aus getrennten Komponenten für die digitale Repräsentation in
                  Form von \href{http://gams.uni-graz.at/o:konde.215}{XML}/\href{http://gams.uni-graz.at/o:konde.178}{TEI}-Dokumenten und verschiedenen weiteren
                  Präsentationsweisen zusammen, die erst zu entwickeln sind. Für das Processing sind
                  zu unterscheiden: a) \textbf{transkriptive}  Annotationen, die
                  bestimmte Transformationsszenarien nahelegen, z. B. für HTML-Codierungen; b)
                  spezielle, erst in einem zweiten Schritt eingeführte \textbf{prozedurale
                  }  Annotationen für bestimmte Präsentationsformen.\\
            
        Grundsätzlich stellt \emph{Musilonline} die Daten sowohl als
                  XML/TEI im Open-Access-Download für die automatisierte Nachnutzung als auch als
                  html-basierte Schnittstellenlösung zur Verfügung. Letztere leitet sich nicht
                  zwingend aus den TEI-Annotationen ab. Optionale Lösungen führen zu Derivaten als
                  Datenbanken, Animations-Grafiken und den emendierten Lesetexten der Buchausgabe
                  (s. Stichwort \href{http://gams.uni-graz.at/o:konde.96}{Hybridedition}). Als
                  Zielobjekte für Transformationsszenarien kommen für die \href{http://gams.uni-graz.at/o:konde.26}{mikrogenetische} Ebene Schriftarten, für die
                  Darstellung der \href{http://gams.uni-graz.at/o:konde.24}{Mesogenese}
                  Tabellen, Grafiken, Animationen und für die \href{http://gams.uni-graz.at/o:konde.23}{Makrogenese} Datenbanken und deren Ausgabe-Optionen
                  in Frage. \\
            
        Als Beispiel für die Vermeidung von prozeduralen Annotationen im
                  TEI-Auszeichnungsschema für die \href{http://gams.uni-graz.at/o:konde.197}{Transkription} des Musil-Nachlasses sei die Verwendung des Elements
                     <note> angeführt.\\
            
        Bsp. 1: Musil, Nachlass, Mappe I/4, S. 10\\
            
        \begin{verbatim}<p>Zusammenfassung der beiden Träume: <note place="margin" resp="author">Besser: den 
nicht allzu interessanten Traum als unklare Erinnerung schildern.</note> Leib 
Verlassen u. doch Leib haben – Ihr Leib ist der ihres Bruders – … </p>\end{verbatim}Mit dem Attribut @resp wird geklärt, dass es sich um ein Autornotat handelt. Für alle
                  Randbemerkungen Musils in seinen Kapitelentwürfen wird diese Annotation
                  verwendet.\\
            
        Bsp. 2a: Musil, Nachlass, Mappe V/4, S. 1\\
            
        \begin{verbatim}<p>Oft dachte Ulrich, daß alles, was er mit Agathe erlebe, eine <note place="margin" 
resp="author"><seg xml:id="NP_83_1">1) U. nimmt das Nachdenken wieder auf.</seg>
<note resp="editor" target="#NP_83_1" n="1">Bleistift.</note></note> wechselseitige
Suggestion und nur unter dem Einfluß der … </p>\end{verbatim}Mit dem Attribut @resp wird geklärt, dass es sich beim ersten <note>
                  um ein Autornotat und beim zweiten um eine editorische Information handelt. Die
                  hier gezeigte Annotationspraxis entspricht dem Ergebnis der Datenmigration aus der
                     \emph{Klagenfurter Ausgabe}. Um die Vermengung von
                  transkriptiver und prozeduraler Annotation zu vermeiden, wird wie folgt
                  umgeformt:\\
            
        Bsp. 2b: Musil, Nachlass, Mappe V/4, S. 1\\
            
        \begin{verbatim}<p>Oft dachte Ulrich, daß alles, was er mit Agathe erlebe, eine <note place="margin" 
resp="author" hand="#hn_1">1) U. nimmt das Nachdenken wieder auf.</note> 
wechselseitige Suggestion und nur unter dem Einfluß der … </p>\end{verbatim}\subsection*{Literatur:}\begin{itemize}\item Fanta, Walter: Ein Schema für das Schreiben. Musils Nachlass als
                              Modell. In: Annotations in Scholarly Editions and Research.
                              Functions, Differentiation, Systematization. Berlin, Boston: 2020, S. 55–83.\item Fanta, Walter: Musil online total. In: Forschungsdesign 4.0. Datengenerierung und
                              Wissenstransfer in interdisziplinärer Perspektive. Dresden: 2019, S. 149–179.\item Lukas, Wolfgang: Archiv – Text - Zeit. Überlegungen zur Modellierung und
                              Visualisierung von Textgenese im analogen und digitalen Medium. In: Textgenese in der digitalen Edition. Berlin, Boston: 2019, S. 23–50.\item Sahle, Patrick: Digitale Editionsformen. Zum Umgang mit der
                              Überlieferung unter den Bedingungen des Medienwandels. Teil 3:
                              Textbegriffe und Recodierung. Norderstedt: 2013.\end{itemize}Dieser Beitrag wurden im Kontext des FWF-Projekts "MUSIL ONLINE – interdiskursiver Kommentar" 
                  (P 30028-G24) verfasst.\subsection*{Software:}\href{http://oxygenxml.com/}{Oxygen}\subsection*{Verweise:}\href{https://gams.uni-graz.at/o:konde.17}{Annotation (Literaturwissenschaft:
                           grundsätzlich)}, \href{https://gams.uni-graz.at/o:konde.28}{Textgenese}, \href{https://gams.uni-graz.at/o:konde.23}{Makrogenese (Fokus:
                           Literaturwissenschaft - Bsp. Musil)}, \href{https://gams.uni-graz.at/o:konde.24}{Mesogenese (Fokus:
                           Literaturwissenschaft - Bsp. Musil)}, \href{https://gams.uni-graz.at/o:konde.26}{Mikrogenese (Fokus:
                           Literaturwissenschaft - Bsp. Musil)}, \href{https://gams.uni-graz.at/o:konde.96}{Hybridedition}, \href{https://gams.uni-graz.at/o:konde.25}{Metadaten (Fokus:
                           Literaturwissenschaft - Bsp. Musil)}, \href{https://gams.uni-graz.at/o:konde.86}{XSLT}\subsection*{Projekte:}\href{http://musilonline.at}{Musil Online}\subsection*{Themen:}Annotation und Modellierung\subsection*{Zitiervorschlag:}Fanta, Walter; Boelderl, Artur R. 2021. Benutzerschnittstelle (Fokus: Literaturwissenschaft – Bsp.
               Musil). In: KONDE Weißbuch. Hrsg. v. Helmut W. Klug unter Mitarbeit von Selina Galka und Elisabeth Steiner im HRSM Projekt "Kompetenznetzwerk Digitale Edition". URL: https://gams.uni-graz.at/o:konde.18\newpage\section*{Bereitstellung von Digitalisaten} \emph{Klug, Helmut W.; helmut.klug@uni-graz.at }\\
        
    Digitalisate können Bilddateien in unterschiedlichen Formaten oder mit \href{http://gams.uni-graz.at/o:konde.149}{OCR}/ \href{http://gams.uni-graz.at/o:konde.224}{HTR} (\emph{Handwritten Text Recognition}) aufbereitete Dokumente im \href{http://gams.uni-graz.at/o:konde.215}{XML}-, PDF-Format usw. sein. \\
            
        Die Bereitstellung von faksimiliertem Quellenmaterial in Form digitaler Bilder ist
                  in der Regel die Aufgabe von Archiven und Bibliotheken, die neben eigenen
                  Digitalisierungsstrategien auch Auftragsdigitalisierungen der eigenen Bestände
                  durchführen. Neben der breiten Zugänglichmachung der Materialien spielen für die
                  Digitalisierung auch konservatorische Aspekte eine Rolle. Die jeweiligen
                  Institutionen setzen unterschiedliche technische Lösungen für diese Aufgabe ein
                  und sind darüber hinaus berechtigt, die Nutzungsbedingungen und -kosten für die
                  Digitalisate festzusetzen. Abhängig von der Ausrichtung der
                  Digitalisierungsbestrebungen (Massendigitalisierung vs. Einzeldigitalisate) kann
                  die Qualität der digitalen Produkte variieren. \\
            
        Mit \href{http://gams.uni-graz.at/o:konde.149}{OCR} oder \href{http://gams.uni-graz.at/o:konde.224}{HTR} erschlossene
                  Dokumente stellen – neben den Bilddaten – auch Textdaten zur Verfügung, die im
                  Rahmen einer Digitalisierungsinitiative (z. B. \emph{Google
                  Books}) oft automatisiert erstellt werden. Solche digitale Textdaten weisen
                  aber eine höhere Fehlerrate auf als die, die in Editionsprojekten z. B. durch
                  manuelle \href{http://gams.uni-graz.at/o:konde.197}{Transkription}
                  veröffentlicht werden.\\
            
        Für die Planung und Erstellung von \href{http://gams.uni-graz.at/o:konde.59}{Digitalen Editionen} wäre es von Vorteil, wenn Gedächtnisinstitutionen
                  möglichst hochwertige Digitalisate (Bild und Text) mit offenen Lizenzen (\href{http://gams.uni-graz.at/o:konde.152}{Open Access}, \href{http://gams.uni-graz.at/o:konde.45}{Creative Commons}) für die Digitalisate und
                  standardisierten Schnittstellen (\href{http://gams.uni-graz.at/o:konde.123}{IIIF}, \href{http://gams.uni-graz.at/o:konde.129}{METS}, \emph{Encoded Archival Description}) für das \emph{\href{http://gams.uni-graz.at/o:konde.10}{Metadata-Harvesting}} sowie mit \emph{\href{http://gams.uni-graz.at/o:konde.12}{Persistent Identifiern}} für einen langfristigen Zugriff auf die Ressourcen anbieten. Wird eine
                  Digitalisierung im Rahmen von Editionsprojekten vom Projektteam selbst
                  durchgeführt, sollten ähnliche Parameter gelten; Richtlinien dazu finden sich
                  z. B. in den DFG-Praxisregeln "Digitalisierung".\\
            
        \subsection*{Literatur:}\begin{itemize}\item Robinson, Peter: The digitization of primary textual sources. Oxford: 1993.\item DFG-Praxisregeln "Digitalisierung", Deutsche Forschungsgemeinschaft: 2016. URL: \url{https://www.dfg.de/formulare/12_151/}.\item Gotscharek, Annette; Reffle, Ulrich; Ringlstetter, Christoph; Schulz, Klaus U.: On Lexical Resources for Digitization of Historical
                              Documents. In: Proceedings of the 9th ACM Symposium on Document
                              Engineering. New York, NY, USA: 2009, S. 193–200.\item Vogeler, Georg: Ein Standard für die Digitalisierung mittelalterlicher
                              Urkunden mit XML. Bericht von einem internationalen Workshop in
                              München 5./6. April 2004. In: AfD 50: 2004, S. 23–34.\item Digitalisierung der schriftlichen historischen Quellen
                              in der Slowakei. In: Alte Archive - Neue Technologien: 2006, S. 257–265.\item Geiss, Jürgen: Bilder über Bilder. Erfahrungen mit der
                              datenbankgestützten Digitalisierung mittelalterlicher Handschriften im
                              Greifswalder Erschließungsprojekt. In: Das Mittelalter 14: 2009, S. 136–145.\item Klimpel, Paul; Rack, Fabian; Weitzmann, John H: Neue rechtliche Rahmenbedingungen für
                              Digitalisierungsprojekte von Gedächtnisinstitutionen: 2017.\end{itemize}\subsection*{Software:}\href{https://iiif.io/}{iiif}\subsection*{Verweise:}\href{https://gams.uni-graz.at/o:konde.37}{Bilddigitalisierungstechniken}, \href{https://gams.uni-graz.at/o:konde.40}{Checkliste Digitalisierung}, \href{https://gams.uni-graz.at/o:konde.44}{Copyright}, \href{https://gams.uni-graz.at/o:konde.61}{Digitalisierungsdienste}, \href{https://gams.uni-graz.at/o:konde.62}{Digitalisierungskosten}, \href{https://gams.uni-graz.at/o:konde.63}{Digitalisierungsrichtlinien
                           (Bilddigitalisierung)}, \href{https://gams.uni-graz.at/o:konde.64}{Digitalisierungsstandards}, \href{https://gams.uni-graz.at/archive/objects/context:konde/methods/sdef:Context/get?mode=workflow}{Workflow Digitalisierung}\subsection*{Projekte:}\href{https://en.wikipedia.org/wiki/List_of_digital_preservation_initiatives}{List of digital preservation Initiatives}, \href{https://www.e-codices.unifr.ch/en}{e-codices -
                           Virtual Manuscript Library of Switzerland}, \href{https://www.fragmentarium.ms}{Fragmentarium -
                           Digital Research Laboratory for Medieval Manuscript Fragments}\subsection*{Themen:}Digitalisierung\subsection*{Zitiervorschlag:}Klug, Helmut W. 2021. Bereitstellung von Digitalisaten. In: KONDE Weißbuch. Hrsg. v. Helmut W. Klug unter Mitarbeit von Selina Galka und Elisabeth Steiner im HRSM Projekt "Kompetenznetzwerk Digitale Edition". URL: https://gams.uni-graz.at/o:konde.36\newpage\section*{Bereitstellung von Forschungsdaten} \emph{Klug, Helmut W.; helmut.klug@uni-graz.at }\\
        
    Im Rahmen moderner digitaler Forschungsvorhaben sollten Forschungsdaten (Quellendaten, Metadaten, Workflows, Programmcode, Zwischenergebnisse), die eine weitere/vertiefende Forschung möglich machen würden, oder die Forschungsergebnisse und -methoden erst nachvollziehbar machen, der Forschungscommunity unter Berücksichtigung der \href{http://gams.uni-graz.at/o:konde.7}{FAIR Prinzipien} zur Verfügung gestellt werden. Gerade digitale Publikationsformate, wie auch die Digitale Edition eines ist, bieten sich dafür an! Hier wird es sogar möglich, Daten im laufenden Arbeitsprozess zu veröffentlichen bzw. veröffentlichte Daten laufend zu aktualisieren (\href{http://gams.uni-graz.at/o:konde.14}{Versionierung}). \\
            
        \subsection*{Literatur:}\begin{itemize}\item DFG: Leitlinien zum Umgang mit Forschungsdaten: 2015. URL: \url{https://www.dfg.de/download/pdf/foerderung/antragstellung/forschungsdaten/richtlinien_forschungsdaten.pdf}.\item nestor Handbuch. Eine keine Enzyklopädie der digitalen Langzeitarchivierung. Hrsg. von Heike Neuroth, Achim Oßwald, Regine Scheffel, Stefan Strathmann und Mathias Jehn. Glückstadt: 2016, URL: \url{urn:nbn:de:0008-2010071949}.\item Noch einmal: Was sind geisteswissenschaftliche Forschungsdaten?. URL: \url{https://dhd-blog.org/?p=5995}\item Wilkinson, Mark D.; Dumontier, Michel; Aalbersberg, IJsbrand Jan: The FAIR Guiding Principles for scientific data management and stewardship. In: Scientific Data 3: 2016, S. 160018.\end{itemize}\subsection*{Software:}\href{https://textgrid.de/}{TextGrid}, \href{http://gams.uni-graz.at/archive/objects/o:gams.doku/methods/sdef:TEI/get?locale=de}{GAMS}, \href{arche.acdh.oeaw.ac.at/}{ARCHE}\subsection*{Verweise:}\href{https://gams.uni-graz.at/o:konde.7}{FAIR Prinzipien}, \href{https://gams.uni-graz.at/o:konde.6}{Digitale Nachhaltigkeit}, \href{https://gams.uni-graz.at/o:konde.36}{Bereitstellung von Digitalisaten}, \href{https://gams.uni-graz.at/o:konde.9}{Lizenzmodelle}\subsection*{Themen:}Digitale Editionswissenschaft, Einführung, Archivierung, Rechtliche Aspekte\subsection*{Zitiervorschlag:}Klug, Helmut W. 2021. Bereitstellung von Forschungsdaten. In: KONDE Weißbuch. Hrsg. v. Helmut W. Klug unter Mitarbeit von Selina Galka und Elisabeth Steiner im HRSM Projekt "Kompetenznetzwerk Digitale Edition". URL: https://gams.uni-graz.at/o:konde.87\newpage\section*{Bilddigitalisierunstechniken} \emph{Klug, Helmut W.; helmut.klug@uni-graz.at }\\
        
    Der Begriff ‘Bilddigitalisierung’ verweist in der Regel auf einen Vorgang, der auf lichtbasierten Verfahren beruht; es werden dabei z. B. Scanner oder Digitalkameras eingesetzt, um Rastergrafiken (Regelfall) oder Vektorgrafiken der historischen Quellen zu erstellen. Durch den Einsatz von unterschiedlichen Auflageflächen (Scanner) und Objektiven (Digitalkamera) kann eine breite Palette an Vorlagenformaten aufgenommen werden. Filterkombinationen (Digitalkamera) ermöglichen es, unterschiedliche Wellenlängen des Lichtes digital abzubilden (Digitalkamera, Hyper-/Multispektralfotografie).\\
            
        Den Prozess zur Erstellung eines digitalen Abbildes eines physischen Objekts bezeichnet man als Analog-Digital-Wandlung, da dabei mithilfe physikalischer Sensoren das Objekt gemessen und die Messwerte als digitale Daten gespeichert werden. (Jannidis/Kohle/Rehbein 2017, S.179f.) Im Zuge dieser Umwandlung müssen die zeit- und wertekontinuierlichen Daten des analogen Signals (d. h. es sind unendlich viele Informationseinheiten verfügbar) in zeit- und wertediskrete digitale Daten (d. h. die Menge der Informationseinheiten sind begrenzt) einer Rastergrafik umgewandelt werden. Die Qualität der digitalen Daten wird durch die Auflösung und die Abtast-/Samplingrate (Audio) bestimmt – je höher die Werte, desto größer sind die Datenmenge und folglich die Datenqualität. Die Auflösung einer Rastergrafik ergibt sich aus den Werten der Bildgröße, die z. B. in ppi (\emph{pixel per inch}) oder Megapixel (Anzahl aller Pixel in der Grafik) angegeben werden kann, und der Farbtiefe, die in Bit angibt, wie groß die Menge der Farbinformationen pro Bildpunkt ist. Das Dateiformat, in dem Rastergrafiken abgespeichert werden, bedingt, ob und wie die Bildinformationen komprimiert werden.\\
            
        \subsection*{Literatur:}\begin{itemize}\item Kersken, Sascha: IT-Handbuch für Fachinformatiker. Der Ausbildungsbegleiter. Bonn: 2013.\item Jannidis, Fotis; Kohle, Hubertus: Digital Humanities. Eine Einführung. Mit Abbildungen und Grafiken Digital Humanities. Hrsg. von  und Malte Rehbein. Stuttgart: 2017.\item DFG-Praxisregeln "Digitalisierung", Deutsche Forschungsgemeinschaft: 2016. URL: \url{https://www.dfg.de/formulare/12_151/}.\end{itemize}\subsection*{Verweise:}\href{https://gams.uni-graz.at/o:konde.60}{Digitalisierung}, \href{https://gams.uni-graz.at/o:konde.63}{Digitalisierungsrichtlinen}, \href{https://gams.uni-graz.at/o:konde.36}{Bereitstellung von Digitalisaten}, \href{https://gams.uni-graz.at/o:konde.6}{Digitale Nachhaltigkeit}, \href{https://gams.uni-graz.at/o:konde.129}{Metadaten Schemata für LZA: METS}, \href{https://gams.uni-graz.at/o:konde.130}{Metadaten Schemata für LZA: PREMIS}\subsection*{Themen:}Metadaten\subsection*{Zitiervorschlag:}Klug, Helmut W. 2021. Bilddigitalisierunstechniken. In: KONDE Weißbuch. Hrsg. v. Helmut W. Klug unter Mitarbeit von Selina Galka und Elisabeth Steiner im HRSM Projekt "Kompetenznetzwerk Digitale Edition". URL: https://gams.uni-graz.at/o:konde.37\newpage\section*{Bildformate} \emph{Klug, Helmut W.; helmut.klug@uni-graz.at }\\
        
    Generell ist die \href{http://gams.uni-graz.at/o:konde.6}{Langzeitarchivierung} von Quellendigitalisaten (wie auch die \href{http://gams.uni-graz.at/o:konde.60}{Digitalisierung}) Aufgabe von Gedächtnisinstitutionen. Sollte dies nicht möglich sein, gibt die DFG bestimmte Regeln für die Praxis in Forschungsprojekten vor (DFG-Praxisregeln “Digitalisierung”). Diese Regeln umfassen u. a. \href{http://gams.uni-graz.at/o:konde.63}{Digitalisierungsrichtlinien} und Informationen zu den passenden Bildformaten. Die Informationen und Empfehlungen beziehen sich auf die Archivierung der digitalen Masterdatei.\\
            
        Die DFG empfiehlt (S. 20f.) als Dateiformat TIFF \emph{uncompressed} in Form von \emph{Baseline} TIFFs, von \emph{Extended} TIFFs wird abgeraten. TIFF \emph{uncompressed} wird empfohlen, weil es sich dabei um einen lange eingeführten Standard (seit den1980er-Jahren) handelt, von dem man annehmen kann, dass er auch weiterhin unterstützt werden wird. \\
            
        Eine qualitativ ähnliche Alternative ist JPEG2000 in der verlustfreien Kompressionsvariante, solange ausschließlich die lizenzfreien Bereiche verwendet werden.\\
            
        Es könnten theoretisch auch die Kompressionsformate TIFF-LZW oder JPEG2000 für die digitale Masterdatei verwendet werden. (Nestor 2016, Kap.17: 9-12) Das Problem von komprimierten Bilddateien ist aber die höhere Anfälligkeit für Datenverluste. Prinzipiell wird JPEG2000 als effizientes Kompressionsformat erachtet, als problematisch gilt aber die geringe Verbreitung und Marktdurchdringung sowie die mangelnde Softwareunterstützung.\\
            
        Proprietäre Kamera-RAW-Formate eignen sich nicht als Archivformate, da diese meist nur mit den entsprechenden Programmen der Hersteller angezeigt werden können. Formate wie JPEG oder PNG eignen sich nur für die Präsentation des Quellenmaterials im Internet. \\
            
        \subsection*{Literatur:}\begin{itemize}\item DFG-Praxisregeln "Digitalisierung", Deutsche Forschungsgemeinschaft: 2016. URL: \url{https://www.dfg.de/formulare/12_151/}.\item JPEG 2000. URL: \url{https://jpeg.org/jpeg2000/}\item JP2 (JPEG 2000 part 1). URL: \url{http://www.nationalarchives.gov.uk/PRONOM/Format/proFormatSearch.aspx?status=detailReport&id=686}\item Tagged Image File Format. URL: \url{http://www.nationalarchives.gov.uk/PRONOM/Format/proFormatSearch.aspx?status=detailReport&id=1099}\item RAW, JPEG and TIFF. URL: \url{https://archive.is/20130112013139/http://www.photo.net/learn/raw/}\item nestor Handbuch. Eine keine Enzyklopädie der digitalen Langzeitarchivierung. Hrsg. von Heike Neuroth, Achim Oßwald, Regine Scheffel, Stefan Strathmann und Mathias Jehn. Glückstadt: 2016, URL: \url{urn:nbn:de:0008-2010071949}.\end{itemize}\subsection*{Verweise:}\href{https://gams.uni-graz.at/o:konde.60}{Digitalisierung}, \href{https://gams.uni-graz.at/o:konde.63}{Digitalisierungsrichtlinien}, \href{https://gams.uni-graz.at/o:konde.36}{Bereitstellung von Digitalisaten}, \href{https://gams.uni-graz.at/o:konde.124}{LZA-Datenformate: Metadatenformate für Bilddateien}, \href{https://gams.uni-graz.at/o:konde.6}{Digitale Nachhaltigkeit}, \href{https://gams.uni-graz.at/o:konde.123}{IIIF}\subsection*{Themen:}Archivierung\subsection*{Zitiervorschlag:}Klug, Helmut W. 2021. Bildformate. In: KONDE Weißbuch. Hrsg. v. Helmut W. Klug unter Mitarbeit von Selina Galka und Elisabeth Steiner im HRSM Projekt "Kompetenznetzwerk Digitale Edition". URL: https://gams.uni-graz.at/o:konde.122\newpage\section*{Briefedition} \emph{Lobis, Ulrich; ulrich.lobis@uibk.ac.at / Wang-Kathrein, Joseph;
                  joseph.wang@uibk.ac.at}\\
        
    Ein Brief ist ein Schriftstück, das von einer Person (Absenderin/Absender oder
                  Schreiberin/Schreiber) an eine andere Person (Empfängerin/Empfänger) gesendet
                  wird. Eine Briefedition ist eine Edition von Briefen. \emph{Prima
                     facie} unterscheidet diese sich von anderen Arten von
                  Lebensdokumenten-Editionen nicht wesentlich. Jedoch gibt, es bedingt durch die
                  Eigenheit des Materials (Briefe), einige Besonderheiten zu berücksichtigen.\\
            
        Zunächst gibt es unterschiedliche Arten und Weisen, Briefe für die Edition
                  auszuwählen. Meistens steht eine Person oder eine Personengruppe im Fokus, die
                  Briefe verfasst oder empfangen hat. Es gibt aber auch Zugänge, die einen
                  bestimmten Zeitraum oder eine geographische Region in den Mittelpunkt stellen.
                  Gibt es eine Hauptperson in einer Edition und bietet diese auch Briefe an, die
                  weder von der Hauptperson verfasst, noch an sie gerichtet sind, so nennt man diese
                  Briefe ‘Drittbriefe’. Viele Briefe haben auch Beilagen – ob diese in die Edition
                  aufgenommen werden, hängt v. a. vom Ziel der Edition ab. Meistens wird auf solche
                  Beilagen nur in Anmerkungen hingewiesen.\\
            
        Da Briefe von einer Person geschrieben und an eine andere geschickt werden, ergibt
                  sich daraus eine kompliziertere Materialbeschaffung. Während die Briefe \emph{an} eine Person meistens in einem Archiv liegen, können die
                  Briefe \emph{von} ihr überall auf der Welt verstreut sein. Gerade
                  von Zeitgenossen oder Personen, deren Ableben noch nicht lange zurückliegt, können
                  Briefe, die als verschollen galten oder unbekannt ware, beispielsweise in
                  Auktionskatalogen oder Antiquariaten auftauchen. Das macht die Materialbeschaffung
                  schwieriger und bringt Ungewissheit in die Planung der Edition.\\
            
        Zusätzlich muss man auf \href{http://gams.uni-graz.at/o:konde.44}{Urheber- und
                     Verwertungsrechte} besonders Acht geben, da zwar dem Schreiber die
                  Urheberrechte zugeschrieben werden, der Empfänger aber auch Rechtsinhaber eines
                  Korrespondenzstücks sein kann. Unter gewissen Umständen erwerben sogar
                  Transkribienten und Erstellerinnen und Ersteller von Digitalisaten eigene
                  Urheberrechte. Somit sind mehrere Parteien in der Rechtsfrage bei einer
                  Veröffentlichung von Briefeditionen involviert.\\
            
        Aus der Eigenheit des Briefs ergibt sich auch, dass besondere \href{http://gams.uni-graz.at/o:konde.25}{Metadaten} aufzunehmen sind. Die \href{http://gams.uni-graz.at/o:konde.178}{TEI}-Richtlinie hat hierfür ein eigenes Element
                  vorgesehen: <correspDesc>(TEI correspDesc). In <correspDesc> werden die Metadaten ereignisbasiert beschrieben.
                  So wird in einem <correspAction> mit dem
                  möglichen @type-Wert \emph{sent} das
                  Absende-Ereignis mit sendender Person, Absendeort und Absendedatum, und in einem
                  anderen <correspAction> mit dem möglichen
                  @type-Wert \emph{received} das
                  Empfang-Ereignis mit empfangender Person, dem Empfangsdatum und dem Empfangsort
                  festgehalten. Zusätzlich können noch Vorgänger- und Folgebriefe in <correspContext> offengelegt werden.\\
            
        Oft kann ein Brief nicht eindeutig datiert werden. Werden die Briefe chronologisch
                  gereiht, so werden diese meistens an jenem Punkt eingereiht, der den spätest
                  möglichen Punkt der Datierung darstellt. Um die Auffindbarkeit von Briefen zu
                  steigern, ist es zudem ratsam, Normdateien für die Metadaten zu verwenden, sodass
                  die maschinlesbaren Metadaten nicht nur die Namen, sondern gleich die \href{http://gams.uni-graz.at/o:konde.147}{Normdaten} der beteiligten Personen
                  liefern.\\
            
        Von einem Brief kann es unterschiedliche Varianten geben. Neben den Vorstufen
                  (z. B. Vorlagen, Skizzen und Entwürfe) kann es auch Kopien geben, die der
                  Schreiber bei sich behält oder gar an andere Personen schickt. Unter Umständen
                  können diese als eigene Briefe in die Edition aufgenommen werden. Außerdem kann
                  ein physisches Korrespondenzstück unter Umständen mehrere Briefe in der Edition
                  rechtfertigen, wenn beispielsweise eine Person einen Brief erhält und diesen
                  später mit Kommentaren an eine andere Person weiterschickt.\\
            
        Ein Vorteil \href{http://gams.uni-graz.at/o:konde.59}{digitalen Edierens}
                  ist, dass Daten aus unterschiedlichen Quellen leichter zusammengetragen werden
                  können. Bei Briefeditionen sieht man das v. a. am Projekt \emph{correspSearch}. Editorinnen und Editoren von verschiedenen Briefeditionen
                  können ihre Metadaten dem Aggregationsdienst \emph{correspSearch}
                  zur Verfügung stellen, sodass Nutzerinnen und Nutzer eines Projekts gezielt nach
                  Briefen suchen können, die von einer Person stammen bzw. an sie gerichtet sind. So
                  können beispielsweise Gegenbriefe in unterschiedlichen Editionen viel leichter
                  aufgefunden werden.\\
            
        \subsection*{Literatur:}\begin{itemize}\item Dumont, Stefan: correspSearch – Connecting Scholarly Editions of
                              Letters. In: Journal of the Text Encoding Initiative: 2016.\item Vanhoutte, Edward; Branden, Ron Van den: Describing, transcribing, encoding, and editing modern
                              correspondence material: a textbase approach Describing, transcribing, encoding, and editing
                              modern correspondence material. In: Literary and Linguistic Computing 24: 2009, S. 77–98.\item TEI element correspDesc. URL: \url{https://tei-c.org/release/doc/tei-p5-doc/de/html/ref-correspDesc.html}\item Dumont, Stefan: Interfaces in Digital Scholarly Editions of
                              Letters. In: Digital Scholarly Editions as Interfaces 12. Norderstedt: 2018, S. 109–131.\item Dumont, Stefan: Kommentieren in digitalen Brief- und
                              Tagebuch-Editionen. In: Annotieren, Kommentieren, Erläutern: Aspekte des
                              Medienwandels. Berlin, Boston: 2020, S. 175–193.\end{itemize}\subsection*{Verweise:}\href{https://gams.uni-graz.at/o:konde.178}{TEI}, \href{https://gams.uni-graz.at/o:konde.59}{Digitale Edition}, \href{https://gams.uni-graz.at/o:konde.25}{Metadaten}, \href{https://gams.uni-graz.at/o:konde.147}{Normdaten}, \href{https://gams.uni-graz.at/o:konde.44}{Urheberrecht}\subsection*{Projekte:}\href{https://correspsearch.net}{correspSearch}, \href{https://www.uibk.ac.at/brenner-archiv/projekte/wittg_briefe/}{Digitale Edition des Gesamtbriefwechsels Ludwig Wittgensteins}, \href{https://www.uibk.ac.at/brenner-archiv/projekte/ficker_briefed/}{Kommentierte Online-Briefedition und Monografie: Umfassende Aufarbeitung
                           des Nachlasses von Ludwig von Ficker}, \href{https://edition.onb.ac.at/sauer-seuffert}{Briefwechsel August Sauer Bernhard Seuffert 1880 bis 1926 digital}, \href{https://www.briefedition.alfred-escher.ch}{Alfred Escher Briefedition}, \href{https://hdl.handle.net/11471/925.20}{Alexander
                           Rollett Briefedition}\subsection*{Themen:}Digitale Editionswissenschaft\subsection*{Zitiervorschlag:}Lobis, Ulrich; Wang-Kathrein, Joseph. 2021. Briefedition. In: KONDE Weißbuch. Hrsg. v. Helmut W. Klug unter Mitarbeit von Selina Galka und Elisabeth Steiner im HRSM Projekt "Kompetenznetzwerk Digitale Edition". URL: https://gams.uni-graz.at/o:konde.39\newpage\section*{CIDOC-Conceptual Reference Model (CRM)} \emph{Pollin, Christopher; christopher.pollin@uni-graz.at}\\
        
    Das konzeptionelle Referenzmodell CIDOC hat das Ziel, ein gemeinsames Verständnis von Information über jede Form von Kulturerbe-Objekten zu fördern, indem es einen erweiterbaren, semantischen Referenzrahmen definiert. Es steht im Selbstverständnis eines \emph{semantic glue}, sprich eines Leitfadens bzw. einer gemeinsamen Sprache, um Kulturerbe-Objekte und ihre Metainformationen formal zu beschreiben sowie den Austausch in Informationssystemen zu verbessern. \\
            
        Das CIDOC-CRM wird als \href{http://gams.uni-graz.at/o:konde.151}{Ontologie} im Sinne des \emph{\href{http://gams.uni-graz.at/o:konde.167}{Semantic Web}} angeboten und fungiert als sogenannte \emph{Top-Level-Ontology}. Die zentralen Klassen sind \emph{E 39 Actor} und \emph{E 70 Thing}, um unterschiedliche Akteurinnen und Akteure und abstrakte sowie physikalische Objekte im Kontext von Kulturerbe-Objekten zu beschreiben. Weiters gibt es übergeordnete Klassen, die räumliche (\emph{E53 Place}) und zeitliche Dimensionen (\emph{E2 Temporal Entity}) beschreiben. CIDOC ist eventbasiert (\emph{E5 Event)} aufgebaut, das heißt Kulturerbe-Objekte werden hinsichtlich der Ereignisse beschrieben, mit denen sie in Verbindung stehen. \\
            
        Weiters ist CIDOC-CRM modular aufgebaut. Es besteht aus einer Basisontologie CRM\emph{base}, die grundlegende Klassen und Beziehungen bereitstellt. Die modularen Erweiterungen dieses Basismodells sind so konzipiert, dass sie Spezialisierungen hinsichtlich Forschungsfragen und Dokumentationen ermöglichen. Beispiele dafür sind FRBR für bibliographische Daten oder CRM\emph{geo} für Daten aus dem Kontext von Geoinformationssystemen.\\
            
        \subsection*{Literatur:}\begin{itemize}\item CIDOC CRM. URL: \url{http://www.cidoc-crm.org}\item Doerr, Martin: The cidoc crm, an ontological approach to schema heterogeneity. In: Dagstuhl Seminar Proceedings. Schloss Dagstuhl-Leibniz-Zentrum für Informatik: 2005.\item Howarth, Lynne C.: FRBR and linked data: connecting FRBR and linked dat. In: Cataloging & Classification Quarterly 50: 2012, S. 763–776.\item Hiebel, Gerald; Doerr, Martin; Eide, Øyvind; Theodoridou, Maria: CRMgeo: A spatiotemporal extension of CIDOC-CRM. In: International Journal on Digital Libraries 18: 2017, S. 271–279.\item Doerr, Martin; Crofts, Nicholas: Electronic communication on diverse data—the role of an object-oriented CIDOC reference model. In: Proceedings of CIDOC’98. Melbourne: 1998.\end{itemize}\subsection*{Verweise:}\href{https://gams.uni-graz.at/o:konde.167}{Semantic Web}, \href{https://gams.uni-graz.at/o:konde.151}{Ontologie}, \href{https://gams.uni-graz.at/o:konde.168}{Semantic-Web-Technologien}\subsection*{Themen:}Metadaten, Archivierung\subsection*{Zitiervorschlag:}Pollin, Christopher. 2021. CIDOC-Conceptual Reference Model (CRM). In: KONDE Weißbuch. Hrsg. v. Helmut W. Klug unter Mitarbeit von Selina Galka und Elisabeth Steiner im HRSM Projekt "Kompetenznetzwerk Digitale Edition". URL: https://gams.uni-graz.at/o:konde.133\newpage\section*{COAR-Prinzipien} \emph{Stigler, Johannes; johannes.stigler@uni-graz.at }\\
        
    Die \emph{Confederation of Open Access Repositories} (COAR) ist
                  eine internationale Vereinigung von Repositorien-Initiativen und Netzwerken aus
                  dem Bibliotheksbereich, deren Ziel es ist, durch globale Vernetzung von \href{http://gams.uni-graz.at/o:konde.152}{Open Access}-Repositorien und
                  internationaler Kooperation die Sichtbarkeit von Forschungsergebnissen zu erhöhen
                  und Infrastrukturen für Open Science und Open Content bereitzustellen und
                  auszubauen. Österreich ist in COAR durch die Bibliotheken der Technischen
                  Universitäten Wien und Graz sowie durch die Universitätsbibliothek der \href{http://gams.uni-graz.at/o:konde.204}{Universität Wien} vertreten.\\
            
        Die Arbeit der Initiative orientiert sich an folgenden Prinzipien:\\
            
        \begin{itemize}\item {\emph{Distribution of control} – Die verteilte Kontrolle oder
                     Steuerung von wissenschaftlichen Ressourcen (Preprints, Postprints,
                     Forschungsdaten, unterstützende Software usw.) und wissenschaftlichen
                     Infrastrukturen wird als wichtiges Prinzip erachtet. Ohne dieses kann eine
                     kleine Anzahl von Akteuren zu viel Kontrolle erlangen und eine quasi
                     monopolistische Position aufbauen.}\item {\emph{Inclusiveness and diversity} – Verschiedene Institutionen
                     und Regionen haben einzigartige und besondere Bedürfnisse und Kontexte (z. B.
                     unterschiedliche Sprache, Politik und Prioritäten). Ein verteiltes Netzwerk von
                     Repositorien soll die unterschiedlichen Bedürfnisse und Kontexte verschiedener
                     Regionen, Disziplinen und Länder widerspiegeln.}\item {\emph{Public good} – Die Technologien, Architekturen und
                     Protokolle, die im Rahmen des globalen Netzwerks für Repositorien erarbeitet
                     und verwendet werden, stehen allen zur Verfügung und basieren auf globalen
                     Standards, soweit diese verfügbar sind.}\item {\emph{Intelligent openness and accessibility} –
                     Wissenschaftliche Ressourcen werden nach Möglichkeit offen und in zugänglichen
                     Formaten zur Verfügung gestellt, um ihren Wert zu steigern und ihre
                     Wiederverwendung zum Nutzen von Wissenschaft und Gesellschaft zu
                     maximieren.}\item {\emph{Sustainability} – Institutionen und
                     Forschungseinrichtungen sind wichtige Teilnehmer am globalen Netzwerk und
                     tragend zur langfristigen \href{http://gams.uni-graz.at/o:konde.6}{Nachhaltigkeit} der Ressourcen
                     bei.}\item {\emph{Interoperability} – Repositorien übernehmen gemeinsame
                     Funktionen und Standards, um die Interoperabilität zwischen Institutionen
                     sicherzustellen und eine gemeinsame Zusammenarbeit mit externen Dienstleistern
                     zu ermöglichen.}\end{itemize}Neben jährlich stattfindenden Konferenzen verfolgt die Initiative ihre Ziele in
                  den folgenden COAR-Working Groups:\\
            
        \begin{itemize}\item {\emph{Repository Interoperability}}\item {\emph{Open Access Language in Licences}}\item {\emph{Usage Statistics and Beyond}}\item {\emph{Linked Data Interest Group}}\item {\emph{Repository Observatory}}\item {\emph{Long Tail of Research Data}}\item {\emph{Libraries for Research Data}}\item {\emph{Joint Task Force on Librarians Competencies}}\item {\emph{Repository Impact and Visibility}}\end{itemize}\subsection*{Projekte:}\href{https://folk.uib.no/hnooh/mufi/}{Medieval
                           Unicode Font Initiative}\subsection*{Themen:}Rechtliche Aspekte, Archivierung, Institutionen\subsection*{Zitiervorschlag:}Stigler, Johannes. 2021. COAR-Prinzipien. In: KONDE Weißbuch. Hrsg. v. Helmut W. Klug unter Mitarbeit von Selina Galka und Elisabeth Steiner im HRSM Projekt "Kompetenznetzwerk Digitale Edition". URL: https://gams.uni-graz.at/o:konde.3\newpage\section*{Checkliste bzw. Prozess der Digitalisierung} \emph{Lenger, Karl; karl.lenger@uni-graz.at }\\
        
    Um Digitalisierungsprojekte erfolgreich und problemlos umsetzen zu können, ist ein
                  breites bibliothekarisches Vorwissen grundlegende Voraussetzung. Im Umgang mit
                  speziellen Beständen – wie zum Beispiel mittelalterlichen Handschriften – werden
                  wegen ihrer spezifischen Beschaffenheit und ihres Wertes als unikales Kulturgut
                  vertiefte Erfahrungen sowohl im Scanprozess als auch im Umgang mit sensiblen
                  Materialien benötigt. Oft haben kleinere Institutionen aufgrund finanzieller oder
                  personeller Ressourcenknappheit nicht die Möglichkeit, sich die nötige Expertise
                  auf dem Gebiet der \href{http://gams.uni-graz.at/o:konde.60}{Digitalisierung}
                  anzueignen. Aus diesem Grund werden von unterschiedlichen Einrichtungen, die
                  bereits über jahrelange Erfahrung in diesen Bereichen verfügen, strukturierte
                  Leitfäden angeboten.\\
            
        Im Handbuch \emph{Einführung in die Digitalisierung von gedrucktem
                     Kulturgut}(Weymann 2010) wird der Prozess der Digitalisierung in sechs Phasen
                  unterteilt. Dieses Handbuch gibt für Anfänger eine praktikable Einführung in die
                  einzelnen Phasen des Prozesses:\\
            
        \begin{itemize}\item { Vorbereitung: Projektdefinition, Dublettenprüfung, Materialzusammenstellung }\item { Festlegung der Rahmenbedingungen: Medieneigenschaften, Konservatorische
                     Prüfung, Festlegung der Verwendung, Festlegung der Scanparameter }\item { Durchführung und Finanzierung: Projektvoraussetzungen, Kostenschätzung,
                     Durchführung, Finanzierung, Abschluss der Organisation }\item { Digitale Erfassung und Aufbereitung: Auswahl der Technik, Scannen,
                     Qualitätskontrolle, Erzeugung eines Masters, Aufbereitung von Derivaten für
                     Präsentationszwecke }\item { Metadatenvergabe und Präsentation: Vergabe von Metadaten,
                     Qualitätskontrolle, Zusammenführen von Daten und Bildern, Präsentation, Zugriff
                     und Export }\item { Rücktransport und Speicherung: Vorbereitung zum Transport, Speicherung,
                     Langzeitsicherung }\end{itemize}Parallel dazu empfiehlt sich die Berücksichtigung folgender Literatur:
                  \emph{DFG-Praxisregeln}(2016), \emph{Bestandsaufnahme zur Digitalisierung von Kulturgut und
                  Handlungsfelder}(2007) und \emph{Rechtliche Rahmenbedingungen der Digitalisierung
                  kulturellen Erbes}(2018).\\
            
        \subsection*{Literatur:}\begin{itemize}\item DFG-Praxisregeln "Digitalisierung", Deutsche Forschungsgemeinschaft: 2016. URL: \url{https://www.dfg.de/formulare/12_151/}.\item Erstellt vom Fraunhofer-Institut für Intelligente Analyse- und
                                 Informationssysteme IAIS im Auftrag des BKM: Bestandsaufnahme zur Digitalisierung von Kulturgut und
                              Handlungsfelder: 2007.\item Ernst, Michael: Rechtliche Rahmenbedingungen der Digitalisierung
                              kulturellen Erbes. In: Bibliotheksdienst 52: 2018, S. 687–697.\item Weymann, Anna; Orozco, Rodrigo A. Luna; Müller, Christoph; Nickolay, Bertram; Schneider, Jan: Einführung in die Digitalisierung von gedrucktem
                              Kulturgut: ein Handbuch für Einsteiger. Berlin: 2010.\end{itemize}\subsection*{Verweise:}\href{https://gams.uni-graz.at/o:konde.60}{Digitalisierung}\subsection*{Themen:}Digitalisierung\subsection*{Zitiervorschlag:}Lenger, Karl. 2021. Checkliste bzw. Prozess der Digitalisierung. In: KONDE Weißbuch. Hrsg. v. Helmut W. Klug unter Mitarbeit von Selina Galka und Elisabeth Steiner im HRSM Projekt "Kompetenznetzwerk Digitale Edition". URL: https://gams.uni-graz.at/o:konde.40\newpage\section*{Citizen Science} \emph{Klug, Helmut W.; helmut.klug@uni-graz.at }\\
        
    Mit \emph{Citizen Science} wird jene wissenschaftliche Arbeitsform bezeichnet, bei der auf die Einbeziehung freiwilliger, nicht wissenschaftlicher Beiträgerinnen und Beiträger für die Generierung von Daten und Wissen gesetzt wird. \emph{Citizen Science} ist ein Aspekt von Wissenschaftskommunikation, da Nicht-Wissenschaftlerinnen und Nicht-Wissenschaftler im Zuge ihrer Arbeit einen tieferen Einblick in wissenschaftliche Arbeitsweisen und Forschung erhalten. Im Gegensatz zum \emph{\href{http://gams.uni-graz.at/o:konde.47}{Crowdsourcing}} stehen bei \emph{Citizen Science} ausschließlich wissenschaftliche Arbeitsschritte und Forschungsfragen im Mittelpunkt. Interessierte Laien bringen, meist unter der Leitung von Forscherinnen und Forschern, intellektuellen Input oder Forschungsdaten in laufende wissenschaftliche Forschungsprojekte ein. Bei \href{http://gams.uni-graz.at/o:konde.59}{digitalen Editionen} ist oft die \href{http://gams.uni-graz.at/o:konde.197}{Transkription} der Quellen ein Aufgabenbereich, der mithilfe von Laien realisiert wird. \\
            
        \subsection*{Literatur:}\begin{itemize}\item Wiggins, Andrea; Crowston, Kevin: From Conservation to Crowdsourcing: A Typology of Citizen Science. In: Proceedings of the 44th Hawaii International Conference on System Sciences (HICSS '10), January 2011: 2011.\item Dürrstein, Hubert: OeAD News. Bildung - Wissenschaft - Forschung - International: Citizen Science: Wir forschen mit.: 2015, URL: \url{https://oead.at/fileadmin/Dokumente/oead.at/KIM/Downloadcenter/Publikationen/Zeitschriften_und_Periodika/oead-news-97_WEB.pdf}.\item Vohland, Katrin; Dickel, Sascha; Ziegler, David; Mahr, Dominik: Virtuelle Bürgerwissenschaft: Digitale Ansätze in Citizen Science Projekten. In: GEWISS Bericht Nr. 2. Deutsches Zentrum für Integrative Biodiversitätsfor- schung (iDiv): 2015.\item Causer, Tim; Terras, Melissa: Digitalization, analysis and editions of textual sources. Crowdsourcing Bentham: Beyond the traditional boundaries of academic history. In: International Journal of Humanities and Arts Computing 8: 2014, S. 46–64.\item Causer, Tim; Tonra, Justin; Wallace, Valerie: Transcription maximized; expense minimized? Crowdsourcing and editing The Collected Works of Jeremy Bentham Transcription maximized; expense minimized?. In: Literary and Linguistic Computing 27: 2012, S. 119–137.\item Bonney, Rick; Phillips, Tina B.; Ballard, Heidi L.; Enck, Jody W.: Can citizen science enhance public understanding of science? In: Public Understanding of Science 25: 2016, S. 2-16.\end{itemize}\subsection*{Software:}\href{http://transcribe-bentham.ucl.ac.uk/td/Transcribe_Bentham}{Bentham Transcription Desk}, \href{https://diyhistory.lib.uiowa.edu}{Civil War Diaries & Letters Transcription Project}, \href{https://github.com/gsbodine/crowd-ed}{Crowd-Ed}, \href{http://www.ala.org.au/get-involved/citizen-science/fielddata-software/}{FieldData}, \href{https://fromthepage.com/}{FromThePage}, \href{http://edgerton-digital-collections.org/notebooks}{Harold "Doc" Edgerton Project}, \href{http://www.digiverso.com/de/products/viewer}{Gobi viewer}, \href{https://islandora.ca/}{Citizen Science, Collaboration}, \href{http://pybossa.com/}{PyBOSSA}, \href{http://github.com/zooniverse/Scribe}{Scribe}, \href{http://scripto.org/}{scripto}, \href{https://transkribus.eu/Transkribus/}{Transkribus}, \href{http://bencrowder.net/coding/unbindery/}{Unbindery}, \href{http://menus.nypl.org/}{What's On the Menu?}, \href{http://en.wikisource.org/wiki/Main_Page}{Wikisource}, \href{https://www.zooniverse.org/}{zooniverse}\subsection*{Verweise:}\href{https://gams.uni-graz.at/o:konde.169}{Social Edition}, \href{https://gams.uni-graz.at/o:konde.47}{Crowdsourcing}, \href{https://gams.uni-graz.at/o:konde.89}{Gamification}\subsection*{Projekte:}\href{https://www.oldweather.org}{Old Weather}, \href{https://www.topothek.at/de/}{Topothek}, \href{http://gastrosophie.sbg.ac.at/salzburg-zu-tisch/}{Salzburg zu tisch}, \href{https://www.citizenscience.gov/#}{citizenscience.gov}\subsection*{Themen:}Einführung, Digitale Editionswissenschaft\subsection*{Zitiervorschlag:}Klug, Helmut W. 2021. Citizen Science. In: KONDE Weißbuch. Hrsg. v. Helmut W. Klug unter Mitarbeit von Selina Galka und Elisabeth Steiner im HRSM Projekt "Kompetenznetzwerk Digitale Edition". URL: https://gams.uni-graz.at/o:konde.41\newpage\section*{Copy-Text-Edition} \emph{Rieger, Lisa; lrieger@edu.aau.at }\\
        
    Der Begriff Copy-Text stammt aus der anglo-amerikanischen Textkritik und
                  bezeichnet einen “aus verschiedenen Textfassungen nach jeweils festgelegten
                  Kriterien erarbeiteten ‘idealen’ Text”. (Plachta 1997, S. 137) Der
                  Terminus wurde 1904 von R. B. McKerrow geprägt und bezeichnete damals jenen Text,
                  den er bei der Edition der Werke von Thomas Nashe (McKerrow 1904) als
                  Basis für seinen Editionstext heranzog. 1939 schlug er zusätzlich die Anwendung
                  der eklektischen Methode (das Hinzuziehen weiterer Überlieferungsträger) vor, um
                  abgesehen vom Copy-Text zusätzliche möglichst autornahe Textstellen in den
                  Editionstext aufnehmen zu können, (Baender 1969, S. 313) denn als
                  beste Wahl für den Copy-Text galt lange der zeitlich gesehen autornächste Text.
                  Basierend auf Verfahren der \href{http://gams.uni-graz.at/o:konde.192}{Textkritik} unterschied Greg (1950, S. 21) bei Abweichungen
                  zwischen Textquellen \emph{substantives} und \emph{accidentals}: Unter \emph{substantives} verstand er jene
                  Lesarten, die unmittelbar die Bedeutungsabsicht des Autors bzw. “the essence of
                  his expression” betreffen, während er als \emph{accidentals} die
                  vorwiegend formale Präsentation des Texts in Form von Rechtschreibung u. Ä.
                  zusammenfasste. Im Fall von \emph{accidentals} solle dabei immer
                  dem Copy-Text Vorrang gewährt werden, während \emph{substantives}
                  erst durch textkritische Methoden überprüft werden sollten. (Greg 1950, S.
                     26) Nach Gabler ergibt sich anhand von Gregs Prinzipien ein
                  „kritisch-eklektischer Text als idealer Text von kumuliert größtmöglicher Nähe zur
                  Autorniederschrift.“ (Gabler 2003, Abs. 10)\\
            
        Während Greg seine Theorie auf gedruckte Bücher mit mehr als einer Fassung, deren
                  Vorpublikationsformen verloren gegangen waren, beschränkte, befürwortete Bowers
                     (1964, S. 226) die Ausweitung dieser Prinzipien auf den Fall von
                  überlieferten Manuskripten. Baender wandte jedoch ein, dass zuvor überprüft werden
                  müsse, ob und in welchem Ausmaß der Autor diese repräsentiert sehen wollte.
                  Dementsprechend postulierte er folgende grundlegende Vorgehensweise: Bei
                  Vorhandensein nur gedruckter Versionen sollte hinsichtlich der\emph{
                     accidentals} weiterhin die erste verlässliche Fassung als maßgebend
                  gewertet sowie für \emph{substantives} die eklektische
                  Vorgehensweise angewandt werden. Bei der zusätzlichen Überlieferung von
                  Manuskripten müssten jedoch Autoritätsentscheidungen auf individueller Basis
                  getroffen werden. (Baender 1969, S. 315- 317)\\
            
        Greetham (1990, S. 15) macht darauf aufmerksam, dass ab der Mitte der
                  80er-Jahre immer mehr Kritik an der Greg-Bowerschen Copy-Text-Methode aufkam – und
                  verweist dabei v.a. auf Jerome J. McGann und Hershel Parker. Parker bezeichnete
                  Greg’s Vorgehen dabei als zu streng und vertrat die Meinung, dass es die Aufgabe
                  editorischen Arbeitens sei, sich nicht auf einen eklektischen Text zu beschränken,
                  sondern in einer Edition auch die verschiedenen Schichten der Autorisation
                  wiederzugeben. McGann begründete mit der Ablehnung des Autors als einzigen Maßstab
                  für die Autorisation eines Werks die Theorie des \emph{social textual
                     criticism}, während sich Hans Walter Gabler in seiner \href{http://gams.uni-graz.at/o:konde.174}{synoptischen} Edition von \emph{Ulysses} an der \href{http://gams.uni-graz.at/o:konde.28}{genetisch}
                  ausgerichteten franco-germanischen Schule orientierte. (Ebd., S.
                     16–17) Durch diese Entwicklungen und weitere Forderungen nach einer
                  soziohistorischen und liberalisierenden Öffnung des konservativen Denkens gegen
                  Ende des 20. Jahrhunderts verlor die Copy-Text-Theorie ihr Monopol innerhalb der
                  angloamerikanischen Buchwissenschaft und \href{http://gams.uni-graz.at/o:konde.192}{Textkritik}. (Gabler 2003, Abs. 13)\\
            
        \subsection*{Literatur:}\begin{itemize}\item Baender, Paul: The Meaning of Copy-Text. In: Studies in Bibliography 22: 1969, S. 311–318.\item Bowers, Fredson: Some Principles for Scholarly Editions of
                              Nineteenth-Century American Authors. In: Studies in Bibliography 17: 1964, S. 223–228.\item A1 - Angloamerikanische Editionswissenschaft. URL: \url{http://www.edkomp.uni-muenchen.de/CD1/frame_edkomp_HWG.html}\item Greg, W. W: The Rationale of Copy-Text. In: Studies in Bibliography 3: 1950, S. 19–36.\item Greetham, David C: Politics and Ideology in Current Anglo-American Textual
                              Scholarship. In: Editio. Internationales Jahrbuch für
                              Editionswissenschaften 4: 1990, S. 1-20.\item McKerrow, Ronald Brunlees: The Works of Thomas Nashe. Edited from the original
                              texts. London: 1904.\item Plachta, Bodo: Editionswissenschaft. Eine Einführung in Methode und
                              Praxis der Edition neuerer Texte Editionswissenschaft: 1997.\end{itemize}\subsection*{Verweise:}\href{https://gams.uni-graz.at/o:konde.192}{Textkritik}, \href{https://gams.uni-graz.at/o:konde.174}{Synopse}, \href{https://gams.uni-graz.at/o:konde.169}{social edition}, \href{https://gams.uni-graz.at/o:konde.90}{genetische Edition}, \href{https://gams.uni-graz.at/o:konde.28}{Textgenese}\subsection*{Themen:}Digitale Editionswissenschaft\subsection*{Lexika}\begin{itemize}\item \href{https://lexiconse.uantwerpen.be/index.php/lexicon/copy-text/}{Lexicon of Scholarly Editing}\end{itemize}\subsection*{Zitiervorschlag:}Rieger, Lisa. 2021. Copy-Text-Edition. In: KONDE Weißbuch. Hrsg. v. Helmut W. Klug unter Mitarbeit von Selina Galka und Elisabeth Steiner im HRSM Projekt "Kompetenznetzwerk Digitale Edition". URL: https://gams.uni-graz.at/o:konde.43\newpage\section*{Correspondence Metadata Interchange Format (CMIF)} \emph{Kurz, Stephan; stephan.kurz@oeaw.ac.at }\\
        
    Das CMIF ist ein unerlässliches Werkzeug bei der Codierung von Briefen in
                  digitalen \href{http://gams.uni-graz.at/o:konde.39}{Briefeditionen}, aber
                  auch darüber hinaus. Es ermöglicht die standardisierte Aufnahme von
                  Korrespondenzmetadaten, die über das Aggregationsservice correspSearch geteilt und auffindbar gemacht werden können. \\
            
        Über das in TEI P5 aufgenommene <teiHeader>-Element <correspDesc>
                   ist CMIF Bestandteil der Richtlinien der \emph{\href{http://gams.uni-graz.at/o:konde.178}{Text Encoding Initiative}}.\\
            
        Typischer Anwendungsfall ist die strukturierte Aufzeichnung von Briefinformationen
                  – unter der Vorgabe, dass zu definieren ist, was ein Brief sei (vgl. dazu
                     immer noch gültig Honnefelder 1975, S. 4f.), was der Datenmodellierung
                  der jeweiligen Editorinnen und Editoren anheimgestellt bleibt. Beispiele dafür
                  sind Absender/Absenderin, Empfänger/Empfängerin, Sende- und Empfangsort und -zeit
                  sowie Verortung im betreffenden Briefwechsel. \\
            
        Das folgende Beispiel aus der Korrespondenz zwischen Schnitzler und Hofmannsthal
            vom 27.07.1891 (unter MIT-Lizenz von Martin Anton Müller auf GitHub veröffentlicht) illustriert dies: \\
            
        \begin{verbatim}
<teiHeader>
    <profileDesc>
        <correspDesc>
            <correspAction type="sent">
            <persName ref="#pmb2121">Schnitzler, Arthur</persName>
            <placeName ref="#pmb50">Wien</placeName>
            <date when="1891-07-27" n="01">27. 7. 1891</date>
            </correspAction>
            <correspAction type="received">
            <persName ref="#pmb11740">Hofmannsthal, 
            Hugo von</persName>
            <placeName ref="#pmb50">Wien</placeName>
            </correspAction>
            <correspContext>
            <ref type="belongsToCorrespondence" target="#pmb11740">Hofmannsthal,
            Hugo von</ref>
            <ref subtype="previous_letter" type="withinCorrespondence" 
            target="1891-07-13_01"> Hofmannsthal an Schnitzler, 13. 7. [1891]</ref>
            <ref subtype="next_letter" type="withinCorrespondence" 
            target="1891-08-11_01">Schnitzler an Hofmannsthal, 11. 8. 1891</ref>
            </correspContext>
        </correspDesc>
     </profileDesc>
</teiHeader>\end{verbatim}Das Format wird von der TEI-SIG \emph{Correspondence} betreut;
                  eine jeweils aktualisierte Dokumentation der Bestandteile von CMIF und der
                  TEI-Elemente stellt die SIG auf der correspSearch-Website
                  zur Verfügung. Insbesondere für die Erstellung von Briefverzeichnissen, die nicht
                  aus \href{http://gams.uni-graz.at/o:konde.59}{digitalen Editionen} extrahiert
                  werden können (etwa: Buchausgaben), wird auch ein Online-Tool angeboten, mit dem
                  Briefverzeichnisse in das CMIF konvertiert werden können: der CMIF Creator. \\
            
        \subsection*{Literatur:}\begin{itemize}\item Stadler, Peter; Illetschko, Marcel; Seifert, Sabine: Towards a Model for Encoding Correspondence in the TEI:
                              Developing and Implementing correspDesc Towards a Model for Encoding Correspondence in the
                              TEI. In: Journal of the Text Encoding Initiative: 2016.\item Dumont, Stefan: Perspectives of the further development of the
                              Correspondence Metadata Interchange Format. In: digiversity. Webmagazin für Informationstechnologie in
                              den Geisteswissenschaften: 2015.\item Dumont, Stefan: correspSearch – Connecting Scholarly Editions of
                              Letters. In: Journal of the Text Encoding Initiative: 2016.\item Dumont, Stefan; Börner, Ingo; Müller, Laackmann, Jonas; Leipold, Dominik; Schneider, Gerlinde: Correspondence Metadata Interchange Format
                              (CMIF). In: Encoding Correspondence. A Manual for Encoding Letters
                              and Postcards in TEI-XML and DTABf. Berlin: 2019.\item Honnefelder, Gottfried: Der Brief im Roman: Untersuchungen zur erzähltechnischen
                              Verwendung des Briefes im deutschen Roman. Bonn: 1975.\end{itemize}\subsection*{Software:}\href{http://oxygenxml.com/}{Oxygen}, \href{http://dcl.ils.indiana.edu/teibp/index.html}{TEI Boilerplate}, \href{https://sourceforge.net/projects/tei-comparator/}{TEI
                           Comparator}, \href{https://teipublisher.com/index.html}{TEI
                           Publisher}, \href{http://www.dnb.de/DE/Standardisierung/GND/gnd_node.html}{GND}, \href{geonames.org}{Geonames}, \href{https://correspsearch.net/creator/index.xql?l=de}{CMIF
                           Creator}\subsection*{Verweise:}\href{https://gams.uni-graz.at/o:konde.178}{TEI}, \href{https://gams.uni-graz.at/o:konde.39}{Briefedition}, \href{https://gams.uni-graz.at/o:konde.25}{Metadaten}, \href{https://gams.uni-graz.at/o:konde.126}{Markup}, \href{https://gams.uni-graz.at/o:konde.59}{Digitale Edition}\subsection*{Projekte:}\href{https://correspsearch.net}{correspSearch}, \href{https://edition-humboldt.de}{edition humboldt
                           digital}, \href{https://github.com/TEI-Correspondence-SIG/CMIF}{TEI-SIG
                           Correspondence}, \href{https://correspsearch.net/index.xql?id=participate_cmi-format}{Correspondence Metadata Interchange-Format: Dokumentation}, \href{https://correspsearch.net/creator/index.xql?l=de}{CMIF Creator:
                           Digitale Briefverzeichnisse erstellen}, \href{https://github.com/acdh-oeaw/schnitzler-briefe/blob/master/data/editions/1891-07-27_01.xml}{Schnitzler-Brief vom 27.07.1891}\subsection*{Themen:}Annotation und Modellierung, Metadaten, Digitale Editionswissenschaft\subsection*{Zitiervorschlag:}Kurz, Stephan. 2021. Correspondence Metadata Interchange Format (CMIF). In: KONDE Weißbuch. Hrsg. v. Helmut W. Klug unter Mitarbeit von Selina Galka und Elisabeth Steiner im HRSM Projekt "Kompetenznetzwerk Digitale Edition". URL: https://gams.uni-graz.at/o:konde.42\newpage\section*{Creative Commons} \emph{Klug, Helmut W.; helmut.klug@uni-graz.at }\\
        
    Creative Commons bedeutet übersetzt ‘künstlerisches Allgemeingut’ und ist der Name
                  einer gemeinnützigen Organisation, die 2001 in den USA gegründet wurde. Sie
                  verfolgt das Ziel, Wissen und Kulturgüter weltweit frei zugänglich zu machen,
                  indem sie Lizenzen und Tools zur Verfügung stellt, die Individualpersonen und
                  Organisationen dabei unterstützen, entsprechende Nutzungsbedingungen zu
                  formulieren. (Creative Commons: What we do)\\
            
        Die Art der Nachnutzung kann vollkommen offenen sein (CC-0; Österreich: CC BY)
                  oder komplett eingeschränkt (Lizenz CC BY-NC-ND). Die von Creative Commons
                  bereitgestellten Lizenzen orientieren sich an den Vorgaben internationaler
                  Handels- und Urheberrechtsabkommen. Bis zur Version 3.0 der Lizenzen war es auch
                  möglich, unter bestimmten individuellen Rechtsvorschriften (meist einzelner
                  Länder) zu lizenzieren (\emph{ported licenses}); die Lizenzen ab
                  Version 4.0 sind primär auf eine internationale Anwendbarkeit ausgerichtet. Alle
                  Lizenzen haben aber internationale Gültigkeit. (Creative Commons: What are
                     the international (“unported”) Creative Commons licenses) Sobald
                  Materialien mit einer CC-Lizenz veröffentlicht worden sind, kann diese \href{http://gams.uni-graz.at/o:konde.119}{Lizenzierung} für die Dauer des \href{http://gams.uni-graz.at/o:konde.44}{Urheberrechts} nicht mehr rückgängig
                  gemacht werden. Es ist aber durchaus möglich, für dasselbe Material zusätzliche
                  individuelle Lizenzen (z. B. für ein kommerzielles Druckvorhaben eines Werkes, das
                  auch unter einer nicht kommerziellen CC-Lizenz veröffentlicht ist) zu vergeben.
                  Für diesen Zweck bieten sich CC+(CCPlus)-Lizenzen an. (Creative Commons: CCPlus) CC-Lizenzen könnten aber auch in Inhalt und Wortlaut verändert
                  werden, dürfen danach allerdings nicht mehr als CC-Lizenzen bezeichnet werden.\\
            
        Zum Auswählen einer passenden Lizenz stellt Creative Commons auf seiner Website
                  ein interaktives Formular zur Verfügung, das neben der Lizenz auch die
                  urheberrechtsrelevanten Daten zum Werk als maschinenlesbare \href{http://gams.uni-graz.at/o:konde.25}{Metadaten} in HTML-Code generieren kann.
                     (Creative Commons: Choose a License) Die damit lizenzierten
                  Inhalte sind dadurch nicht nur mithilfe von Suchmaschinen, die CC-Inhalte
                  berücksichtigen, speziell als solche auffindbar, sondern der Klick auf den durch
                  den HTML-Code erstellten CC-Button generiert auch die entsprechenden
                  bibliografischen Daten zum Referenzieren des Werkes. \\
            
        \subsection*{Literatur:}\begin{itemize}\item Creative Commons: What We Do.: [Ohne Datum].\item Creative Commons: What are the international (“unported”) Creative Commons
                              licenses, and why does CC offer “ported” licenses?: [Ohne Datum]. URL: \url{https://creativecommons.org/faq/#what-are-the-international-unported-creative-commons-licenses-and-why-does-cc-offer-ported-licenses}.\item Creative Commons: CCPlus: 2017-12-21. URL: \url{https://wiki.creativecommons.org/wiki/CCPlus}.\item Creative Commons: Choose a License: [Ohne Datum]. URL: \url{https://creativecommons.org/choose/}.\item DFG-Praxisregeln "Digitalisierung", Deutsche Forschungsgemeinschaft: 2016. URL: \url{https://www.dfg.de/formulare/12_151/}.\end{itemize}\subsection*{Software:}\href{Vectr}{CC Lizenzgenerator}\subsection*{Verweise:}\href{https://gams.uni-graz.at/o:konde.119}{Lizenzierung}, \href{https://gams.uni-graz.at/o:konde.9}{Allgemein: Lizenzmodelle}, \href{https://gams.uni-graz.at/o:konde.44}{Urheberrecht}\subsection*{Projekte:}\href{https://creativecommons.org}{Creative
                           Commons}, \href{https://creativecommons.org/use-remix/cc-licenses/}{Creative
                           Commons Licenses}, \href{https://www.fwf.ac.at/de/forschungsfoerderung/open-access-policy/}{FWF Open Access Policy}, \href{https://creativecommons.org/faq/}{Creative
                           Commons Frequently Asked Questions}, \href{https://wiki.creativecommons.org/wiki/Main_Page}{Creative Commons
                           Wiki}\subsection*{Themen:}Rechtliche Aspekte\subsection*{Zitiervorschlag:}Klug, Helmut W. 2021. Creative Commons. In: KONDE Weißbuch. Hrsg. v. Helmut W. Klug unter Mitarbeit von Selina Galka und Elisabeth Steiner im HRSM Projekt "Kompetenznetzwerk Digitale Edition". URL: https://gams.uni-graz.at/o:konde.45\newpage\section*{Crowdsourcing} \emph{Eder, Elisabeth; elisabeth.eder@aau.at / Krieg-Holz, Ulrike; ulrike.krieg-holz@aau.at }\\
        
    \emph{Crowdsourcing} bezeichnet das Auslagern von traditionell intern durchgeführten Aufgaben an eine undefinierte Menge von externen Personen – meistens über das Internet. (Hammon und Hippner 2012) Die mögliche Heterogenität und Größe der Gruppe birgt großes Potential, kann sie doch unter bestimmten Bedingungen kollektiv Resultate erbringen, die einzelne Individuen nicht erzielen können. (Leihmeister 2010) Mögliche Ziele von \emph{Crowdsourcing}-Projekten sind z. B. das Sammeln von Ideen und Wissen oder die Erfüllung von einfachen, maschinell jedoch nicht realisierbaren Aufgaben. Die Mitarbeit der Partizipantinnen und Partizipanten an \emph{Crowdsourcing}-Projekten kann sowohl bezahlt als auch unbezahlt sein und auf verschiedenen Motivationen basieren (Geld, Reputation, Hilfsbereitschaft, Spaß  usw.). (Hammon und Hippner 2012) Dementsprechend verwenden Organisationen, die mit \emph{Crowdsourcing} arbeiten, auch unterschiedliche Motivationsstrategien (siehe z. B. \emph{\href{http://gams.uni-graz.at/o:konde.89}{Gamification}}). Zu bekannten Plattformen, die \emph{Crowdsourcing} für Organisationen anbieten und für diese bezahlte Microjobs an sogenannte \emph{Crowdworker} vergeben, zählen unter anderem \emph{Amazon Mechanical Turk}, \emph{FigureEight} und die deutsche Firma \emph{Clickworker}. Diese Microjobs sind jedoch nur bedingt an spezifische Problemstellungen anpassbar, weshalb vielfach eigene \emph{Crowdsourcing}-Möglichkeiten entwickelt werden. Als Beispiel für ein Projekt mit eigener \emph{Crowdsourcin}g-Plattform sei insbesondere auf \emph{Transkribus} verwiesen.\\
            
        \subsection*{Literatur:}\begin{itemize}\item Hammon, Larissa; Hippner, Hajo: Crowdsourcing. In: Business & Information Systems Engineering 4: 2012, S. 163–166.\item Howe, Jeff: Crowdsourcing. How the Power of the Crowd is Driving the Future of Business. New York, NY, USA: 2008.\item Leimeister, Jan Marco: Kollektive Intelligenz. In: Wirtschaftsinformatik 52: 2010, S. 239–242.\end{itemize}\subsection*{Software:}\href{https://transkribus.eu/Transkribus/}{Transkribus}, \href{http://transcribe-bentham.ucl.ac.uk/td/Transcribe_Bentham}{Bentham Transcription Desk}, \href{https://fromthepage.com/}{FromThePage}, \href{https://islandora.ca/}{Citizen Science, Collaboration}, \href{http://menus.nypl.org/}{What's On the Menu?}, \href{http://en.wikisource.org/wiki/Main_Page}{Wikisource}\subsection*{Verweise:}\href{https://gams.uni-graz.at/o:konde.89}{Gamification}, \href{https://gams.uni-graz.at/o:konde.104}{Collaboration}, \href{https://gams.uni-graz.at/o:konde.41}{Citizen Science}\subsection*{Projekte:}\href{https://www.mturk.com/}{Amazon Mechanical Turk}, \href{https://www.figure-eight.com/}{FigureEight}, \href{https://www.clickworker.de/}{Clickworker}\subsection*{Themen:}Digitalisierung, Digitale Editionswissenschaft\subsection*{Zitiervorschlag:}Eder, Elisabeth; Krieg-Holz, Ulrike. 2021. Crowdsourcing. In: KONDE Weißbuch. Hrsg. v. Helmut W. Klug unter Mitarbeit von Selina Galka und Elisabeth Steiner im HRSM Projekt "Kompetenznetzwerk Digitale Edition". URL: https://gams.uni-graz.at/o:konde.47\newpage\section*{DHA-Ontologie} \emph{Stigler, Johannes; johannes.stigler@uni-graz.at }\\
        
    Diese von der Arbeitsgruppe Archivierung unter KONDE entwickelte \href{http://gams.uni-graz.at/o:konde.151}{Ontologie} dient der formalen Beschreibung von analogen und digitalen Sammlungen sowie Objekten im Kontext der DH-Forschung und damit im Speziellen auch von \href{http://gams.uni-graz.at/o:konde.59}{Digitalen Editionen}. Die Grundüberlegungen basieren auf Prämissen des \emph{Europeana Data Models}, erweitern dessen Möglichkeiten jedoch vor allem um Aspekte der standardisierten, vertiefenden Beschreibung von Inhaltsdaten.\\
            
        Die Ontologie unterstützt die Interoperabilität, Wiederverwendbarkeit und das zentrale \emph{harvesting} von Objekten aus unterschiedlichen Forschungsdatenrepositorien unter der Annahme der \href{http://gams.uni-graz.at/o:konde.6}{Langzeitarchivierung} der Daten an verteilten Standorten, auch mit unterschiedlichen Repositoriumsinfrastrukturen. Dabei werden aber eben nicht nur klassische, deskriptive \href{http://gams.uni-graz.at/o:konde.25}{Metadaten} in den Fokus genommen. So existieren auch Klassen zur Primärdatenschema-übergreifenden \href{http://gams.uni-graz.at/o:konde.137}{Modellierung} von Inhaltsdaten (\emph{named entities} u. a.), z. B. zur Anreicherung des Suchraumes eines \emph{OAI-PMH Service Provides} (dha:\emph{AnnotationProperty}) mit Angaben wie Personennamen, Ortsbezeichnungen, Werktiteln u. v. m. aus den Primärdaten des digitalen Objektes. Darüber hinaus können Repositoriumsdienste (\emph{Services}) und Schnittstellen, z. B. Dienste zur Erzeugung von Präsentationsformen von Objektinhalten, formal beschrieben werden (dha:\emph{DisseminationService}).\\
            
        \subsection*{Verweise:}\href{https://gams.uni-graz.at/o:konde.151}{Ontologie}, \href{https://gams.uni-graz.at/o:konde.6}{Langzeitarchivierung}, \href{https://gams.uni-graz.at/o:konde.25}{Metadaten}\subsection*{Themen:}Archivierung, Institutionen\subsection*{Projekte:}\href{https://github.com/acdh-oeaw/dha-ontology}{DHA-Ontologie}, \href{http://visualdataweb.de/webvowl/#iri=https://raw.githubusercontent.com/acdh-oeaw/dha-ontology/master/dha-ontology.owl}{Visualisierung der DHA-Ontologie in WebVOWL}, \href{https://pro.europeana.eu/page/edm-documentation}{Europeana Data Model}\subsection*{Zitiervorschlag:}Stigler, Johannes. 2021. DHA-Ontologie. In: KONDE Weißbuch. Hrsg. v. Helmut W. Klug unter Mitarbeit von Selina Galka und Elisabeth Steiner im HRSM Projekt "Kompetenznetzwerk Digitale Edition". URL: https://gams.uni-graz.at/o:konde.5\newpage\section*{Data Mining} \emph{Lobis, Ulrich; ulrich.lobis@uibk.ac.at / Wang-Kathrein, Joseph; joseph.wang@uibk.ac.at}\\
        
    \emph{Data Mining} kann als "process of discovering interesting patterns and knowledge from large amounts of data [...] [where] data sources can include databases, data warehouses, the Web, other information repositories, or data that are streamed into the system dynamically" (Han/Kamber 2005, Kap. 1.10., S. 39) betrachtet werden. Es handelt sich um einen Sammelbegriff, der eng mit \emph{Machine Learning} und mit \emph{Data Science} verknüpft ist. Üblicherweise wird \emph{Data Mining} auf bestehende Daten angewandt. Für die DH als Teil der bevorzugt textorientierten Geisteswissenschaften sind vor allem \emph{\href{http://gams.uni-graz.at/o:konde.194}{Text Mining}} und \emph{Web Mining} von großem Interesse.\\
            
        Für die Anwendung von \emph{Data Mining} stehen zahlreiche spezialisierte, kommerzielle und freie Softwarelösungen zur Verfügung, wobei in erster Linie auf \emph{Java}, \emph{R} und \emph{Python} zu verweisen ist.  \\
            
        \subsection*{Literatur:}\begin{itemize}\item Han, Jiawei; Kamber, Micheline; Pai, Jian: Data Mining: Concepts and Techniques. Amsterdam [u.a.]: 2005.\end{itemize}\subsection*{Software:}\href{https://www.nltk.org/}{Natural Language Toolkit (nltk)}, \href{https://scikit-learn.org/}{scikit-learn}, \href{https://rapidminer.com}{RapidMiner Studio}, \href{https://www.cs.waikato.ac.nz/ml/weka/}{Weka}, \href{https://www.r-project.org}{R}\subsection*{Themen:}Datenanalyse\subsection*{Zitiervorschlag:}Lobis, Ulrich; Wang-Kathrein, Joseph. 2021. Data Mining. In: KONDE Weißbuch. Hrsg. v. Helmut W. Klug unter Mitarbeit von Selina Galka und Elisabeth Steiner im HRSM Projekt "Kompetenznetzwerk Digitale Edition". URL: https://gams.uni-graz.at/o:konde.48\newpage\section*{Datenaggregation} \emph{Klug, Helmut W.; helmut.klug@uni-graz.at }\\
        
    Datenaggregation bezeichnet den (automatisierten) Vorgang der Datenzusammenführung aus unterschiedlichen Datenbeständen, um daraus ein neues oder erweitertes Datenset zu erstellen. Ziel der Datenaggregation ist es in der Regel, große Datensets für überblicksartige Analysen (\emph{\href{http://gams.uni-graz.at/o:konde.48}{Data Mining}}) anzulegen. Bekannte Aggregatoren sind z. B. \emph{Europeana} (\href{http://gams.uni-graz.at/o:konde.25}{Metadaten} von Sammlungsgegenständen europäischer Kultureinrichtungen) oder \emph{correspSearch} (Metadaten von \href{http://gams.uni-graz.at/o:konde.39}{Briefeditionen}): “The web service correspSearch aggregates the metadata provided by different scholarly editions and displays them via a graphical user interface for a centralized search as well as via an API for automated queries and further use.” (Dumont 2019, S.121)\\
            
        Die Bereitstellung von aggregationsfähigen Daten und Metadaten schafft die Grundlage für die Nachnutzung von \href{http://gams.uni-graz.at/o:konde.59}{Digitalen Editionen} (z. B. für die Korpuserstellung) . Datenaggregation ist auch oft ein Anliegen von nationalen Forschungsnetzwerken. \\
            
        \subsection*{Literatur:}\begin{itemize}\item Dumont, Stefan: Interfaces in Digital Scholarly Editions of Letters. In: Digital Scholarly Editions as Interfaces. Norderstedt: 2018, S. 109–131.\item Burdick, Anne; Drucker, Johanna; Lunefeld, Peter; Presner, Todd; Schnapp, Jeffrey: Digital_humanities. Cambridge, London: 2012.\item Burr, Elisabeth: DHD 2016. Modellierung, Vernetzung, Visualisierung. Konferenzabstracts, URL: \url{http://dhd2016.de/}.\end{itemize}\subsection*{Software:}\href{https://iiif.io/}{iiif}, \href{http://projectmirador.org/}{Mirador}, \href{https://www.w3.org/RDF/}{RDF}\subsection*{Verweise:}\href{https://gams.uni-graz.at/o:konde.10}{Metadata-Harvesting}, \href{https://gams.uni-graz.at/o:konde.42}{CMIF}, \href{https://gams.uni-graz.at/o:konde.7}{FAIR-Prinzipien}, \href{https://gams.uni-graz.at/o:konde.129}{METS}, \href{https://gams.uni-graz.at/o:konde.225}{Metadaten}\subsection*{Projekte:}\href{https://www.europeana.eu/portal/de}{Europeana}, \href{https://correspsearch.net}{correspSearch}\subsection*{Themen:}Datenanalyse\subsection*{Zitiervorschlag:}Klug, Helmut W. 2021. Datenaggregation. In: KONDE Weißbuch. Hrsg. v. Helmut W. Klug unter Mitarbeit von Selina Galka und Elisabeth Steiner im HRSM Projekt "Kompetenznetzwerk Digitale Edition". URL: https://gams.uni-graz.at/o:konde.49\newpage\section*{Datenmodell “Hyperdiplomatische Transkription”} \emph{Klug, Helmut W.; helmut.klug@uni-graz.at / Böhm, Astrid;
                  astrid.boehm@uni-graz.at }\\
        
    Eine hyperdiplomatische \href{http://gams.uni-graz.at/o:konde.197}{Transkription} versucht, die historische Quelle möglichst detailreich bis
                  hin zur Teilzeichenebene (z. B. Superskripte) bzw. unter Berücksichtigung der
                  Quellentopographie (Verortung der Informationseinheiten in einem digitalen Abbild
                  der Quelle) in ein modernes Zeichensystem zu übertragen. In Grazer Projekten (\emph{Mittelalterlabor}, \emph{Cooking recipes of the
                     Middle Ages}) wird nach einer hyperdiplomatischen Transkriptionsmethode
                  gearbeitet, die auf die “Grazer dynamische Editionsmethode”
                     (Hofmeister-Winter 2003) zurückgeht: Das Ziel ist eine
                  graphematisch möglichst fein differenzierte Transkription, eine sogenannte
                  “deskriptive Transkription” (Feigs 1979, Teil 1) bzw.
                  “Basistransliteration” (Hofmeister-Winter 2003, S. 101), die sowohl
                  sprach- und literaturwissenschaftlichen als auch geschichtswissenschaftlichen
                  Ansprüchen genügt (Böhm/Klug 2020). Das im Anschluss vorgestellte
                  Modellierungsmodell bildet diese Absicht ab.\\
            
        Jegliches Zeicheninventar, das nicht mit Elementen der ASCII-Code-Chart
                  dargestellt werden kann, wird mithilfe der \href{http://gams.uni-graz.at/o:konde.178}{TEI} innerhalb der <encodingDesc> im Detail beschrieben und über das <g>-Element in den Fließtext des TEI-XML
                  eingebunden. Mittels des TEI-XML-Textmodells wird die historische Quelle
                  deskriptiv dargestellt: Seiten-, Spalten- und Zeilenlayout werden als
                  makrostrukturelle Elemente der Quelle erhalten. Als Elemente der Mikrostruktur
                  werden Überschriften, Initialen, Rubrizierungen, Strichelungen, Unterstreichungen
                  etc. mit den entsprechenden TEI-Elementen und Attributwerten ausgezeichnet.
                  Marginalien werden, wenn sie Teil des Haupttextes sind, ihrem Charakter nach
                  annotiert (z. B. als Einfügung) und an der passenden Stelle in den Text integriert
                  oder mit dem Element <note> kommentiert.
                  Kustoden werden mit dem Element <note>
                  modelliert und mit einem beschreibenden Kommentar versehen. Auf Wort- und
                  Graphebene werden Revisionen, Abbreviaturen sowie editorische Rekonstruktionen mit
                  den entsprechenden TEI-Elementen modelliert. Der Quellentext wird
                  teilzeichengetreu wiedergegeben. Obwohl man eine Differenzierung der Modellierung
                  sehr tiefgreifend anlegen könnte, sollte bei allen Projekten auf die
                  Zweckmäßigkeit der Annotierung geachtet werden: Wenn gewisse Details (z. B. die
                  konkrete Ausführung eines Superskripts tilde-, strich-, punktförmig usw.) nur
                  eingeschränkt in die Transkription übernommen werden, sollten das Datenmodell
                  sowie die Annotation zumindest das Auffinden und eine spätere Feindifferenzierung
                  dieser Daten ermöglichen.\\
            
        Im Rahmen der <charDecl> des TEI-Dokuments
                  wird ein für die Quelle typisches Zeichensonderinventar angelegt: Die Zeichen in
                  der \emph{Character Declaration} werden auf Basis eines de
                  facto-Community-Standards – der \emph{Medieval Unicode Font
                     Initiative} (MUFI) – beschrieben. Die Sonderinventarliste gliedert sich
                  dabei in alphabetische Zeichen, Diakritika, Kürzungszeichen und
                  Interpunktionszeichen und folgt bei der Hierarchisierung der Einträge
                  weitestgehend dem Beschreibungsmodell von Handschrift, wie es im Rahmen des
                  DigiPal-Projekts entwickelt wurde. (Stokes 2011)\\
            
        In der praktischen Umsetzung mit TEI-XML kann das Element <charDecl> als Kindelemente <char> haben; mithilfe des Elements <glyph> wird der Allograph näher beschrieben: So wird als \emph{Character} das „LATIN SMALL LETTER I“ als <i> geführt,
                  als Allograph das „LATIN SMALL LETTER DOTLESS I“ <ı>. Die unserem Modell
                  zugrunde liegende Hierarchie kann aufgrund der Einschränkungen der TEI (<char>- und <glyph>-Elemente können nur auf gleicher Ebene und nicht
                  verschachtelt geschrieben werden) nur linear und nicht hierarchisch realisiert
                  werden. Eine Zeichengruppe wird in der \emph{Character
                     Declaration} folgendermaßen beschrieben:\\
            
        \begin{verbatim}<char xml:id="i">
    <charName>LATIN SMALL LETTER I</charName>
    <mapping/>
</char>
<glyph corresp="#i" xml:id="inodot" ana="allograph"
resp="https://bora.uib.no/handle/1956/10699" source="p.48">
    <glyphName>LATIN SMALL LETTER DOTLESS I</glyphName>
    <mapping type="normalized">i</mapping>
    <mapping type="transcription">i2</mapping>
    <mapping type="unicode_codepoint" subtype="LatExtA">0131</mapping>
    <mapping type="encoding" subtype="html_entity">&#305;</mapping>
    <mapping type="encoding" subtype="unicode_symbol">ı</mapping>
</glyph>\end{verbatim}In den Einträgen zu den einzelnen individuellen Allographenbeschreibungen werden
                  mithilfe von XML-Attributen und den entsprechenden Attributwerten folgende
                  Informationen abgebildet: Der Wert des @corresp-Attributs
                  weist auf eine mögliche hierarchische Verknüpfung hin. Jeder Eintrag erhält einen
                  eindeutigen Identifikator, der auf den \emph{Entity Name} der MUFI
                     \emph{Character Recommendation} zurückgeht. Jeder Eintrag wird
                  systematisch beschrieben als Allograph, Abbreviatur, Superskript oder Satzzeichen.
                  Abbreviaturen wiederum werden nochmals in allgemeines Kürzungszeichen, Brevigraph
                  oder Kontraktion differenziert. Die Benennung und Beschreibung der Zeichen beruht
                  in der Regel auf der MUFI, deshalb wird mit entsprechenden Attributen
                  (@resp, @source) auch auf diese
                  Quelle mit Seitenreferenz verwiesen. Ist ein Zeichen oder eine Zeichenkombination
                  dort nicht vorhanden, wird dessen Name analog zu den Unicode-Namen aufgebaut. Das
                  Element <mapping> gibt die Darstellung eines
                  Zeichens in unterschiedlichen Kontexten an: Aufgenommen sind die jeweiligen Werte
                  des \emph{Unicode Codepoint} und für die Darstellung die HTML-\emph{Entity} und das Unicodesymbol. Zusätzlich sind für die
                  automatisierte Verarbeitung noch das proprietäre Markup, das im
                  Transkriptionsworkflow verwendet wird, und die normalisierte Zeichendarstellung
                  hinterlegt. Zeichen und Zeichenkombinationen, die im ASCII-Zeichensatz vorhanden
                  sind, werden nicht im Detail beschrieben.\\
            
        Der damit teilweise bis auf Teilzeichenebene modellierte und annotierte Text steht
                  dann zur Weiterbe- und -verarbeitung bereit: Die entsprechenden Informationen zu
                  den Zeichen werden mit dem <g>-Element und
                  dem Attribut @ref, das als Wert die jeweilige
                  @xml:id des Eintrags in der \emph{Character
                     Declaration} hat, im Text modelliert. \\
            
        
                           w<g ref="#inodot">i</g>ldw
                        wildwFür die Erstellung von hyperdiplomatischen Transkriptionen nach diesem
                  Transkriptionsmodell gibt es einen fertig konzipierten und erprobten Workflow, der
                  auf mithilfe von \emph{Transkribus} und proprietären \href{http://gams.uni-graz.at/o:konde.126}{Markup} erstellten Transkriptionen
                  aufbaut und durch \href{http://gams.uni-graz.at/o:konde.86}{XSL}-Transformationen TEI-XML-Dateien erstellt. Für eine Publikation
                  derartiger Transkriptionen in \href{http://gams.uni-graz.at/o:konde.70}{GAMS}
                  gibt es Templates, mithilfe derer ein Archivprojekt umgesetzt werden kann.\\
            
        \subsection*{Literatur:}\begin{itemize}\item Die (hyper-)diplomatische Transkription und ihre
                              Erkenntnispotentiale. URL: \url{https://www.hsozkult.de/event/id/termine-42210}\item Böhm, Astrid; Klug, Helmut W: Quellenorientierte Aufbereitung historischer Texte im
                              Rahmen digitaler Editionen: Das Problem der Transkription in
                              mediävistischen Editionsprojekten. In: Digitale Methoden und Objekte in Forschung und Vermittlung der mediävistischen Disziplinen. Akten der Tagung Bamberg, 08.-10. November 2018t: 2020, S. 51–72.\item Hofmeister-Winter, Andrea: Das Konzept einer „Dynamischen Edition" dargestellt an
                              der Erstausgabe des „Brixner Dommesnerbuches" von Veit Feichter (Mitte
                              16. Jh.). Göppingen: 2003.\item Feigs, Wolfgang: Deskriptive Edition auf Allograph-, Wort- und
                              Satzniveau, demonstriert an handschriftlich überlieferten,
                              deutschsprachigen Briefen von H. Steffens. Bern (u.a.): 1979.\item CfI: Die (hyper-)diplomatische Transkription und ihre
                              Erkenntnispotentiale. URL: \url{https://dhd-blog.org/?p=12369}\end{itemize}\subsection*{Software:}\href{http://transcribo.org/en/}{Transcribo}, \href{https://transkribus.eu/Transkribus/}{Transkribus}\subsection*{Verweise:}\href{https://gams.uni-graz.at/o:konde.197}{Transkription}, \href{https://gams.uni-graz.at/o:konde.66}{Diplomatische Transkription}, \href{https://gams.uni-graz.at/o:konde.198}{Transkriptionsrichtlinien}, \href{https://gams.uni-graz.at/o:konde.199}{Transkriptionswerkzeuge}, \href{https://gams.uni-graz.at/o:konde.65}{Diplomatische Edition}\subsection*{Projekte:}\href{https://gams.uni-graz.at/corema}{CoReMA -
                           Cooking Recipes of the Middle Ages}, \href{http://gams.uni-graz.at/context:malab}{Mittelalterlabor}, \href{https://folk.uib.no/hnooh/mufi/}{Medieval
                           Unicode Font Initiative}, \href{http://www.digipal.eu}{http://www.digipal.eu}\subsection*{Themen:}Digitale Editionswissenschaft, Annotation und Modellierung\subsection*{Zitiervorschlag:}Klug, Helmut W.; Böhm, Astrid. 2021. Datenmodell “Hyperdiplomatische Transkription”. In: KONDE Weißbuch. Hrsg. v. Helmut W. Klug unter Mitarbeit von Selina Galka und Elisabeth Steiner im HRSM Projekt "Kompetenznetzwerk Digitale Edition". URL: https://gams.uni-graz.at/o:konde.50\newpage\section*{Datenmodell “eventSearch”} \emph{Fritze, Christiane; christiane.fritze@onb.ac.at / Klug, Helmut W.;
                     helmut.klug@uni-graz.at / Kurz, Stephan; stephan.kurz@oeaw.ac.at; Steindl,
                     Christoph; christoph.steindl@onb.ac.at }\\
        
    Ein Event definiert sich – möglichst allgemein gehalten – als \textbf{was}  geschah \textbf{wann}  und mit \textbf{wem/was} . In nahezu allen historischen, aber auch literarischen Quellen
                     sind Events überliefert. Ein einheitliches Datenmodell für die \href{http://gams.uni-graz.at/o:konde.17}{Annotation} von Events würde den
                     Aufbau einer Eventdatenbank/-Schnittstelle vorantreiben, mithilfe derer
                     Ereignisse, die in edierten Quellen überliefert und annotiert sind,
                     zusammengeführt werden können. Ein prototypisches Anwendungsbeispiel ist correspSearch.\\
            
        Da sich die \href{http://gams.uni-graz.at/o:konde.178}{TEI} als de
                     facto-Standard für die \href{http://gams.uni-graz.at/o:konde.137}{Modellierung} von historischen Texten herauskristallisiert hat, und die
                     TEI-Community die Annotation von Events immer wieder diskutiert, wurde für
                     Eventdaten das nachstehende Modell entwickelt, das sich grundlegend an den
                     Vorgaben der TEI für Namen, Daten, Personen und Orte (TEI Guidelines, Ch.
                        13) orientiert, dabei aber auch neue Wege beschreitet. Das \emph{Codesnippet} zeigt einen möglichst vollständig annotierten
                     Eintrag <event> im Abschnitt <listEvent> des
                     <teiHeader>; die in Blockbuchstaben geschriebenen Attribute
                     sind im TEI-\href{http://gams.uni-graz.at/o:konde.166}{Schema} noch nicht
                     vorgesehen und müssen zur Zeit durch die Anpassung der Schemata ermöglicht
                     werden:\\
            
        \begin{verbatim}<event
                        att.datable.*="1950-05-05"
                        where="placeName"
                        ATT.DURATION.*="duration.w3c [e.g. PT45M]"
                        xml:id="some_ID"
                        source="URI"
                        type="Type of the event"
                        <head>[short title]</head>
                        <label type="generated">[What happened.]</label>
                        <desc><!-- Further information as you like/need and as
                        is TEI compliant. May contain further linking between 
                        entities involved in the event. --></desc>
                        </event>\end{verbatim}Für die kürzeste Variante ein Event zu beschreiben reicht dieser Code:\\
            
        \begin{verbatim}<event when-iso="1950-05-05">
                        <label>[What happened.]</label>
                        </event>\end{verbatim}Das für eine einheitliche Beschreibung von Events konzipierte Datenmodell baut
                     also weitestgehend auf den Vorgaben der TEI auf (ein Beispiel einer Edition, die <event>s nach \href{http://gams.uni-graz.at/o:konde.178}{TEI}P5 3.6.0 für die Verlistung von Sitzungsdaten und Tagesordnungen verwendet: Edition der Ministerratsprotokolle). Eine Erweiterung der TEI wird
                     aber angestrebt und in einem Kooperationsprojekt von ÖNB, ÖAW und ZIM
                     vorangetrieben:\\
            
        \begin{itemize}\item {Ergänzung des <event>-Elements um das Attribut @dur/@dur-iso}\item {Einführung des Elements <eventName>}\item {Vereinheitlichung des Datenmodells für <event>,
                        <person>, <org>,
                        <place>}\end{itemize}\subsection*{Literatur:}\begin{itemize}\item Dumont, Stefan: correspSearch – Connecting Scholarly Editions of
                                 Letters. In: Journal of the Text Encoding Initiative: 2016.\item 13 Names, Dates, People, and Places. URL: \url{https://tei-c.org/release/doc/tei-p5-doc/en/html/ND.html}\item Recreating history through events. URL: \url{https://zenodo.org/record/3447298#.XrPOkBMzbUI}\item DHd 2020. Book of Abstracts: Spielräume. Digital
                                 Humanities zwischen Modellierung und Interpretation. Hrsg. von  und Christof Schöch, URL: \url{https://doi.org/10.5281/zenodo.3666690}.\end{itemize}\subsection*{Verweise:}\href{https://gams.uni-graz.at/o:konde.17}{Annotation}, \href{https://gams.uni-graz.at/o:konde.178}{TEI}, \href{https://gams.uni-graz.at/o:konde.137}{Modellierung}, \href{https://gams.uni-graz.at/o:konde.133}{CIDOC CRM}, \href{https://gams.uni-graz.at/o:konde.42}{CMIF}\subsection*{Projekte:}\href{https://correspsearch.net}{correspSearch}, \href{https://labs.onb.ac.at/gitlab/digital-editions/eventSearch}{ONB
                              Gitlab: Eventsearch}\subsection*{Themen:}Annotation und Modellierung\subsection*{Zitiervorschlag:}Fritze, Christiane; Klug, Helmut W.; Kurz, Stephan. 2021. Datenmodell “eventSearch”. In: KONDE Weißbuch. Hrsg. v. Helmut W. Klug unter Mitarbeit von Selina Galka und Elisabeth Steiner im HRSM Projekt "Kompetenznetzwerk Digitale Edition". URL: https://gams.uni-graz.at/o:konde.53\newpage\section*{Datenmodell: Kalender} \emph{Raunig, Elisabeth; elisabeth.raunig@uni-graz.at}\\
        
    In Kalendertafeln wird eine überaus große Menge an Informationen meist in
                  tabellarischer Form vermittelt, die Reihenfolge der unterschiedlichen Inhalte kann
                  dabei variieren. Informationen in Kalendern können daher Spalte für Spalte von
                  oben nach unten gelesen werden oder Zeile für Zeile von links nach rechts: Die
                  Zeitrechnung im Mittelalter richtet sich nach dem Sonnenjahr mit 365 Tagen und
                  allen 4 Jahren einem Schaltjahr. Das Sonnenjahr wird für die christliche
                  Festrechnung mit dem Mondumlauf verbunden, daraus konnte der Festkalender
                  berechnet werden, weil im julianischen Kalender alle 19 Jahre die Mondphasen
                  wieder an denselben Tagen auftreten. Die Nummerierung der Jahre von 1 bis 19 nach
                  dem Mondzyklus nennt sich Goldene Zahl. 28 Jahre dauert wiederum der Sonnenzyklus,
                  alle 28 Jahre sollen daher die Wochentage mit den Monatsdaten übereinstimmen.
                  Deshalb wurden die Wochentage mit den Buchstaben A-G bezeichnet, diese werden
                  Tagesbuchstaben genannt oder auch als Ferialzählung bezeichnet. Der Buchstabe, der
                  auf den ersten Sonntag im Jahr fällt, wird Sonntagsbuchstabe genannt. Eine weitere
                  Buchstabengruppe sind die Lunarbuchstaben, diese stehen für die Mondphasen und
                  umfassen die Buchstaben A-U/V bzw. A-T. Nachdem Anfang des 6. Jahrhunderts der
                  Frühlingsanfang mit dem 21. März festgelegt wurde, wurde das Osterfest mit dem
                  ersten Sonntag nach dem ersten Vollmond nach dem 21. März festgelegt, woran alle
                  anderen Festtage ausgerichtet wurden und der Festkalender berechnet wurde. Diese
                  Berechnungsmethoden scheinen somit häufig in den Kalendern auf, wir finden die
                  Goldene Zahl, Ferialzählung (inkludiert Sonntagsbuchstabe), Lunarbuchstaben und
                  die Festtage. Hinzu kommen noch die Heiligentage, die von Diözese zu Diözese
                  variieren können und die römische Tageszählung mit Kalenden, Iden und Nonen. Es
                  können jedoch noch diverse andere Spalten hinzukommen und die Heiligentage können
                  in unterschiedlicher Form aufscheinen, zum Beispiel als Buchstabe in Cisiojanus
                  Form oder ausgeschrieben in einer Textspalte. Kalender können aber zusätzlich auch
                  noch diözesenspezifische Notizen in der Tabellenstruktur aufweisen. Dieser
                  umfangreiche Informationsgehalt sowie die Bedeutung jeder einzelnen Zelle müssen
                  in einer Modellierung berücksichtigt werden.\\
            
        Eine mögliche Modellierung könnte mit der TEI daher so aussehen:\\
            
        Gehen wir davon aus, dass der Jahreskalender einer Quelle im <body>-Element modelliert wird, dann kann ein
                  Monat durch ein <div> repräsentiert werden
                  und auch jeder Tag wird als <div> modelliert.
                  Und wir gehen weiter davon aus, einen Kalender vor uns zu haben, der die
                  Heiligentage als Cisiojanus verzeichnet und zusätzlichen Text beinhaltet.\\
            
        \begin{verbatim}<div type="day" ana="grote_celeb" corresp="#Feast_jesu_dominicircumcisio" n="--01-01">
    <div type="Intervallzone">
        <ab type="Lunarzahl"/>
        <ab type="römisch"/>
        <ab type="Goldenezahl"/>
        <ab type="Ferialzählung">A</ab> <ab
        type="Cisiojanus">Ci.</ab>
    </div> 
    <div type="Kalendereintrag">
        <ab>Text</ab> 
    </div> 
</div>\end{verbatim}Dieses Beispiel zeigt also den 1. Jänner, angegeben als @n-Attribut mit dem Wert “--01-01”mit der Ferialzählung A und “Ci”, dem
                  Beginn des Cisiojanus, das für das Fest \emph{Domini circumcisio}
                  steht. \\
            
        Wäre kein Cisiojanus vorhanden, könnte auch die Spalte mit dem Heiligentag in das
                     <div> des @type “Kalendereintrag” überführt werden, oder als eigenes <div> realisiert werden:\\
            
        \begin{verbatim}<div type="day" ana="grote_celeb" corresp="#Feast_jesu_dominicircumcisio" n="--01-01">
    <div type="Intervallzone">
        <ab type="Lunarzahl"/>
        <ab type="römisch"/>
        <ab type="Goldenezahl"/>
        <ab type="Ferialzählung">A</ab>
        <ab type="Heiliger">Ci.</ab>
    </div>
</div>\end{verbatim}Die Feste und Heiligen aus einer derartigen Kalendertranskription könnten
                  zusätzlich mit Heiligen- oder Festdatenbanken verknüpft werden, wie im
                  vorliegenden Beispiel im Tages-<div> mit den
                  Attributen @ana und @corresp realisiert. \\
            
        Das vorliegende Beispiel könnte weiter in \href{http://gams.uni-graz.at/o:konde.131}{RDF} umgewandelt und mit dem RDF Modell von \emph{Grotefend digital} verknüpft werden: Der Wert im Attribut
                     @ana ‘grote-celeb’ bedeutet, dass ein Tag in diesem Kalender einer
                  <http://gams.uni-graz.at/o:grotefend.ontology/Celebration> entspricht. Im
                  Attribut @corresp ist die ID des Heiligenfestes \emph{Domini
                     circumcisio}, das in der Grotefend Ontologie folgende URI hat:
                        <http://gams.uni-graz.at/o:grotefend.ontology/Feast\_jesu\_dominicircumcisio>.
                  In Verbindung mit dem Herkunftsort des Kalenders (z. B. im Header des
                  TEI-Dokuments modelliert) und des Datums des Festes wird das zu einer
                  <Celebration>: Wenn der Herkungftsort zum Beispiel Tegernsee wäre, dann
                  lautete die URI: <http://gams.uni-graz.at/o:grotefend.ontology/Feast\_jesu\_dominicircumcisio\_Tegernsee\_--01-01>.\\
            
        Die daraus entstandenen Ressourcen können so mit den bereits bestehenden Instanzen
                  von \emph{Grotefend digital} verknüpft werden, sind eindeutig
                  identifizierbar und darüberhinaus in das \href{http://gams.uni-graz.at/o:konde.8}{Linked Open Data} Web eingebunden.\\
            
        \subsection*{Literatur:}\begin{itemize}\item Böhm, Astrid: Das iatromathematische Hausbuch des Codex ÖNB, 3085
                              (fol. 1r39v): stoffgeschichtliche Einordnung, dynamisch-mehrstufige
                              Edition und Glossar. Graz: 2014.\item Borst, Arno: Computus. München: 1999.\item Roos - Heikkila, Teemu - Tuomas: Evaluating methods for computer-assisted stemmatology
                              using artificial benchmark data sets. In: LLC 24,4: 2009, S. 417–433.\item Hoffmann, Andreas: Die Anfänge des Heiligenkalenders. In: Der Kalender: 2001.\item Brincken, Anna-Dorothee von den: Historische Chronologie des Abendlandes. Stuttgart: 2000.\end{itemize}\subsection*{Verweise:}\href{https://gams.uni-graz.at/o:konde.131}{RDF}, \href{https://gams.uni-graz.at/o:konde.137}{Modellierung}, \href{https://gams.uni-graz.at/o:konde.178}{TEI}, \href{https://gams.uni-graz.at/o:konde.8}{Linked Open Data}\subsection*{Projekte:}\href{https://gams.uni-graz.at/grotefend}{Grotefend digital}\subsection*{Themen:}Annotation und Modellierung\subsection*{Zitiervorschlag:}Raunig, Elisabeth. 2021. Datenmodell: Kalender. In: KONDE Weißbuch. Hrsg. v. Helmut W. Klug unter Mitarbeit von Selina Galka und Elisabeth Steiner im HRSM Projekt "Kompetenznetzwerk Digitale Edition". URL: https://gams.uni-graz.at/o:konde.51\newpage\section*{Datenvisualisierung} \emph{Galka, Selina; selina.galka@uni-graz.at }\\
        
    Datenvisualisierung eignet sich einerseits, um komplexere Sachverhalte darzustellen, die visuell verständlicher werden, andererseits aber auch, um eine große Menge an Daten besser analysieren zu können. (Rehbein 2017, S. 228)\\
            
        Bei allen Formen von Informationsvisualisierung sollte darauf geachtet werden, die Daten transparent und unverfälscht zu zeigen und Visualisierungen außerdem nur dann einzusetzen, wenn es auch wirklich sinnvoll erscheint; also wenn eine reichhaltige, mehrdimensionale oder komplexe Datengrundlage vorliegt. In der Forschung dienen Datenvisualisierungen z. B. der Präsentation, aber auch der konfirmativen oder explorativen Analyse. Die Visualisierung als Präsentation hat zum Ziel, Forschungsergebnisse anschaulich darzustellen. (Rehbein 2017, S. 331) Visualisierungen ermöglichen aber auch, Erkenntnisse überhaupt erst zu gewinnen: Bei der explorativen Analyse wird beispielsweise versucht, Strukturen, Muster oder Auffälligkeiten in den Daten zu erkennen; so ist dies oft im \emph{\href{http://gams.uni-graz.at/o:konde.48}{Data Mining}} der Fall. Darüber hinaus können Visualisierungen auch künstlerisch eingesetzt werden, z. B. als Mittel zum Storytelling, oder sie können mit Simulationen verbunden werden. (Rehbein 2017, S. 331f.)\\
            
        Je nach vorliegenden Daten bieten sich unterschiedliche visuelle Strukturen und Repräsentationsmöglichkeiten an: für Daten mit räumlichem Bezug beispielsweise Datenkarten, für Daten mit temporalem Bezug u. a. Zeitreihen. Außerdem gibt es mehrere Arten von Diagrammen, wie Linien-, Punkt- oder Streudiagramme; für die Darstellung von Beziehungen eignen sich Graphe oder Baumstrukturen. Visualisierungen können darüber hinaus statisch oder interaktiv gestaltet werden. (Rehbein 2017, S. 334f.)\\
            
        \subsection*{Literatur:}\begin{itemize}\item Sinclair, Stéfan; Ruecker, Stan; Radzikowska, Milena: Information Visualization for Humanities Scholars. In: Literary Studies in the Digital Age: 2013.\item Rehbein, Malte: Informationsvisualisierung. In: Digital Humanities. Eine Einführung. Stuttgart: 2017, S. 328–342.\item Tufte, Edward R: The Visual Display of Quantitative Information. Cheshire: 2001.\item Drucker, Johanna: Graphesis: Visual Forms of Knowledge Production Graphesis. Cambridge, Massachusetts: 2014.\item Drucker, Johanna: Visualization and Interpretation: Humanistic Approaches to Display Visualization and Interpretation. Cambridge, Massachusetts: 2020.\end{itemize}\subsection*{Verweise:}\href{https://gams.uni-graz.at/o:konde.210}{Visualisierungstools}, \href{https://gams.uni-graz.at/o:konde.48}{Data Mining}, \href{https://gams.uni-graz.at/o:konde.74}{Dramennetzwerkanalyse}\subsection*{Themen:}Datenanalyse\subsection*{Zitiervorschlag:}Galka, Selina. 2021. Datenvisualisierung. In: KONDE Weißbuch. Hrsg. v. Helmut W. Klug unter Mitarbeit von Selina Galka und Elisabeth Steiner im HRSM Projekt "Kompetenznetzwerk Digitale Edition". URL: https://gams.uni-graz.at/o:konde.54\newpage\section*{Denkmäleredition} \emph{Galka, Selina; selina.galka@uni-graz.at }\\
        
    Im Rahmen einer Denkmäleredition oder -ausgabe werden Texte oder Werke
                  zusammengefasst, denen historische Bedeutsamkeit zugeschrieben wird. Man
                  orientiert sich dabei also an der Qualität und dem Einfluss der Texte oder Werke
                  und bestimmt einen thematisch geschlossenen Bereich. \\
            
        Die Denkmäleredition spielt vor allem im Bereich der \href{http://gams.uni-graz.at/o:konde.139}{Musikedition} eine bedeutende Rolle.
                  Denkmälerausgaben folgen denselben Standards und Ansprüchen wie beispielsweise
                     \href{http://gams.uni-graz.at/o:konde.91}{Gesamtausgaben} oder \href{http://gams.uni-graz.at/o:konde.93}{Historisch-kritische Ausgaben}; das
                  bedeutet, die Quellen werden verzeichnet und diskutiert, es werden
                  Editionsrichtlinien vorgegeben und ein kritischer \href{http://gams.uni-graz.at/o:konde.32}{Apparat} eingerichtet. In der Regel gibt es auch
                  einen einführenden Text zum historischen Umfeld. (Horn 2015, S. 704)\\
            
        Die ersten Denkmälerausgaben finden sich um 1800 im Bereich der Musik. Im 20.
                  Jahrhundert ist die ‘nationale’ Denkmälerausgabe eine wichtige Ausprägung.
                  (Horn 2015, S. 710) Die nationale Orientierung dieser Ausgaben muss
                  nicht grundsätzlich auf nationalistische Motive zurückführen, sondern ist auch der
                  damals noch geringeren Mobilität und medialen Austauschmöglichkeiten geschuldet.
                     (Horn 2015, S. 711)\\
            
        \subsection*{Literatur:}\begin{itemize}\item Horn, Wolfgang: Denkmälerausgaben. In: Musikeditionen im Wandel der Geschichte. Berlin, Boston: 2015, S. 204–740.\item Denkmälerausgaben. URL: \url{https://www.musiklexikon.ac.at/ml/musik_D/Denkmaelerausgabe.xml}\end{itemize}\subsection*{Verweise:}\href{https://gams.uni-graz.at/o:konde.139}{Musikedition}, \href{https://gams.uni-graz.at/o:konde.91}{Gesamtausgabe}, \href{https://gams.uni-graz.at/o:konde.93}{Historisch-kritische Ausgabe}, \href{https://gams.uni-graz.at/o:konde.32}{Apparat}\subsection*{Projekte:}\href{http://www.opera.adwmainz.de/informationen.html}{OPERA - Spektrum
                           des europäischen Musiktheaters in Einzeleditionen}\subsection*{Themen:}Digitale Editionswissenschaft\subsection*{Lexika}\begin{itemize}\item \href{https://edlex.de/index.php?title=Denkm%C3%A4leredition}{Edlex:
                           Editionslexikon}\end{itemize}\subsection*{Zitiervorschlag:}Galka, Selina. 2021. Denkmäleredition. In: KONDE Weißbuch. Hrsg. v. Helmut W. Klug unter Mitarbeit von Selina Galka und Elisabeth Steiner im HRSM Projekt "Kompetenznetzwerk Digitale Edition". URL: https://gams.uni-graz.at/o:konde.55\newpage\section*{Design Digitaler Editionen} \emph{Klug, Helmut W.; helmut.klug@uni-graz.at }\\
        
    Digitale Editionen werden mittlerweile weitestgehend im Internet publiziert,
                  sodass der Begriff ‘Design’ in diesem Zusammenhang oft nur mit Webdesign und
                  Interfacedesign in Verbindung gebracht wird. Im weitesten Sinne müssen aber auch
                  das grundlegende Projektdesign, die wissenschaftliche Ausrichtung der Edition
                     (\href{http://gams.uni-graz.at/o:konde.76}{Editionstyp}), die \href{http://gams.uni-graz.at/o:konde.137}{Datenmodellierung}, die Planung von
                  Disseminationsstrategien (\href{http://gams.uni-graz.at/o:konde.138}{Hybrid},
                     \emph{\href{http://gams.uni-graz.at/o:konde.152}{Open
                     Access}} etc.) im und neben dem Internet und die Lizenzierung der
                  Inhalte Teil eines Designprozesses Digitaler Editionen sein. \\
            
        \begin{itemize}\item {Projektdesign: Digitale Editionen werden in der Regel als Kooperationen
                     zwischen ‘Fachwissenschaft’ und ‘Technik’ im Rahmen eines finanzierten
                     Forschungsprojekts durchgeführt. Die Gründung von DH-Zentren und
                     Ausbildungsangebote im DH-Bereich tragen mittlerweile bedingt dazu bei, dass
                     diese Trennung von Kompetenzen abgebaut wird. Nichtsdestotrotz müssen für die
                     Erstellung einer Digitalen Edition unterschiedliche Ziele, Kompetenzen,
                     Forschungsinteressen, Terminkalender usw. berücksichtigt werden. Projektdesign
                     kann daher weitestgehend mit Projektplanung und -management gleichgesetzt
                     werden und umfasst neben der organisatorischen Zeit- und Finanzplanung
                     inhaltlich zumindest die nachstehenden Themen:}\item { Editionstyp: Der avisierte \href{http://gams.uni-graz.at/o:konde.76}{Editionstyp} und die intendierten Benutzerinnen- und \href{http://gams.uni-graz.at/o:konde.148}{Benutzergruppe} bilden die
                     Entscheidungsgrundlagen für die Erstellung (z. B. \href{http://gams.uni-graz.at/o:konde.60}{Digitalisierung}) und Aufbereitung (z. B. \href{http://gams.uni-graz.at/o:konde.137}{Modellierung}, \href{http://gams.uni-graz.at/o:konde.16}{Analysemethoden}) der
                     Daten.}\item { Datenmodellierung: Gibt es neben dem Ziel, eine Edition zu erstellen,
                     zusätzliche Forschungsfragen, müssen diese bei der \href{http://gams.uni-graz.at/o:konde.137}{Modellierung} und \href{http://gams.uni-graz.at/o:konde.17}{Annotation} der Daten zusätzlich zur abstrahierten
                     Darstellung der Quelle berücksichtigt werden.}\item { Disseminationsstrategie: Digitale Editionen wurden auf digitalen
                     Datenträgern und werden nach wie vor im Internet publiziert; bei speziellen
                     Editionstypen wie der \href{http://gams.uni-graz.at/o:konde.96}{Hybridedition} kann es auch Printderivate oder andere gedruckte
                     Begleitmaterialien, die aus den Editionsdaten erstellt werden, geben. Es muss
                     also entschieden werden, welche Daten wie und unter welchen Bedingungen
                     veröffentlicht werden: Die Daten können von \href{http://gams.uni-graz.at/o:konde.36}{Quellendigitalisaten} über \href{http://gams.uni-graz.at/o:konde.25}{Metadaten} und \href{http://gams.uni-graz.at/o:konde.87}{Forschungsdaten} bis hin zum Editionstext,
                     Apparat, Kommentar etc. alle im Rahmen des Editionsprojekts erstellten und
                     gesammelten Daten umfassen. Die digitalen Daten können von ihren Urhebern mit jeweils eigenen \href{http://gams.uni-graz.at/o:konde.118}{Lizenzen} veröffentlicht werden.
                     Bei Printerzeugnissen bestimmt in der Regel der \href{http://gams.uni-graz.at/o:konde.208}{Verlag}, wie die Rechtssituation rund um die
                     Publikation gelöst wird. }\item {Layout und Webdesign: Im Publikationsprozess ist es sowohl bei der
                     Erstellung von gedruckten wie Digitalen Editionen üblich, dass die Gestaltung
                     der Publikation von Editorinnen und Editoren bzw. dem Editionsteam übernommen
                     wird. Für Printeditionen bestimmt in der Regel bereits das Medium und die
                     Tradition einen großen Teil der Gestaltung. Digitale Editionen bieten dagegen –
                     vor allem auch durch die rasante Entwicklung der Webtechnologien –
                     kontinuierlich Raum für Experimente. Nichtsdestotrotz bestimmen auch hier die
                     Publikationsumgebung (z. B. \href{http://gams.uni-graz.at/o:konde.67}{ARCHE}, \href{http://gams.uni-graz.at/o:konde.68}{DHPLUS}, \href{http://gams.uni-graz.at/o:konde.70}{GAMS}) und die Erwartungshaltung
                     der Benutzerinnen und Benutzer (Designelemente, \href{http://gams.uni-graz.at/o:konde.80}{Elemente
                        digitaler Editionen}) die Möglichkeiten in der Umsetzung. }\item { Qualitätssicherung: Diese Aufgabe begleitet ein Editionsprojekt vom Beginn
                     der Projektplanung bis zum Projektabschluss und erhält in den einzelnen Phasen
                     und in Bezug auf die diversen Produkte und Zwischenprodukte immer wieder
                     unterschiedliche Bedeutung. Ein simples Mittel der Qualitätssicherung im
                     Projektmanagement wäre z. B. das Gantt-Chart, in dem alle Projektphasen,
                     Aufgaben, die Aufgabenverteilung, milestones, deliverables usw. im Verhältnis
                     zur Projektdauer definiert sind. Ein jeder Editionstyp verlangt nach bestimmten
                        \href{http://gams.uni-graz.at/o:konde.80}{Elementen}, die als
                     Qualitätsmerkmale vorhanden sein müssen. Datenqualität kann durch entsprechende
                     Richtlinien (wie in der Digitalisierung) oder durch den Einsatz von Schemata
                        (\href{http://gams.uni-graz.at/o:konde.137}{Modellierung} und \href{http://gams.uni-graz.at/o:konde.17}{Annotation}) gewährleistet werden.
                     Publikationen müssen generell den \href{http://gams.uni-graz.at/o:konde.7}{FAIR-Prinzipien} folgen und bestehenden wissenschaftlichen Standards
                     entsprechen, die \emph{\href{http://gams.uni-graz.at/o:konde.206}{Usability}} von Webseiten kann durch ausgiebiges \emph{\href{http://gams.uni-graz.at/o:konde.206}{User-}}
                     und \emph{\href{http://gams.uni-graz.at/o:konde.78}{Editor-Testing}} gewährleistet werden. }\end{itemize}\subsection*{Literatur:}\begin{itemize}\item Ermolaev, Natalia; Munson, Rebecca; Li, Xinyi; Siemens, Ray; Kaufman, Micki; Siemens, Lynne; Boyd, Jason: Project Management For The Digital Humanities
                              (Abstract). URL: \url{https://dh2018.adho.org/en/project-management-for-the-digital-humanities/}.\item Reed, Ashley: Managing an Established Digital Humanities Project:
                              Principles and Practices from the Twentieth Year of the William Blake
                              Archive. In: DHQ 8: 2014.\item Portny, Stanley E.: Projektmanagement für Dummies: 2016.\end{itemize}\subsection*{Literatur:}\begin{itemize}\item Scholger, Walter: Urheberrecht und offene Lizenzen im wissenschaftlichen
                              Publikationsprozess. In: Publikationsberatung an Universitäten Ein
                              Praxisleitfaden zum Aufbau publikationsunterstützender
                              Services. Bielefeld: 2020.\item Martinez, Merisa; Dillen, Wout; Bleeker, Elli; Sichani, Anna-Maria; Kelly, Aodhán: Refining our conceptions of ‘access’ in digital
                              scholarly editing: Reflections on a qualitative survey on inclusive
                              design and dissemination. Refining our conceptions of ‘access’ in digital
                              scholarly editing. In: Variants. The Journal of the European Society for
                              Textual Scholarship: 2019, S. 41–74.\item Klingner, Jens; Lühr, Merve; , : Forschungsdesign 4.0 Datengenerierung und
                              Wissenstransfer in interdisziplinärer Perspektive: 2019, URL: \url{https://nbn-resolving.de/urn:nbn:de:bsz:14-qucosa2-359184}.\item Ralle, Inga Hanna: Maschinenlesbar — menschenlesbar. Über die grundlegende
                              Ausrichtung der Edition. In: editio. Internationales Jahrbuch für
                              Editionswissenschaft 30: 2016, S. 144–156.\end{itemize}\subsection*{Literatur:}\begin{itemize}\item Rosselli Del Turco, Roberto: After the Editing is Done. Designing a Graphic User
                              Interface for Digital Editions. In: Digital Medievalist 7: 2011.\item Sinclair, Stéfan; Ruecker, Stan; Radzikowska, Milena: Information Visualization for Humanities
                              Scholars. In: Literary Studies in the Digital Age: 2013.\item Rockwell, Geoffrey; Ruecker, Stan; Windsor, Jennifer; Ilovan, Mihaela; Sondheim, Daniel: The Face of Interface: Studying Interface to the
                              Scholarly Corpus and Edition The Face of Interface. In: Scholarly and Research Communication 3: 2013.\item Lambertz, Michael: Darf wissenschaftliches Design in DH-Projekten emotional
                              ansprechen? (Abstract). In: DHd 2016. Modellierung, Vernetzung, Visualisierung. Die
                              Digital Humanities als fächerübergreifendes Forschungsparadigma.
                              Konferenzabstracts: 2016, S. 343–345.\item Drucker, Johanna: Performative Materiality and Theoretical Approaches to
                              Interface. In: dhq 7: 2013.\item Kirschenbaum, Matthew G.: Editing the Interface. Textual Studies and First
                              Generation Electronic Objects. In: Text 14: 2002, S. 15–51.\item Interfacing the Edition. URL: \url{http://www2.iath.virginia.edu/bpn2f/1866/interface.html}\item Pierazzo, Elena: Digital scholarly editing: theories, models and
                              methods. Farnham: 2015, URL: \url{http://hal.univ-grenoble-alpes.fr/hal-01182162}.\item Warwick, Claire: Studying users in digital humanities. In: Digital Humanities in Practice. London: 2012, S. 1-21.\end{itemize}\subsection*{Literatur:}\begin{itemize}\item Leitlinien zur Sicherung guter wissenschaftlicher
                              Praxis. Kodex, Gruppe Chancengleichheit, Wissenschaftliche Integrität und
                                 Verfahrensgestaltung: 2019. URL: \url{https://www.dfg.de/download/pdf/foerderung/rechtliche_rahmenbedingungen/gute_wissenschaftliche_praxis/kodex_gwp.pdf}.\item . In: Criteria for Reviewing Scholarly Digital Editions,
                              version 1.1 | Institut für Dokumentologie und Editorik: 2014.\item DFG-Praxisregeln "Digitalisierung", Deutsche Forschungsgemeinschaft: 2016. URL: \url{https://www.dfg.de/formulare/12_151/}.\end{itemize}\subsection*{Verweise:}\href{https://gams.uni-graz.at/o:konde.84}{Farbdesign}, \href{https://gams.uni-graz.at/o:konde.99}{Interface Design Cycle}, \href{https://gams.uni-graz.at/o:konde.164}{Responsive Design}, \href{https://gams.uni-graz.at/o:konde.207}{User-centered Design}\subsection*{Themen:}Einführung, Digitale Editionswissenschaft\subsection*{Zitiervorschlag:}Klug, Helmut W. 2021. Design Digitaler Editionen. In: KONDE Weißbuch. Hrsg. v. Helmut W. Klug unter Mitarbeit von Selina Galka und Elisabeth Steiner im HRSM Projekt "Kompetenznetzwerk Digitale Edition". URL: https://gams.uni-graz.at/o:konde.56\newpage\section*{Design to Test} \emph{Stoff, Sebastian; sebastian.stoff@uni-graz.at }\\
        
    Testbarkeit ist eine wesentliche Voraussetzung für die Wartbarkeit jeglicher Programme. (Rau, 2016)\emph{Design to Test} fordert dabei bereits bei der Erstellung von Software und Datenmaterial nicht nur die Berücksichtigung, sondern vor allem eine harmonische Einbindung der Erfordernisse von seriöser Softwarequalitätssicherung. Es soll im gesamten Entwicklungsprozess guten Entwicklungspraktiken wie KISS (\emph{Keep it simple, stupid!}), YAGNI (\emph{You aren’t gonna need it.}) und DRY (\emph{Don’t repeat yourself.}) gefolgt werden, die neben einer allgemein höheren und besseren Wartbarkeit des Codes auch eine einfachere Testbarkeit erzielen. (Goll, 2018)\\
            
        Daten und Quellcode sollen von Anfang an ‘wartbar’ entworfen und nicht erst in einem abschließenden Prozess auf Fehler kontrolliert werden. Es gilt, Fehler als (im Entwicklungsprozess) wiederkehrende Abweichungen des Soll- vom Ist-Zustand zu verstehen, die nicht bloß durch ein abschließendes Test- und Konsolidierungsverfahren behoben werden können. \emph{Design to Test} ist also im Kontext von Daten- und Softwarelebenszyklusmodellen zu sehen und versteht sich als Entwicklungsphilosophie. Sie steht im Gegensatz zu linearen Vorgehensmodellen, die das Testen als ausschließlich abschließende, einmalige qualitätssichernde Maßnahme ansehen. \\
            
        \subsection*{Literatur:}\begin{itemize}\item Rau, Karl Heinz: Agile objektorientierte Software-Entwicklung: 2016, URL: \url{https://link.springer.com/book/10.1007%2F978-3-658-00776-8}.\item Goll, Joachim: Entwurfsprinzipien und Konstruktionskonzepte der Softwaretechnik: 2018, URL: \url{doi.org/10.1007/978-3-658-20055-8}.\end{itemize}\subsection*{Themen:}Software und Softwareentwicklung\subsection*{Zitiervorschlag:}Stoff, Sebastian. 2021. Design to Test. In: KONDE Weißbuch. Hrsg. v. Helmut W. Klug unter Mitarbeit von Selina Galka und Elisabeth Steiner im HRSM Projekt "Kompetenznetzwerk Digitale Edition". URL: https://gams.uni-graz.at/o:konde.57\newpage\section*{Digitale Edition} \emph{Rieger, Lisa; lrieger@edu.aau.at }\\
        
    Eine Digitale Edition ist eine digitale, wissenschaftlich aufbereitete Ausgabe von literarischen oder historischen Texten, die als Quellen für weitere, vorwiegend geisteswissenschaftliche Forschungen einem möglichst breiten Nutzerkreis zur Verfügung gestellt werden. Während die zentrale Aufgabe einer kritischen Edition früher die Ermittlung des ursprünglichen Autortextes war, werden heute je nach Forschungsinteresse auch andere Zielsetzungen verfolgt, wie z. B. die Analyse des Entstehungsprozesses oder die Darstellung der Varianzbreite. Digitale Editionen unterscheiden sich von gedruckten Editionen u. a. durch das Wegfallen der Umfangsbegrenzung, ihre leichte Veränderbarkeit und die gesteigerten Möglichkeiten der Vernetzung von Informationen. (Jannidis/Kohle/Rehbein 2017, S. 237–240)\\
            
        Die neu entstandenen Möglichkeiten der Datenaufbereitung und -darstellung haben auch zu Kritik an Digitalen Editionen geführt. Klassische Editionswissenschaftlerinnen und Editionswissenschaftler fürchteten, dass Digitale Editionen aufgrund eines zu einseitigen Fokus auf innovativen Technologien und durch die Art der digitalen Präsentation ihre inhaltliche Qualität einbüßen könnten. Dementsprechend ist es gerade bei großen Datenmengen und auf Algorithmen basierenden Analysen wichtig, die methodologischen Ansätze der traditionellen Editionswissenschaften und \href{http://gams.uni-graz.at/o:konde.192}{Textkritik} zu beachten und zu verstehen. (Earhart 2012, S. 24 ff.)\\
            
        In Österreich wird derzeit in den Aufbau einer nachhaltigen Infrastruktur für Digitale Editionen investiert, was u. a. der aktuelle Strategieplan der \href{http://gams.uni-graz.at/o:konde.153}{Österreichischen Nationalbibliothek}(Österreichische Nationalbibliothek o. J.) und die Förderung des Kompetenznetzwerks Digitale Edition (KONDE) erkennen lassen. Durch die Schaffung einer gemeinsamen, auf Modulen basierenden technischen Basis am \href{http://gams.uni-graz.at/o:konde.217}{Zentrum für Informationsmodellierung} der Universität Graz sollen die Standardkomponenten ohne größeren Aufwand von einer Vielzahl von unterschiedlichen Editionen genutzt werden können und so Ressourcen gespart werden.  (Fritze 2019, S. 435–438)\\
            
        \subsection*{Literatur:}\begin{itemize}\item Assmann, Bernhard: Digitale Edition im Internet, oder: Hätte Ranke einen Scanner benutzt? In: Historical Social Research/Historische Sozialforschung 21: 1996, S. 136–139.\item Earhart, Amy E: The Digital Edition and the Digital Humanities. In: Textual Cultures 7 (1): 2012, S. 18–28.\item Jannidis, Fotis; Kohle, Hubertus: Digital Humanities. Eine Einführung. Mit Abbildungen und Grafiken Digital Humanities. Hrsg. von  und Malte Rehbein. Stuttgart: 2017.\item Fritze, Christiane: Wohin mit der digitalen Edition? Ein Beitrag aus der Perspektive der Österreichischen Nationalbibliothek Wohin mit der digitalen Edition?. In: Bibliothek - Forschung und Praxis 43 (3): 2019, S. 432–440.\item Kamzelak, Roland: Hypermedia - Brauchen wir eine neue Editionswissenschaft? Hypermedia In: Computergestützte Text-Edition. Beihefte zu Editio 12: 1999, S. 119–126.\item Digitale Editionen an der Österreichischen Nationalbibliothek. URL: \url{https://edition.onb.ac.at/start/o:ode.home/methods/sdef:TEI/get}\item Sahle, Patrick: Digitale Editionsformen. Zum Umgang mit der Überlieferung unter den Bedingungen des Medienwandels. Teil 1: Das typografische Erbe. Norderstedt: 2013.\item Sahle, Patrick: Digitale Editionsformen. Zum Umgang mit der Überlieferung unter den Bedingungen des Medienwandels. Teil 2: Befunde, Theorie und Methodik. Norderstedt: 2013.\item Sahle, Patrick: Digitale Editionsformen. Zum Umgang mit der Überlieferung unter den Bedingungen des Medienwandels. Teil 3: Textbegriffe und Recodierung. Norderstedt: 2013.\end{itemize}\subsection*{Verweise:}\href{https://gams.uni-graz.at/o:konde.192}{Textkritik}, \href{https://gams.uni-graz.at/o:konde.153}{Österreichische Nationalbibliothek}, \href{https://gams.uni-graz.at/o:konde.217}{ZIM}, \href{https://gams.uni-graz.at/o:konde.80}{Elemente digitaler Editionen}, \href{https://gams.uni-graz.at/o:konde.83}{Faksimileausgabe}, \href{https://gams.uni-graz.at/o:konde.93}{historisch-kritische Edition}, \href{https://gams.uni-graz.at/o:konde.96}{Hybridedition}, \href{https://gams.uni-graz.at/o:konde.63}{Digitalisierungsrichtlinien}, \href{https://gams.uni-graz.at/o:konde.54}{Datenvisualisierung}, \href{https://gams.uni-graz.at/o:konde.48}{Datamining}, \href{https://gams.uni-graz.at/o:konde.102}{Kataloge digitaler Editionen}\subsection*{Themen:}Einführung, Digitale Editionswissenschaft\subsection*{Projekte:}\href{http://www.digitale-edition.at}{KONDE - Kompetenznetzwerk Digitale Edition}\subsection*{Lexika}\begin{itemize}\item \href{https://edlex.de/index.php?title=Digitale_Edition}{Edlex: Editionslexikon}\item \href{https://wiki.helsinki.fi/display/stemmatology/Edition%2C+digital}{Parvum Lexicon Stemmatologicum}\item \href{https://lexiconse.uantwerpen.be/index.php/lexicon/edition-digital/}{Lexicon of Scholarly Editing}\end{itemize}\subsection*{Zitiervorschlag:}Rieger, Lisa. 2021. Digitale Edition. In: KONDE Weißbuch. Hrsg. v. Helmut W. Klug unter Mitarbeit von Selina Galka und Elisabeth Steiner im HRSM Projekt "Kompetenznetzwerk Digitale Edition". URL: https://gams.uni-graz.at/o:konde.59\newpage\section*{Digitale Musikedition} \emph{Galka, Selina; selina.galka@uni-graz.at}\\
        
    Unter dem Begriff ‘Digitale Musikedition’ können unterschiedliche Varianten von Editionen verstanden werden:\\
            
        Johannes Kepper unterschied 2011 im Hinblick auf die bis dahin verfügbaren Editionen und Projekte zwischen “elektronischen Editionen”, die bereits bestehende Inhalte in das digitale Medium bringen und die technischen Möglichkeiten wie Suchfunktion, Verlinkung, Sortierung oder flexiblere Präsentationsmöglichkeiten nutzen (vgl. dazu auch \href{http://gams.uni-graz.at/o:konde.96}{Hybridedition}), und “Digitalen Editionen”, die als \emph{born-digital} verstanden werden können und auf Musikcodierung, Notentext- und Quellenabbildungen zurückgreifen. Diese Art von Digitalen Editionen ließ sich bisher vor allem im Bereich der älteren Musik sehr gut umsetzen, da der Zeichenvorrat zu dieser Zeit noch begrenzt war. (Kepper 2011, S. 167ff. und Glossar aus dem Projekt ‘Beethovens Werkstatt’: Digitale Musikedition)\\
            
        Die heute verbreiteten Codierungsysteme für Musikeditionen sind Music\href{http://gams.uni-graz.at/o:konde.215}{XML} und \href{http://gams.uni-graz.at/o:konde.226}{MEI} (\emph{Music Encoding Initiative}). MEI wird in Anlehnung an die \href{http://gams.uni-graz.at/o:konde.178}{TEI} seit 1999 von Perry Roland entwickelt und legt den Fokus auf westeuropäische Kunstmusik; MusicXML wurde von Michael Good entworfen (2004) und dient ebenfalls der Codierung moderner westlicher Notenschrift.\\
            
        Die Digitale Musikedition orientiert sich zwar an den digitalen Editionstechniken der Textwissenschaften, allerdings unterscheidet sich die Edition von Notentext doch in einem wesentlichen Aspekt: Das musikalische Zeichensystem ist teilweise durch starke Doppel- und Mehrdeutigkeiten geprägt und schon allein der Notentext nur eine schriftliche Repräsentation des Klangereignisses und defizitär. (Veit 2005, S. 11–14)\\
            
        Musikeditionen können grundsätzlich zwei Schwerpunkte haben: einerseits die wissenschaftliche Ausgabe, also beispielsweise Darstellung der Varianten, andererseits spielt aber auch der Praxisbezug eine große Rolle, z. B. die Möglichkeit der Ausgabe von nur einzelnen, ausgewählten Stimmen, MIDI-Wiedergabe, Transposition, Metronom oder Notengenerierung (z. B. als PDF). Digitale Musikeditionen, die diese Möglichkeiten ausschöpfen und auf codierten Quellen basieren, sind bisher nur zum Teil vorhanden.\\
            
        \subsection*{Literatur:}\begin{itemize}\item Kepper, Johannes: Musikedition im Zeichen neuer Medien. Historische Entwicklung und gegenwärtige Perspektiven musikalischer Gesamtausgaben Musikedition im Zeichen neuer Medien. Norderstedt: 2011, URL: \url{https://kups.ub.uni-koeln.de/6639/}.\item Acquavella-Rauch, Stefanie: (Musik)Edition im ›digitalen Zeitalter‹ – Versuch einer Verortung konzeptioneller und struktureller Veränderungen. In: Beitragsarchiv des Internationalen Kongresses der Gesellschaft für Musikforschung. »Wege der Musikwissenschaft« - 2016 Stand und Perspektiven musikwissenschaftlicher Digital-Humanities-Projekte. Mainz: 2018.\item Musikeditionen im Wandel der Geschichte. Hrsg. von Reinmar Emans und Ulrich Krämer. Berlin, Boston: 2015.\item Veit, Joachim: Musikwissenschaft und Computerphilologie – eine schwierige Liaison? In: Jahrbuch für Computerphilologie 7: 2005, S. 67–92.\item Veit, Joachim: Digitale Edition zwischen Experiment und Standardisierung. Internationale Tagung im Heinz-Nixdorf-Museums-Forum Paderborn, 6.-8. Dezember 2007. In: editio 22: 2008, S. 232–240.\item Veit, Joachim: Digitale Edition und Noten-Text. Vermittlungs- und Erkenntnisfortschritt? In: Im Dickicht der Texte. Editionswissenschaft als interdisziplinäre Grundlagenforschung, hg. v. Gesa Dane, Jörg Jungmayr u. Marcus Schotte: 2013, S. 233–266?.\item Veit, Joachim: Musikedition 2.0: Das 'Aus' für den Edierten Notentext? In: editio 29: 2015, S. 70–84.\item Music Encoding Initiative. URL: \url{https://music-encoding.org}\item Glossar aus dem Projekt "Beethovens Werkstatt": Digitale Musikedition. URL: \url{https://beethovens-werkstatt.de/glossary/digitale-musikedition/}\end{itemize}\subsection*{Software:}\href{https://www.edirom.de/edirom-projekt/index.html}{Edirom}, \href{https://www.humdrum.org}{Humdrum}, \href{https://www.sibelius.at}{Sibelius}, \href{https://musescore.com}{MuseScore}, \href{https://www.finalemusic.com}{Finale}\subsection*{Verweise:}\href{https://gams.uni-graz.at/o:konde.121}{LZA-Datenformate: Audio- bzw. Musik- und Videoformate}, \href{https://gams.uni-graz.at/o:konde.215}{XML}, \href{https://gams.uni-graz.at/o:konde.59}{Digitale Edition}, \href{https://gams.uni-graz.at/o:konde.226}{MEI}\subsection*{Projekte:}\href{http://www.digital-musicology.at/de-at/}{Digitale Musikwissenschaft Österreich}, \href{https://www.edirom.de/edirom-projekt/digitale-musikedition/}{Edirom - Digitale Musikedition}, \href{https://www.cmme.org}{The CMEE Project}, \href{https://www.dimused.uni-tuebingen.de}{DiMusEd}, \href{https://beethovens-werkstatt.de}{Beethovens Werkstatt}, \href{https://freischuetz-digital.de}{Freischütz digital}, \href{https://bargheer.edirom.de}{Carl Louis Bargheer: Fiedellieder plus. Digitale Edition}, \href{https://sarti-edition.de}{Sarti-Edition}\subsection*{Themen:}Einführung, Digitale Editionswissenschaft\subsection*{Zitiervorschlag:}Galka, Selina. 2021. Digitale Musikedition. In: KONDE Weißbuch. Hrsg. v. Helmut W. Klug unter Mitarbeit von Selina Galka und Elisabeth Steiner im HRSM Projekt "Kompetenznetzwerk Digitale Edition". URL: https://gams.uni-graz.at/o:konde.139\newpage\section*{Digitale Nachhaltigkeit} \emph{Stigler, Johannes; johannes.stigler@uni-graz.at }\\
        
    Der Begriff der Nachhaltigkeit, der vorwiegend im Zusammenhang mit ökologischen Themen verwendet wird, erhält im Kontext des ‘Digitalen’ eine erweiterte Bedeutung und meint den bewussten Umgang mit digitalen Ressourcen, die dann nachhaltig verwaltet werden, wenn ihr Nutzen für die Gesellschaft maximiert wird, sodass die digitalen Bedürfnisse gegenwärtiger und zukünftiger Generationen gleichermaßen erfüllt bleiben. (vgl. Dapp 2013)\\
            
        Digitale Ressourcen in diesem Sinn repräsentieren (a) Wissensprodukte und kulturelle Artefakte, die als Text, Bild, Audio, Video oder Software digital vorliegen und deren digitale Repräsentationen (b) transparente Strukturen aufweisen, (c) durch semantische (Meta-)Daten erschlossen und (d) gegen Datenverlust redundant an verschiedenen Orten gespeichert sind. (vgl. auch Stürmer 2017)\\
            
        Digitale Nachhaltigkeit ist somit auch ein wichtiger Faktor in der Akzeptanz des ‘digitalen Weges’ in der Fachcommunity der Editorinnen und Editoren. Schwierig macht die Erfüllung dieser Maxime eine vorwiegend durch kommerzielle Interessen bestimmte Grundcharakteristik der Informationstechnologie: der ständige Wandel, das ständig Neue als bestimmende Prämisse der technologischen Entwicklung. Sie erfordert spezifische Strategien in der Konzeption von nachhaltigen Langzeitstrategien für die Virtualisierung des kulturellen Erbes, dem natürlich auch \href{http://gams.uni-graz.at/o:konde.59}{Editionen} zuzurechnen sind.\\
            
        Wichtige Pfeiler einer solchen digitalen Nachhaltigkeitsstrategie sind die Verwendung von (a) offenen, dokumentierten, gleichermaßen maschinen- und menschenlesbaren (Meta-)Datenformaten, (b) offene Daten und Inhalte, (c) die Verwendung von offener und freier Software (\emph{open source}) für die Prozessierung dieser Daten (vgl. auch die \emph{Open Definition} der \emph{Open Knowledge Foundation}) und (d) nötige technische und institutionelle Infrastrukturen, die sich der Kuration des digitalen Kulturerbes unter sich ständig ändernden technologischen Bedingungen annehmen und die idealerweise im Umfeld der klassischen Gedächtnisinstitutionen angesiedelt sind.\\
            
        Zur Speicherung von Textdaten hat sich mittlerweile das Schema der \emph{Text Encoding Initiative} (\href{http://gams.uni-graz.at/o:konde.178}{TEI}) als Quasi-Standard in den Geisteswissenschaften durchgesetzt. Zunehmend existieren auch standardisierte Vokabularien und Metadatenstandards zur systematischen Tiefenerschließung digitaler Objekte (z. B. \href{http://gams.uni-graz.at/o:konde.131}{RDF}, \href{http://gams.uni-graz.at/o:konde.132}{SKOS}, \href{http://gams.uni-graz.at/o:konde.131}{OWL}). \href{http://gams.uni-graz.at/o:konde.9}{Lizenzmodelle} regeln den Zugang und die Verwendung des digital vorliegenden Kulturerbes (z. B. \href{http://gams.uni-graz.at/o:konde.45}{Creative Commons}). Eine Vielzahl von Open Source-Initiativen entwickelt Software, die sich als Basis für eine Nachhaltigkeitsstrategie für das Kulturerbe sehr gut eignen (z. B. \href{http://gams.uni-graz.at/o:konde.69}{FEDORA}). Am wenigsten ausgeprägt sind noch institutionelle Trägerstrukturen, auch wenn hier über EU-Mittel geförderte, internationale Langzeit-Infrastrukturprojekte wie CLARIN oder DARIAH oder die Diskussion um eine nationale Forschungsdateninfrastruktur (NFDI) in der BRD wichtige Impulse für deren Etablierung liefern.\\
            
        \subsection*{Verweise:}\href{https://gams.uni-graz.at/o:konde.4}{nestor Kompetenznetzwerk}, \href{https://gams.uni-graz.at/o:konde.7}{FAIR Prinzipien}, \href{https://gams.uni-graz.at/o:konde.11}{OAIS RM}, \href{https://gams.uni-graz.at/o:konde.13}{Trusted Repository & Zertifizierungen}, \href{https://gams.uni-graz.at/o:konde.178}{TEI}, \href{https://gams.uni-graz.at/o:konde.132}{SKOS}, \href{https://gams.uni-graz.at/o:konde.131}{RDF, RDFS, OWL}, \href{https://gams.uni-graz.at/o:konde.3}{COAR Prinzipien}\subsection*{Projekte:}\href{https://www.clarin-d.net/de/}{CLARIN-D}, \href{http://opendefinition.org/}{Open Definition}, \href{https://de.dariah.eu/}{DARIAH-DE}, \href{https://www.youtube.com/watch?v=x3Cvn1vNQ98&feature=youtu.be}{National Research Data Infrastructure (NFDI)}\subsection*{Literatur:}\begin{itemize}\item Dapp, Marcus: Digitale Nachhaltigkeit« als Strategisches Prinzip für das 21. Jh.: 2013, URL: \url{http://digisus.com/blog/wp-content/uploads/2013/06/Marcus-Dapp-Digitale-Nachhaltigkeitsstrategie-OSSTage-2013-06-20.pdf}.\item Digitale Nachhaltigkeit: Digitale Gemeingüter für die Wissensgesellschaft der Zukunft. URL: \url{https://www.parldigi.ch/de/2017/07/digitale-nachhaltigkeit/}\end{itemize}\subsection*{Themen:}Einführung, Archivierung, Digitale Editionswissenschaft\subsection*{Lexika}\begin{itemize}\item \href{https://edlex.de/index.php?title=Langzeitarchivierung}{Edlex: Editionslexikon}\end{itemize}\subsection*{Zitiervorschlag:}Stigler, Johannes. 2021. Digitale Nachhaltigkeit. In: KONDE Weißbuch. Hrsg. v. Helmut W. Klug unter Mitarbeit von Selina Galka und Elisabeth Steiner im HRSM Projekt "Kompetenznetzwerk Digitale Edition". URL: https://gams.uni-graz.at/o:konde.6\newpage\section*{Digitalisierung} \emph{Rieger, Lisa; lrieger@edu.aau.at }\\
        
    Digitalisierung ist ein umfassender Begriff, der sich – je nach Umfang der Definition – sowohl auf die digitale Umwandlung bzw. Durchführung von Informationen und Kommunikation, die digitale Modifikation von Instrumenten, Geräten und Fahrzeugen als auch auf die digitale Revolution bzw. digitale Wende allgemein beziehen kann. (Bendel o. J.) Sowohl der \emph{Duden} als auch das \emph{Oxford English Dictionary} beschränken sich bei ihren Definitionen jedoch auf den Vorgang der Umwandlung von Daten und Informationen (Duden o. J.) bzw. Text, Bilder oder Ton  in ein digitales Format, das von Computern bearbeitet werden kann (Oxford English Dictionary o. J.). Das zentrale Ziel der Digitalisierung ist dabei der Einsatz fortgeschrittener Technologien in sämtlichen Prozessen, Produkten und Services. (Parida 2018, S. 23)\\
            
        Für die Digitalisierung literarischer Werke und ähnlicher Texte gibt es zwei wesentliche Strategien: Bei der Image-Erschließung wird eine Buchseite durch Scannen erfasst. Dabei entsteht zwar ein authentisches Textbild, ohne nachträgliche Indizierung von Überschriften oder inhaltlichen Einheiten u. Ä. kann der Computer jedoch nicht mit dem Inhalt arbeiten. Dies ermöglicht erst eine Volltexterfassung, bei der jedes Einzelzeichen eines Textes erfasst und dementsprechend auch in einer Suche gefunden werden kann. Aufgrund des erhöhten Aufwandes ist dieses Verfahren jedoch wesentlich kostspieliger. Mit Hilfe automatischer Texterkennungsprogramme (\href{http://gams.uni-graz.at/o:konde.149}{OCR}) können diese beiden Verfahren miteinander verknüpft werden – hier kommt es jedoch bei unvollständigen und graphisch komplexeren Texten vermehrt zu Problemen. (Drummer et al. 2012, 196)\\
            
        Um eine einheitliche Verfahrensweise zu erreichen, werden von größeren Forschungsgesellschaften Leitfäden für den Digitalisierungsprozess herausgegeben. Ein Beispiel hierfür sind die \emph{Praxisregeln „Digitalisierung“} der Deutschen Forschungsgemeinschaft, die darauf abzielen, durch Standards zur Nachhaltigkeit von Projekten beizutragen sowie auch Personen ohne Vorkenntnisse die Arbeit zu erleichtern. (Deutsche Forschungsgemeinschaft 2016)\\
            
        Einen Schwerpunkt auf die zu beachtenden \href{http://gams.uni-graz.at/o:konde.223}{rechtlichen Aspekte} bei der Digitalisierung gemeinfreier Werke legt der Leitfaden von Kreutzer (Kreutzer 2011), der sich jedoch auf die Rechtslage in Deutschland bezieht. In Österreich wird die Digitalisierung hauptsächlich in Zusammenhang mit Wirtschaft und Industrie thematisiert, weshalb auf der Homepage der Österreichischen Forschungsförderungsgesellschaft (Österreichische Forschungsförderungsgesellschaft o. J.) derzeit kein der Digitalisierung literarischer Werke und/oder ähnlicher Texte gewidmeter Leitfaden zu finden ist.\\
            
        \subsection*{Literatur:}\begin{itemize}\item Digitalisierung. URL: \url{https://wirtschaftslexikon.gabler.de/definition/digitalisierung-54195}\item digitalisieren. In: Duden.\item DFG-Praxisregeln "Digitalisierung", Deutsche Forschungsgemeinschaft: 2016. URL: \url{https://www.dfg.de/formulare/12_151/}.\item Drummer, Sven; Michaelis, Frank; Schlaefer, Michael: Zur Digitalisierung historischer Wörterbücher. In: Lexikos 8: 1998, S. 194–222.\item Kreutzer, Till: Digitalisierung gemeinfreier Werke durch Bibliotheken. Leitfaden von Dr. Till Kreutzer, i.e. - Büro für informationsrechtliche Expertise Berlin Digitalisierung gemeinfreier Werke durch Bibliotheken, Hochschulbibliothekszentrum des Landes Nordrhein-Westfalen: 2011. URL: \url{https://irights.info/wp-content/uploads/userfiles/Digitalisierungsleitfaden.pdf}.\item Den Forschungs- und Innovationsstandort Österreich stärken. URL: \url{https://www.ffg.at/}\item digitalization Lexico.com. In: Lexico.com. Powered by Oxford.\item Parida, Vinit: Digitalization. In: Adressing Societal Challenges. Lulea: 2018, S. 23–38.\end{itemize}\subsection*{Verweise:}\href{https://gams.uni-graz.at/o:konde.59}{Digitale Edition}, \href{https://gams.uni-graz.at/o:konde.80}{Elemente digitaler Editionen}, \href{https://gams.uni-graz.at/o:konde.6}{Digitale Nachhaltigkeit}, \href{https://gams.uni-graz.at/o:konde.197}{Transkription}, \href{https://gams.uni-graz.at/o:konde.40}{Checkliste Digitalisierung}, \href{https://gams.uni-graz.at/o:konde.61}{Digitalisierungsdienste}, \href{https://gams.uni-graz.at/o:konde.62}{Digitalisierungskosten}, \href{https://gams.uni-graz.at/o:konde.63}{Digitalisierungsrichtlinien}, \href{https://gams.uni-graz.at/o:konde.149}{OCR}, \href{https://gams.uni-graz.at/o:konde.64}{Digitalisierungsstandards}, \href{https://gams.uni-graz.at/o:konde.36}{Bereitstellung von Digitalisaten}, \href{https://gams.uni-graz.at/o:konde.38}{Forschungsinstitut Brenner Archiv}, \href{https://gams.uni-graz.at/o:konde.201}{Universität Innsbruck}\subsection*{Themen:}Digitalisierung\subsection*{Lexika}\begin{itemize}\item \href{https://edlex.de/index.php?title=Digitalisierung}{Edlex: Editionslexikon}\item \href{https://lexiconse.uantwerpen.be/index.php/lexicon/digitization/}{Lexicon of Scholarly Editing}\end{itemize}\subsection*{Zitiervorschlag:}Rieger, Lisa. 2021. Digitalisierung. In: KONDE Weißbuch. Hrsg. v. Helmut W. Klug unter Mitarbeit von Selina Galka und Elisabeth Steiner im HRSM Projekt "Kompetenznetzwerk Digitale Edition". URL: https://gams.uni-graz.at/o:konde.60\newpage\section*{Digitalisierung: Rechtliches} \emph{Scholger, Walter; walter.scholger@uni-graz.at }\\
        
    \href{http://gams.uni-graz.at/o:konde.59}{Digitale Editionen} beinhalten fast immer neben Transkripten und anderen Materialien auch Faksimiles der Originaldokumente, üblicherweise in Form von \href{http://gams.uni-graz.at/o:konde.36}{Digitalisaten}. Der (rechtliche) Status solcher Digitalisate ist in Österreich bislang ungeklärt.\\
            
        Urheberrechtlich ist in den meisten Fällen davon auszugehen, dass eine Fotografie als ‘Werk der bildenden Künste’ bzw. Lichtbildwerk gilt, da die Schwelle der Schöpfungshöhe bei Lichtbildwerken sehr niedrig angesetzt wird und jeder Schnappschuss darunter fällt. Selbst in Fällen, in denen die Schöpfungshöhe nicht erreicht wird, besteht jedoch ein Leistungsschutzrecht für Lichtbilder (UrhG §74), d. h. das Lichtbild ist als Leistung des Fotografen bzw. der Fotografin geschützt, selbst wenn es sich nicht um ein Werk handelt. \\
            
        Unklar ist, wie in diesem Sinne Digitalisate, also Scans, zu bewerten sind: Grundsätzlich ist die Reprografie als ‘ein der Photographie ähnliches Verfahren’ zu bewerten, womit an der Reproduktion selbst Leistungsschutzrechte der jeweiligen Bearbeiterin bzw. des jeweiligen Bearbeiters entstehen würden. \\
            
        Nach anderer Auffassung stellt ein Scan – insbesondere wenn es sich dabei um eine maschinelle Reproduktion handelt, die keine technische oder fachliche Expertise erfordert – lediglich eine Vervielfältigung dar, an der selbst keine Rechte entstehen. \\
            
        Die \emph{EU-Urheberrechtsrichtlinie 2019/790} über das Urheberrecht und verwandte Schutzrechte im digitalen Binnenmarkt sieht jedenfalls vor, dass Vervielfältigungen von Werken der bildenden Künste, die gemeinfrei (also außerhalb der urheberrechtlichen Schutzfristen) sind, ebenfalls gemeinfrei bleiben. Die Umsetzung der Richtlinie in nationales Recht muss bis 7. Juni 2021 erfolgen.\\
            
        \subsection*{Literatur:}\begin{itemize}\item Klimpel, Paul; Fack, Fabian; Weitzmann, John H: Handreichung  – Neue rechtliche Rahmenbedingungen für Digitalisierungsprojekte von Gedächtnisinstitutionen. Berlin: 2017, URL: \url{doi:10.12752/2.0.002.3.}.\item Klimpel, Paul; Weitzmann, John H: Forschen in der digitalen Welt. Juristische Handreichung für die Geisteswissenschaften. Göttingen: 2014.\item Scholger, Walter: Urheberrecht und offene Lizenzen im wissenschaftlichen Publikationsprozess. In: Publikationsberatung an Universitäten.Ein Praxisleitfaden zum Aufbau publikationsunterstützender Services. Bielefeld: 2020, S. 123–147.\item RICHTLINIE  (EU)  2019/  790  DES  EUROPÄISCHEN  PARLAMENTS  UND  DES  RATES  -  vom  17. April  2019  -  über  das  Urheberrecht  und  die  verwandten  Schutzrechte  im  digitalen  Binnenmarkt  und  zur  Änderung  der  Richtlinien  96/  9/  EG  und  2001/  29/  EG. URL: \url{https://eur-lex.europa.eu/legal-content/DE/TXT/PDF/?uri=CELEX:32019L0790}.\end{itemize}\subsection*{Verweise:}\href{https://gams.uni-graz.at/o:konde.44}{Urheberrecht}, \href{https://gams.uni-graz.at/o:konde.222}{Freie Werknutzungen}\subsection*{Themen:}Rechtliche Aspekte\subsection*{Zitiervorschlag:}Scholger, Walter. 2021. Digitalisierung: Rechtliches. In: KONDE Weißbuch. Hrsg. v. Helmut W. Klug unter Mitarbeit von Selina Galka und Elisabeth Steiner im HRSM Projekt "Kompetenznetzwerk Digitale Edition". URL: https://gams.uni-graz.at/o:konde.223\newpage\section*{Digitalisierungsdienste} \emph{Lenger, Karl; karl.lenger@uni-graz.at }\\
        
    Für die Massendigitalisierung gibt es sehr viele Privatunternehmen, die ein entsprechendes Service anbieten. Im speziellen Bereich der Altbestandsdigitalisierung hingegen existieren nur wenige Handschriftenzentren, welche die dafür erforderliche technische Infrastruktur und das nötige Know-how bieten können. \\
            
        In Österreich hat sich das Grazer Digitalisierungszentrum an der Universitätsbibliothek, auch auf Grund seines selbst entwickelten Grazer Kameratisches sowie dem sogenannten ‘Traveller’, zu einem Kompetenzzentrum für externe und Inhouse-Aufträge speziell in der Handschriftendigitalisierung entwickelt. Gerade im Umgang mit sensiblen Kulturgütern gilt es, besondere Kriterien zu erfüllen. Hierfür bieten Handschriftenzentren wie das in Graz aufgrund der Eigenbestände zum Beispiel alarmgesicherte Tresore, die für die Aufbewahrung von Sonderbeständen speziell klimatisiert werden.   \\
            
        Weitere Digitalisierungsdienste in Österreich sind an der Österreichischen Nationalbibliothek, der Steiermärkischen Landesbibliothek, dem Steiermärkischen Landesarchiv, der oberösterreichischen Landesbibliothek, dem oberösterreichischen Landesarchiv, der Universitätsbibliothek Salzburg sowie an der Universitäts- und Landesbibliothek Tirol zu finden.\\
            
        In Deutschland gibt es eine Vielzahl von Handschriftenzentren, die die Durchführung von Erschließungs- und Digitalisierungsprojekten anbieten – unter anderen das Münchener Digitalisierungszentrum (MDZ), das Göttinger Digitalisierungszentrum (GDZ) oder das Digitalisierungszentrum der UB Heidelberg, um stellvertretend nur einige wenige zu nennen.\\
            
        \subsection*{Literatur:}\begin{itemize}\item Der Traveller TCCS 4232. URL: \url{https://vestigia.uni-graz.at/de/arbeitsbereiche-projekte/technologieentwicklung/der-traveller-tccs-4232/}\end{itemize}\subsection*{Verweise:}\href{https://gams.uni-graz.at/o:konde.40}{Checkliste Digitalisierung}, \href{https://gams.uni-graz.at/o:konde.62}{Digitalisierungskosten}, \href{https://gams.uni-graz.at/o:konde.63}{Digitalisierungsrichtlinien}, \href{https://gams.uni-graz.at/o:konde.64}{Digitalisierungsstandards}, \href{https://gams.uni-graz.at/o:konde.60}{Digitalisierung}, \href{https://gams.uni-graz.at/o:konde.153}{Österreichische Nationalbibliothek}\subsection*{Themen:}Digitalisierung\subsection*{Zitiervorschlag:}Lenger, Karl. 2021. Digitalisierungsdienste. In: KONDE Weißbuch. Hrsg. v. Helmut W. Klug unter Mitarbeit von Selina Galka und Elisabeth Steiner im HRSM Projekt "Kompetenznetzwerk Digitale Edition". URL: https://gams.uni-graz.at/o:konde.61\newpage\section*{Digitalisierungskosten} \emph{Lenger, Karl; karl.lenger@uni-graz.at }\\
        
    Bei der Berechnung der Kosten eines Digitalisierungsprojektes spielen zahlreiche
                  Faktoren eine Rolle, die es am Beginn der Planung eines Projektes zu
                  berücksichtigen gilt. Grundsätzlich sind die Kosten für Personal und der Aufbau
                  der technischen Infrastruktur bzw. des Know-hows am kostenintensivsten. \\
            
        Ein gut ausgestattetes Digitalisierungszentrum verfügt über entsprechendes
                  Know-how und moderne Technologien. Bei der Wahl eines Scanners sollte gerade im
                  Rahmen der \href{http://gams.uni-graz.at/o:konde.60}{Digitalisierung} von
                  alten Kulturgütern immer der konservatorische Aspekt beachtet werden. Im Grazer
                  Digitalisierungszentrum spricht man deswegen auch von objektschonender
                  Digitalisierung. \\
            
        Die Kosten sind also stark von der Materialität der zu digitalisierenden Vorlagen
                  abhängig. Zudem muss bei sensiblen Materialien der restauratorische Zustand
                  berücksichtigt werden, da diese gegebenenfalls bereits vor dem
                  Digitalisierungsprozess aufwändigen Maßnahmen unterzogen werden müssen.
                  Folgekosten, die zum Beispiel durch Wartung von Software, Hardware sowie
                  Speicherung und \href{http://gams.uni-graz.at/o:konde.6}{Langzeitarchivierung} oder spätere Datenmigration entstehen, sind in den
                  meisten Fällen meist nur grob abschätzbar. \\
            
        Kostenfaktoren sind u. a.:
                  (siehe auch ETH-Bibliothek, S. 5)\\
            
        \begin{itemize}\item {
                     Personalaufwand für das Scannen 
                  }\item {
                     Technische Infrastruktur (Scanner, Kamera, Software,
                        Datenbanken etc.) 
                  }\item {
                     Lizenzkosten für Bildbearbeitungs- und Erschließungsprogramme
                        (\href{http://gams.uni-graz.at/o:konde.149)}{OCR})
                     
                  }\item {
                     Beschaffenheit der Vorlage (gebunden, aufgeschnitten, Spezialformate)
                  }\item {Scanparameter (Auflösung) 
                  }\item {Onlinepräsentation (IT-Infrastruktur, Plattform, Support,
                        Metadatenerfassung) 
                  }\item {Transport und Versicherungskosten 
                  }\end{itemize}Ein Beispiel für eine Kostenberechnung für die Digitalisierung mittelalterlicher
                  Handschriften findet man im Masterplan der Digitalisierung
                     mittelalterlicher Handschriften in deutschen Bibliotheken. (Punkt 8,
                     S. 10–11)\\
            
        \subsection*{Literatur:}\begin{itemize}\item ETH-Bibliothek: Best Practices Digitalisierung: 2016. URL: \url{https://www.library.ethz.ch/de/content/download/19891/509651/version/9/file/best-practices-digitalisierung.pdf}.\item Digitalisierung mittelalterlicher Handschriften in
                              deutschen Bibliotheken. Masterplan. Bayerische Staatsbibliothek: 2015.\end{itemize}\subsection*{Verweise:}\href{https://gams.uni-graz.at/o:konde.6}{Digitale Nachhaltigkeit}, \href{https://gams.uni-graz.at/o:konde.40}{Checkliste Digitalisierung}, \href{https://gams.uni-graz.at/o:konde.61}{Digitalisierungsdienste}, \href{https://gams.uni-graz.at/o:konde.63}{Digitalisierungsrichtlinien}, \href{https://gams.uni-graz.at/o:konde.64}{Digitalisierungsstandards}, \href{https://gams.uni-graz.at/o:konde.60}{Digitalisierung}\subsection*{Themen:}Digitalisierung\subsection*{Zitiervorschlag:}Lenger, Karl. 2021. Digitalisierungskosten. In: KONDE Weißbuch. Hrsg. v. Helmut W. Klug unter Mitarbeit von Selina Galka und Elisabeth Steiner im HRSM Projekt "Kompetenznetzwerk Digitale Edition". URL: https://gams.uni-graz.at/o:konde.62\newpage\section*{Digitalisierungsrichtlinien (Bilddigitalisierung)} \emph{Klug, Helmut W. ; helmut.klug@uni-graz.at }\\
        
    Bei der Bilddigitalisierung historischer Quellen müssen laut den \emph{DFG-Praxisregeln Digitalisierung}(S. 14–20) eine Reihe von Aspekten beachtet werden, die grundlegende Maxime könnte aber lauten: „Als Untergrenze sollte daher eine Scanauflösung für die Digitalisate gewählt werden, bei der die Details einer Vorlage vollständig in einer gleichgroßen Reproduktion wiedergegeben werden können.“ (S. 15) Zu beachten sind die technischen Parameter Auflösung und Farbtiefe.\\
            
        Auflösung: Als Standardauflösung sollten immer 300 dpi veranschlagt werden. Abhängig von der Vorlage muss dieser Wert individuell angepasst werden. Diese Anpassung richtet sich einerseits nach der Größe der Vorlage, aber auch nach deren Detailreichtum, z. B.:\\
            
        \begin{itemize}\item {Textwerke, Grafiken und fotografische Aufsichtvorlagen: 300 dpi, Ziel: 1:1-Abbildung/Druck.}\item {Kleinbildnegativ: 3000 dpi, Ziel: Druckrepräsentation in 24x36 cm.}\item {Fotografie (10x13): 800 dpi, Ziel: 1:1-Abbildung/Druck}\item {Großvorlagen (z. B. Plakate) für hohen Betrachtungsabstand: 150 dpi, Ziel: 1:1-Abbildung/Druck}\end{itemize}Die technischen Voraussetzungen des Aufnahmemediums sollten grundsätzlich so gut wie möglich ausgenutzt werden, da dies immer einen Mehrwert in der späteren Weiterver/bearbeitung der Digitalisate bringt.\\
            
        Farbtiefe: Höhere Farbtiefe schafft eine größere Farbdifferenzierung, was bewirkt, dass bei einer späteren Bearbeitung weniger Farbinformationen verloren gehen: „Für die Sicherung des finalen digitalen Master ist eine Farbtiefe von 8 Bit pro Kanal, d. h. 24 Bit ausreichend, da die heute gängigen Ausgabe- und Anzeigegeräte nur eine Tonwertwiedergabe mit 8-Bit-Differenzierung unterstützen.“ (DFG-Praxisregeln Digitalisierung, S.17)\\
            
        Für die Erstellung digitaler Archivmaterialien aus Textwerken usw. wird folgender Workflow bei digitalfotografischer Reproduzierung vorgeschlagen: \\
            
        \begin{itemize}\item {Erstellung der digitalen Kopie im Rohdatenformat der Kamera (RAW) bei maximaler Größe und einer Farbtiefe von 14 oder 16 Bit pro Farbkanal unter Zuhilfenahme eines Graukeils oder einer Farbpalette.}\item {Bildkorrekturen sollten an diesen Originaldaten mit der Software des Kameraherstellers gemacht werden. }\item {Bildbearbeitungen sollen sich ausschließlich auf Farb- und Tonwertkorrekturen beschränken und werden bei 16 Bit Farbtiefe/Kanal vorgenommen.}\item {Konvertierung des Originals vom Kamera-RAW in eine Masterdatei mit 24 Bit/RGB oder 8 Bit/Graustufen als archivtaugliche Datei.}\end{itemize}\subsection*{Literatur:}\begin{itemize}\item DFG-Praxisregeln "Digitalisierung", Deutsche Forschungsgemeinschaft: 2016. URL: \url{https://www.dfg.de/formulare/12_151/}.\item Digital Humanities. Eine Einführung. Hrsg. von Fotis Jannidis, Hubertus Kohle und Malte Rehbein. Stuttgart: 2017, URL: \url{https://doi.org/10.1007%2f978-3-476-05446-3}.\end{itemize}\subsection*{Verweise:}\href{https://gams.uni-graz.at/o:konde.60}{Digitalisierung}, \href{https://gams.uni-graz.at/o:konde.64}{Digitalisierungsstandards}, \href{https://gams.uni-graz.at/o:konde.36}{Bereitstellung von Digitalisaten}, \href{https://gams.uni-graz.at/o:konde.6}{Digitale Nachhaltigkeit}, \href{https://gams.uni-graz.at/o:konde.40}{Checkliste Digitalisierung}, \href{https://gams.uni-graz.at/o:konde.61}{Digitalisierungsdienste}, \href{https://gams.uni-graz.at/o:konde.62}{Digitalisierungskosten}\subsection*{Themen:}Digitalisierung\subsection*{Zitiervorschlag:}Klug, Helmut W. 2021. Digitalisierungsrichtlinien (Bilddigitalisierung). In: KONDE Weißbuch. Hrsg. v. Helmut W. Klug unter Mitarbeit von Selina Galka und Elisabeth Steiner im HRSM Projekt "Kompetenznetzwerk Digitale Edition". URL: https://gams.uni-graz.at/o:konde.63\newpage\section*{Digitalisierungsstandards} \emph{Lenger, Karl; karl.lenger@uni-graz.at }\\
        
    In Österreich existieren im Gegensatz zu Deutschland (DFG-Praxisregeln Digitalisierung) noch keine einheitlichen Digitalisierungsstandards, sondern lediglich Empfehlungen. Digitale Bilder können in unterschiedlichen Dateiformaten gespeichert werden. Gebräuchliche Dateiformate in der Digitalisierung sind: TIFF, JPEG, JPEG2000, PDF, PDF/A. \\
            
        Für das Internet empfiehlt es sich zum Beispiel, ein sogenanntes Nutzungsderivat, also ein kleines Präsentationsformat wie etwa JPG oder PDF/A zu verwenden. Eine Masterdatei sollte aber immer als Ausgangsbasis in Form eines unkomprimierten Baseline-TIFFs erstellt werden. Denn eine verlustbehaftete Kompression lässt sich nicht mehr rückgängig machen. Neben TIFFs wird von der DFG ein Master in Form eines verlustfreien JPEG2000-Formats empfohlen. (DFG-Praxisregeln Digitalisierung, S. 6ff.)\\
            
        Da die Farbtreue des Digitalisates für Forschungsfragen meist ein bedeutendes Kriterium darstellt, empfiehlt es sich, der Aufnahme einen standardisierten Graukeil, Farbkeil oder auch einen Colorchecker beizulegen. Kamerabasierende Scanner bieten zusätzlich die Möglichkeit, digitale Bilder in einem herstellerbedingten Rohdatenformat (RAW-Format) in der maximalen Größe und einer Farbtiefe von 14 oder 16 Bit pro Farbkanal zu erzeugen. Jedoch ist das RAW-Format auf Grund der unterschiedlichen Kamerahersteller kein empfohlenes Format für die  \href{http://gams.uni-graz.at/o:konde.6}{Langzeitarchivierung}. Für die Anzahl der Bildpunkte pro Längeneinheit (Bildauflösung) empfiehlt die ETH Zürich (Best Practice Digitalisierung, S. 13–15):\\
            
        \begin{itemize}\item {300 dpi für Graustufen- und Farbvorlagen}\item {400 dpi für spezielle Vorlagen wie Handschriften, Drucke oder Karten mit filigranem Inhalt}\item {600 dpi für bitonale Scans (Schwarz-Weiss-Vorlagen)}\item {Höhere Auflösungen: nur sinnvoll für Spezialanwendungen wie zum Beispiel die Untersuchung von Papierstrukturen oder die Bilddigitalisierung (3000 bis 4000 dpi bei Negativen und Dias)}\end{itemize}Hinsichtlich der Präsentation und Referenzierbarkeit der Digitalisate mittelalterlicher Handschriften sowie der begleitenden \href{http://gams.uni-graz.at/o:konde.25}{Metadaten} existiert in Österreich sowie auch in Deutschland noch kein einheitlicher Standard. Mindeststandards für die Präsentation der Digitalisate sind aber in den Praxisregeln Digitalisierung für DFG-geförderte Digitalisierungsprojekte vorgegeben. (DFG-Praxisregeln, S.  6-7)\\
            
        \subsection*{Literatur:}\begin{itemize}\item DFG-Praxisregeln "Digitalisierung", Deutsche Forschungsgemeinschaft: 2016. URL: \url{https://www.dfg.de/formulare/12_151/}.\item ETH-Bibliothek: Best Practices Digitalisierung: 2016. URL: \url{https://www.library.ethz.ch/de/content/download/19891/509651/version/9/file/best-practices-digitalisierung.pdf}.\end{itemize}\subsection*{Verweise:}\href{https://gams.uni-graz.at/o:konde.60}{Digitalisierung}, \href{https://gams.uni-graz.at/o:konde.36}{Bereitstellung von Digitalisaten}, \href{https://gams.uni-graz.at/o:konde.6}{Digitale Nachhaltigkeit}, \href{https://gams.uni-graz.at/o:konde.40}{Checkliste Digitalisierung}, \href{https://gams.uni-graz.at/o:konde.61}{Digitalisierungsdienste}, \href{https://gams.uni-graz.at/o:konde.62}{Digitalisierungskosten}, \href{https://gams.uni-graz.at/o:konde.63}{Digitalisierungsrichtlinien}\subsection*{Themen:}Digitalisierung\subsection*{Zitiervorschlag:}Lenger, Karl. 2021. Digitalisierungsstandards. In: KONDE Weißbuch. Hrsg. v. Helmut W. Klug unter Mitarbeit von Selina Galka und Elisabeth Steiner im HRSM Projekt "Kompetenznetzwerk Digitale Edition". URL: https://gams.uni-graz.at/o:konde.64\newpage\section*{Diplomatische Edition} \emph{Rieger, Lisa; lrieger@edu.aau.at }\\
        
    Bei der diplomatischen Edition handelt es sich um die “gedruckte Ausgabe einer
                  Handschrift, die den genauen Zeilen- und Seiteninhalt des Originals beibehält” und
                  lediglich Abkürzungen aufschlüsselt (Best 1991, S. 113). Werden bei
                  einem diplomatischen Abdruck keinerlei Eingriffe durch den Herausgeber
                  vorgenommen, stellt dieser - als Gegenpol zur Darstellung der gesamten \href{http://gams.uni-graz.at/o:konde.28}{Textgenese} - eine „Extremposition
                  der \href{http://gams.uni-graz.at/o:konde.192}{textkritischen} Praxis“ dar.
                     (Nünning 2013, S. 740) Die Grenze zwischen diplomatischen Abdruck
                  und normalisierter Ausgabe kann, je nach Umfang der Eingriffe, oft sehr schmal
                  sein - weshalb sie auch nicht als Gegensätze, sondern “Stufen eines linearen
                  Editionsprozesses” gesehen werden sollten. (Hofmeister 2005: 8)\\
            
        Als Grundprinzip der Textkonstitution werden weder die Autorintention noch die
                  Erfüllung des Autorwillens herangezogen, sondern die möglichst genaue Darstellung
                  des Überlieferungsträgers angestrebt. Damit kann u. a. das Ziel verfolgt werden,
                  der Benutzerin oder dem Benutzer der Ausgabe die Textgenese nachvollziehbar zu
                  machen, wie es in der Edition der Fontane-Notizbücher bezweckt wird
                     (Jannidis/Kohle/Rehbein 2017, S. 235). Oft wird auch der Abbildung
                  eines \href{http://gams.uni-graz.at/o:konde.83}{Faksimiles} nur noch eine
                     \href{http://gams.uni-graz.at/o:konde.66}{diplomatische Transkription}
                  hinzugefügt (Plachta 1997, S. 42).\\
            
        \subsection*{Literatur:}\begin{itemize}\item Best, Otto: Handbuch literarischer Fachbegriffe. Definitionen und
                              Beispiele. Überarbeitete und erweiterte Ausgabe Handbuch literarischer Fachbegriffe. Frankfurt am Main: 1991.\item Jannidis, Fotis; Kohle, Hubertus: Digital Humanities. Eine Einführung. Mit Abbildungen und
                              Grafiken Digital Humanities. Hrsg. von  und Malte Rehbein. Stuttgart: 2017.\item Hofmeister, Andrea: Textkritik als Erkenntnisprozeß. In: Editio 19: 2005, S. 1-9.\item Textkritik. In: Metzler Lexkion Literatur- und Kulturtheorie. Ansätze -
                              Personen - Grundbegriffe 5., aktualisierte und erweiterte
                                 Auflage. Stuttgart, Weimar: 2013, S. 740–741.\item Plachta, Bodo: Editionswissenschaft. Eine Einführung in Methode und
                              Praxis der Edition neuerer Texte Editionswissenschaft: 1997.\end{itemize}\subsection*{Verweise:}\href{https://gams.uni-graz.at/o:konde.28}{Textgenese}, \href{https://gams.uni-graz.at/o:konde.192}{Textkritik}, \href{https://gams.uni-graz.at/o:konde.83}{Faksimileausgabe}, \href{https://gams.uni-graz.at/o:konde.66}{Diplomatische Transkription}, \href{https://gams.uni-graz.at/o:konde.50}{Datenmodell hyperdiplomatische
                           Transkription}\subsection*{Projekte:}\href{https://fontane-nb.dariah.eu/index.html}{Theodor Fontane: Notizbücher. Digitale genetisch-kritische und
                           kommentierte Edition}\subsection*{Themen:}Einführung, Digitale Editionswissenschaft\subsection*{Lexika}\begin{itemize}\item \href{https://wiki.helsinki.fi/display/stemmatology/Edition%2C+diplomatic}{Parvum Lexicon Stemmatologicum}\item \href{https://lexiconse.uantwerpen.be/index.php/lexicon/edition-diplomatic/}{Lexicon of Scholarly Editing}\end{itemize}\subsection*{Zitiervorschlag:}Rieger, Lisa. 2021. Diplomatische Edition. In: KONDE Weißbuch. Hrsg. v. Helmut W. Klug unter Mitarbeit von Selina Galka und Elisabeth Steiner im HRSM Projekt "Kompetenznetzwerk Digitale Edition". URL: https://gams.uni-graz.at/o:konde.65\newpage\section*{Diplomatische Transkription} \emph{Klug, Helmut W.; helmut.klug@uni-graz.at }\\
        
    Eine diplomatische Transkription versucht die historische Quelle so layout- und zeichengetreu wie möglich abzubilden. (Kümper 2014, S. 84f.) Diese detaillierte \href{http://gams.uni-graz.at/o:konde.197}{Transkription} – manchmal auch zur ‘hyperdiplomatischen Transkription’ gesteigert – ist aus germanistischer Perspektive für die sprachhistorische Forschung von großer Relevanz. (Bein 2000, S. 92) Die diplomatische Dokumentation der Eigenheiten einer Handschrift macht aber auch detaillierte paläografische Untersuchungen möglich. Der gesteigerte Erkenntnisgewinn mithilfe einer detaillierten Quellentextbefundung steht auch im Zentrum des \emph{\href{http://gams.uni-graz.at/o:konde.72}{Documentary Editing}}. \\
            
        Um eine lesefreundliche Ausgabe eines auf diese Weise transkribierten Textes zu erstellen, müssen \href{http://gams.uni-graz.at/o:konde.146}{Normalisierungsrichtlinien} aufgestellt werden. Gerade eine \href{http://gams.uni-graz.at/o:konde.59}{Digitale Edition} bietet sich aufgrund der Trennung von Daten und Textrepräsentation an, (hyper)diplomatische Transkriptionen zur Verfügung zu stellen. \\
            
        \subsection*{Literatur:}\begin{itemize}\item Hofmeister, Wernfried; Hofmeister-Winter, Andrea: Schriftzüge unter der High-Tech-Lupe: Theoretische Grundlagen und erste praktische Ergebnisse des Grazer Pilotprojekts DAmalS (‚Datenbank zur Authentifizierung mittelalterlicher Schreiberhände‘). In: editio 22: 2008, S. 90–117.\item Hofmeister-Winter, Andrea: Die Grammatik der Schreiberhände. Versuch einer Klärung der Schreiberfrage anhand der mehrstufig-dynamischen Neuausgabe der Werke Hugos von Montfort. In: Edition und Sprachgeschichte. Baseler Fachtagung 2.-4. März 2005. Tübingen: 2007, S. 89–116.\item Kümper, Hiram: Materialwissenschaft Mediävistik. Wien, Köln, Weimar: 2014.\item Bein, Thomas: Die mediävistische Edition und ihre Methoden. In: Text und Edition: Positionen und Perspektiven. Berlin: 2000, S. 81–98.\item Sahle, Patrick: Digitale Editionsformen. Zum Umgang mit der Überlieferung unter den Bedingungen des Medienwandels. Teil 1: Das typografische Erbe. Norderstedt: 2013.\item Sahle, Patrick: Digitale Editionsformen. Zum Umgang mit der Überlieferung unter den Bedingungen des Medienwandels. Teil 3: Textbegriffe und Recodierung. Norderstedt: 2013.\end{itemize}\subsection*{Software:}\href{https://www.annotationstudio.org/}{Annotation Studio}, \href{http://transcribe-bentham.ucl.ac.uk/td/Transcribe_Bentham}{Bentham Transcription Desk}, \href{https://diyhistory.lib.uiowa.edu}{Civil War Diaries & Letters Transcription Project}, \href{http://cte.oeaw.ac.at/}{Classical Text Editor}, \href{https://github.com/gsbodine/crowd-ed}{Crowd-Ed}, \href{https://wiki.tei-c.org/index.php/CWRC-Writer}{CWRC-Writer}, \href{https://ecdosis.rocks/Home/}{Ecdosis}, \href{http://www.bbaw.de/telota/software/ediarum}{Ediarum}, \href{https://www.e-laborate.nl/en/}{eLaborate}, \href{http://linhd.es/en/}{EVI-Lindh}, \href{https://www.nch.com.au/scribe/index.html}{Express Scribe}, \href{https://fromthepage.com/}{FromThePage}, \href{http://edgerton-digital-collections.org/notebooks}{Harold "Doc" Edgerton Project}, \href{https://islandora.ca/}{Citizen Science, Collaboration}, \href{http://www.mom-wiki.uni-koeln.de/}{Itineranova-Editor}, \href{https://kcl-ddh.github.io/kiln/}{KILN}, \href{http://lombardpress.org/}{LombardPress}, \href{https://manuscriptdesk.uantwerpen.be/md/Main_Page}{Manuscript Desk}, \href{https://www.archives.gov/citizen-archivist/missions}{Citizen Archivist Dashboard}, \href{http://ntvmr.uni-muenster.de/de/manuscript-workspace}{NTVMR (manuscript workspace)}, \href{http://code.google.com/p/openscribe/}{Open Scribe}, \href{http://oxygenxml.com/}{Oxygen}, \href{https://github.com/oxygenxml/TEI-Facsimile-Plugin}{Oxygen-TEI-Facsimile-Plugin}, \href{http://www.fabiovitali.it/filologia/}{PhiloEditor}, \href{http://pybossa.com/}{PyBOSSA}, \href{http://github.com/zooniverse/Scribe}{Scribe}, \href{http://scripto.org/}{scripto}, \href{http://t-pen.org/TPEN/}{T-Pen}, \href{https://textgrid.de/}{TextGrid}, \href{https://www.textlab.org/about/}{TextLab}, \href{https://textualcommunities.org/app/}{Textual Communities}, \href{http://transcribo.org/en/}{Transcribo}, \href{https://transkribus.eu/Transkribus/}{Transkribus}, \href{http://www.tustep.uni-tuebingen.de/}{TUSTEP}, \href{http://bencrowder.net/coding/unbindery/}{Unbindery}, \href{https://docs.google.com/document/d/1QsFodbmuOld4ZAmnURR2tKewE1tgRo1cGxpaIUy92Mw/edit}{EditMOM3}, \href{http://wlt.synat.pcss.pl/}{Virtual Transcription Laboratory}, \href{http://menus.nypl.org/}{What's On the Menu?}, \href{http://en.wikisource.org/wiki/Main_Page}{Wikisource}, \href{http://community.ancestry.co.uk/awap}{World Archives Project}, \href{https://www.zooniverse.org/}{zooniverse}, \href{http://www.teitok.org/index.php?action=about}{TEITOK}, \href{https://www.digitisation.eu}{IMPACT Tools and Data}, \href{https://3pc.de/}{Refine!Editor}\subsection*{Verweise:}\href{https://gams.uni-graz.at/o:konde.198}{Transkriptionsrichtlinien}, \href{https://gams.uni-graz.at/o:konde.199}{Transkriptionswerkzeuge}, \href{https://gams.uni-graz.at/o:konde.197}{Transkription}, \href{https://gams.uni-graz.at/o:konde.50}{Datenmodell "hyperdiplomatische Transkription"}\subsection*{Projekte:}\href{http://gams.uni-graz.at/context:epistles}{St Patrick's epistles: Transcriptions of the seven medieval manuscript witnesses}\subsection*{Themen:}Digitalisierung, Digitale Editionswissenschaft\subsection*{Zitiervorschlag:}Klug, Helmut W. 2021. Diplomatische Transkription. In: KONDE Weißbuch. Hrsg. v. Helmut W. Klug unter Mitarbeit von Selina Galka und Elisabeth Steiner im HRSM Projekt "Kompetenznetzwerk Digitale Edition". URL: https://gams.uni-graz.at/o:konde.66\newpage\section*{Dissemination-Services: ARCHE} \emph{Ďurčo, Matej; matej.durco@oeaw.ac.at}\\
        
    ARCHE (\emph{A Resource Centre for the HumanitiEs}) ist ein Service für die stabile \href{http://gams.uni-graz.at/o:konde.6}{Langzeitarchivierung} und Dissemination von digitalen Forschungsdaten und -ressourcen für die österreichische geisteswissenschaftliche Forschungsgemeinschaft. ARCHE speichert Daten aus dem gesamten Bereich der Geisteswissenschaften.\\
            
        Die Hauptaufgabe von ARCHE besteht darin, geisteswissenschaftlich Forschenden einen einfachen und nachhaltigen Zugang zu \href{http://gams.uni-graz.at/o:konde.87}{digitalen Forschungsdaten und -ressourcen} zu ermöglichen. Dies geschieht zum einen durch die \href{http://gams.uni-graz.at/o:konde.6}{langfristige Erhaltung digitaler Daten} und die Förderung der Nutzung von \href{http://gams.uni-graz.at/o:konde.152}{Open Access}- und \emph{\href{http://gams.uni-graz.at/o:konde.8}{Open Data}}-Richtlinien. Zum anderen bietet ARCHE eine Reihe digitaler Komponenten, um die archivierten Ressourcen leicht auffindbar und nutzbar zu machen:\\
            
        \begin{itemize}\item {Browser, der generische Navigation und Suche in den archivierten Sammlungen ermöglicht, }\item {OAI-PMH-Schnittstelle, die maschinenlesbare \href{http://gams.uni-graz.at/o:konde.25}{Metadaten} in mehreren Formaten bereitstellt, }\item {Dissemination-Services, eine wachsende Anzahl offener Webapplikationen, die auf die Verarbeitung und Anzeige bestimmter Datentypen oder -formate spezialisiert sind. }\end{itemize}Ein typisches Beispiel für ein Dissemination-Service ist die Transformierung von \href{http://gams.uni-graz.at/o:konde.178}{TEI} (oder einer anderen \href{http://gams.uni-graz.at/o:konde.215}{XML}-Variante) in HTML oder PDF. Es können aber auch geographische Daten in eine Karte integriert oder Graf-Daten als interaktives Netzwerk visualisiert werden. \\
            
        Diese Dienste werden als entsprechende alternative Darstellungsformen in der Detailansicht der jeweiligen digitalen Objekte in ARCHE angeboten. Die Verknüpfung für welche Objekte, welche Dienste verfügbar sind, ist dynamisch, basierend auf bestimmten Attributen der Objekte (wie z. B. MIME-\emph{type}) im ARCHE-System konfiguriert.\\
            
        Folgende Dissemination-Services sind bereits implementiert:\\
            
        \begin{itemize}\item {CMDI2HTML \emph{access}}\item {RAW CMDI \emph{access}}\item {OWL2HTML}\item {\emph{View image}}\item {IIIF \emph{Endpoint}}\item {3D \emph{viewer}}\item {\emph{View on map}}\item {TEI2HTML}\item {WMS \emph{Endpoint}}\item {WFS \emph{Endpoint}}\item {TEI \emph{to} HTML \emph{conversion Endpoint}}\item {\emph{Thumbnail service}}\end{itemize}\subsection*{Literatur:}\begin{itemize}\item Trognitz, Martina; Ďurčo, Matej: One Schema to Rule them All. The Inner Workings of the Digital Archive ARCHE. In: Mitteilungen der Vereinigung Österreichischer Bibliothekarinnen und Bibliothekare 71: 2018, S. 217–231.\item Trognitz, Martina; Ďurčo, Matej: Shapeshifting Digital Language Resources - Dissemination Services on ARCHE. In: CLARIN2019 Book of Abstracts 2019. Leipzig.\end{itemize}\subsection*{Software:}\href{arche.acdh.oeaw.ac.at/}{ARCHE}\subsection*{Verweise:}\href{https://gams.uni-graz.at/o:konde.6}{Digitale Nachhaltigkeit}, \href{https://gams.uni-graz.at/o:konde.70}{GAMS}, \href{https://gams.uni-graz.at/o:konde.68}{DHPlus}, \href{https://gams.uni-graz.at/o:konde.69}{Fedora}\subsection*{Projekte:}\href{https://arche.acdh.oeaw.ac.at/browser/technical-setup}{ARCHE: Technical Setup}, \href{https://arche.acdh.oeaw.ac.at/browser/oeaw_detail/45647}{3-D-Objekt in der ARCHE}\subsection*{Themen:}Archivierung, Institutionen\subsection*{Zitiervorschlag:}Ďurčo, Matej. 2021. Dissemination-Services: ARCHE. In: KONDE Weißbuch. Hrsg. v. Helmut W. Klug unter Mitarbeit von Selina Galka und Elisabeth Steiner im HRSM Projekt "Kompetenznetzwerk Digitale Edition". URL: https://gams.uni-graz.at/o:konde.67\newpage\section*{Disseminations-Services: Geisteswissenschaftliches Asset Management System
               (GAMS)} \emph{Stigler, Johannes; johannes.stigler@uni-graz.at}\\
        
    GAMS ist ein \href{http://gams.uni-graz.at/o:konde.11}{OAIS}-konformes
                  Repositorium zur Verwaltung, Publikation und \href{http://gams.uni-graz.at/o:konde.6}{Langzeitarchivierung} von Forschungsdaten im
                  Allgemeinen und \href{http://gams.uni-graz.at/o:konde.59}{Digitalen
                     Editionen} im Speziellen. Die Softwarelösung auf Open Source-Basis
                  orchestriert die Funktionalitäten einer Vielzahl von einschlägigen
                  Softwareprojekten zu einem Ecosystem für geisteswissenschaftliche Forschungsdaten.
                  Dazu wurde am \href{http://gams.uni-graz.at/o:konde.217}{Zentrum für
                     Informationsmodellierung} der Universität Graz eine Client-Komponente
                  entwickelt, die intelligente Masseningest- und Verwaltungsabläufe unterstützt. Die
                  Entwicklung von GAMS orientiert sich an den Prämissen der \href{http://gams.uni-graz.at/o:konde.7}{FAIR}-Prinzipien für das Forschungsdatenmanagement und
                  stellt sich dem Anspruch nicht nur der nachhaltigen Bereitstellung von
                  (Text-)Daten, sondern auch der dauerhaften Erhaltung der darauf basierenden
                  Prozesse. So tritt neben den konservatorischen Fokus im Planungsansatz auch das
                  Bemühen um die Sicherstellung einer nachhaltigen Verfügbarkeit der
                  Funktionalitäten rund um die gespeicherten Texte und Quellenmaterialien. \\
            
        GAMS wurde 2014 erstmalig und 2018 wieder als vertrauenswürdiges digitales Archiv
                  nach den Richtlinien des \emph{Core Trust Seals} zertifiziert.
                  Neben einer Vielzahl von Partnerinstitutionen nutzt auch die \href{http://gams.uni-graz.at/o:konde.153}{Österreichische Nationalbibliothek} dieses Framework
                  zur Publikation von Digitalen Editionen, die in ihrem Hause entstehen. Es ist frei
                  nach dem Open Source-Prinzip verfügbar und als DARIAH-Beitrag Österreichs
                  registriert.\\
            
        Die ersten Objekte wurden 2004 unter GAMS erstellt. Das Repositorium enthält
                  derzeit über 100.000 digitale Objekte, die in rund 80 verschiedenen
                  wissenschaftlichen, Drittmittel-geförderten Koope­rationsprojekten entstanden
                  sind. 2019 wurde damit begonnen, dem technischen Wandel dieses Zeitraums
                  entsprechend, die Verfügbarkeit dieser Inhalte für die nächsten 15 Jahre
                  sicherzustellen. Ein internes Migrationsprojekt stellt sich der Aufgabe, den
                  Technologiestack des Repositoriums so zu erneuern und \emph{plug-and-play} auszutauschen, dass an den Interface-Komponenten der
                  einzelnen Projekte keinerlei manuelle Adaptionsarbeiten notwendig wer­den. Diese
                  neue Version von GAMS unterstützt eine clusterbasierte Systemarchitektur auf Basis
                  von Docker und Kubernetes, bietet auch clientseitig viele neue Funktionalitäten
                  und wird Ende 2020 verfügbar sein.\\
            
        Portal: https://gams.uni-graz.at\\
            
        Technologien: \href{http://gams.uni-graz.at/o:konde.69}{FEDORA Commons},
                  Apache Tomcat, Apache Solr, Apache Cocoon, Blazegraph (Graph Database), LORIS
                     (\href{http://gams.uni-graz.at/o:konde.123}{IIIF} Imageserver), Proai
                  OAI-PMH-Provider, LaTeX u.a \\
            
        Unterstützte Standards: \href{http://gams.uni-graz.at/o:konde.178}{TEI}, MEI,
                  LIDO, BIBTEX, \href{http://gams.uni-graz.at/o:konde.131}{RDF}, \href{http://gams.uni-graz.at/o:konde.132}{SKOS}, \href{http://gams.uni-graz.at/o:konde.131}{OWL}, \href{http://gams.uni-graz.at/o:konde.129}{METS}, MODS uvm.\\
            
        Dokumentation: https://gams.uni-graz.at/doku\\
            
        Download: https://github.com/acdh/cirilo\\
            
        PID-Service: https://hdl.handle.net\\
            
        Zertifizierung: CoreTrustSeal\\
            
        \subsection*{Literatur:}\begin{itemize}\item Stigler, Johannes; Steiner, Elisabeth: GAMS – Eine Infrastruktur zur Langzeitarchivierung und
                              Publikation geisteswissenschaftlicher Forschungsdaten. In: Mitteilungen der Vereinigung Österreichischer
                              Bibliothekarinnen und Bibliothekare 71: 2018, S. 207–216.\end{itemize}\subsection*{Software:}\href{https://duraspace.org/fedora/}{Fedora}, \href{http://gams.uni-graz.at/archive/objects/o:gams.doku/methods/sdef:TEI/get?locale=de}{GAMS}, \href{https://iiif.io/}{iiif}, \href{https://www.blazegraph.com/}{BlazeGraph}, \href{http://lucene.apache.org/solr/}{Solr}, \href{https://www.cis.uni-muenchen.de/~schmid/tools/TreeTagger/}{TreeTagger}, \href{https://www.w3.org/RDF/}{RDF}\subsection*{Verweise:}\href{https://gams.uni-graz.at/o:konde.6}{Digitale Nachhaltigkeit}, \href{https://gams.uni-graz.at/o:konde.11}{OAIS RM}, \href{https://gams.uni-graz.at/o:konde.123}{IIIF}, \href{https://gams.uni-graz.at/o:konde.10}{Metadata-Harvesting}, \href{https://gams.uni-graz.at/o:konde.7}{FAIR-Prinzipien}, \href{https://gams.uni-graz.at/o:konde.217}{Zentrum für Informationsmodellierung}, \href{https://gams.uni-graz.at/o:konde.227}{Universität Graz}\subsection*{Themen:}Archivierung, Institutionen\subsection*{Zitiervorschlag:}Stigler, Johannes. 2021. Disseminations-Services: Geisteswissenschaftliches Asset Management
               System (GAMS). In: KONDE Weißbuch. Hrsg. v. Helmut W. Klug unter Mitarbeit von Selina Galka und Elisabeth Steiner im HRSM Projekt "Kompetenznetzwerk Digitale Edition". URL: https://gams.uni-graz.at/o:konde.70\newpage\section*{Disseminations-Services: dhPLUS} \emph{Hörmann, Richard; richard.hoermann@sbg.ac.at}\\
        
    dhPLUS ist eine Plattform der \href{http://gams.uni-graz.at/o:konde.203}{Universität Salzburg} (Paris Lodron Universität Salzburg: PLUS), die die \href{http://gams.uni-graz.at/o:konde.6}{Langzeitarchivierung} (LZA) von Digital Humanities(DH)-Projekten sicherstellen soll. Sie ist vorzugsweise für Projekte der Universität Salzburg gedacht, steht aber darüber hinaus der nationalen und internationalen DH-Community und ihren Projekten offen.   	        	\\
            
        dhPLUS ist seit 1.1.2020 in Betrieb; der derzeit stattfindende Aufbau ist der Beitrag der PLUS zum KONDE-Projekt, aus deren Mitteln die Stelle eines \emph{Repository Developers} finanziert wird.\\
            
        dhPLUS war von Beginn an als ein Service der IT-Abteilung der PLUS (ITS) geplant, d. h. dass nach dem Releasedatum die für den Betrieb von dhPLUS zuständigen Stellen nicht aus Drittmitteln, sondern wie jedes andere Service des ITS aus Mitteln der Universität finanziert werden. Als Service des ITS nutzt dhPLUS auch die dort vorhandene IT-Infrastruktur, insbesondere die ORACLE-DB mit dem integrierten Archivsystem.\\
            
        Der Aufbau der Plattform entspricht dem \href{http://gams.uni-graz.at/o:konde.11}{OAIS-Modell}: Die Einspeisung der Daten in das Archivsystem erfolgt beim Ingest großer Datenmengen automatisiert, die Eingabe und die Bearbeitung einzelner Datensätze geschieht über Eingabemasken. Die Daten sind im Archivsystem in Digitalen Objekten abgelegt, die einen eindeutigen und persistenten Identifier (\href{http://gams.uni-graz.at/o:konde.12}{PID}) haben und grundlegende \href{http://gams.uni-graz.at/o:konde.25}{Metadaten} über das Digitale Objekt und über die darin enthaltenen Datensätze strukturiert und standardisiert bereitstellen. Jede Änderung an den Daten wird im Archivsystem versioniert. Der User-Zugang zu den Daten erfolgt in der Dissemination einerseits über eine Konvertierung der Daten in Standard-Formate wie .html, .pdf oder .xml, andererseits in der Bereitstellung von Applikationen, mit denen der User verschiedene Funktionen über den Daten ausführen kann. Angepasst an den generischen Aufbau von DH-Projekten wird nicht nur wie im OAIS-Modell die Datenschicht, sondern auch die Funktionsschicht archiviert. Für die LZA wichtig ist auch die dritte Art von Digitalen Objekten, die die (technischen und wissenschaftlichen) Dokumentationen eines Projektes beinhalten.\\
            
        dhPLUS ist als Plattform auf Kontinuität und Stabilität hin ausgelegt. Zugleich sind die Informationstechnologien (IT) ein Bereich, der zu den kompetitivsten und innovativsten gehört. Es ergibt sich die Herausforderung, den Betrieb der Plattform auf Dauer zu gewährleisten und zugleich mit der rasanten technischen Entwicklung Schritt zu halten. Bei begrenzten finanziellen und personalen Ressourcen gelingt das nur mit einer weitgehenden Standardisierung der eingesetzten IT. Die diesbezügliche Strategie von dhPLUS ist, so weit wie möglich international anerkannte Standards zu übernehmen und nur dort, wo dies nicht der Fall ist, auf weniger allgemeine Lösungen zurückzugreifen. Proprietäre Entwicklungen sind als Extremfall möglichst zu vermeiden.\\
            
        Standardisierung wird auf verschiedenen Ebenen umgesetzt: In Bezug auf die Dateiformate werden alle Daten nach dem Ingest im Format .xml im Archivsystem abgelegt. Ausnahmen sind Projektdaten, die nicht in .xml, sondern nur in PDF/A gesichert werden können, und Pre-Ingest-Daten, die für die Rekonstruktion der Archivdaten nützlich sein können.\\
            
        Eine weitere Vereinheitlichung betrifft die Auslegung von dhPLUS als \emph{\href{http://gams.uni-graz.at/o:konde.8}{Linked Data}}-Platform (LDP). Das bedeutet, dass alle im Archivsystem befindlichen Daten und Metadaten, die zur funktionellen Weiterverarbeitung verwendet werden, in \href{http://gams.uni-graz.at/o:konde.131}{RDF} modelliert sind, wobei RDF in \href{http://gams.uni-graz.at/o:konde.215}{XML} serialisiert ist. RDF ist ein Standard-Framework des \emph{\href{http://gams.uni-graz.at/o:konde.167}{Semantic Web}} und die Voraussetzung für LDP. Mit LDP ist es möglich, die Daten- und Metadatensätze der Digitalen Objekte via HTTP-\emph{requests} zugänglich zu machen.\\
            
        In RDF ist auch die weitere Content-Modellierung der Daten und Metadaten ausgeführt. Die Datensätze der bibliographischen, biographischen und topographischen Metadaten werden in eine CIDOC-Struktur eingefügt und mit \href{http://gams.uni-graz.at/o:konde.147}{Normdaten} aus \href{http://gams.uni-graz.at/o:konde.131}{OWL}-\href{http://gams.uni-graz.at/o:konde.151}{Ontolologien} der \href{http://gams.uni-graz.at/o:konde.109}{GND} und der \href{http://gams.uni-graz.at/o:konde.112}{WikiData} gespeist. Die bibliographischen Metadaten werden in BIBFRAME modelliert, dem sich abzeichnenden Nachfolgestandard für das derzeit in den Bibliothekssystemen vorherrschende MARC21-XML-Format. Die MARC-Datensätze werden über Schnittstellen automatisiert aus den Bibliothekssystemen bezogen, nach BIBFRAME konvertiert und abgelegt. Über diese Schnittstellen geschieht auch ein automatisierter Abgleich der Datensätze. Gleiches gilt für die GND- und WikiData-Normdaten.\\
            
        Zur Validierung der RDF-Daten wird SHACL auf dhPLUS implementiert. SHACL ist ein 2017 von der W3C herausgegebener Standard, der das Problem löst, dass es für RDF-Daten bisher kein mit dem XML-\href{http://gams.uni-graz.at/o:konde.166}{Schema} vergleichbares Regelwerk gab und OWL dafür nur bedingt geeignet war. Mit SHACL können Datensätze auf Regelkonformität geprüft, neue RDF-\emph{triples} erzeugt und SPARQL-\emph{snippets} dort eingefügt werden, wo SHACL nicht ausreicht. Auf dhPLUS werden bestehende OWL-Ontologien nicht durch SHACL ersetzt, sondern in eine Kombination mit SHACL-Dateien integriert.\\
            
        Die meisten DH-Projekte weisen Volltexte auf, die in TEI-XML modelliert sind. Auf dhPLUS werden diese archiviert und mit einer RDF-Repräsentation ergänzt. Zugleich werden alle TEI-Elemente und ihre RDF-Repräsentanten mit eindeutigen xml-IDs getaggt, so dass in den RDF-Statements die Informationen über die TEI und die ID enthalten sind. Annotationen werden über das \emph{Web Annotation Vocabulary} mit den Statements verbunden. Das Resultat ist ein flexibles Annotationstool, mit dem ein \emph{\href{http://gams.uni-graz.at/o:konde.171}{Stand off-Markup}} realisiert werden kann, indem (theoretisch) beliebig viele Annotationsebenen mit den RDF-Repräsentanten der TEI-Elemente verbunden werden.\\
            
        In der technischen Umsetzung ist dhPLUS als modulares System realisiert. Eine Folge davon ist, dass die ORACLE DB des ITS, an die die Plattform angebunden ist, kein systemimmanenter Teil von dhPLUS ist, sondern die Verbindung über ein Modul geschieht, das ersetzt werden kann, wenn die ORACLE DB durch eine andere Infrastruktur abgelöst wird. Gleiches gilt für die \href{http://gams.uni-graz.at/o:konde.69}{FEDORA}-Architektur, die von den großen österreichischen Repositorien verwendet wird. dhPLUS ist kein FEDORA-Repositorium, es emuliert aber mit Hilfe von Modulen die FEDORA-Architektur, um die Zusammenarbeit und den Austausch mit den Repositorien der anderen österreichischen Universitäten zu erleichtern.\\
            
        Kern des modularen Aufbaus von dhPLUS ist eine Microservice-Architektur. Prozesse des Ingest, der Archivierung und der Dissemination werden über Microservices abgewickelt. Jedes Microservice läuft in einem eigenen Docker-Container, der bei einer Fehlfunktion nur dieses eine Service, aber nicht die Plattform insgesamt zum Shutdown bringt. Ein Fehler-Monitoring-System kann die Fehlfunktion entdecken, das Service kontrolliert abschalten und als Ersatz ein anderes Service starten. Die Module sind skalierbar: Arbeitet ein Modul eine Aufgabe ab und tritt ein \emph{request} nach der gleichen Aufgabe ein, kann ein weiteres Modul gestartet werden, das die gleiche Aufgabe übernimmt. Entspricht ein Microservice nicht mehr dem Stand der Technik oder muss aus einem anderen Grund ersetzt werden, kann das neue Modul im Testsystem geprüft und bei Erfolg in das Produktivsystem übernommen werden, ohne dass der Betrieb unterbrochen werden muss.\\
            
        Der unterbrechungsfreie Betrieb bei gleichzeitiger Übernahme innovativer Entwicklungen ist ein Vorteil des modularen Aufbaus der Plattform und ein weiterer wichtiger Baustein für die Realisierung einer LZA von DH-Projekten. Vom Standpunkt des dhPLUS-Teams aus ist eine Langzeitarchivierung, die dadurch gewährleistet wird, dass es an einem Standort eine Plattform gibt, auf der Projekte dauerhaft serviciert werden, eine wichtige Voraussetzung, aber sie reicht nicht aus. Ein unwahrscheinlicher, aber möglicher Ausfall dieses Standortes und der Plattform würde den Verlust der Projekte und ihrer Daten bedeuten. Um einen solchen Fall zu verhindern, ist ein \emph{deployment} der Plattform erforderlich, das in Zukunft etwa durch eine österreichweite Aufteilung der Plattform auf die verschiedenen Standorte erreicht werden könnte. Ein erster Schritt in diese Richtung könnte die Entwicklung einer Austauschontologie zwischen den Repositorien Österreichs sein, deren Konzeption im KONDE-Projekt bereits initiiert wurde.\\
            
        \subsection*{Literatur:}\begin{itemize}\item Hörmann, Richard; Schlager, Daniel: Saving Digital Humanities. In: digital humanities austria 2018. empowering researchers. Wien.\end{itemize}\subsection*{Software:}\href{https://jena.apache.org/}{Apache Jena}, \href{https://getbootstrap.com/}{Bootstrap}, \href{http://www.dnb.de/DE/Standardisierung/GND/gnd_node.html}{GND}, \href{http://www.handle.net/}{Handle}, \href{https://iiif.io/}{iiif}, \href{https://www.latex-project.org/}{latex}, \href{https://www.w3.org/RDF/}{RDF}, \href{http://lucene.apache.org/solr/}{Solr}, \href{https://www.wikidata.org/wiki/Wikidata:Main_Page}{Wikidata}\subsection*{Verweise:}\href{https://gams.uni-graz.at/o:konde.11}{OAIS}, \href{https://gams.uni-graz.at/o:konde.167}{Semantic Web}, \href{https://gams.uni-graz.at/o:konde.171}{stand off markup}, \href{https://gams.uni-graz.at/o:konde.52}{Datenmodell MHDBDB}, \href{https://gams.uni-graz.at/o:konde.5}{DHA-Ontologie}, \href{https://gams.uni-graz.at/o:konde.6}{Digitale Nachhaltigkeit}, \href{https://gams.uni-graz.at/o:konde.70}{GAMS}, \href{https://gams.uni-graz.at/o:konde.69}{Fedora}\subsection*{Themen:}Archivierung, Institutionen\subsection*{Zitiervorschlag:}Hörmann, Richard. 2021. Disseminations-Services: dhPLUS. In: KONDE Weißbuch. Hrsg. v. Helmut W. Klug unter Mitarbeit von Selina Galka und Elisabeth Steiner im HRSM Projekt "Kompetenznetzwerk Digitale Edition". URL: https://gams.uni-graz.at/o:konde.68\newpage\section*{Distant Reading, Close Reading, Scalable Reading} \emph{Zeppezauer-Wachauer, Katharina; katharina.wachauer@sbg.ac.at }\\
        
    “Difficult; but too interesting not to give it a try.”(Franco Moretti 2013,
                     S. 165)\\
            
        Der mittlerweile zum Standard­repertoire der Digital Humanities zählende Begriff
                  des \emph{Distant Reading} wurde erstmals im Jahr 2000 von Franco
                  Moretti verwendet. Ursprünglich war er von Moretti als pointierte Polemik gegen
                  das \emph{Close Reading} konzipiert worden (“we know how to read
                  texts, now let’s learn how not to read them“ (Moretti 2000, S. 57)),
                  weil sich ein Vorurteil der Geistes­wissenschaften hartnäckig hielt und hält: dass
                  wissenschaftliche Analy­sen, die mithilfe des Computers generiert würden, von
                  minderem Wert seien. Primäres Ziel des Zugangs von Moretti ist es zum einen,
                  althergebrachte kanonische Überlegungen und Theorien unter Berücksichtigung einer
                  größeren Anzahl an Vergleichsdaten zu überprüfen, zum anderen sollen textimmanente
                  Gegebenheiten mittels quantitativer Erhebung neu bewertet werden. Auch Matthew L.
                  Jockers beschreibt in seinem Buch \emph{Macroanalysis}, “how a new
                  method of studying large collections of digital material can help us to understand
                  and contextualize the individual works within those collections\emph{”}(Jockers 2013, S. 32).\\
            
        Weder Moretti noch Jockers noch die gegenwärtigen Digital Humanities wollen die
                  sorgfältige Lektüre, das \emph{Close Reading}, ersetzen. Vielmehr
                  ist es ihre Absicht, einen zusätzlichen Blick auf Sachverhalte zu gewähren, die
                  ohne eine objektive ‘Draufsicht‘ auf die Quellen vielleicht nie erkannt werden
                  wür­den.\\
            
        Zentral für die interdisziplinäre Vernetzung von Daten und Forschenden ist es, die
                  quantitative Methode des \emph{Distant Reading} auch für andere,
                  nicht textbasierte Disziplinen gangbar zu machen. Überlegungen zum Bereich der
                  Bild- und Kunstwissenschaften liefert etwa Isabella Nicka, die analog dazu die
                  Methode des \emph{Distant Viewing} postuliert und den \emph{Distant Reading}-Ansatz auch für Untersuchungen von Bildern
                  anregt. (Nicka 2019, S. 100–103)\\
            
        Dem Switchen zwischen hermeneutischen und empirischen Verfahren wurde durch Martin
                  Mueller programmatischer Charakter verliehen. Mueller fordert dazu auf, zwischen
                  Nah- und Fernsicht, zwischen Mikro- und Makroanalyse, zwischen \emph{Close}- und \emph{Distant Reading} umzuschalten, wann
                  immer es notwendig ist, und die neuen quantitativen Methoden mit erprob­ten
                  hermeneutischen zu kombinieren. Er bezeichnet diese Strategie als \emph{Scalable Reading}. (Mueller 2010)\\
            
        \subsection*{Literatur:}\begin{itemize}\item Jockers, Matthew Lee: Macroanalysis. Digital methods and literary
                              history. Urbana, Chicago: 2013.\item Moretti, Franco: Conjectures on World Literature. In: New Left Review 1: 2000, S. 54–68.\item Moretti, Franco: Distant Reading. London, New York: 2013.\item Scalable Reading dedicated to DATA: digitally assisted
                              text analysis. URL: \url{}\item Nicka, Isabella: Object Links in/zu Bildern mit REALonline
                              analysieren. In: Object Links – Dinge in Beziehung (Formate – Forschungen
                              zur Materiellen Kultur. Wien: 2019, S. 95–126.\end{itemize}\subsection*{Software:}\href{https://github.com/philkon/InCritApp}{InCritApp}, \href{https://code.google.com/archive/p/topic-modeling-tool/}{topic-modelling-tool}, \href{https://www.nltk.org/}{Natural Language Toolkit
                           (nltk)}, \href{http://cltk.org/}{Classical Language Toolkit
                           (cltk)}\subsection*{Verweise:}\href{https://gams.uni-graz.at/o:konde.16}{Analysemethoden}, \href{https://gams.uni-graz.at/o:konde.145}{NLP}, \href{https://gams.uni-graz.at/o:konde.194}{Textmining}, \href{https://gams.uni-graz.at/o:konde.141}{NER}, \href{https://gams.uni-graz.at/o:konde.74}{Dramennetzwerkanalyse}\subsection*{Themen:}Datenanalyse, Natural Language Processing\subsection*{Zitiervorschlag:}Zeppezauer-Wachauer, Katharina. 2021. Distant Reading, Close Reading, Scalable Reading. In: KONDE Weißbuch. Hrsg. v. Helmut W. Klug unter Mitarbeit von Selina Galka und Elisabeth Steiner im HRSM Projekt "Kompetenznetzwerk Digitale Edition". URL: https://gams.uni-graz.at/o:konde.71\newpage\section*{Documentary Editing} \emph{Vogeler, Georg; georg.vogeler@uni-graz.at}\\
        
    \emph{Documentary Editing} ist ein in den USA entwickeltes
                  Editionsverfahren, das zwar häufig in \href{http://gams.uni-graz.at/o:konde.160}{Quelleneditionen} verwendet wird, aber eine davon unabhängige Logik
                  entwickelt. Das Editionsverfahren bevorzugt die Dokumentation des textlichen
                  Befundes des originalen Textzeugen. Thomas Tanselle ist der Begründer des
                  Verfahrens, das sich von den auswählenden Verfahren (\emph{Eclectic
                     Edition}) der \href{http://gams.uni-graz.at/o:konde.43}{Copy-Text-Theorie} ebenso wie von \href{http://gams.uni-graz.at/o:konde.146}{normalisierenden} und modernisierenden Verfahren
                  abgrenzte (Tanselle 1978). Es legt besonderen Wert darauf, konsequent
                  Orthographie, Grammatik, Groß- und Kleinschreibung, Abkürzungen, Streichungen und
                  Interpunktion jedes einzelnen Textzeugen zu erhalten. Stillschweigende
                  Emendationen mißachten in dieser Logik die Erkenntnisse, die sich aus unsicherer
                  Orthographie oder kreativem Gebrauch von Interpunktion über die Autoren ziehen
                  lassen. Die jüngste Ausgabe des US-amerikanischen \emph{Guide to
                     Documentary Edition}(Kline/Perdue 2013) schlägt für viele paläographische Phänomene eine
                  symbolische Notation vor und für die verbliebenen die Verwendung von \href{http://gams.uni-graz.at/o:konde.17}{Annotationen}. Dieser \emph{Guide} ist jedoch weniger radikal als die von Tanselle und
                  von Theoretikern wie Jerome McGann weiterentwickelte Editionstheorie (McGann
                     1983), so dass heute die Bezeichnung \emph{Documentary
                     Editing} jede Edition von textlichen ‘Überresten’ im Sinne der
                  Quellentypologie Johann Gustav Droysens (1937, S. 38–50) meint, die
                  besonderen Wert auf die präzise Wiedergabe des linguistischen und paläographischen
                  Befundes legt (vgl. dazu auch \href{http://gams.uni-graz.at/o:konde.66}{diplomatische Transkription}).\\
            
        \subsection*{Literatur:}\begin{itemize}\item Tanselle, G. Thomas: The Editing of Historical Documents. In: Studies in Bibliography 31: 1978, S. 1-56.\item Rosenberg, Bob: Documentary Editing. In: Electronic textual editing. New York: 2006, S. 92–104.\item Kline, Mary Jo; Perdue, Susan: A Guide to Documentary Editing. Baltimore, London: 2013, URL: \url{https://gde.upress.virginia.edu/00C-gde.html}.\item McGann, Jerome J: A Critique of Modern Textual Criticism: 1983, URL: \url{http://books.google.be/books?hl=en&lr=&id=_EKipG36yiAC&oi=fnd&pg=PR9&dq=a+critique+of+modern+textual+criticism&ots=7Y7BJCrn3d&sig=YQZMVRwkOQ5FF_YKzmk6R-jp2fE}.\end{itemize}\subsection*{Software:}\href{http://evt.labcd.unipi.it/}{EVT}, \href{http://oxygenxml.com/}{Oxygen}, \href{https://github.com/oxygenxml/TEI-Facsimile-Plugin}{Oxygen-TEI-Facsimile-Plugin}, \href{http://scripto.org/}{scripto}, \href{http://t-pen.org/TPEN/}{T-Pen}, \href{https://textgrid.de/}{TextGrid}, \href{https://www.textlab.org/about/}{TextLab}, \href{https://textualcommunities.org/app/}{Textual
                           Communities}, \href{http://transcribo.org/en/}{Transcribo}, \href{https://transkribus.eu/Transkribus/}{Transkribus}, \href{http://www.tustep.uni-tuebingen.de/}{TUSTEP}, \href{https://docs.google.com/document/d/1QsFodbmuOld4ZAmnURR2tKewE1tgRo1cGxpaIUy92Mw/edit}{EditMOM3}, \href{http://wlt.synat.pcss.pl/}{Virtual
                           Transcription Laboratory}\subsection*{Verweise:}\href{https://gams.uni-graz.at/o:konde.65}{Diplomatische Edition}, \href{https://gams.uni-graz.at/o:konde.66}{diplomatische Transkription}, \href{https://gams.uni-graz.at/o:konde.197}{Transkription}, \href{https://gams.uni-graz.at/o:konde.198}{Transkriptionsrichtlinien}, \href{https://gams.uni-graz.at/o:konde.199}{Transkriptionswerkzeuge}\subsection*{Themen:}Digitale Editionswissenschaft\subsection*{Lexika}\begin{itemize}\item \href{https://edlex.de/index.php?title=Documentary_editing}{Edlex: Editionslexikon}\item \href{https://wiki.helsinki.fi/display/stemmatology/Edition%2C+documentary}{Parvum Lexicon Stemmatologicum}\item \href{https://lexiconse.uantwerpen.be/index.php/lexicon/documentary-editing/}{Lexicon of Scholarly Editing}\end{itemize}\subsection*{Zitiervorschlag:}Vogeler, Georg. 2021. Documentary Editing. In: KONDE Weißbuch. Hrsg. v. Helmut W. Klug unter Mitarbeit von Selina Galka und Elisabeth Steiner im HRSM Projekt "Kompetenznetzwerk Digitale Edition". URL: https://gams.uni-graz.at/o:konde.72\newpage\section*{Dramennetzwerk} \emph{Geiger, Bernhard C.; geiger@ieee.org }\\
        
    Ein Dramennetzwerk ist ein \href{http://gams.uni-graz.at/o:konde.144}{Netzwerk}, dessen Knotenmenge die Menge (bzw. eine Obermenge, in einem bipartiten Netzwerk) der Figuren eines Dramas ist. Wie in jedem anderen Netzwerk stellen die Kanten die Relationen zwischen den Figuren dar. Interessant ist, dass ein Drama auf verschiedene Arten als Netzwerk dargestellt werden kann (Trilcke 2013, Sec. II.2):\\
            
        \begin{itemize}\item {Als einfaches Netzwerk: Z. B. können Kanten zwischen zwei Figuren darstellen, dass diese Figuren gleichzeitig gemeinsam auf der Bühne standen bzw. in der selben Szene aufgetreten sind.}\item {Als gewichtetes Netzwerk: Z. B. kann das Kantengewicht proportional zur Anzahl der Szenen sein, in denen die Figuren gemeinsam aufgetreten sind.}\item {Als bipartites Netzwerk: Z. B. kann eine Kante zwischen einer Szene und einer Figur darstellen, dass diese Figur in jener Szene aufgetreten ist.}\item {Als gerichtetes Netzwerk: Z. B. kann eine gerichtete Kante von Figur A nach Figur B bedeuten, dass A mit B sprach bzw. dass B nach A einen Sprechakt hatte.}\item {Als signiertes Netzwerk: Z. B. kann eine positiv (negativ) gewichtete Kante von Figur A nach Figur B darstellen, dass A mit B in einem Freundschaftsverhältnis steht (verfeindet ist). (Nalisnick; Baird 2013)}\end{itemize}Dramennetzwerke dienen nicht nur der graphischen Darstellung eines Dramas, sondern ermöglichen auch eine quantitative Analyse (\href{http://gams.uni-graz.at/o:konde.74}{Dramennetzwerkanalyse}). Ferner können sie sowohl manuell (durch \emph{\href{http://gams.uni-graz.at/o:konde.71}{close reading}}, Moretti 2011) als auch automatisiert (z. B. durch XML-\emph{parsing}) erzeugt werden. Verständlicherweise liefern diese beiden Ansätze durchaus unterschiedliche Netzwerke. Inwieweit die automatisierte Erzeugung eines Dramennetzwerkes sich auf die Ergebnisse der Dramennetzwerkanalyse auswirkt, ist Gegenstand aktueller Forschung. (Fischer et al. 2018, Klimashevskaia et al. 2020)\\
            
        \subsection*{Literatur:}\begin{itemize}\item Nalisnick, Eric T.; Baird, Henry S.: Extracting Sentiment Networks from Shakespeare's Plays. In: Proc. 12th International Conference on Document Analysis and Recognition 12th International Conference on Document Analysis and Recognition (ICDAR). Washington, DC, USA: 2013, S. 758–762.\item Moretti, Franco: Network Theory, Plot Analysis. In: Literary Lab Pamphlet 2: 2011.\item Trilcke, Peer: Social Network Analysis (SNA) als  Methode einer  extempirischen Literaturwissenschaft. In: Empirie in der Literaturwissenschaft. Münster: 2013, S. 201–247.\item Fischer, Frank; Trilcke, Peer; Kittel, Christopher; Milling, Carsten; Skorinkin, Daniil: To Catch a Protagonist: Quantitative Dominance Relations in German-Language Drama (1730–1930). In: Digital Humanities. Mexico City: 2018.\item . In: “To be or not to be central” – On the Stability of Network Centrality Measures in Shakespeare’s “Hamlet”.\end{itemize}\subsection*{Software:}\href{https://d3js.org}{D3js}, \href{https://gephi.org/}{Gephi}, \href{https://nodegoat.net/}{Node Goat}, \href{https://public.tableau.com/s/}{Tableau}\subsection*{Verweise:}\href{https://gams.uni-graz.at/o:konde.144}{Netzwerk}, \href{https://gams.uni-graz.at/o:konde.74}{Dramennetzwerkanalyse}, \href{https://gams.uni-graz.at/o:konde.71}{close reading}, \href{https://gams.uni-graz.at/o:konde.16}{Analysemethoden}\subsection*{Themen:}Datenanalyse, Digitale Editionswissenschaft\subsection*{Zitiervorschlag:}Geiger, Bernhard C. 2021. Dramennetzwerk. In: KONDE Weißbuch. Hrsg. v. Helmut W. Klug unter Mitarbeit von Selina Galka und Elisabeth Steiner im HRSM Projekt "Kompetenznetzwerk Digitale Edition". URL: https://gams.uni-graz.at/o:konde.73\newpage\section*{Dramennetzwerkanalyse} \emph{Geiger, Bernhard C.; geiger@ieee.org }\\
        
    In der Dramennetzwerkanalyse werden Methoden der Netzwerkanalyse auf \href{http://gams.uni-graz.at/o:konde.73}{Dramennetzwerke} angewendet. Konkret
                  werden gewisse Kenngrößen des Netzwerks als ganzes bzw. einzelner Knoten
                  (Figuren) und Kanten (Beziehungen zwischen Figuren) berechnet, gegenübergestellt
                  und zur Beantwortung literaturwissenschaftlicher Fragestellungen
                  operationalisiert.\\
            
        Wichtige Kenngrößen (siehe u. a. Thurner et al. 2018, Kap. 4) eines
                  Dramennetzwerks sind, neben der Anzahl seiner Figuren und Beziehungen, z. B. die
                  Dichte, die das Verhältnis zwischen der Anzahl der Figuren und der Anzahl der
                  Beziehungen charakterisiert, und der Durchmesser, also die Distanz zwischen den am
                  weitesten voneinander entfernten Figuren. Für einzelne Figuren kann man z. B. den
                  Knotengrad, also die Anzahl der mit ihr in Beziehung stehenden Figuren, den
                  Clusteringkoeffizienten, welcher beschreibt, wie stark die mit der jeweiligen
                  Figur in Beziehung stehenden Figuren ihrerseits untereinander in Beziehung stehen,
                  sowie diverse Zentralitätsmaße – z. B. beschreibt die \emph{betweenness centrality}, wie stark eine Figur als Mittler zwischen anderen
                  Figuren auftritt – berechnen. Diese figurenspezifischen Kenngrößen kann man erneut
                  mitteln bzw. deren Verteilung analysieren. So haben z. B. \emph{small
                     worlds} einen hohen mittleren Clusteringkoeffizienten und eine kurze
                  mittlere Distanz zwischen zwei Figuren (Thurner et al. 2018, Kap. 4.5.3;
                     Trilcke et al. 2016). Schließlich kann man in Dramennetzwerken Gruppen
                  stark miteinander interagierender Figuren detektieren; das Stichwort dazu lautet
                     \emph{community detection}(Geiger/Amjad 2017).\\
            
        In der wissenschaftlichen Analyse werden die automatisiert gewonnenen
                  Charakteristika (Kenngrößen, Community-Struktur bzw. Vorliegen der \emph{small world}-Eigenschaft) innerhalb eines größeren Korpus
                  miteinander (Trilcke et al. 2016, Fig. 1) oder mit den
                  Charakteristika realer sozialer Netzwerke verglichen bzw. vor einem
                  literaturwissenschaftlichen Hintergrund interpretiert (Moretti 2011).
                  Hinsichtlich der Beantwortung literaturwissenschaftlicher Fragestellungen bzw. des
                  explorativen Werts der Analyse von aus Dramen und Romanen gewonnen Daten sei z. B.
                  auf Sec. IV in Trilcke, 2013 verwiesen. Exemplarisch kann angeführt werden, dass die
                  Dramennetzwerkanalyse durchaus automatisiert zwischen der offenen und
                  geschlossenen Dramenform unterscheiden kann; so haben die ‘einfachen Netzwerke’
                     (\href{http://gams.uni-graz.at/o:konde.73}{Dramennetzwerk}) geschlossener
                  Dramen eine höhere Dichte und eine niedrigere Standardabweichung über die
                  Knotengrade ihrer Figuren (Trilcke 2013, Sec. III). Ferner wurde in
                  Fischer et al., 2018 gezeigt, dass die aus dem Netzwerk berechnete relative
                  Zentralität einer Figur mit ihrer relativen Bühnen(sprech)präsenz korreliert.\\
            
        \subsection*{Literatur:}\begin{itemize}\item Thurner, Stefan; Hanel, Rudolf; Klimek, Peter: Introduction to the Theory of Complex Systems. Oxford, New York: 2018.\item Moretti, Franco: Network Theory, Plot Analysis. In: Literary Lab Pamphlet 2: 2011.\item Trilcke, Peer: Social Network Analysis (SNA) als Methode einer
                              extempirischen Literaturwissenschaft. In: Empirie in der Literaturwissenschaft. Münster: 2013, S. 201–247.\item Fischer, Frank; Trilcke, Peer; Kittel, Christopher; Milling, Carsten; Skorinkin, Daniil: To Catch a Protagonist: Quantitative Dominance Relations
                              in German-Language Drama (1730–1930). In: Digital Humanities. Mexico City: 2018.\item Trilcke, Peer; Fischer, Frank; Göbel, Mathias; Kampkaspar, Dario: Theatre Plays as 'Small Worlds'? Network Data on the
                              History and Typology of German Drama, 1730–1930. In: Digital Humanities. Krakow: 2016, S. 385–387.\item Geiger, Bernhard C.; Amjad, Rana Ali: Community Detection in Shakespeare’s Plays. Classic and
                              Information-Theoretic Approaches. Potsdam: 2017.\end{itemize}\subsection*{Software:}\href{https://d3js.org}{D3js}, \href{https://gephi.org/}{Gephi}, \href{https://nodegoat.net/}{Node Goat}, \href{https://public.tableau.com/s/}{Tableau}\subsection*{Verweise:}\href{https://gams.uni-graz.at/o:konde.144}{Netzwerk}, \href{https://gams.uni-graz.at/o:konde.73}{Dramennetzwerk}, \href{https://gams.uni-graz.at/o:konde.71}{close reading}, \href{https://gams.uni-graz.at/o:konde.16}{Analysemethoden}\subsection*{Themen:}Datenanalyse, Digitale Editionswissenschaft\subsection*{Zitiervorschlag:}Geiger, Bernhard C. 2021. Dramennetzwerkanalyse. In: KONDE Weißbuch. Hrsg. v. Helmut W. Klug unter Mitarbeit von Selina Galka und Elisabeth Steiner im HRSM Projekt "Kompetenznetzwerk Digitale Edition". URL: https://gams.uni-graz.at/o:konde.74\newpage\section*{Dublin Core Metadata Initiative (DCMI)} \emph{Pollin, Christopher; christopher.pollin@uni-graz.at }\\
        
    DCMI ist eine Organisation, die Innovationen im \href{http://gams.uni-graz.at/o:konde.25}{Metadaten}-Design und Best Practices in der gesamten Metadaten-Ökologie unterstützt. Die DCMI stellt das \emph{Dublin Core Schema} (DC) zur Verfügung und entwickelt dieses Schema weiter. DC ist ein kompaktes Vokabular zur Beschreibung von basalen Metadaten von digitalen Ressourcen, kann aber auch zur Beschreibung von physischen Ressourcen (z. B. Buch) verwendet werden. Durch seine reduzierte, offene und flexible Konzeption unterstützt DC die Interoperabilität von Daten und findet daher im Bereich von \emph{\href{http://gams.uni-graz.at/o:konde.8}{Linked Data}} Verwendung.\\
            
        Den Kern dieses Vokabulars bildet das \emph{Dublin Core Metadata Element Set }(DCMES) mit 15 Kategorien, das durch den URL http://purl.org/dc/elements/1.1 repräsentiert wird. Diese Kategorien erlauben es, beteiligte Akteure (\emph{creator, contributor}), deskriptive (\emph{description, title, language} etc.) und administrative (\emph{rights}) Metadaten eines Objektes zu beschreiben. Mit \emph{Qualified Dublin Core} wird DCMES um restiktivere, von DCMES abgeleitete Kategorien erweitert und sowohl DCMES als auch \emph{Qualified} werden zusammen über den URL http://purl.org/dc/terms referenziert. So ist die Kategorie \emph{accessRights} ein sogenanntes \emph{refinement} von \emph{rights} und beschreibt explizit die Zugangsrechte eines Objektes; es ist nicht mehr nur ein Feld, in dem rechtliche Information in einem allgemeineren Sinne beschrieben werden kann.\\
            
        Die Semantik von DC ist unabhängig von der Syntax und kann daher in unterschiedlichen Beschreibungssprachen serialisiert werden. Weiters ist jede Kategorie optional, wiederholbar und unabhängig von jeglicher Sequenz.\\
            
        \subsection*{Literatur:}\begin{itemize}\item Dublin Core Metadata Initiative. URL: \url{https://dublincore.org}\item DCMI Metadata Terms. URL: \url{https://www.dublincore.org/specifications/dublin-core/dcmi-terms/}\item Weibel, Stuart: The Dublin Core: a simple content description model for electronic resources. In: Bulletin of the American Society for Information Science and Technology 24: 1997, S. 9-11.\end{itemize}\subsection*{Verweise:}\href{https://gams.uni-graz.at/o:konde.25}{Metadaten}, \href{https://gams.uni-graz.at/o:konde.8}{Linked Open Data}, \href{https://gams.uni-graz.at/o:konde.129}{METS}\subsection*{Themen:}Metadaten, Archivierung\subsection*{Zitiervorschlag:}Pollin, Christopher. 2021. Dublin Core Metadata Initiative (DCMI). In: KONDE Weißbuch. Hrsg. v. Helmut W. Klug unter Mitarbeit von Selina Galka und Elisabeth Steiner im HRSM Projekt "Kompetenznetzwerk Digitale Edition". URL: https://gams.uni-graz.at/o:konde.128\newpage\section*{EXIF / XMP} \emph{Vogeler, Georg; georg.vogeler@uni-graz.at}\\
        
    \href{http://gams.uni-graz.at/o:konde.25}{Metadaten} können auf verschiedene Art und Weise in Bilddateien eingebunden werden. Der älteste und am weitesten unterstützte Standard ist EXIF (\emph{Exchangeable Image File Format}). EXIF stammt von der \emph{Japan Electronic and Information Technology Industries Association} und ist 2010 in seiner jüngsten Version (2.32 mit kontinuierlichen redaktionellen Anpassungen bis 2019) festgelegt worden. EXIF speichert die Metadaten im sogenannten ‘Header’ von JPEG- und TIFF-Dateien, der eine festgelegte Größe hat. EXIF speichert insbesondere technische Metadaten des Aufnahmegeräts, Datum und Zeit, knappe textliche Informationen zu Copyright (8298) und zum Inhalt (Bildbeschreibung: 010e, Urheber: 013B) als \emph{Plain Text} in den ‘EXIF-Tags’. Das Format wird umfangreich von Bildbearbeitungssoftware unterstützt.\\
            
        Flexibler und zunehmend weiter verbreitet ist XMP (\emph{Extensible Metadata Platform}), ein Bildbeschreibungsstandard, der im Jahr 2001 von \emph{Adobe} zusammen mit einer Programmbibliothek (XMP \emph{Toolkit} SDK) veröffentlicht wurde und seit 2012 als ISO 16684 standardisiert ist. XMP bettet Metadaten als \href{http://gams.uni-graz.at/o:konde.131}{RDF}/\href{http://gams.uni-graz.at/o:konde.215}{XML} in die Bilddateien ein. Das Datenmodell beruht auf einer offenen Liste von Schlüssel-/Wert-Paaren. Es empfiehlt die Verwendung von \emph{\href{http://gams.uni-graz.at/o:konde.128}{Dublin Core}}-Eigenschaften als Kernelemente der Beschreibung, erlaubt aber die Verwendung frei gewählter Vokabularien.\\
            
        \subsection*{Literatur:}\begin{itemize}\item Camera & Imaging Products Association: Exchangeable image file format for digital still cameras: Exif Version 2.32: 2010. URL: \url{http://www.cipa.jp/std/documents/e/DC-X008-Translation-2019-E.pdf}.\item ISO 16684: Graphic technology — Extensible metadata platform (XMP) specification, Part 1: Data model, serialization and core properties. URL: \url{https://www.iso.org/standard/75163.html}\item ISO 16684: Graphic technology — Extensible metadata platform (XMP) specification, Part 2: Description of XMP schemas using RELAX NG. URL: \url{https://www.iso.org/standard/57422.html}\end{itemize}\subsection*{Verweise:}\href{https://gams.uni-graz.at/o:konde.25}{Metadaten}, \href{https://gams.uni-graz.at/o:konde.124}{Metadatenformate für Bilddateien}, \href{https://gams.uni-graz.at/o:konde.128}{DCMI}, \href{https://gams.uni-graz.at/o:konde.101}{IPTC}\subsection*{Projekte:}\href{https://www.adobe.com/devnet/xmp.html}{XMP Toolkit SDK}\subsection*{Themen:}Metadaten, Archivierung\subsection*{Zitiervorschlag:}Vogeler, Georg. 2021. EXIF / XMP. In: KONDE Weißbuch. Hrsg. v. Helmut W. Klug unter Mitarbeit von Selina Galka und Elisabeth Steiner im HRSM Projekt "Kompetenznetzwerk Digitale Edition". URL: https://gams.uni-graz.at/o:konde.81\newpage\section*{Editionstext} \emph{Rieger, Lisa; lrieger@edu.aau.at }\\
        
    Der Editionstext oder edierte Text ist „jener Text, der das literarische Werk präsentiert“ (Nutt-Kofoth 2007, S. 5). Er wird vom Editor nach der Sichtung sämtlicher Textträger und ihrer kritischen Prüfung auf Autorisation und Bedeutung für die Textentwicklung ausgewählt. Dabei können auch mehrere Fassungen als edierter Text präsentiert werden, die Entscheidung des Editors muss jedoch immer begründet werden. (Plachta 1997, S. 14; Scheibe 1986, S. 123)\\
            
        Seit den Siebzigerjahren des 20. Jahrhunderts gilt in der neugermanistischen Edition ein weitgehender Konsens, dass alle Fassungen eines Werkes, die es in seiner historischen Entwicklung präsentieren, als gleichwertig zu betrachten sind. (Zeller 1989, S. 9 f.) Damit war die prinzipielle Bevorzugung einer Fassung früher oder auch später Hand im Allgemeinen veraltet. Zeitgleich führte das neue, strukturalistische Verständnis des ‘Textfehlers’ zu einer starken Abnahme editorischer Eingriffe in den Text und der Verbannung von Diskussionen unsicherer Stellen in den textkritischen \href{http://gams.uni-graz.at/o:konde.32}{Apparat}. Mit zunehmendem Fokus auf die \href{http://gams.uni-graz.at/o:konde.28}{Textgenese} und die Möglichkeiten \href{http://gams.uni-graz.at/o:konde.174}{synoptischer Darstellungsformen} schwand die Bedeutung des Editionstextes zugunsten der des Apparates als Präsentation des gesamten Textbestandes. (Nutt-Kofoth 2004, S. 46–49)\\
            
        Im Laufe der Geschichte gab es – in engem Zusammenhang mit der gerade vorherrschenden Strömung innerhalb der \href{http://gams.uni-graz.at/o:konde.192}{Textkritik} – immer wieder andere Auffassungen davon, auf welche Weise die Konstitution des Editionstextes zu erfolgen habe. Beginnend bei Karl Lachmann 1845, der für die Konstitution des Editionstexts die kritische Bewertung sämtlicher überlieferter Texte forderte und zur Wiederherstellung des ‘echten’ Textes  die Operationen der Emendation und Konjekturen empfahl, galt zu Beginn des 20. Jahrhunderts im anglo-amerikanischen Sprachraum der Grundsatz des ‚besten‘ Textes, der dem Editor Freiräume für die Normierung von Orthografie und Interpunktion ließ. Diese Methode wurde von amerikanischen Editoren wenig später zur \emph{\href{http://gams.uni-graz.at/o:konde.43}{Copy-Text-Methode}} weiterentwickelt, die mit Hilfe von \emph{substantives} und \emph{accidentals} autornahe Varianten in eine frühe Textform einarbeitete – somit aber lediglich die Idealvorstellung eines Textes erstellte. Die \emph{\href{http://gams.uni-graz.at/o:konde.46}{critique génétique}} in Frankreich wiederum stellte den Schreibprozess in den Vordergrund, nicht einen bestimmten edierten Text. (Plachta 2012, S. 14–25) In den letzten Jahren beeinflussen aber auch zunehmend technische Veränderungen und neue Darstellungsmöglichkeiten im Rahmen von \href{http://gams.uni-graz.at/o:konde.59}{Digitalen Editionen} die Gestaltung des Editionstextes. (Jannidis/Kohle/Rehbein 2017, S. 240)\\
            
        \subsection*{Literatur:}\begin{itemize}\item Jannidis, Fotis; Kohle, Hubertus: Digital Humanities. Eine Einführung. Mit Abbildungen und Grafiken Digital Humanities. Hrsg. von  und Malte Rehbein. Stuttgart: 2017.\item Nutt-Kofoth, Rüdiger: Vom Schwinden der neugermanistischen Textkritik. Zu Geschichte, Gegenwart und Zukunft eines editorischen Zentralbegriffs Vom Schwinden der neugermanistischen Textkritik. In: Editio 18: 2004, S. 38–55.\item Nutt-Kofoth, Rüdiger: Editionsphilologie Editionsphilologie. In: Handbuch Literaturwissenschaft. Gegenstände - Konzepte - Institutionen.. Stuttgart, Weimar: 2007, S. 1-27.\item Plachta, Bodo: Editionswissenschaft. Eine Einführung in Methode und Praxis der Edition neuerer Texte Editionswissenschaft: 1997.\item Plachta, Bodo: Wie international ist die Editionswissenschaft? Ein Blick in ihre Geschichte Wie international ist die Editionswissenschaft?. In: Editio 26: 2012, S. 13–29.\item Scheibe, Siegfried: Editionstechnik Editionstechnik. In: Wörterbuch der Literaturwissenschaft 1. Auflage. Leipzig: 1986, S. 123–124.\item Zeller, Siegfried: Fünfzig Jahre neugermanistische Edition. Zur Geschichte und künftigen Aufgabe der Textologie Fünfzig Jahre neugermanistische Edition. In: Editio 3: 1989, S. 1-17.\end{itemize}\subsection*{Verweise:}\href{https://gams.uni-graz.at/o:konde.32}{Apparat}, \href{https://gams.uni-graz.at/o:konde.28}{Textgenese}, \href{https://gams.uni-graz.at/o:konde.174}{Synopse}, \href{https://gams.uni-graz.at/o:konde.192}{Textkritik}, \href{https://gams.uni-graz.at/o:konde.43}{Copy Text Edition}, \href{https://gams.uni-graz.at/o:konde.46}{critique génétique}, \href{https://gams.uni-graz.at/o:konde.59}{Digitale Edition}, \href{https://gams.uni-graz.at/o:konde.146}{Normalisierung}\subsection*{Themen:}Einführung, Digitale Editionswissenschaft\subsection*{Lexika}\begin{itemize}\item \href{https://edlex.de/index.php?title=Edierter_Text}{Edlex: Editionslexikon}\end{itemize}\subsection*{Zitiervorschlag:}Rieger, Lisa. 2021. Editionstext. In: KONDE Weißbuch. Hrsg. v. Helmut W. Klug unter Mitarbeit von Selina Galka und Elisabeth Steiner im HRSM Projekt "Kompetenznetzwerk Digitale Edition". URL: https://gams.uni-graz.at/o:konde.75\newpage\section*{Editionstypen} \emph{Klug, Helmut W.; helmut.klug@uni-graz.at }\\
        
    Die wissenschaftliche Edition einer historischen Quelle bietet deren Aufbereitung für ein einschlägiges Publikum nach disziplinenspezifischen Grundsätzen. Da die Geschichte des wissenschaftlichen Edierens Jahrhunderte zurückreicht, haben sich charakteristische Editionstypen (z. B. \href{http://gams.uni-graz.at/o:konde.93}{historisch-kritische Edition}, \href{http://gams.uni-graz.at/o:konde.116}{Leseausgabe} etc.) herausgebildet, die jeweils unterschiedliche editorische Vorgehensweisen verlangen. Viele Editionstypen sind noch sehr stark dem Druckparadigma verhaftet, manche profitieren aber von den Möglichkeiten der digitalen Umsetzung: z. B. \href{http://gams.uni-graz.at/o:konde.33}{Archivausgabe}, \href{http://gams.uni-graz.at/o:konde.39}{Briefedition} usw.\\
            
        \subsection*{Literatur:}\begin{itemize}\item Ott, Wilhelm; Gabler, Hans Walter; Sappler, Paul: EDV-Fibel für Editoren. Im Auftrag und in Zusammenarbeit mit der Arbeitsgemeinschaft Philosophischer Editionen der Allgemeinen Gesellschaft für Philosophie in Deutschland. Tübingen: 1982.\item Runow, Holger: Wem nützt was? Mediävistische Editionen (auch) vom Nutzer aus gedacht. In: editio. Internationales Jahrbuch für Editionswissenschaft 28: 2014, S. 50–57.\item Sahle, Patrick: Digitale Editionsformen. Zum Umgang mit der Überlieferung unter den Bedingungen des Medienwandels. Teil 1: Das typografische Erbe. Norderstedt: 2013.\end{itemize}\subsection*{Software:}\href{http://evt.labcd.unipi.it/}{EVT}, \href{http://gams.uni-graz.at/archive/objects/o:gams.doku/methods/sdef:TEI/get?locale=de}{GAMS}, \href{http://lombardpress.org/}{LombardPress}, \href{https://textualcommunities.org/app/}{Textual Communities}\subsection*{Verweise:}\href{https://gams.uni-graz.at/o:konde.33}{Archivausgabe}, \href{https://gams.uni-graz.at/o:konde.39}{Briefedition}, \href{https://gams.uni-graz.at/o:konde.43}{Copy Text Edition / copy text}, \href{https://gams.uni-graz.at/o:konde.46}{critique génétique}, \href{https://gams.uni-graz.at/o:konde.55}{Denkmäleredition}, \href{https://gams.uni-graz.at/o:konde.59}{Digitale Edition}, \href{https://gams.uni-graz.at/o:konde.65}{diplomatische Edition}, \href{https://gams.uni-graz.at/o:konde.72}{documentary editing}, \href{https://gams.uni-graz.at/o:konde.83}{Faksimileausgabe/edition}, \href{https://gams.uni-graz.at/o:konde.85}{Filmedition}, \href{https://gams.uni-graz.at/o:konde.88}{Fragmentedition}, \href{https://gams.uni-graz.at/o:konde.95}{Hörspieledition}, \href{https://gams.uni-graz.at/o:konde.90}{Genetische Edition}, \href{https://gams.uni-graz.at/o:konde.91}{Gesamtausgabe}, \href{https://gams.uni-graz.at/o:konde.93}{Historisch-kritische Edition / Ausgabe}, \href{https://gams.uni-graz.at/o:konde.96}{Hybridedition}\subsection*{Themen:}Einführung, Digitale Editionswissenschaft\subsection*{Zitiervorschlag:}Klug, Helmut W. 2021. Editionstypen. In: KONDE Weißbuch. Hrsg. v. Helmut W. Klug unter Mitarbeit von Selina Galka und Elisabeth Steiner im HRSM Projekt "Kompetenznetzwerk Digitale Edition". URL: https://gams.uni-graz.at/o:konde.76\newpage\section*{Editionstypografie} \emph{Neuber, Frederike; frederike.neuber@bbaw.de }\\
        
    Die Auseinandersetzungen mit \href{http://gams.uni-graz.at/o:konde.200}{Typografie} als Informationssystem einer Edition kann man unter dem Begriff ‘Editionstypografie’ zusammenfassen. Das sich im deutschsprachigen Raum vor allem durch einschlägige Publikationen konstituierende Feld beschäftigt sich mit der typografischen Repräsentation historischer Quellen in einer Edition. (Reuß 2006; Falk/Mattenklott 2007; Falk/Rahn 2016) Ein Diskussionsfeld der Editionstypografie ist die Wahl der Typografie in Bezug auf die edierte Quelle und damit verbunden beispielsweise die Frage, inwiefern die Wahl einer bestimmten Schriftart typografische Nähe oder Distanz zur edierten Quelle schafft. (Forssman/Rahn 2016, Brüning 2016) Zum anderen spielen aber auch typografische Systeme zur Rekodierung der Quelle eine Rolle. Das umfasst etwa die typografische Konzeption editorischer Erschließungselemente (z. B. kritischer \href{http://gams.uni-graz.at/o:konde.32}{Apparat}) und die Entwicklung von typografischen Markierungssystemen. (Nutt-Kofoth 2016; Röcken 2008, S. 45) Schließlich sind in der Diskussion um Editionstypografie nicht zuletzt auch ästhetische Fragen der Buchgestaltung relevant. (Falk/Mattenklott 2007; Peschken 2017) Das vornehmlich im deutschsprachigen Raum besetzte Feld widmet sich bisher vorrangig der Typografie gedruckter Editionen, während die Webtypografie digitaler Editionen kaum berücksichtigt wird.\\
            
        \subsection*{Literatur:}\begin{itemize}\item Brüning, Gerrit: Fraktur oder Antiqua? Typographie und Zeichentreue als editorisches Problem. In: Typographie & Literatur. Frankfurt am Main: 2016, S. 335–348.\item Ästhetische Erfahrung und Edition. Hrsg. von Rainer Falk und Thomas Rahn. Tübingen: 2007.\item Forssman, Friedrich; Rahn, Thomas: Gemäßigte Mimesis. Spielräume und Grenzen einer eklektischen Editionstypographie. In: Typographie & Literatur. Frankfurt am Main: 2016, S. 369–386.\item Nutt-Kofoth, Rüdiger: Typographie als Informationssystem. Zum Layout der neugermanistischen Edition. In: Typographie & Literatur. Frankfurt am Main: 2016, S. 249–368.\item Peschken, Martin: Sudelblätter in Halbleinen – oder wie ästhetisch ist eine wissenschaftliche Edition. In: Ästhetische Erfahrung und Edition. Tübingen: 2007, S. 213–232.\item Reuß, Roland: Edition & Typographie. Frankfurt am Main: 2006.\item Röcken, Per: Was ist – aus editorischer Sicht – Materialität? Versuch einer Explikation des Ausdrucks und einer sachlichen Klärung Was ist – aus editorischer Sicht – Materialität?. In: editio 22: 2008, S. 22–46.\end{itemize}\subsection*{Verweise:}\href{https://gams.uni-graz.at/o:konde.200}{Typografie}, \href{https://gams.uni-graz.at/o:konde.221}{Paläotypie}\subsection*{Themen:}Digitale Editionswissenschaft\subsection*{Zitiervorschlag:}Neuber, Frederike. 2021. Editionstypografie. In: KONDE Weißbuch. Hrsg. v. Helmut W. Klug unter Mitarbeit von Selina Galka und Elisabeth Steiner im HRSM Projekt "Kompetenznetzwerk Digitale Edition". URL: https://gams.uni-graz.at/o:konde.77\newpage\section*{Editor-testing} \emph{Bleier, Roman; roman.bleier@uni-graz.at }\\
        
    Bei der Entwicklung von Webanwendungen sind Softwareentwicklerinnen und
                  -entwickler und Nutzerinnen und Nutzer zentrale Beteiligte und das \emph{\href{http://gams.uni-graz.at/o:konde.206}{User-testing}} ein wichtiger Schritt in der \href{http://gams.uni-graz.at/o:konde.182}{Qualitätssicherung}, um die Anwendung an die Bedürfnisse ihrer Nutzer
                  anpassen zu können. Obwohl auch \href{http://gams.uni-graz.at/o:konde.59}{digitale
                     Editionen} Webanwendungen sind, läuft ihrer Entwicklung meist anders:
                  User-testing spielt noch immer nicht die zentrale Rolle, die sie laut dem \emph{\href{http://gams.uni-graz.at/o:konde.99}{Interface Design Cycle}} spielen könnte und sollte. (Caria/Mathiak 2018) Der Einfluss
                  der Editorinnen und Editoren, die nicht nur Entwicklerinnen und Entwickler von
                  Inhalten sind, sondern das Editionsprojekt üblicherweise auch leiten und ein sehr
                  stark persönliches Interesse an der erfolgreichen Umsetzung haben, ist zentral.
                  Editorinnen und Editoren nehmen Einfluss auf die Entwicklung der Editionsdaten
                  (z. B. das \href{http://gams.uni-graz.at/o:konde.178}{TEI}-\href{http://gams.uni-graz.at/o:konde.137}{Datenmodell}, Dokumentation,
                  Entwicklung von Paratexten), sie sind aber oft auch zentral bei der Realisierung
                  des User-\href{http://gams.uni-graz.at/o:konde.98}{Interfaces}. Die
                  Benutzerfreundlichkeit spielt eine wichtige Rolle, User-Interfaces von Digitalen
                  Editionen übermitteln darüber hinaus aber auch eine Botschaft und den
                  wissenschaftlichen Standpunkt der Editorinnen und Editoren. Dillen beschreibt das
                  User-Interface etwa als neuen Paratext für Digitale Editionen, welcher die Aufgabe
                  hat, Benutzerinnen und Benutzer zu den gesuchten Informationen zu führen.
                     (Dillen 2018, S. 37–42) Die argumentative Rolle von
                  User-Interfaces wurde von Andrews und van Zundert in einem Konferenzbeitrag an der
                  Universität Graz diskutiert (Andrews und van Zundert 2018).\\
            
        Mit Editor-testing ist die bei der Entwicklung von Digitalen Editionen bereits
                  übliche Praxis gemeint, dass Editorinnen und Editoren als eine eigene User-Gruppe
                  die Edition testen. In manchen Editionsprojekten ist das der einzige Test von
                  User-Interface und Funktionalität; in anderen Projekten wird diese Art von Test
                  als Bestandteil der Entwicklungsphase der Edition ausgeführt. Da Editorinnen und
                  Editoren beständig an der Edition mitarbeiten und auch nach Veröffentlichung der
                  Edition diese für die eigene Forschung nutzen, ist eine Einbindung dieser Gruppe
                  in die Entwicklung und Testphase des User-Interfaces durchaus sinnvoll. Durch die
                  eigene Einbindung ins Projekt, die Entwicklung der Edition und ihre Erfahrung mit
                  den Editionsdaten sind Editorinnen und Editoren für die Testphase natürlich sehr
                  wichtig. Ein \emph{Editor-testing} sollte aber auf keinen Fall das
                     \emph{User-testing} mit projektexternen Benutzerinnen und
                  Benutzern ersetzen, da die kontinuierliche Nähe zum Editionsprojekt die
                  Editorinnen und Editoren, ebenso wie die Entwicklerinnen und Entwickler, für
                  manche Nutzungsszenarien und Mängel in der Informationsdarbietung blind macht.\\
            
        \subsection*{Literatur:}\begin{itemize}\item Andrews, Tara L.; van Zundert, Joris J: What Are You Trying to Say? The Interface as an Integral
                              Element of Argument What Are You Trying to Say?. In: Digital Scholarly Editions as Interfaces 12: 2018, S. 3-33.\item Caria, Federico; Mathiak, Brigitte: A Hybrid Focus Group for the Evaluation of Digital
                              Scholarly Editions of Literary Authors. In: Digital Scholarly Editions as Interfaces 12. Norderstedt: 2018, S. 267–285.\item Dillen, Wout: The Editor in the Interface: Guiding the User through
                              Texts and Images The Editor in the Interface. In: Digital Scholarly Editions as Interfaces 12. Norderstedt: 2018, S. 35–59.\end{itemize}\subsection*{Verweise:}\href{https://gams.uni-graz.at/o:konde.205}{Usability}, \href{https://gams.uni-graz.at/o:konde.207}{User-centered Design}\subsection*{Themen:}Interfaces, Digitale Editionswissenschaft\subsection*{Zitiervorschlag:}Bleier, Roman. 2021. Editor-testing. In: KONDE Weißbuch. Hrsg. v. Helmut W. Klug unter Mitarbeit von Selina Galka und Elisabeth Steiner im HRSM Projekt "Kompetenznetzwerk Digitale Edition". URL: https://gams.uni-graz.at/o:konde.78\newpage\section*{Einführung: Was ist XML/TEI?} \emph{Lisa Rieger; lrieger@edu.aau.at }\\
        
    \href{http://gams.uni-graz.at/o:konde.215}{XML} (\emph{eXtensible }\emph{Markup}\emph{Language}) ist ein Standard, der vom World Wide
                  Web-Konsortium (W3C) als Vereinfachung der älteren SGML (\emph{Standard Generalized Markup Language}, ISO 8879) zur strukturierten
                  Darstellung von Dokumenten und Daten, v. a. auch für den Gebrauch im Zusammenhang
                  mit dem Internet, entwickelt wurde. Auf der W3C-Homepage (W3C o. J.)
                  finden sich sowohl einführende Erläuterungen und Tutorials als auch sämtliche
                  Dokumente, in denen die Standards definiert werden:\\
            
        \begin{itemize}\item {\emph{Extensible Markup Language} (XML) 1.0 (Fifth Edition)
                        (W3C Recommendation 2008)}\item {\emph{Namespaces} in XML 1.0 (Third Edition) (W3C
                        Recommendation 2009a)}\item {XML \emph{Inclusions} (XInclude) Version 1.0 (Second
                     Edition) (W3C Recommendation 2006)}\item {XML \emph{Information Set }(Second Edition) (W3C
                        Recommendation 2004)}\item {xml:id Version 1.0 (W3C Recommendation 2005)}\item {XML \emph{Fragment Interchange}(W3C Recommendation 2001)}\item {XML \emph{Base} (Second Edition) (W3C Recommendation
                        2009b)}\item {\emph{Associating Style Sheets with XML documents }1.0 (Second
                     Edition) (W3C Recommendation 2010)}\end{itemize}Die Hauptbausteine eines XML-Dokuments sind Elemente, die aus einem öffnenden Tag
                  (<) und einem schließenden Tag (/>) bestehen. Der Text zwischen den Tags
                  gilt dann als Teil des Elements und unterliegt seinen Bestimmungen. Elemente ohne
                  Text zwischen öffnendem und schließendem Tag sind leere Elemente (z. B.
                     <empty/>), die dem Text Zusatzinformationen hinzufügen.
                  Attribute können ein Element genauer bestimmen. Sie werden im Starttag der
                  Elementbezeichnung nachgestellt und ihr Wert wird nach einem Gleichheitszeichen
                  durch Anführungszeichen begrenzt (<Substantiv
                  Kasus=“Nominativ“>Edition</Substantiv>). (Eckstein 2000, S. 6
                     ff.) Zudem können XML-Dokumente noch Deklarationen, Kommentare,
                  Zeichenreferenzen und Verarbeitungsanweisungen enthalten. Damit ein XML-Dokument
                  gültig ist, muss es sowohl hinsichtlich seiner physikalischen Struktur wohlgeformt
                  als auch, vorgegeben durch den Dokumenttyp, valide sein (vgl. \href{http://gams.uni-graz.at/o:konde.166}{Schema}). (Doss 2000, S. 43–46)\\
            
        Das \href{http://gams.uni-graz.at/o:konde.178}{TEI}(TEI Consortium a) ist ein Konsortium, das auf Basis eines
                  erweiterbaren XML-Schemas eine Markup-Sprache – ebenfalls TEI genannt – zur
                  Auszeichnung der Eigenschaften von Dokumenten im Bereich der Geistes- und
                  Sozialwissenschaften entwickelt und mit einer Open-Source-Lizenz veröffentlicht
                  hat. Der Schwerpunkt liegt dabei auf der Beschreibung von Primärquellen für
                  Forschung und Analysen. Wie bei XML werden Leitfäden (aktuellste Version: P5 \emph{Guidelines}) und andere wichtige Dokumente (TEI
                     Consortium b) direkt auf der Homepage zur Verfügung gestellt. TEI
                  erweist sich gerade für literaturwissenschaftliche Arbeiten wie die \href{http://gams.uni-graz.at/o:konde.59}{Edition} von Texten als besonders
                  geeignet, da es sowohl die Repräsentation der Zeichensequenz und Textstruktur als
                  auch das Annotieren editionswissenschaftlicher und dokumentenbezogener \href{http://gams.uni-graz.at/o:konde.25}{Metadaten} ermöglicht und zudem
                  langfristig und plattformübergreifend nutzbar ist. (Schöch 2016, S.
                     335)\\
            
        \subsection*{Literatur:}\begin{itemize}\item Bader, Winfried: Was ist die Text Encoding Initiative (TEI)? Was ist die Text Encoding Initiative (TEI)? In: Computergestützte Text-Edition. Beihefte zu
                              editio 12: 1999, S. 9-20.\item Box, Don; Skonnard, Aaron; Lam, John: Essential XML. XML für Softwareentwicklung. Deutsche
                              Übersetzung von Mathias Born und Michael Tamm Essential XML. München: 2001.\item Doss, George: XML-Tipps. Deutsche Übersetzung von Adreas
                              Fieback XML-Tipps. Frankfurt: 2000.\item Eckstein, Robert: XML kurz & gut. Übersetzung von Nick Klever XML kurz & gut. Köln (u. a.): 2000.\item Jele, Harald: XML-TEI-kodierte Texte editieren: 2013. URL: \url{https://ubdocs.aau.at/open/voll/fodok/ub/AC11224010.pdf}.\item Jones, Christopher A.; Drake Jr., Fred L.: Python & XML. Deutsche Übersetzung von Dinu C.
                              Gherman Python & XML. Köln (u. a.): 2002.\item Klettke, Maike; Meyer, Holger: XML & Datenbanken. Konzepte, Sprachen und
                              Systeme XML & Datenbanken. Heidelberg: 2003.\item Schöch, Christof: Ein digitales Textformat für die
                              Literaturwissenschaften. Die Richtlinien der Text Encoding Initiative
                              und ihr Nutzen für Textedition und Textanalyse Ein digitales Textformat für die
                              Literaturwissenschaften. In: Romanische Studien: 2016, S. 325–364.\item TEI: Text Encoding Initiative. URL: \url{https://tei-c.org/}\item TEI: Guidelines. URL: \url{https://tei-c.org/guidelines/}\item XML Technology. URL: \url{https://www.w3.org/standards/xml/}\item Associating Style Sheets with XML documents 1.0 (Second
                              Edition). W3C Recommendation 28 October 2010 Associating Style Sheets with XML documents 1.0
                              (Second Edition). URL: \url{https://www.w3.org/TR/xml-stylesheet/}\item Namspaces in XML 1.0 (Third Edition). W3C Recommendation
                              8 December 2009 Namespaces in XML 1.0 (Third Edition). URL: \url{https://www.w3.org/TR/xml-names/}\item XML Base (Second Edition). W3C Recommendations 28
                              January 2009 XML Base (Second Edition). URL: \url{https://www.w3.org/TR/2009/REC-xmlbase-20090128/}\item Extensible Markup Language (XML) 1.0 (Fifth Edition).
                              W3C Recommendation 26 November 2008 Extensible Markup Language (XML) 1.0 (Fifth
                              Edition). URL: \url{https://www.w3.org/TR/2008/REC-xml-20081126/}\item XML Inclusions (XInclude) Version 1.0 (Second Edition).
                              W3C Recommendation 15 November 2006 XML Inclusions (XInclude) Version 1.0 (Second
                              Edition). URL: \url{https://www.w3.org/TR/2006/REC-xinclude-20061115/}\item xml:id Version 1.0. W3C Recommendation 9 September
                              2005 xml:id Version 1.0. URL: \url{https://www.w3.org/TR/xml-id/}\item XML Information Set (Second Edition). W3C Recommendation
                              4 February 2004 XML Information Set (Second Edition). URL: \url{https://www.w3.org/TR/2004/REC-xml-infoset-20040204/}\item XML Fragment Interchange. W3C Candidate Recommendation
                              12 February 2001 XML Fragment Interchange. URL: \url{https://www.w3.org/TR/xml-fragment}\end{itemize}\subsection*{Software:}\href{https://wiki.tei-c.org/index.php/CWRC-Writer}{CWRC-Writer}, \href{http://oxygenxml.com/}{Oxygen}, \href{https://github.com/oxygenxml/TEI-Facsimile-Plugin}{Oxygen-TEI-Facsimile-Plugin}, \href{http://dcl.ils.indiana.edu/teibp/index.html}{TEI Boilerplate}, \href{https://sourceforge.net/projects/tei-comparator/}{TEI
                           Comparator}, \href{https://teipublisher.com/index.html}{TEI
                           Publisher}, \href{https://code.activestate.com/ppm/Text-TEI-Collate/}{Text-TEI-collate}, \href{http://www.teitok.org/index.php?action=about}{TEITOK}\subsection*{Verweise:}\href{https://gams.uni-graz.at/o:konde.59}{Digitale Edition}, \href{https://gams.uni-graz.at/o:konde.6}{Digitale Nachhaltigkeit}, \href{https://gams.uni-graz.at/o:konde.115}{Lemmatisierung}, \href{https://gams.uni-graz.at/o:konde.152}{Open Access}, \href{https://gams.uni-graz.at/o:konde.17}{Annotation}, \href{https://gams.uni-graz.at/o:konde.195}{Textmodellierung}, \href{https://gams.uni-graz.at/o:konde.25}{Metadaten}, \href{https://gams.uni-graz.at/o:konde.29}{Annotationsstandards}, \href{https://gams.uni-graz.at/o:konde.126}{Markup}, \href{https://gams.uni-graz.at/o:konde.178}{TEI}, \href{https://gams.uni-graz.at/o:konde.156}{Part of Speech Tagging}, \href{https://gams.uni-graz.at/o:konde.176}{Tagger}, \href{https://gams.uni-graz.at/o:konde.177}{TagSets}, \href{https://gams.uni-graz.at/o:konde.212}{WebLicht}, \href{https://gams.uni-graz.at/o:konde.216}{xTokenizer}, \href{https://gams.uni-graz.at/o:konde.15}{Alternative Textkodierung}\subsection*{Themen:}Einführung, Metadaten, Digitale Editionswissenschaft\subsection*{Zitiervorschlag:}Rieger, Lisa. 2021. Einführung: Was ist XML/TEI?. In: KONDE Weißbuch. Hrsg. v. Helmut W. Klug unter Mitarbeit von Selina Galka und Elisabeth Steiner im HRSM Projekt "Kompetenznetzwerk Digitale Edition". URL: https://gams.uni-graz.at/o:konde.79\newpage\section*{Elemente Digitaler Editionen} \emph{Klug, Helmut W.; helmut.klug@uni-graz.at }\\
        
    Der Kern einer (Digitalen) Edition ist die historische Quelle in ihrer jeweiligen Repräsentation, also der \href{http://gams.uni-graz.at/o:konde.75}{Editionstext}. Um diesen herum gruppieren sich unterschiedliche Ergänzungen bzw. Erweiterungen, die den Bereichen Editionswissenschaft, Datenmodellierung, Archivierung und Webdesign zuzuordnen sind. Die unterschiedlichen Elemente können methodisch, technisch oder nach beliebigen anderen Kriterien (z. B. Genettes Paratexttheorie) beschrieben und gruppiert werden. Als klassische Elemente Digitaler Editionen kann man aus editionswissenschaftlicher Sicht u. A. nennen (Bleier/Klug 2020, S. 100f.):\\
            
        \begin{itemize}\item {Archivlösung}\item {Benutzungshinweise}\item {Bibliografie}\item {\href{http://gams.uni-graz.at/o:konde.44}{Copyright} / rechtliche Hinweise}\item {Editionsdaten als Download}\item {Editionsrichtlinien}\item {Einleitung (kurzer überblicksartiger Text, der das Editionsprojekt und die Edition vorstellt)}\item {Finanzierung (Text oder Logos zur Benennung von Geldgebern)}\item {Hyperlink-Materialien (externe, per Hyperlinks verknüpfte Daten)}\item {\href{http://gams.uni-graz.at/o:konde.34}{Kommentare}}\item {Kontaktinformationen}\item {Materialien (zur Ergänzung von Edition und Kommentar, z. B. digitale Faksimiles)}\item {Modellbeschreibung (formale und/oder verbale Beschreibung des Textmodells)}\item {Projektbeschreibung (Ziele, Methoden usw. des Editionsprojekts)}\item {Quellenbeschreibung (Beschreibung der zugrundeliegenden historischen Quelle)}\item {Social Input (Benutzerinnen und Benutzer können Inhalte generieren)}\item {Suchfunktionen}\item {Textanalysen (jegliche Art von automatisierter Verarbeitung des Editionstextes)}\item {Überlieferung}\item {Zitierhinweis(e)}\end{itemize}Legt man die Definition von (2014) zugrunde, kann man als Genre-definierende Elemente Aspekte rund um Archivierung, Modellbeschreibungen, Analyse- und Interaktionsanwendungen betrachten.\\
            
        \subsection*{Literatur:}\begin{itemize}\item Bleier, Roman; Klug, Helmut W.: Funktion und Umfang des Kommentars in Digitalen Editionen mittelalterlicher Texte: Eine Bestandsaufnahme. In: Annotieren, Kommentieren, Erläutern. Aspekte des Medienwandels. Berlin/Boston: 2020, S. 97–112.\item Pierazzo, Elena: Digital scholarly editing: theories, models and methods. Farnham: 2015, URL: \url{http://hal.univ-grenoble-alpes.fr/hal-01182162}.\item Buzzoni, Marina: A Protocol for Scholarly Digital Editions? The Italian Point of View. In: Digital Scholarly Editing Theories and Practices: 2016, S. 59–82.\item . In: Criteria for Reviewing Scholarly Digital Editions, version 1.1 | Institut für Dokumentologie und Editorik: 2014.\end{itemize}\subsection*{Verweise:}\href{https://gams.uni-graz.at/o:konde.98}{Interface}, \href{https://gams.uni-graz.at/o:konde.32}{Apparat}, \href{https://gams.uni-graz.at/o:konde.75}{Editionstext}, \href{https://gams.uni-graz.at/o:konde.106}{Konkordanz}, \href{https://gams.uni-graz.at/o:konde.6}{Digitale Nachhaltigkeit}, \href{https://gams.uni-graz.at/o:konde.174}{Synopse}, \href{https://gams.uni-graz.at/o:konde.31}{API}, \href{https://gams.uni-graz.at/o:konde.35}{Barrierefreies Design}, \href{https://gams.uni-graz.at/o:konde.54}{Datenvisualisierung}, \href{https://gams.uni-graz.at/o:konde.56}{Design}, \href{https://gams.uni-graz.at/o:konde.113}{Lagenvisualisierung}, \href{https://gams.uni-graz.at/o:konde.92}{Handschriftenbeschreibung}, \href{https://gams.uni-graz.at/o:konde.137}{Modellierung}, \href{https://gams.uni-graz.at/o:konde.220}{Zitiervorschlag}, \href{https://gams.uni-graz.at/o:konde.44}{Urheberrecht}, \href{https://gams.uni-graz.at/o:konde.9}{Lizenzmodelle}, \href{https://gams.uni-graz.at/o:konde.147}{Normdaten}\subsection*{Themen:}Digitale Editionswissenschaft\subsection*{Zitiervorschlag:}Klug, Helmut W. 2021. Elemente Digitaler Editionen. In: KONDE Weißbuch. Hrsg. v. Helmut W. Klug unter Mitarbeit von Selina Galka und Elisabeth Steiner im HRSM Projekt "Kompetenznetzwerk Digitale Edition". URL: https://gams.uni-graz.at/o:konde.80\newpage\section*{FAIR-Prinzipien} \emph{Stigler, Johannes; johannes.stigler@uni-graz.at 	 			}\\
        
    Die FAIR \emph{Guiding Principles for scientific data management and stewardship} wurden 2016 veröffentlicht und besagen, dass Forschungsdaten auffindbar (\emph{Findable}), zugänglich (\emph{Accessible}), interoperabel (\emph{Interoperable}) und wiederverwendbar (\emph{Re-usable}) sein sollen. Sie bilden die Grundlage für eine disziplinen- und länderübergreifende Nachnutzung von Forschungsdaten. \\
            
        Die Auffindbarkeit von Forschungsdaten setzt voraus, dass diese mit auch maschinenlesbaren Metadaten und \href{http://gams.uni-graz.at/o:konde.12}{persistenten Identifikatoren} versehen sind. Um den Zugang zu Forschungsdaten langfristig zu gewährleisten, ist es weiters notwendig, diese so aufzubewahren, dass sie (oder zumindest die dazugehörigen \href{http://gams.uni-graz.at/o:konde.25}{Metadaten}) mittels anerkannter, standardisierter Kommunikationsprotokolle (z. B. OAI-PMH) leicht abgerufen werden können. Auch sollte anhand standardisierter \href{http://gams.uni-graz.at/o:konde.9}{Lizenzmodelle} (z. B. \href{http://gams.uni-graz.at/o:konde.45}{Creative Commons}) eindeutig beschrieben werden, unter welchen Bedingungen die Forschungsdaten zugänglich sind. \\
            
        Daten werden als interoperabel bezeichnet, wenn sie ausgetauscht, interpretiert und in einer (semi-)automatisierten Weise mit anderen Datensätzen kombiniert werden können. Dies erfordert die Verwendung von standardisierten, idealerweise \href{http://gams.uni-graz.at/o:konde.215}{XML}-basierten Datenformaten für die Primärdaten und die Verwendung von Metadaten, die auf kontrollierten Vokabularen, Klassifikationen, \href{http://gams.uni-graz.at/o:konde.151}{Ontologien} oder Thesauri basieren und ihrerseits den FAIR-Prinzipien unterliegen.\\
            
        Um sicherzustellen, dass Forschungsdaten in zukünftigen Forschungsvorhaben wiederverwendet werden können, sollten ihre Metadaten eine ausführliche und detaillierte Beschreibung des Entstehungszusammenhangs liefern. Beispielsweise sollte das \href{http://gams.uni-graz.at/o:konde.178}{TEI}-Dokument einer \href{http://gams.uni-graz.at/o:konde.59}{Digitalen Edition} idealerweise auch das bei seiner Entstehung zur Anwendung gebrachte \href{http://gams.uni-graz.at/o:konde.198}{Editionsregelwerk} in einer formalisierten Weise enthalten. Auf diese Weise können Forschende, die diese Daten nachnutzen möchten, besser einschätzen, ob diese tatsächlich für ihr Forschungsvorhaben geeignet sind. \\
            
        GO FAIR ist eine weltweit (\emph{global}) agierende Initiative, die prinzipiell jedem offen (\emph{open}) steht. Unter ihrem Dach kommen Einzelpersonen, Institutionen und Organisationen zusammen, um ihre Bemühungen hinsichtlich eines FAIRen Umgangs mit Forschungsobjekten aufeinander abzustimmen.\\
            
        \subsection*{Literatur:}\begin{itemize}\item Wilkinson, Mark D.; Dumontier, Michel; Aalbersberg, IJsbrand Jan: The FAIR Guiding Principles for scientific data management and stewardship. In: Scientific Data 3: 2016, S. 160018.\item FAIR Principles. URL: \url{https://www.go-fair.org/imprint/}\end{itemize}\subsection*{Verweise:}\href{https://gams.uni-graz.at/o:konde.6}{Digitale Nachhaltigkeit}, \href{https://gams.uni-graz.at/o:konde.12}{Persistent Identifier}, \href{https://gams.uni-graz.at/o:konde.9}{Lizenzmodelle}, \href{https://gams.uni-graz.at/o:konde.45}{Creative Commons}, \href{https://gams.uni-graz.at/o:konde.215}{XML}, \href{https://gams.uni-graz.at/o:konde.151}{Ontologie}, \href{https://gams.uni-graz.at/o:konde.178}{TEI}, \href{https://gams.uni-graz.at/o:konde.59}{Digitale Edition}, \href{https://gams.uni-graz.at/o:konde.107}{Kontrollierte Vokabularien: GeoNames}, \href{https://gams.uni-graz.at/o:konde.108}{Getty}, \href{https://gams.uni-graz.at/o:konde.109}{GND}, \href{https://gams.uni-graz.at/o:konde.147}{Normdaten allgemein}, \href{https://gams.uni-graz.at/o:konde.111}{VIAF}, \href{https://gams.uni-graz.at/o:konde.112}{Wikidata}\subsection*{Projekte:}\href{https://www.go-fair.org/}{FAIR-Initiative}\subsection*{Themen:}Einführung, Archivierung, Digitale Editionswissenschaft, Rechtliche Aspekte\subsection*{Zitiervorschlag:}Stigler, Johannes. 2021. FAIR-Prinzipien. In: KONDE Weißbuch. Hrsg. v. Helmut W. Klug unter Mitarbeit von Selina Galka und Elisabeth Steiner im HRSM Projekt "Kompetenznetzwerk Digitale Edition". URL: https://gams.uni-graz.at/o:konde.7\newpage\section*{FEDORA (Flexible Extensible Digital Object Repository Architecture)} \emph{Stigler, Johannes; johannes.stigler@uni-graz.at }\\
        
    FEDORA ist ein Open Source-Projekt, mit dessen Hilfe man digitale Objekte in elektronischen Archiven verwalten und zugänglich machen kann. Es ist nicht zu verwechseln mit der \emph{Red Hat Linux Distribution} gleichen Namens. FEDORA ist mit den Qualitätskriterien der \emph{Open Archives Initiative} (OAI) für \href{http://gams.uni-graz.at/o:konde.6}{Langzeitarchivierung} kompatibel und damit zur Archivierung geisteswissenschaftlicher Forschungsdaten sehr gut geeignet. Ursprünglich von der Cornell University und der University of Virginia mit der finanziellen Unterstützung der Andrew W. Mellon Foundation entwickelt, wird das Open Source-Projekt heute koordiniert durch LYRASIS – einer Non-Profit-Organisation aus dem amerikanischen Bibliotheksbereich – und DURASPACE.\\
            
        FEDORA implementiert seit Version 4.x die \emph{Linked Data Platform}-Empfehlungen des W3C und bietet einen flexibel erweiterbaren Rahmen für die Speicherung, Verwaltung und Dissemination komplexer, digitaler Objekte. Dabei unterstützt FEDORA die Zusammenfassung lokaler und verteilter Inhalte zu digitalen Objekten und die Zuordnung von Diensten zu Objekten. Auf diese Weise kann ein Objekt über mehrere zugängliche Präsentationsformen verfügen, die dynamisch erstellt werden können. Die Architektur basiert auf einem generischen \href{http://gams.uni-graz.at/o:konde.131}{RDF}-Datenmodell, das in der Lage ist, Beziehungen zwischen Objekten und ihren Komponenten darzustellen. Abfragen für diese Beziehungen werden über einen RDF-\emph{Triple-Store} realisiert. Die Architektur ist als Webservice implementiert, wobei alle Aspekte der komplexen Objektarchitektur und der zugehörigen Verwaltungsfunktionen über REST- und SOAP-Schnittstellen verfügbar gemacht werden. Möglichkeiten zur \href{http://gams.uni-graz.at/o:konde.14}{Versionierung} der digitalen Inhalte des Repositoriums, Transaktionssicherung und \emph{Messaging Pipeline}-gesteuerte CRUD-Workflows bilden weitere Merkmale der Systemarchitektur des Frameworks. Die Implementierung ist als Open Source-Software verfügbar und bildet die Grundlage für eine Vielzahl von Endbenutzeranwendungen für digitale Bibliotheken, Archive, institutionelle Forschungsdatenrepositorien und Lernobjekte.\\
            
        \subsection*{Literatur:}\begin{itemize}\item Lagoze, Carl; Payette, Sandy; Shin, Edwin; Wilper, Chris: Fedora: an architecture for complex objects and their relationships. In: International Journal on Digital Libraries 6: 2006, S. 124–138.\item Linked Data Plattform 1.0. URL: \url{https://www.w3.org/TR/ldp/}\end{itemize}\subsection*{Verweise:}\href{https://gams.uni-graz.at/o:konde.6}{Digitale Nachhaltigkeit}, \href{https://gams.uni-graz.at/o:konde.131}{RDF}, \href{https://gams.uni-graz.at/o:konde.14}{Versionierung}, \href{https://gams.uni-graz.at/o:konde.114}{Langzeitarchivierung}\subsection*{Projekte:}\href{https://duraspace.org/fedora/}{FEDORA}, \href{http://openarchives.org/}{Open Archives Initiative}, \href{http://lyrasis.org}{Lyrasis}, \href{https://duraspace.org/fedora/}{DURASPACE}\subsection*{Themen:}Archivierung\subsection*{Zitiervorschlag:}Stigler, Johannes. 2021. FEDORA (Flexible Extensible Digital Object Repository Architecture). In: KONDE Weißbuch. Hrsg. v. Helmut W. Klug unter Mitarbeit von Selina Galka und Elisabeth Steiner im HRSM Projekt "Kompetenznetzwerk Digitale Edition". URL: https://gams.uni-graz.at/o:konde.69\newpage\section*{Facettierte Suche} \emph{Galka, Selina; selina.galka@uni-graz.at }\\
        
    Eine facettierte Suche ermöglicht das Filtern eines Bestandes nach Kriterien bzw. Facetten, wobei die Suche immer weiter verfeinert werden kann.\\
            
        Im Gegensatz zum \emph{Drill-down}, der hierarchisch erfolgt, und bei dem somit immer nur weiter im Rahmen einer vorgegebenen Struktur navigiert werden kann, können bei der Facettensuche einzelne Kriterien kombiniert werden; außerdem können vorherige Kriterien beibehalten oder auch weggeschaltet werden.\\
            
        Die Facettennavigation bietet mehrere Vorteile: Nutzerinnen und Nutzer müssen über keine genauen Kenntnisse der Entitäten oder deren Zusammenhänge verfügen, die Navigation ist sehr flexibel und man kann unterschiedliche Perspektiven auf den Bestand einnehmen. (Kwasnick 1999, S. 41) Vor allem die Visualisierung der Suche ist wichtig, kann man doch Muster oder Auffälligkeiten entdecken, die eventuell neue Erkenntnisse erzeugen können (vgl. \href{http://gams.uni-graz.at/o:konde.54}{Datenvisualisierung}). (Kwasnick 1999, S. 42)\\
            
        Herausfordernd kann jedoch die Definition der Facetten sein – dies erfordert umfangreiches Wissen über die Objekte – und auch die Visualisierung des Systems kann sich als schwierig erweisen. (Kwasnick 1999, S. 41f.)\\
            
        \subsection*{Literatur:}\begin{itemize}\item Kwasnick, Barbara H.: The role of classification in knowledge representation and discovery. In: Library Trends 48, S. 22–47.\end{itemize}\subsection*{Software:}\href{http://lucene.apache.org/solr/}{Solr}\subsection*{Verweise:}\href{https://gams.uni-graz.at/o:konde.211}{Volltextsuche}, \href{https://gams.uni-graz.at/o:konde.98}{Interface}, \href{https://gams.uni-graz.at/o:konde.54}{Datenvisualisierung}\subsection*{Themen:}Interfaces\subsection*{Zitiervorschlag:}Galka, Selina. 2021. Facettierte Suche. In: KONDE Weißbuch. Hrsg. v. Helmut W. Klug unter Mitarbeit von Selina Galka und Elisabeth Steiner im HRSM Projekt "Kompetenznetzwerk Digitale Edition". URL: https://gams.uni-graz.at/o:konde.82\newpage\section*{Faksimileausgabe} \emph{Rieger, Lisa; lrieger@edu.aau.at }\\
        
    Bei einem Faksimile handelt es sich um „die getreue Abbildung/Nachbildung einer
                  Vorlage, z. B. von Handschriften, älteren Druckwerken, Zeichnungen“ (Best
                     1991, S. 154). Während die Herstellung von Faksimiles bei
                  mittelalterlichen Handschriften schon lange Teil der anerkannten editorischen
                  Tradition ist (Plachta 1997, S. 22 f.), gewann sie in der
                  neugermanistischen Editorik erst mit der wachsenden Bedeutung der \href{http://gams.uni-graz.at/o:konde.28}{Textgenese} an Stellenwert.
                  Faksimiles ermöglichen die Beschäftigung mit der Materialität eines Textträgers,
                  ohne dafür ein Archiv aufsuchen zu müssen. Auch wenn sie die Textträger im
                  Verhältnis zu ihrer physischen Form immer noch in einem reduzierten Umfang
                  darstellen, ersetzen oder zumindest unterstützen Faksimiles das aufwendige
                  Beschreibungsverfahren. Im Zuge der erweiterten Faksimilierung wurde mit der
                  Ausgabe von Büchners Woyzeck (Schmid 1981), die Faksimiles der Handschriften
                  in Originalgröße und -faltung inkl. \href{http://gams.uni-graz.at/o:konde.66}{diplomatischer Umschrift} enthält, sogar die \href{http://gams.uni-graz.at/o:konde.33}{Archivausgabe} als neuer Ausgabetyp geschaffen.
                     (Nutt-Kofoth 2007, S. 19) In \href{http://gams.uni-graz.at/o:konde.59}{Digitalen Editionen} werden Faksimiles oft in \href{http://gams.uni-graz.at/o:konde.174}{synoptischen Darstellungen} parallel
                  zum Text angezeigt (Jannidis/Kohle/Rehbein 2017, S. 235), da es im
                  digitalen Medium wesentlich einfacher ist, \href{http://gams.uni-graz.at/o:konde.36}{Digitalisate bereitzustellen} als im Druck.\\
            
        \subsection*{Literatur:}\begin{itemize}\item Best, Otto: Handbuch literarischer Fachbegriffe. Definitionen und
                              Beispiele. Überarbeitete und erweiterte Ausgabe Handbuch literarischer Fachbegriffe. Frankfurt am Main: 1991.\item Nutt-Kofoth, Rüdiger: Editionsphilologie Editionsphilologie. In: Handbuch Literaturwissenschaft. Gegenstände - Konzepte -
                              Institutionen.. Stuttgart, Weimar: 2007, S. 1-27.\item Jannidis, Fotis; Kohle, Hubertus: Digital Humanities. Eine Einführung. Mit Abbildungen und
                              Grafiken Digital Humanities. Hrsg. von  und Malte Rehbein. Stuttgart: 2017.\item Plachta, Bodo: Editionswissenschaft. Eine Einführung in Methode und
                              Praxis der Edition neuerer Texte Editionswissenschaft: 1997.\item Büchner, Georg: Woyzeck: Faksimileausgabe der Handschriften Woyzeck. Hrsg. von  und Gerhard Schmid. Leipzig: 1981.\end{itemize}\subsection*{Verweise:}\href{https://gams.uni-graz.at/o:konde.28}{Textgenese}, \href{https://gams.uni-graz.at/o:konde.65}{diplomatische Edition}, \href{https://gams.uni-graz.at/o:konde.66}{diplomatische Transkription}, \href{https://gams.uni-graz.at/o:konde.33}{Archivausgabe}, \href{https://gams.uni-graz.at/o:konde.59}{Digitale Edition}, \href{https://gams.uni-graz.at/o:konde.174}{Synopse}, \href{https://gams.uni-graz.at/o:konde.32}{Apparat}\subsection*{Themen:}Digitale Editionswissenschaft\subsection*{Lexika}\begin{itemize}\item \href{https://edlex.de/index.php?title=Faksimileausgabe}{Edlex: Editionslexikon}\end{itemize}\subsection*{Zitiervorschlag:}Rieger, Lisa. 2021. Faksimileausgabe. In: KONDE Weißbuch. Hrsg. v. Helmut W. Klug unter Mitarbeit von Selina Galka und Elisabeth Steiner im HRSM Projekt "Kompetenznetzwerk Digitale Edition". URL: https://gams.uni-graz.at/o:konde.83\newpage\section*{Farbdesign} \emph{Galka, Selina; selina.galka@uni-graz.at }\\
        
    Auf einer Website sollten Farben bewusst und gezielt eingesetzt werden. Grundsätzlich sollte nur eine begrenzte Anzahl an Farben verwendet werden – Konzepte der Farbtheorie, wie der Farbkreis, der Komplementärfarben abbildet, können hilfreich sein, um eine harmonische Farbgebung zu erreichen; auch Farbassoziationen und -wirkungen können bei der Auswahl beachtet werden. (Seibert/Hoffmann 2008, S. 257)\\
            
        Bei der Auswahl der Farben kann nach unterschiedlichen Konzepten vorgegangen werden, beispielsweise kann auf Farben von vorliegenden Bildern, die auch in die Website integriert werden sollen, zurückgegriffen werden, außerdem finden sich im Internet zahlreiche vordefinierte Farbschemata. (Seibert/Hoffmann 2008, S. 258ff.)\\
            
        Die Barrierefreieit der Website sollte beachtet werden, da viele Menschen eine Sehschwäche oder -behinderung haben (vgl. \href{http://gams.uni-graz.at/o:konde.35}{Barrierefreies Design}). Hier bietet sich die Verwendung von kontrastreichen Farbschemata an. (Seibert/Hoffmann 2008, S. 262f.) Bei der Farbwahl spielen ästhetische Gesichtspunkte eine Rolle, aber auch Dienstleisteridentität oder Benutzerfreundlichkeit. (Beaird 2008, S. 45)\\
            
        \subsection*{Literatur:}\begin{itemize}\item Seibert, Björn; Hoffmann, Manuela: Professionelles Webdesign mit (X)HTML und CSS. Bonn: 2008.\item Wirth, Thomas: Missing Links. Über gutes Webdesign. München, Wien: 2004.\item Beaird, Jason: Gelungenes Webdesign. Eine praktische Einführung in die Prinzipien der Webseitengestaltung. Heidelberg: 2008.\end{itemize}\subsection*{Verweise:}\href{https://gams.uni-graz.at/o:konde.35}{Barrierefreies Design}, \href{https://gams.uni-graz.at/o:konde.205}{Usability}, \href{https://gams.uni-graz.at/o:konde.56}{Design Digitaler Editionen}\subsection*{Themen:}Interfaces\subsection*{Zitiervorschlag:}Galka, Selina. 2021. Farbdesign. In: KONDE Weißbuch. Hrsg. v. Helmut W. Klug unter Mitarbeit von Selina Galka und Elisabeth Steiner im HRSM Projekt "Kompetenznetzwerk Digitale Edition". URL: https://gams.uni-graz.at/o:konde.84\newpage\section*{Filmedition} \emph{Rieger, Lisa; lrieger@edu.aau.at }\\
        
    Unter einer Filmedition versteht man in der Wissenschaft „ein von einem/einer
                  Herausgeber/in bzw. Restaurator/in nach wissenschaftlichen Prinzipien
                  wiederhergestelltes bzw. restauriertes und öffentlich verfügbar gemachtes Filmwerk
                  eines fremden Werkschöpfers bzw. eines Kollektivs von Werkschöpfern
                  (wissenschaftliche Edition bzw. kritische Edition).“ (Bohn 2013, S.
                     346–350) Aufgrund der benötigten philologischen sowie medien- und
                  filmtechnischen Kompetenzen fordert die Filmedition eine starke interdisziplinäre
                  Zusammenarbeit. Das primäre Ziel der Edition älterer Filme liegt in der
                  Herstellung einer hochauflösenden digitalen Videofassung, mit der durch Vergleiche
                  zwischen allen zugänglichen Archivmaterialien so weit wie möglich die
                  Uraufführungsfassung rekonstruiert werden soll. Ähnlich wie bei \href{http://gams.uni-graz.at/o:konde.93}{historisch-kritischen
                     Literatureditionen} wird in historisch-kritischen Filmeditionen das
                  überlieferte Material kommentiert zur Verfügung gestellt, editorische
                  Entscheidungen offengelegt sowie betroffene Filmstellen kritisch kommentiert.
                     (Keitz 2013, S. 15 f.)\\
            
        Die vorherrschende Meinung zur Filmrestaurierung sieht aufgrund unterschiedlicher
                  Faktoren das tatsächliche Original als verloren an und geht davon aus, dass im
                  besten Fall ein neues, digital bearbeitetes Original hergestellt werden kann, für
                  dessen Restaurierung folgende Leitlinien gelten:\\
            
        \begin{itemize}\item {Restaurierungsentscheidungen müssen transparent dokumentiert werden,}\item {die Rekonstruktion soll so originalgetreu wie möglich erfolgen,}\item {historische Spuren sollen erhalten bleiben und}\item {Entscheidungen sollen reversibel sein. (Keitz 2013, S. 18 ff.)}\end{itemize}Nach Kanzog (2004, S. 56 ff.) hat man es auch bei der Restaurierung
                  von Filmen mit \href{http://gams.uni-graz.at/o:konde.192}{textkritischen} Problemen zu tun, wobei er unter Text dabei „die Gesamtheit
                  aller audiovisuellen Äußerungsakte und Informationen" versteht (Kanzog 2004,
                     S. 58). Besonders wichtig sei dabei, zwischen Varianten und bloßen
                  Differenzen zu unterscheiden sowie Korruptelen und Fremdeingriffe zu erkennen und
                  zu eliminieren. (Kanzog 2004, S. 64) Ein größeres Problem als die
                  Rekonstruktion des Films ist jedoch die Darstellung der Film\href{http://gams.uni-graz.at/o:konde.28}{genese}: Da nicht verwendete Takes oft vernichtet
                  werden, greift man dabei oft auf die Rezeption des Films zurück. (Kanzog
                     2010, S. 216)\\
            
        In Abhandlungen und Workshops aufgezeigte aktuelle Themengebiete innerhalb der
                  kritischen Filmedition beschäftigen sich u. a. mit dem Fehlen verlässlicher
                  Standards bei der Überführung analoger Filme in ein digitales Format, dem Umgang
                  mit nicht-filmischem Material in der Filmgenese, den technischen Möglichkeiten bei
                  der \href{http://gams.uni-graz.at/o:konde.137}{Modellierung} digitaler
                  Filmeditionen sowie bisher vernachlässigten Teilbereichen, wie z. B. die Rolle der
                  Filmmusik. (Keitz et al. 2019, S. 174 ff.) Zudem rückte in den
                  letzten Jahren auch der Vorgänger des Films, die historische Projektionskunst des
                  18. und 19. Jahrhunderts, ins Interesse der Forschung, was u. a. zur Einrichtung
                  der Online-Forschungsplattform \emph{eLaterna} führte.
                     (Vogl-Bienek 2018, S. 105 f.)\\
            
        \subsection*{Literatur:}\begin{itemize}\item Bohn, Anna: Filmedition Filmedition. In: Denkmal Film. Band II: Kulturlexikon Filmerbe 2. Wien (u. a.): 2013, S. 346–350.\item Kanzog, Klaus: Der 'richtige Text': Universaler Anspruch -
                              unterschiedliche Wege. Ein mediengeschichtlicher Exkurs Der 'richtige Text'. In: Editio. Internationales Jahrbuch für
                              Editionswissenschaft 18: 2004, S. 56–68.\item Kanzog, Klaus: Darstellung der Filmgenese in einer kritischen
                              Filmedition Darstellung der Filmgenese in einer kritischen
                              Filmedition. In: Editio. Internationales Jahrbuch für
                              Editionswissenschaft 24: 2010, S. 215–222.\item Keitz, Ursula von: Historisch-kritische Filmedition - ein
                              interdisziplinäres Szenario Historisch-kritische Filmedition. In: Editio. Internationales Jahrbuch für
                              Editionswissenschaft 27: 2013, S. 15–37.\item Keitz, Ursula von; Lukas, Wolfgang; Nutt-Kofoth, Rüdiger; Stiasny, Philipp: Kritische Film- und Literaturedition. Perspektiven einer
                              transdisziplinären Editionswissenschaft. Internationale Tagung an der
                              Akademie der Künste Berlin, 17.-19- Januar 2019 Kritische Film- und LIteraturedition. In: Editio. Internationales Jahrbuch für
                              Editionswissenschaft 33: 2019, S. 173–177.\item Vogl-Bienek, Ludwig: eLaterna - Digitale Editionen von Werken der
                              historischen Projektionskunst eLaterna. In: Editio. Internationales Jahrbuch für
                              Editionswissenschaft 32: 2018, S. 104–118.\end{itemize}\subsection*{Verweise:}\href{https://gams.uni-graz.at/o:konde.93}{historisch-kritische Edition}, \href{https://gams.uni-graz.at/o:konde.192}{Textkritik}, \href{https://gams.uni-graz.at/o:konde.34}{Kommentar}, \href{https://gams.uni-graz.at/o:konde.28}{Textgenese}, \href{https://gams.uni-graz.at/o:konde.80}{Elemente digitaler Editionen}, \href{https://gams.uni-graz.at/o:konde.59}{Digitale Edition}, \href{https://gams.uni-graz.at/o:konde.139}{Musikedition}, \href{https://gams.uni-graz.at/o:konde.121}{Audio- Musik und
                           Videoformate}\subsection*{Themen:}Digitale Editionswissenschaft\subsection*{Zitiervorschlag:}Rieger, Lisa. 2021. Filmedition. In: KONDE Weißbuch. Hrsg. v. Helmut W. Klug unter Mitarbeit von Selina Galka und Elisabeth Steiner im HRSM Projekt "Kompetenznetzwerk Digitale Edition". URL: https://gams.uni-graz.at/o:konde.85\newpage\section*{Forschungsinstitut Brenner-Archiv} \emph{Lobis, Ulrich; ulrich.lobis@uibk.ac.at / Wang-Kathrein, Joseph; joseph.wang@uibk.ac.at}\\
        
    Das Forschungsinstitut Brenner-Archiv ging 1964 aus dem Archiv der Zeitschrift \emph{Der Brenner} (herausgegeben von Ludwig von Ficker) hervor und wurde 1979 Teil der \href{http://gams.uni-graz.at/o:konde.201}{Universität Innsbruck}.\\
            
        Neben den über 280 Nachlässen, Teilnachlässen und Sammlungen befinden sich im Brenner-Archiv derzeit auch ca. 30.000 Bücher (v. a. aus den Nachlassbibliotheken) und über 300 Zeitschriften. Da das Archiv nach dem 3-Säulen-Modell (Sammeln, Forschen, Vermitteln) ausgerichtet ist, wurden und werden zahlreiche Forschungsprojekte durchgeführt, die die Materialien aufarbeiten und auch der Öffentlichkeit zur Verfügung stellen.\\
            
        2003 wurde mit der \href{http://gams.uni-graz.at/o:konde.60}{Digitalisierung} von Beständen begonnen, wobei der Schwerpunkt am Anfang vor allem auf den empfindlichsten und wichtigsten Archivalien lag. Seit 2010 beschäftigt sich die ‘Arbeitsgruppe Archiv’ (AGA) mit der Erstellung von Richtlinien und Workflows für das Archivieren; hierbei kümmert sie sich auch darum, dass die Digitalisierung nach den aktuellen Standards durchgeführt wird.\\
            
        Neben seiner Einbindung und Zusammenarbeit in die Strukturen der Universität Innsbruck ist das Archiv auch national und international tätig. Es ist nicht nur ein Mitglied des KOOP-LITERA, eines Netzwerkes aus Institutionen aus Deutschland, Luxemburg, Österreich und der Schweiz, die Literaturnachlässe archivieren; das Brenner-Archiv hat zudem nationale wie internationale Kooperationspartner, wie das Franz und Franziska Jägerstätter-Institut der Katholischen Privat-Universität Linz, das Wittgenstein-Archiv der Universität Bergen (Norwegen) und das Centrum für Informations- und Sprachverarbeitung der Ludwig-Maximilians-Universität München. Das Archiv ist auch Teil des HRSM-Projektes KONDE.\\
            
        \subsection*{Literatur:}\begin{itemize}\item Zeitmesser – 100 Jahre „Brenner“. Hrsg. von  und  . Innsbruck: 2010.\end{itemize}\subsection*{Projekte:}\href{https://www.uibk.ac.at/brenner-archiv/projekte/wittg_briefe/}{Digitale Edition des Gesamtbriefwechsels Ludwig Wittgensteins}, \href{https://www.uibk.ac.at/brenner-archiv/projekte/ficker_briefed/}{Kommentierte Online-Briefedition und Monografie: Umfassende Aufarbeitung des Nachlasses von Ludwig von Ficker}, \href{https://www.uibk.ac.at/brenner-archiv/projekte/lit_karte_tirol/}{Literatur-Land-Karte Tirol / Südtirol: Präsentation und Visualisierung von Autoren und Texte im Bezug auf Orte im Raum Tirol und Südtirol}, \href{https://www.uibk.ac.at/brenner-archiv/projekte/arunda/}{Arunda.40: Internetdokumentation: Dokumentation aller Ausgaben und Präsentation der Zeitschrift}, \href{https://www.uibk.ac.at/brenner-archiv/projekte/trottzusolz/}{Digitalisierung des Nachlasses von Heinrich von Trott zu Solz: digitale Aufarbeitung für des Nachlasses für weitere Projekte}\subsection*{Themen:}Institutionen\subsection*{Zitiervorschlag:}Lobis, Ulrich; Wang-Kathrein, Joseph. 2021. Forschungsinstitut Brenner-Archiv. In: KONDE Weißbuch. Hrsg. v. Helmut W. Klug unter Mitarbeit von Selina Galka und Elisabeth Steiner im HRSM Projekt "Kompetenznetzwerk Digitale Edition". URL: https://gams.uni-graz.at/o:konde.38\newpage\section*{Fragmentedition} \emph{Klug, Helmut W.; helmut.klug@uni-graz.at }\\
        
    Als Fragmente werden historische Quellen bezeichnet, die nicht zusammenhängend und nur in Teilen überliefert sind. Grob kann man hier zwischen zur Wiederverwertung in der Buchbinderei und zum Teilverkauf zerschnittenen Fragmenten unterscheiden. Die Fragmentforschung hat noch vor der Edition der historischen Quelle mehrere andere Probleme zu lösen: das Auffinden, das Katalogisieren und das Zusammenführen von Fragmenten. \\
            
        Eine digitale Arbeitsumgebung kann diese Aufgaben im Bereich der Dokumentation und Beschreibung (z. B. \href{http://gams.uni-graz.at/o:konde.178}{TEI}<msFrag>), während der Identifikation (z. B. \emph{\href{http://gams.uni-graz.at/o:konde.47}{Crowdsourcing}}), im Zuge der Rekonstruktion (z. B. \href{http://gams.uni-graz.at/o:konde.123}{iiif} Manifeste) und der Bereitstellung in virtuellen Bibliotheken (z. B. \emph{Fragmentarium}) unterstützen.\\
            
        \subsection*{Literatur:}\begin{itemize}\item TEI: Guidelines. URL: \url{https://tei-c.org/guidelines/}\item . In: The Promise of Digital Fragmentology: 2015.\end{itemize}\subsection*{Software:}\href{https://iiif.io/}{iiif}\subsection*{Verweise:}\href{https://gams.uni-graz.at/o:konde.123}{iiif}, \href{https://gams.uni-graz.at/o:konde.47}{Crowdsourcing}\subsection*{Projekte:}\href{https://www.fragmentarium.ms}{Fragmentarium - Digital Research Laboratory for Medieval Manuscript Fragments}, \href{https://gams.uni-graz.at/corema}{CoReMA - Cooking Recipes of the Middle Ages}, \href{https://brokenbooks.omeka.net/}{Broken Books}\subsection*{Themen:}Einführung, Digitale Editionswissenschaft\subsection*{Zitiervorschlag:}Klug, Helmut W. 2021. Fragmentedition. In: KONDE Weißbuch. Hrsg. v. Helmut W. Klug unter Mitarbeit von Selina Galka und Elisabeth Steiner im HRSM Projekt "Kompetenznetzwerk Digitale Edition". URL: https://gams.uni-graz.at/o:konde.88\newpage\section*{Freie Werknutzungen} \emph{Scholger, Walter; walter.scholger@uni-graz.at }\\
        
    Wissenschaftlerinnen und Wissenschaftler finden sich üblicherweise sowohl in der
                  Rolle der Urheberin bzw. des Urhebers eigener als auch in der Rolle der Benutzerin
                  bzw. des Benutzers fremder Werke wieder. \\
            
        Die Verwertungsrechte an Werken liegen grundsätzlich einzig und allein bei der
                  Urheberin oder dem Urheber. Das wiederum würde den Zugang zu wissenschaftlichen
                  Erkenntnissen erheblich einschränken, weswegen der Gesetzgeber bereits im
                  Urheberrechtsgesetz eine Reihe von ‘Freien Werknutzungen’, also gesetzlich
                  festgelegten Ausnahmen vom \href{http://gams.uni-graz.at/o:konde.44}{Urheberrecht}, festgelegt hat, die wie Lizenzen zu betrachten sind (vgl.
                     \href{http://gams.uni-graz.at/o:konde.45}{Lizenzmodelle}).\\
            
        Im Folgenden werden jene Regelungen angeführt, die im Kontext digitaler
                  wissenschaftlicher Publikationen am häufigsten zur Anwendung kommen:\\
            
        Forscherinnen und Forscher dürfen im Rahmen ihrer wissenschaftlichen Tätigkeit
                  fremde Werke zu ihrem eigenen Gebrauch in digitaler Weise vervielfältigen, solange
                  die Vervielfältigung keinem kommerziellen Zweck dient – eine Vervielfältigung in
                  einer erwerbsorientierten Publikation ist also nicht zulässig. Wird eine
                  rechtswidrig hergestellte oder der Öffentlichkeit zur Verfügung gestellte
                  Vervielfältigung einer Quelle verwendet, ist der Gebrauch ebenfalls rechtswidrig.
                  Zu beachten ist hierbei, dass diese Ausnahme (UrhG §42 (2)) lediglich für die
                  Vervielfältigung (d. h. den Download) gilt, aber keine Zurverfügungstellung (d. h.
                  Upload) gestattet!\\
            
        Die häufigste freie Werknutzung im digitalen Publikationsprozess ist das
                  wissenschaftliche Zitat. Auch bei einem Zitat (UrhG §42f.) handelt es sich
                  zunächst einmal stets um die Vervielfältigung eines fremden Werkes. Diese ist
                  insbesondere dann zulässig, wenn einzelne fremde Werke “in ein die Hauptsache
                  bildendes wissenschaftliches Werk” (UrhG §42f. Abs1) aufgenommen werden. Derselben
                  Systematik folgend dürfen Werke der bildenden Künste (Grafiken, Fotografien, …)
                  “nur zur Erläuterung des Inhaltes” (UrhG §42f. Abs1) in ein solches Werk
                  aufgenommen werden. Anders als bei der zuvor genannten Ausnahme ist hier neben der
                  Vervielfältigung auch die Zurverfügungstellung gestattet.\\
            
        Eine besondere Stellung im wissenschaftlichen Bereich nimmt das
                  Zweitveröffentlichungsrecht (UrhG §37a), das Urheberinnen und Urhebern
                  wissenschaftlicher Beiträge gestattet, ihre Beiträge “nach Ablauf von zwölf
                  Monaten seit der Erstveröffentlichung in der akzeptierten Manuskriptversion
                  öffentlich zugänglich zu machen, soweit dies keinem gewerblichen Zweck dient”
                  (UrhG §37a), sofern sie eine Reihe zuvor genannter Bedingungen
                  erfüllen.\\
            
        \subsection*{Literatur:}\begin{itemize}\item Burgstaller, Peter: Urheberrecht für Lehrende: Ein Leitfaden für die Praxis
                              mit 80 Fragen und Antworten. Aktuelles Urheberrecht. Wien: 2017.\item Klimpel, Paul; Weitzmann, John H: Forschen in der digitalen Welt. Juristische Handreichung
                              für die Geisteswissenschaften. Göttingen: 2014.\item Scholger, Walter: Urheberrecht und offene Lizenzen im wissenschaftlichen
                              Publikationsprozess. In: Publikationsberatung an Universitäten.Ein
                              Praxisleitfaden zum Aufbau publikationsunterstützender
                              Services. Bielefeld: 2020, S. 123–147.\item Walter, Michael M.: UrhG mit den Novellen 2009-2015, Internationales
                              Privatrecht, Urheberrechtliche EU-Richtlinien: Mit der neueren
                              Rechtsprechung der österreichischen Gerichte und des Gerichtshofs der
                              Europäischen Union. Urheber- und Verwertungsgesellschaftenrecht‚ 15:
                              Textausgabe mit Kurzkommentaren 1. Wien: 2015.\item RIS - Urheberrechtsgesetz - Bundesrecht konsolidiert,
                              Fassung vom 20.01.2021. URL: \url{https://www.ris.bka.gv.at/GeltendeFassung.wxe?Abfrage=Bundesnormen&Gesetzesnummer=10001848}\end{itemize}\subsection*{Verweise:}\href{https://gams.uni-graz.at/o:konde.44}{Urheberrecht}, \href{https://gams.uni-graz.at/o:konde.119}{Lizenzierung}, \href{https://gams.uni-graz.at/o:konde.45}{Lizenzmodelle}\subsection*{Themen:}Rechtliche Aspekte\subsection*{Zitiervorschlag:}Scholger, Walter. 2021. Freie Werknutzungen. In: KONDE Weißbuch. Hrsg. v. Helmut W. Klug unter Mitarbeit von Selina Galka und Elisabeth Steiner im HRSM Projekt "Kompetenznetzwerk Digitale Edition". URL: https://gams.uni-graz.at/o:konde.222\newpage\section*{Gamification} \emph{Moser, Gerda E.; gerda.moser@aau.at / Galka, Selina; selina.galka@uni-graz.at }\\
        
    Das Design eines Video- oder Computerspiels ist handlungsorientiert, fokussiert auf die Aktivitäten der Spielerinnen und Spieler, die durch ihr Agieren entlang der Pfade oder Regeln des Spiels in Emotionen versetzt werden, Neues lernen, Erfahrungen sammeln und vertiefen, ihr Wissen anreichern oder im Rahmen von Belohnungssystemen Punkte und Meisterschaften gewinnen (‘sich hochleveln’). Die Akteurinnen und Akteure treten als Individuen oder Mitglieder einer Gruppe auf, agieren als natürliche Personen oder werden repräsentiert durch einen Stellvertreter (Avatar), der Persönlichkeitsmerkmale mit ihnen teilt oder Seiten der Persönlichkeit hervorhebt, die im realen Leben weniger eine Rolle spielen. Videospiele basieren auf freiwilliger Teilnahme, ermöglichen ein Sich-Erproben im Rahmen von kognitiven und emotionalen Prozessen, die durch das Agieren der Nutzerinnen und Nutzer selbst in Gang gesetzt werden. Videospiele sprechen Motivation und Engagement der Spielerinnen und Spieler an, wobei sie eine intensive Form der Teilhabe sowie Bindung an das Geschehen ermöglichen (Immersion, Kollaboration). \\
            
        Unter dem Stichwort Gamification können Versuche subsumiert werden, Merkmale und Mechanismen von Videospielen in das reale Leben zu übertragen, um dort vergleichbare Effekte zu erzielen. So sollen etwa durch Marketingmaßnahmen, die sich am Design von Videospielen orientieren, Kundenbindungen intensiviert werden. Oder es werden in Anlehnung an Videospiele Tools für den Unterricht entwickelt, die Motivation und Engagement der Lernenden erhöhen sollen. Die Grenzen von Gamification könnten sich als die Grenzen jedweder spielerischen Tätigkeit selbst erweisen. Wirkt das Spiel aufgenötigt oder wird es als Zwang empfunden oder erscheint es allgegenwärtig, verliert sich sein Reiz.\\
            
        Gamification kann so auch eine Rolle beim Editionsprozess spielen (Saklofske u.a. 2016), indem das Konzept z. B. als Motivationsstrategie bei \emph{\href{http://gams.uni-graz.at/o:konde.47}{Crowdsourcing}} oder \emph{\href{http://gams.uni-graz.at/o:konde.41}{Citizien Science}} eingesetzt wird (vgl. auch \emph{\href{http://gams.uni-graz.at/o:konde.169}{Social Edition}}).\\
            
        \subsection*{Literatur:}\begin{itemize}\item Eyal, Nir: Hooked: How to Build Habit-Forming Products: 2014.\item Kim, Sangkyun; Song, Kibong; Lockee, Barbara; Burton, John: Gamification in Learning and Education. Enjoy Learning Like Gaming: 2018.\item Stampfl, Nora S.: Die verspielte Gesellschaft. Gamification oder Leben im Zeitalter des Computerspiels: 2016.\item Sylvester, Tynan: Designing Games. A Guide to Engineering Experiences: 2013.\item Saklofske, Jon; Belojevic, Nina; Christie, Alex; Sapach, Sonja; Simpson, John: Gaming the Edition: Modelling Scholarly Editions through Videogame Frameworks. In: Digital Literary Studies 1: 2016, S. 15–39.\item Pierazzo, Elena: Digital scholarly editing: theories, models and methods. Farnham: 2015, URL: \url{http://hal.univ-grenoble-alpes.fr/hal-01182162}.\end{itemize}\subsection*{Verweise:}\href{https://gams.uni-graz.at/o:konde.47}{Crowdsourcing}, \href{https://gams.uni-graz.at/o:konde.169}{Social Edition}, \href{https://gams.uni-graz.at/o:konde.41}{Citizen Science}\subsection*{Projekte:}\href{https://www.oldweather.org}{Old Weather}, \href{http://menus.nypl.org}{What's on the menu?}\subsection*{Zitiervorschlag:}Moser, Gerda E.; Galka, Selina. 2021. Gamification. In: KONDE Weißbuch. Hrsg. v. Helmut W. Klug unter Mitarbeit von Selina Galka und Elisabeth Steiner im HRSM Projekt "Kompetenznetzwerk Digitale Edition". URL: https://gams.uni-graz.at/o:konde.89\newpage\section*{Genetische Edition} \emph{Galka, Selina; selina.galka@uni-graz.at }\\
        
    Zum einen handelt es sich bei der Genetischen Edition um eine Editionsform, die sich auf die entstehungsgeschichtlichen Aspekte eines Textes oder Werkes konzentriert; zum anderen ist diese Art der Edition auch wesentlicher Bestandteil einer \href{http://gams.uni-graz.at/o:konde.93}{historisch-kritischen Edition}. Man versucht dabei die Änderungen eines Textes chronologisch festzumachen und gleichzeitig auch zu klassifizieren, also festzustellen, ob es sich um Tilgungen, Einfügungen, Umstellungen etc. handelt. (Kamzelak 2018)\\
            
        Wenn man sich mit der Entstehungsgeschichte eines Textes beschäftigt, kann das Interesse auf die Schreibprozesse des Autors fokussiert werden (vgl. \emph{\href{http://gams.uni-graz.at/o:konde.46}{critique génétique}}) oder auf die Entstehung und Veränderung des Textes selbst, wie es bei der historisch-kritischen Edition der Fall ist, und bei der am Ende oft ein \href{http://gams.uni-graz.at/o:konde.75}{edierter Text}, eine Lesefassung, zur Verfügung gestellt wird. (Kamzelak 2018) Die Textgenese kann auf unterschiedlichen Ebenen betrachtet werden: auf der \href{http://gams.uni-graz.at/o:konde.23}{Makro-}, \href{http://gams.uni-graz.at/o:konde.24}{Meso-} und \href{http://gams.uni-graz.at/o:konde.26}{Mikroebene} (vgl. auch \href{http://gams.uni-graz.at/o:konde.28}{Annotation: Textgenese}). (Kamzelak 2018)\\
            
        Wie die Umsetzung einer genetischen Edition aussieht, hängt vom Textbegriff ab, der ihr zugrunde gelegt wird – dabei kann es sich um einen ‘traditionellen’ Textbegriff handeln, der von einem intendierten Text ausgeht, wobei dieser dann neben einem Variantenapparat steht, oder um einen ‘dynamischen’ Textbegriff, der von Gunter Martens geprägt wurde und die unterschiedlichen Varianten eines Textes als zentral einstuft. (Kamzelak 2018) Im Endeffekt spielen die Varianten eines Textes bei der genetischen Edition eine große Rolle, was dazu führte, dass neue Form des \href{http://gams.uni-graz.at/o:konde.32}{Apparats} entwickelt wurden, wie z. B. die \href{http://gams.uni-graz.at/o:konde.174}{Synopse}. (Kamzelak 2018)\\
            
        Die digitale Umsetzung einer genetischen Edition bietet mehrere Vorteile. Zunächst können die  Varianten eines Textes dynamisch mit variablen Ansichten realisiert werden. Außerdem werden die Texte durch die Modellierung und Auszeichnung mit \href{http://gams.uni-graz.at/o:konde.215}{XML} bzw. \href{http://gams.uni-graz.at/o:konde.178}{TEI} maschinenlesbar, austauschbar und nicht zuletzt ergeben sich dadurch auch neue Auswertungsmöglichkeiten. Die TEI ermöglicht zudem auch die Auszeichnung von textgenetischen Zusammenhängen. (Kamzelak 2018)\\
            
        \subsection*{Literatur:}\begin{itemize}\item Genetische Edition. URL: \url{http://edlex.de/index.php?title=Genetische_Edition}\item Textgenese in der digitalen Edition. Hrsg. von Anke Bosse und Walter Fanta. Berlin: 2019.\item Textgenese und digitales Edieren. Wolfgang Koeppens "Jugend" im Kontext der Editionsphilologie. Hrsg. von Katharina Krüger, Elisabetta Mengaldo und Eckhard Schumacher. Berlin, Boston: 2016.\item Zwerschina, Hermann: Variantenverzeichnung, Arbeitsweise des Autors und Darstellung der Textgenese. In: Text und Edition - Positionen und Perspektiven. Hg. von Rüdiger Nutt-Kofoth, Bodo Plachta, H.T.M. van Vliet und Hermann Zwerschina: 2000, S. 203–230.\item Martens, Gunter; Zeller, Hans: Textgenetische Edition. Tübingen: 1998.\item Texte und Varianten. Probleme ihrer Edition und Interpretation. Hrsg. von Hans Zeller und Gunter Martens. München: 1971.\end{itemize}\subsection*{Verweise:}\href{https://gams.uni-graz.at/o:konde.93}{historisch-kritische Ausgabe / Edition}, \href{https://gams.uni-graz.at/o:konde.46}{critique génétique}, \href{https://gams.uni-graz.at/o:konde.27}{Annotation: Textgenese}, \href{https://gams.uni-graz.at/o:konde.26}{Annotation: Mikrogenese}, \href{https://gams.uni-graz.at/o:konde.24}{Annotation: Mesogenese}, \href{https://gams.uni-graz.at/o:konde.23}{Annotation: Makrogenese}, \href{https://gams.uni-graz.at/o:konde.174}{Synopse}, \href{https://gams.uni-graz.at/o:konde.32}{Apparat}\subsection*{Projekte:}\href{https://beethovens-werkstatt.de}{Beethovens Werkstatt}, \href{https://gams.uni-graz.at/context:skerbisch}{Hartmut Skerbisch: Digitale Edition seiner Notizbücher 1969-2008}, \href{https://fontane-nb.dariah.eu/index.html}{Theodor Fontane: Notizbücher. Digitale genetisch-kritische und kommentierte Edition}\subsection*{Themen:}Einführung, Digitale Editionswissenschaft\subsection*{Lexika}\begin{itemize}\item \href{https://edlex.de/index.php?title=Genetische_Edition}{Edlex: Editionslexikon}\item \href{https://lexiconse.uantwerpen.be/index.php/lexicon/genetic-edition/}{Lexicon of Scholarly Editing}\end{itemize}\subsection*{Zitiervorschlag:}Galka, Selina. 2021. Genetische Edition. In: KONDE Weißbuch. Hrsg. v. Helmut W. Klug unter Mitarbeit von Selina Galka und Elisabeth Steiner im HRSM Projekt "Kompetenznetzwerk Digitale Edition". URL: https://gams.uni-graz.at/o:konde.90\newpage\section*{GeoNames} \emph{Steiner, Christian; christian.steiner@uni-graz.at }\\
        
    \emph{GeoNames.org} ist eine geographische Datenbank, die über
                  verschiedene Webdienste unter einer \href{http://gams.uni-graz.at/o:konde.45}{Creative-Commons}-Lizenz verfügbar und zugänglich ist. Die Datenbank
                  enthält über 25.000.000 geografische Namen mit über 11.800.000 \emph{features}. Alle \emph{features} sind in eine von neun
                  Featureklassen und weiter in einen von 645 Featurecodes unterteilt. Neben den
                  Ortsnamen in verschiedenen Sprachen sind auch Breiten- und Längengrade,
                  Höhenangaben, Bevölkerungszahl, administrative Unterteilung und Postleitzahlen
                  gespeichert. Alle Koordinaten verwenden das \emph{World Geodetic
                     System 1984} (WGS84).\\
            
        Der Kern der Datenbank von \emph{GeoNames.org} wird von offiziellen öffentlichen Quellen bereitgestellt,
                  deren Qualität variieren kann. Über eine Wiki-Benutzeroberfläche werden die
                  Benutzer aufgefordert, die Datenbank manuell zu bearbeiten und zu verbessern,
                  indem sie Namen hinzufügen oder korrigieren, bestehende \emph{features} verschieben oder neue \emph{feature}s
                  hinzufügen.Jedes \emph{feature} von \emph{GeoNames.org} wird als Web-Ressource repräsentiert, die durch einen
                  persistenten URI identifiziert wird. Dieser URI ermöglicht auch den Zugriff auf
                  eine \href{http://gams.uni-graz.at/o:konde.131}{RDF}-Beschreibung des \emph{feature} unter Verwendung von Elementen der \emph{GeoNames}-\href{http://gams.uni-graz.at/o:konde.151}{Ontologie}.\\
            
        \emph{GeoNames} spielt bei der \href{http://gams.uni-graz.at/o:konde.17}{Annotation} von geografischen Angaben in \href{http://gams.uni-graz.at/o:konde.59}{Digitalen Editionen} und im Hinblick
                  auf \emph{\href{http://gams.uni-graz.at/o:konde.8}{Linked Open Data}} eine wichtige Rolle.\\
            
        \subsection*{Literatur:}\begin{itemize}\item GeoNames. URL: \url{https://www.geonames.org/}\item Singh, Sanket Kumar; Rafiei, Davood: Strategies for Geographical Scoping and Improving a
                              Gazetteer. In: Proceedings of the 2018 World Wide Web Conference on
                              World Wide Web - WWW '18 the 2018 World Wide Web Conference. Lyon, France: 2018, S. 1663–1672.\end{itemize}\subsection*{Software:}\href{https://geobrowser.de.dariah.eu/}{Dariah
                           Geobrowser}, \href{geonames.org}{Geonames}, \href{https://nodegoat.net/}{Node Goat}, \href{https://www.w3.org/RDF/}{RDF}\subsection*{Verweise:}\href{https://gams.uni-graz.at/o:konde.147}{Normdaten}, \href{https://gams.uni-graz.at/o:konde.167}{Semantic Web}, \href{https://gams.uni-graz.at/o:konde.109}{GND}, \href{https://gams.uni-graz.at/o:konde.111}{VIAF}, \href{https://gams.uni-graz.at/o:konde.112}{Wikidata}, \href{https://gams.uni-graz.at/o:konde.108}{Getty}\subsection*{Themen:}Annotation und Modellierung\subsection*{Zitiervorschlag:}Steiner, Christian. 2021. GeoNames. In: KONDE Weißbuch. Hrsg. v. Helmut W. Klug unter Mitarbeit von Selina Galka und Elisabeth Steiner im HRSM Projekt "Kompetenznetzwerk Digitale Edition". URL: https://gams.uni-graz.at/o:konde.107\newpage\section*{Gesamtausgabe} \emph{Galka, Selina; selina.galka@uni-graz.at }\\
        
    Eine Gesamtausgabe ist eine Ausgabe sämtlicher Werke einer Person, z. B. eines Autors oder Komponisten. Neben den Werken, die während der Lebenszeit geschaffen wurden, können auch andere Zeugnisse der Person, wie Briefe oder Tagebücher, veröffentlicht werden. (Sahle 2013, S. 189)\\
            
        Je nachdem, wie die Werke wissenschaftlich aufbereitet werden, kann noch weiter zwischen unterschiedlichen Ausgabeformen differenziert werden, wie \href{http://gams.uni-graz.at/o:konde.116}{Leseausgaben} oder \href{http://gams.uni-graz.at/o:konde.93}{historisch-kritischen Ausgaben}. Die Bezeichnungen für unterschiedliche Editionstypen können sich teilweise überschneiden, so kann mit ‘\href{http://gams.uni-graz.at/o:konde.213}{Werkausgabe}’ oder ‘Gesammelte Werke’ auch eine Gesamtausgabe gemeint sein.\\
            
        \subsection*{Literatur:}\begin{itemize}\item Gesamtausgabe. URL: \url{https://de.wikipedia.org/wiki/Gesamtausgabe}\item Göttsche, Dirk: Ausgabentypen und Ausgabenbenutzer. In: Text und Edition - Positionen und Perspektiven. Berlin: 2000, S. 37–64.\item Sahle, Patrick: Digitale Editionsformen. Zum Umgang mit der Überlieferung unter den Bedingungen des Medienwandels. Teil 1: Das typografische Erbe. Norderstedt: 2013.\end{itemize}\subsection*{Verweise:}\href{https://gams.uni-graz.at/o:konde.116}{Leseausgabe}, \href{https://gams.uni-graz.at/o:konde.93}{historisch-kritische Ausgabe}, \href{https://gams.uni-graz.at/o:konde.59}{Digitale Edition}, \href{https://gams.uni-graz.at/o:konde.213}{Werkausgabe}\subsection*{Projekte:}\href{https://weber-gesamtausgabe.de/de/Index}{Carl Maria von Weber: Gesamtausgabe}, \href{https://www.freud-edition.net}{Sigmund Freud Edition: historisch-kritische Ausgabe}, \href{http://www.nietzschesource.org/#eKGWB}{Nietzsche Source - Digitale kritische Gesamtausgabe}, \href{https://www.arendteditionprojekt.de}{Hannah Arendt. Kritische Gesamtausgabe}, \href{https://gams.uni-graz.at/context:ohad}{Ödön von Horváth. Historisch-kritische Ausgabe - Digitale Edition}\subsection*{Themen:}Einführung, Digitale Editionswissenschaft\subsection*{Lexika}\begin{itemize}\item \href{https://edlex.de/index.php?title=Gesamtausgabe}{Edlex: Editionslexikon}\end{itemize}\subsection*{Zitiervorschlag:}Galka, Selina. 2021. Gesamtausgabe. In: KONDE Weißbuch. Hrsg. v. Helmut W. Klug unter Mitarbeit von Selina Galka und Elisabeth Steiner im HRSM Projekt "Kompetenznetzwerk Digitale Edition". URL: https://gams.uni-graz.at/o:konde.91\newpage\section*{Getty Vocabulary Program} \emph{Steiner, Christian; christian.steiner@uni-graz.at}\\
        
    Das \emph{Getty Vocabulary Program} ist eine Abteilung innerhalb des \emph{Getty Research Institute} am \emph{Getty Center} in Los Angeles, Kalifornien. Es erstellt und pflegt die von Getty kontrollierten Vokabularien \emph{Art and Architecture Thesaurus }(AAT), \emph{Union List of Artist Names}  (ULAN) und \emph{Getty Thesaurus of Geographic Names }(TGN). Sie entsprechen den ISO- und NISO-Normen für den Aufbau von Thesauri. \\
            
        Die Getty-Vokabulare sind die wichtigsten Referenzen für die Kategorisierung von Kunstwerken, Architektur, materieller Kultur sowie von Namen von Künstlerinnen und Künstlern, Architektinnen und Architekten sowie geografischen Namen, wobei bei letzteren \emph{\href{http://gams.uni-graz.at/o:konde.107}{GeoNames}} eine immer größere Rolle spielt. \\
            
        Diese Datenbanken sind das Lebenswerk vieler Menschen und sind weiterhin wichtige Beiträge zum Informationsmanagement und zur Dokumentation von Kulturerbe. Sie enthalten Begriffe, Namen und andere Informationen über Menschen, Orte, Dinge und Konzepte in Bezug auf Kunst, Architektur und materielle Kultur. Sie können online auf der Getty-Website kostenlos abgerufen werden. Im Rahmen Digitaler Editionen können die Thesauri als Normdaten zur \href{http://gams.uni-graz.at/o:konde.17}{Annotation} der Editionstexte verwendet werden.\\
            
        \subsection*{Literatur:}\begin{itemize}\item Getty Vocabularies (Getty Research Institute). URL: \url{https://www.getty.edu/research/tools/vocabularies/}\item Siegfried, Susan: An Analysis of Search Terminology Used by Humanities Scholars: The Getty Online Searching Project Report Number 1 An Analysis of Search Terminology Used by Humanities Scholars. In: The Library Quarterly 63: 1993, S. 1-39.\end{itemize}\subsection*{Verweise:}\href{https://gams.uni-graz.at/o:konde.147}{Normdaten}, \href{https://gams.uni-graz.at/o:konde.167}{Semantic Web}, \href{https://gams.uni-graz.at/o:konde.109}{GND}, \href{https://gams.uni-graz.at/o:konde.107}{Geonames}, \href{https://gams.uni-graz.at/o:konde.111}{VIAF}, \href{https://gams.uni-graz.at/o:konde.112}{Wikidata}\subsection*{Themen:}Annotation und Modellierung\subsection*{Zitiervorschlag:}Steiner, Christian. 2021. Getty Vocabulary Program. In: KONDE Weißbuch. Hrsg. v. Helmut W. Klug unter Mitarbeit von Selina Galka und Elisabeth Steiner im HRSM Projekt "Kompetenznetzwerk Digitale Edition". URL: https://gams.uni-graz.at/o:konde.108\newpage\section*{HTR} \emph{Mühlberger, Günter; guenter.muehlberger@uibk.ac.at }\\
        
    Die automatisierte Erkennung von Handschrift (\emph{Handwritten Text
                     Recognition}) beruht in \emph{Transkribus} auf exakt
                  demselben Verfahren bzw. derselben \emph{Engine}, die auch für die
                  Druckschriftenerkennung (\href{http://gams.uni-graz.at/o:konde.149}{OCR})
                  angewendet wird. Allerdings müssen die neuronalen Netze mit einer Reihe
                  zusätzlicher Herausforderungen zurechtkommen, da der Standardisierungsgrad bei
                  Handschriften insgesamt wesentlich geringer ist als bei Druckschrift. Konkret
                  bedeutet dies, dass wesentlich mehr Trainingsdaten notwendig sind, um diese
                  Aufgabe bewältigen zu können. Das macht sich insbesondere bei großen Modellen
                  bemerkbar, die zum Beispiel für eine Epoche von 100 oder mehr Jahren gute
                  Ergebnisse erzielen sollen. \\
            
        Der einfachste Fall, der jedoch bei der Erstellung von digitalen Editionen häufig
                  auftreten wird, ist gegeben, wenn ein Modell für einen einzelnen Schreiber
                  trainiert werden soll. Hier reichen schon relativ wenige Seiten aus, um gute
                  Ergebnisse erzielen zu können. Als Beispiel führen wir die Tagebücher von Andreas
                  Okopenko an, die im Rahmen des KONDE-Projekts in \emph{Transkribus} trainiert wurden. Hier lässt sich auch gut der Fortschritt
                  der letzten Jahre dokumentieren. Das erste Modell, das im Frühjahr 2018 erzeugt
                  wurde, weist eine Fehlerquote von 10,17 Prozent am Validierungsset auf. Mit der
                  weiterentwickelten Engine hingegen wird auf den identischen Trainingsdaten eine
                  Fehlerquote von 3,61 Prozent erreicht. Das Trainingsset besitzt in beiden Fällen
                  20.782 Wörter, geht man von 200 Wörtern pro Seite aus, dann liegen also nicht mehr
                  als ca. 100 Seiten Trainingsmaterial zugrunde.
               \\
            
        Abbildung: Beispiel Texterkennung - Andreas Okopenko: Fehlerquote auf dieser Seite: 1,76%\\
            
        Eines der größten Modelle für historische Handschriften in \emph{Transkribus} wurde vom Nationalarchiv der Niederlande zusammen mit dem
                  Stadtarchiv Amsterdam erstellt. Das Modell umfasst ca. 7.000 Seiten bzw. 1.384.893
                  Wörter und erzielt auf dem Validierungsset 5,67 Prozent. Die zugrundeliegenden
                  Trainingsdaten wurden auf Basis einer Zufallsstichprobe aus mehreren Millionen
                  Seiten des 18. Jahrhunderts ausgewählt. Das Modell enthält hunderte
                  unterschiedliche Schreiber und kann daher mit einer Vielzahl unterschiedlicher
                  Schreibstile umgehen.\\
            
        \\
            
        Abbildung: Beispiel Texterkennung - Niederländisches Dokument\\
            
        Ganz ähnliche Ergebnisse können auch mit den Kurrentmodellen in \emph{Transkribus} erzielt werden, die auf ähnlichen Datenmengen beruhen. Hier
                  wurden mehrere tausend Seiten Kurrentschrift aus dem 17. bis 20. Jahrhundert
                  zugrunde gelegt. Der Schwerpunkt liegt allerdings auf dem späten 19. Jahrhundert.
                  Die Modelle sind in \emph{Transkribus} frei verfügbar.\\
            
        Zusammenfassend lässt sich sagen, dass mit dem Einsatz moderner Methoden der
                  Texterkennung bei historischen Druckschriften nahezu fehlerlose Transkriptionen
                  erzielt werden können. Bei historischen Handschriften sind die Ergebnisse noch
                  deutlich fehlerhafter, trotzdem lassen sich mit überschaubarem Aufwand auch für
                  Handschriften Modelle trainieren, die sowohl die Transkription beschleunigen, als
                  auch die Durchsuchbarkeit großer Dokumentenmengen ermöglichen. \\
            
        \subsection*{Literatur:}\begin{itemize}\item Mühlberger, Günter; Colutto, Sebastian; Kahle, Philipp: Handwritten Text Recognition (HTR) of Historical
                              Documents as a Shared Task for Archivists, Computer Scientists and
                              Humanities Scholars: The Model of a Transcription & Recognition
                              Platform (TRP).\item Strauß, Tobias; Weidemann, Max; Michael, Johannes; Leifert, Gundram; Grüning, Tobias; Labahn, Roger: System Description of CITlab's Recognition &
                              Retrieval Engine for ICDAR2017 Competition on Information Extraction
                              in Historical Handwritten Records. In: arXiv:1804.09943 [cs]: 2018.\item READ-COOP SCE: Public Models in Transkribus: 2020. URL: \url{https://readcoop.eu/transkribus/public-models/}.\end{itemize}\subsection*{Software:}\href{https://readcoop.eu/wp-content/uploads/2018/12/D7.9_HTR_NN_final.pdf}{HTR+}, \href{https://github.com/jpuigcerver/PyLaia}{PyLaia}, \href{https://github.com/tesseract-ocr/}{Tesseract}, \href{https://github.com/tmbarchive/ocropy}{The
                           OCRopus OCR System and Related Software}, \href{https://github.com/githubharald/SimpleHTR}{SimpleHTR}, \href{https://transkribus.eu/Transkribus/}{Transkribus}\subsection*{Verweise:}\href{https://gams.uni-graz.at/o:konde.149}{OCR}, \href{https://gams.uni-graz.at/o:konde.66}{Diplomatische Transkription}, \href{https://gams.uni-graz.at/o:konde.197}{Transkription}, \href{https://gams.uni-graz.at/o:konde.199}{Transkriptionswerkzeuge}\subsection*{Projekte:}\href{https://edition.onb.ac.at/okopenko/context:okopenko/methods/sdef:Context/get}{Tagebücher Andreas Okopenko}, \href{https://www.uibk.ac.at/projects/noscemus/}{Noscemus}, \href{http://newseye.eu/}{Newseye}, \href{https://readcoop.eu/transkribus/}{Transkribus}\subsection*{Themen:}Digitalisierung\subsection*{Zitiervorschlag:}Mühlberger, Günter. 2021. HTR. In: KONDE Weißbuch. Hrsg. v. Helmut W. Klug unter Mitarbeit von Selina Galka und Elisabeth Steiner im HRSM Projekt "Kompetenznetzwerk Digitale Edition". URL: https://gams.uni-graz.at/o:konde.224\newpage\section*{Handschriftenbeschreibung} \emph{Galka, Selina; selina.galka@uni-graz.at }\\
        
    Eine detaillierte und analysierende Handschriftenbeschreibung kann zum besseren
                  Verständnis des überlieferten Textes beitragen, außerdem auch zu dessen Datierung
                  oder Identifizierung. Bei der Beschreibung mittelalterlicher Handschriften gibt es
                  allgemeine Richtlinien, wie z. B. die Berücksichtigung folgender Aspekte:\\
            
        \begin{itemize}\item {                  Standort mit Angabe der Signatur
}\item {Provenienz}\item {Beschreibstoff mit ausführlicher Beschreibung materieller Elemente (z.  B. Wasserzeichen)}\item {Schrift und Schreiberhände}\item {Blatt- und Lagenzählung}\item {Format (angegeben in Höhe x Breite in Millimetern)}\item {                  Einrichtung der Handschrift (z.  B. Spaltengliederung, Zeilenzahl pro Seite, Umfang der Initialen, Absetzung der Verse)
}\item {Ausstattung der Handschrift (z.  B. Bilder, Initialen, Wappen, Ranken, Zierstriche)}\item {Einband}\item {Schreibdialekt (Kamzelak 2016)}\end{itemize}Bei \href{http://gams.uni-graz.at/o:konde.59}{Digitalen Editionen} bzw. \href{http://gams.uni-graz.at/o:konde.137}{Modellierung} und \href{http://gams.uni-graz.at/o:konde.17}{Annotation} von Handschriften wird
                  die Handschriftenbeschreibung auch in den \href{http://gams.uni-graz.at/o:konde.25}{Metadaten} der jeweiligen Handschrift abgebildet. Die
                     \href{http://gams.uni-graz.at/o:konde.178}{TEI} stellt dafür das Element
                     <msDesc> (\emph{manuscript
                     description}) bereit. (TEI: 10 Manuscript Description 2020)\\
            
        \subsection*{Literatur:}\begin{itemize}\item Handschriftenbeschreibung Edlex. URL: \url{http://www.edlex.de/index.php?title=Handschriftenbeschreibung}\item TEI: 10 Manuscript Description. URL: \url{https://www.tei-c.org/release/doc/tei-p5-doc/en/html/MS.html}\item Driscoll, Matthew James: P5MS: A general purpose tags for manuscript
                              description. In: Digital Medievalist 2: 2006.\item Henzel, Katrin: Zur Praxis der Handschriftenbeschreibung. Am Beispiel
                              des Modells der historisch-kritischen Edition von Goethes
                              Faust. In: Vom Nutzen der Editionen. Zur Bedeutung moderner
                              Editorik für die Erforschung von Literatur- und
                              Kulturgeschichte. Berlin, Boston: 2015, S. 75–95.\end{itemize}\subsection*{Verweise:}\href{https://gams.uni-graz.at/o:konde.103}{Kodikologie}, \href{https://gams.uni-graz.at/o:konde.155}{Paläographie}, \href{https://gams.uni-graz.at/o:konde.25}{Metadaten}, \href{https://gams.uni-graz.at/o:konde.179}{TEI msDesc}, \href{https://gams.uni-graz.at/o:konde.59}{Digitale Edition}\subsection*{Themen:}Einführung\subsection*{Software:}\href{https://github.com/leoba/VisColl}{Viscoll}\subsection*{Zitiervorschlag:}Galka, Selina. 2021. Handschriftenbeschreibung. In: KONDE Weißbuch. Hrsg. v. Helmut W. Klug unter Mitarbeit von Selina Galka und Elisabeth Steiner im HRSM Projekt "Kompetenznetzwerk Digitale Edition". URL: https://gams.uni-graz.at/o:konde.92\newpage\section*{Historisch-kritische Ausgabe/Edition} \emph{Galka, Selina; selina.galka@uni-graz.at }\\
        
    Bei der historisch-kritischen Edition oder auch historisch-kritischen Ausgabe
                  handelt es sich um einen Editionstyp, der im 19. Jahrhundert entwickelt und im 20.
                  Jahrhundert weiter spezifiziert wurde. (Plachta 2013, S. 12)\\
            
        Das Attribut ‘historisch’ zielt darauf ab, dass bei diesem Editionstyp der
                  historische Entstehungsprozess eines Textes umfassend dokumentiert und erläutert
                  wird. Das bedeutet, dass sämtliche Materialien eines Textes gesichtet und in Bezug
                  auf ihre Entstehung in Beziehung gesetzt werden. Der ‘kritische’ Aspekt betrifft
                  die \href{http://gams.uni-graz.at/o:konde.192}{Textkritik} – sämtliche
                  Textträger müssen überprüft und ausgewertet werden, um einen \href{http://gams.uni-graz.at/o:konde.75}{edierten Text} zu konstituieren und die Textgenese zu
                  dokumentieren (variante Textteile unterschiedlicher Fassungen, Korrekturschichten,
                  Entstehungsstufen). (Plachta 2013, S. 13f.)\\
            
         Folgende Bestandteile sind laut Plachta elementar für eine historisch-kritische
                  Ausgabe:\\
            
        \begin{itemize}\item {1. Möglichst vollständige Präsentation der Texte}\item {2. Präsentation aller zum Text erhaltenen Textträger in nachvollziehbarer
                     Form}\item {3. Gleichberechtigte Darstellung aller Fassungen eines Textes oder
                     Werkes}\item {4. Nicht alle Text- oder Werkfassungen müssen vollständig repräsentiert
                     werden; es ist möglich, Varianten in einem Variantenapparat zu
                     verzeichnen}\item {5. Wiedergabe der Textentstehung (genetischer Apparat)}\item {6. Abdruck aller Materialien wie z. B. Notizen oder Exzerpte}\item {7. Beschreibung der erhaltenen Textträger bzw. der erschließbar verlorenen
                     Textträger}\item {8. Wiedergabe aller Dokumente zur Entstehung der Textgeschichte}\item {9. Beschreibung der Wirkungsgeschichte eines Textes oder Werkes}\item {10. Kommentierung von Sachbezügen (z. B. aus biographischer oder
                     historischer Perspektive) (Plachta 2013, S. 14f.)}\end{itemize}Die Umsetzung im Rahmen einer \href{http://gams.uni-graz.at/o:konde.59}{Digitalen
                     Edition} ermöglicht neue und erweiterte Analyse- und
                  Darstellungsmöglichkeiten der Texte.\\
            
        \subsection*{Literatur:}\begin{itemize}\item Plachta, Bodo: Editionswissenschaft. Eine Einführung in Methode und
                              Praxis der Edition neuerer Texte. Stuttgart: 2013.\item Sahle, Patrick: Digitale Editionsformen. Zum Umgang mit der
                              Überlieferung unter den Bedingungen des Medienwandels. Teil 1: Das
                              typografische Erbe. Norderstedt: 2013.\item Scheibe, Siegfried: Zu einigen Grundprinzipien einer historisch-kritischen
                              Ausgabe. In: Texte und Varianten. Probleme ihrer Edition und
                              Interpretation. Hg. von Gunter Martens und Hans Zeller: 1971, S. 1-44.\item Göttsche, Dirk: Ausgabentypen und Ausgabenbenutzer. In: Text und Edition - Positionen und Perspektiven. Berlin: 2000, S. 37–64.\item Nutt-Kofoth, Rüdiger: Editionswissenschaft. In: Methodengeschichte der Germanistik: 2009, S. 109–132.\end{itemize}\subsection*{Verweise:}\href{https://gams.uni-graz.at/o:konde.59}{Digitale Edition}, \href{https://gams.uni-graz.at/o:konde.34}{Kommentar}, \href{https://gams.uni-graz.at/o:konde.32}{Apparat}, \href{https://gams.uni-graz.at/o:konde.174}{Synopse}, \href{https://gams.uni-graz.at/o:konde.75}{Editionstext}, \href{https://gams.uni-graz.at/o:konde.116}{Leseausgabe}, \href{https://gams.uni-graz.at/o:konde.173}{Studienausgabe}, \href{https://gams.uni-graz.at/o:konde.213}{Werkausgabe}, \href{https://gams.uni-graz.at/o:konde.91}{Gesamtausgabe}\subsection*{Projekte:}\href{https://www.arthur-schnitzler.de}{Arthur
                           Schnitzler digital: Digitale historisch-kritische Edition (Werke
                           1905-1931)}, \href{http://faustedition.net/}{Faustedition}\subsection*{Themen:}Einführung, Digitale Editionswissenschaft\subsection*{Lexika}\begin{itemize}\item \href{https://edlex.de/index.php?title=Historisch-kritische_Ausgabe}{Edlex: Editionslexikon}\item \href{https://lexiconse.uantwerpen.be/index.php/lexicon/edition-historical-critical/}{Lexicon of Scholarly Editing}\end{itemize}\subsection*{Zitiervorschlag:}Galka, Selina. 2021. Historisch-kritische Ausgabe/Edition. In: KONDE Weißbuch. Hrsg. v. Helmut W. Klug unter Mitarbeit von Selina Galka und Elisabeth Steiner im HRSM Projekt "Kompetenznetzwerk Digitale Edition". URL: https://gams.uni-graz.at/o:konde.93\newpage\section*{Historische Korpora} \emph{Resch, Claudia; claudia.resch@oeaw.ac.at }\\
        
    Historische Korpora sind strukturierte Sammlungen von Daten älterer Sprachstufen, wobei der Begriff ‘historisch’ vage bleibt. Jost Gippert (2015, S. 9) bezieht ihn etwa auf altertümliche, alte oder mittelalterliche, jedenfalls aber auf nicht zeitgemäße Sprachstadien. In der Unterscheidung zu gegenwartssprachlichen Korpora ist, so Carmen Scherer (2014, S. 26), die zeitliche Nähe oder Distanz zur Gegenwart relevant: „Je größer der zeitliche Abstand zwischen der Entstehung eines Textes und der Gegenwart wird, umso eher ist ein Text und das Korpus, in dem er enthalten ist, als historisch einzustufen.“ Verallgemeinernd spricht Claudia Claridge (2008, S. 242) von „periods before the present-day language“ und ergänzt, dass diese etwa eine Generation vor der gegenwärtigen enden.\\
            
        Die Ausgewogenheit von historischen Korpora ist immer von der Überlieferung abhängig und dadurch oftmals eingeschränkt. Im Gegensatz zu gegenwartssprachlichen Sprachdatensammlungen sind historische Korpora meist von geringerem Umfang. Nicht nur muss das Material für den Korpusaufbau meist erst in maschinenlesbare Form gebracht werden, auch bei der \href{http://gams.uni-graz.at/o:konde.17}{Annotation} der historischen Daten stehen Wissenschaftlerinnen und Wissenschaftler vor nicht geringen Herausforderungen: Die variantenreiche und -tolerante Schreibweise vieler älterer Sprachstufen, verbunden mit anderen Flexionsmustern und syntaktischen Konventionen, erschwert die automatische Verarbeitung und Erschließung des Materials, das aus heutiger Sicht häufig zu den \emph{non-standard}-Varietäten gezählt wird. In den letzten Jahren ist deshalb verstärkt versucht worden, spezifische \href{http://gams.uni-graz.at/o:konde.177}{Tagsets} zu entwickeln bzw. bestehende Kategoriensysteme zu erweitern und automatische Annotationsverfahren an ältere Sprachstufen anzupassen.\\
            
        Anhand verlässlich annotierter historischer Korpora lassen sich einerseits bestehende Erkenntnisse überprüfen und andererseits neue syntaktische, semantische, pragmatische, lexikologische oder andere sprachliche Phänomene (auch im diachronen Verlauf) untersuchen. Das Potential historischer Korpora bleibt allerdings bei weitem nicht nur auf die historische Sprachwissenschaft beschränkt, sondern könnte in den kommenden Jahren noch viel stärker für korpusbasierte Ansätze ihrer Nachbardisziplinen genützt werden, wie etwa der Literaturwissenschaft, der Geschichte, der Philosophie oder der Theologie.\\
            
        Einen sehr guten Überblick über die derzeit verfügbaren historischen Korpora bietet die von CLARIN kuratierte "Historical Corpora"-Liste.\\
            
        \subsection*{Literatur:}\begin{itemize}\item New Methods in Historical Corpora. Hrsg. von Paul Bennett, Richard Whitt, Silke Scheible und Martin Durrell. Tübingen: 2013.\item Claridge, Claudia: Historical Corpora. In: Corpus Linguistics. An International Handbook 1. Berlin, Boston: 2008, S. 242–259.\item Gippert, Jost: Preface. In: Historical Corpora. Challenges and Perspectives. Tübingen: 2015, S. 9-12.\item Kroymann, Emil; Thiebes, Sebastian; Lüdeling, Anke; Leser, Ulf: Eine vergleichende Analyse von historischen und diachronen digitalen Korpora. Technical Report 174 des Instituts für Informatik der Humboldt-Universität zu Berlin: 2004. URL: \url{https://www2.informatik.hu-berlin.de/sam/preprint/kroymann174.pdf}.\item Pettersson, Eva; Megyesi, Beáta: The HistCorp Collection of Historical Corpora and Resources. In: Proceedings of the Digital Humanities in the Nordic Countries 3rd Conference. University of Helsinki: 2018, S. 306–320.\item Scherer, Carmen: Korpuslinguistik. Heidelberg: 2014.\item Historische Korpuslinguistik. Hrsg. von Renata Szczepaniak, Stefan Hartmann und Lisa Dücker. Berlin, Boston: 2019.\end{itemize}\subsection*{Verweise:}\href{https://gams.uni-graz.at/o:konde.29}{Annotationsstandards}, \href{https://gams.uni-graz.at/o:konde.48}{Data Mining}, \href{https://gams.uni-graz.at/o:konde.60}{Digitalisierung}, \href{https://gams.uni-graz.at/o:konde.106}{Konkordanz}, \href{https://gams.uni-graz.at/o:konde.115}{Lemmatisierung}, \href{https://gams.uni-graz.at/o:konde.126}{Markup}, \href{https://gams.uni-graz.at/o:konde.25}{Metadaten}, \href{https://gams.uni-graz.at/o:konde.145}{NLP}, \href{https://gams.uni-graz.at/o:konde.146}{Normalisierung}, \href{https://gams.uni-graz.at/o:konde.156}{Part-of-Speech-Tagging}, \href{https://gams.uni-graz.at/o:konde.177}{Tagsets}, \href{https://gams.uni-graz.at/o:konde.198}{Transkriptionsrichtlinien}, \href{https://gams.uni-graz.at/o:konde.211}{Volltextsuche}\subsection*{Projekte:}\href{http://www.deutschestextarchiv.de/}{Deutsches Textarchiv}, \href{https://digipress.digitale-sammlungen.de/}{DigiPress}, \href{https://www.ids-mannheim.de/cosmas2/}{Cosmas II
}, \href{https://acdh.oeaw.ac.at/abacus/}{Austrian Baroque Corpus (ABaC:us)}, \href{http://mhdbdb.sbg.ac.at/}{Mittelhochdeutsche Begriffsdatenbank (MHDBDB)}, \href{https://www.deutschdiachrondigital.de/}{Deutsch Diachron Digital (DDD)}, \href{https://www.linguistics.rub.de/rem/}{Referenzkorpus Mittelhochdeutsch (ReM);}, \href{https://www.linguistik.hu-berlin.de/de/institut/professuren/korpuslinguistik/forschung/ridges-projekt}{RIGDES-Projekt (Register in Diachronic German Science)}, \href{https://korpora.zim.uni-duisburg-essen.de/FnhdC/}{Bonner Frühneuhochdeutsch Korpus (FnhdC)}, \href{https://www.clarin.eu/resource-families/historical-corpora}{CLARIN: Historische Korpora}\subsection*{Themen:}Einführung, Natural Language Processing, Digitale Editionswissenschaft\subsection*{Zitiervorschlag:}Resch, Claudia. 2021. Historische Korpora. In: KONDE Weißbuch. Hrsg. v. Helmut W. Klug unter Mitarbeit von Selina Galka und Elisabeth Steiner im HRSM Projekt "Kompetenznetzwerk Digitale Edition". URL: https://gams.uni-graz.at/o:konde.94\newpage\section*{Hybridedition} \emph{Fritze, Christiane; christiane.fritze@onb.ac.at }\\
        
    Die Hybridedition ist eine wissenschaftliche Edition, die sowohl aus einer \href{http://gams.uni-graz.at/o:konde.59}{Digitalen Edition} als auch aus einer gedruckten Edition besteht. Grundlage für beide Präsentationsmedien sind ein und dieselben Dateien – \href{http://gams.uni-graz.at/o:konde.197}{Transkriptionen}, Register, \href{http://gams.uni-graz.at/o:konde.34}{Kommentare} und \href{http://gams.uni-graz.at/o:konde.32}{Apparate}. Meist ist der digitale Teil einer Hybridedition umfassender als die Druckausgabe, die fallweise ganz bestimmte Funktionen erfüllen soll, beispielsweise eine \href{http://gams.uni-graz.at/o:konde.173}{Studienausgabe} oder Publikumsausgabe des verfügbaren Materials darstellt. Die Druckausgabe hat meist eine eigene Einleitung, die sich von der der Digitalen Edition unterscheidet. Außerdem beinhalten die meisten Druckausgaben der Hybrideditionen faksimilierte Seiten nur zu illustrativen Zwecken, wohingegen die Präsentation digitaler Faksimiles wesentlicher Bestandteil Digitaler Editionen ist.\\
            
        \subsection*{Literatur:}\begin{itemize}\item Kocher, Ursula: Vom Nutzen der Hybridedition. In: editio 33: 2019, S. 82–93.\item Caria, Federico; Mathiak, Brigitte: A Hybrid Focus Group for the Evaluation of Digital Scholarly Editions of Literary Authors. In: Digital Scholarly Editions as Interfaces 12. Norderstedt: 2018, S. 267–285.\item Schopper, Daniel; Wallnig, Thomas; Wang, Victor: “Don’t worry, we are also doing a book!” – A Hybrid Edition of the Correspondence of Bernhard and Hieronymus Pez OSB [submitted to peer review for a volume on digital scholarly editing in Austria, ed. Helmut Klug].\end{itemize}\subsection*{Software:}\href{https://textgrid.de/}{TextGrid}\subsection*{Verweise:}\href{https://gams.uni-graz.at/o:konde.59}{Digitale Edition}, \href{https://gams.uni-graz.at/o:konde.117}{Liste der Hybrideditionen}, \href{https://gams.uni-graz.at/o:konde.208}{Verlage, die Hybrideditionen unterstützen}, \href{https://gams.uni-graz.at/o:konde.138}{Motivation zur Hybridedition}, \href{https://gams.uni-graz.at/o:konde.22}{Lesetext (Hybridedition)}\subsection*{Themen:}Einführung, Digitale Editionswissenschaft\subsection*{Lexika}\begin{itemize}\item \href{https://edlex.de/index.php?title=Hybrid-Edition}{Edlex: Editionslexikon}\end{itemize}\subsection*{Zitiervorschlag:}Fritze, Christiane. 2021. Hybridedition. In: KONDE Weißbuch. Hrsg. v. Helmut W. Klug unter Mitarbeit von Selina Galka und Elisabeth Steiner im HRSM Projekt "Kompetenznetzwerk Digitale Edition". URL: https://gams.uni-graz.at/o:konde.96\newpage\section*{Hybridedition (Lesetext)} \emph{Boelderl, Artur R.; artur.boelderl@aau.at / Fanta, Walter;
                  walter.fanta@aau.at}\\
        
    Als \href{http://gams.uni-graz.at/o:konde.96}{Hybridedition} bezeichnet man
                  eine Edition, die sowohl gedruckte als auch digitale Anteile aufweist; erstere
                  können, müssen aber nicht in Buchform vorliegen, letztere sind in der Regel online
                  verfügbar und zunehmend per \href{http://gams.uni-graz.at/o:konde.152}{Open
                     Access} zugänglich. Die Bezeichnung als solche ist insofern ungenau, als
                  sie medienwissenschaftlich im Grunde die Zusammenführung verschiedener Medien in
                  einem Träger meint, während sie editionswissenschaftlich die Parallelität mehrerer
                  Träger (z. B. Buch und Internet) bedeutet, die sich auf dieselben Quellen
                  beziehen, wobei diese Träger in einem komplementären Verhältnis zueinander stehen,
                  d. h. die digitale (Online-)Komponente ergänzt und übertrifft (\emph{enhanced}) die Print-Komponente, insofern ihre Darstellungs- wie
                  Navigationsmöglichkeiten über die des Drucks hinausgehen. In welcher Weise die
                  Schnittstelle bzw. das Zusammenspiel der verschiedenen Komponenten organisiert
                  wird, ist abhängig vom zu edierenden Material einerseits sowie von den
                  editorischen Zielen andererseits und wird bis dato von den jeweiligen
                  Editionsprojekten unterschiedlich gelöst.\\
            
        So bietet \emph{Musilonline} in Verbindung mit der \href{http://gams.uni-graz.at/o:konde.91}{Gesamtausgabe} in Buchform in zweiter
                  Linie auch den Komfort der wissenschaftlichen Nutzung in einer zeitgemäßen Form:
                  In der Buchausgabe werden die Textstufen und Textschichten des Manuskripts in
                  Gestalt eines emendierten Texts präsentiert, online aber können die Streichungen,
                  Einfügungen und Randanmerkungen des Autors eingesehen werden. Die Schritte zur
                  Textkonstitution auf der Basis der Nachlassmanuskripte (Auswahl, Emendationen,
                  Konjekturen) werden auf diese Weise durch den Druck und die digitale
                  Online-Präsentation der Lesetexte transparent. Das betrifft allerdings nur die
                  Texte des Status ‘Manuskripttyp’ (= Entwurf in der jeweils letzten Fassung). Vom
                  umfangreichen Notizmaterial Musils (ca. 60 % des Nachlasskorpus) wird kein
                  Lesetext erzeugt. Die Generierung der Lesetexte erfolgt auch nicht in einem
                  automatisierten Workflow auf der Basis der \href{http://gams.uni-graz.at/o:konde.17}{TEI-Annotationen}, da es sich bei diesen um
                  transkriptive und nicht prozeduale Markups handelt (vgl. \href{http://gams.uni-graz.at/o:konde.18}{Benutzerschnittstelle}). Die Hybridlösung im Fall von
                     \emph{Musilonline} liefert ein Modell für eine Nachnutzung,
                  deren Druckkomponente in erster Linie auf das breite Publikum und die literarische
                  Lektüre zielt.\\
            
        Auch der Lesetext wird als digitale Repräsentation in \href{http://gams.uni-graz.at/o:konde.215}{XML}/\href{http://gams.uni-graz.at/o:konde.178}{TEI}-Dokumenten hinterlegt. Deren Annotationen stellen dreierlei Verweise
                  her: a) die Werk- oder Kapiteltitel sind mit dem textgenetischen Dossier (TGD)
                  verknüpft; b) die Annotation von Seitenumbrüchen verweist auf Erstausgaben und
                  posthume Editionen sowie auf die Manuskriptseiten der Nachlasstranskription; c)
                  außerdem sind in den aus dem Druck erzeugten Lesetexten Druckvarianten bzw.
                  handschriftliche Korrektureintragungen des Autors annotiert.\\
            
        Bsp. ad a: Musil, Fortsetzung der Druckfahnen-Kapitel 1937-1942, Kapitelprojekt
                     \emph{Nachtgespräch}\\
            
        \begin{verbatim}<div type="chapter-project" xml:id="moe3_fdf_nac"
corresp="tgd.xml#moe3_fdf_nac_3">
<head><seg type="chapter-num">59.</seg>
<seg type="chapter-title">Nachtgespräch</seg></head>\end{verbatim}Bsp. ad b: Musil, \emph{Der Mann ohne Eigenschaften}, Bd. 2\\
            
        \begin{verbatim}Der Tee, zu dem sie sich getroffen hatten, war in ein
<pb ed="#ea_moe2" n="33"/> unregelmäßiges und vorzeitiges 
Abendbrot übergegangen, weil Ulrich übermüdet war
und darum gebeten hatte, denn er wollte früh zu Bett gehn, um sich für
<pb ed="#ra_1978" n="686"/> den nächsten Tag auszuschlafen,\end{verbatim}Bsp. ad c: Musil, \emph{Der Mann ohne Eigenschaften}, Bd. 2\\
            
        \begin{verbatim}ein wenig Zurechtweisung in der unbekümmerten Wahl dieses
<app><lem>Kleidungsstücks</lem><rdg
wit="#ea_moe2">Kleidungstücks</rdg><rdg
wit="#ra_1978">Kleidungstücks</rdg><rdg
wit="#ka_2009">Kleidungstücks</rdg><rdg
wit="#ga_2016_17">Kleidungsstücks</rdg><note type="textcrit"
corresp="#var_case09"/></app>, obwohl das Gefühl, seine Schwester
werde schon\end{verbatim}\subsection*{Literatur:}\begin{itemize}\item Fanta, Walter: Musil online total. In: Forschungsdesign 4.0. Datengenerierung und
                              Wissenstransfer in interdisziplinärer Perspektive. Dresden: 2019, S. 149–179.\end{itemize}Dieser Beitrag wurden im Kontext des FWF-Projekts "MUSIL ONLINE – interdiskursiver Kommentar" 
                  (P 30028-G24) verfasst.\subsection*{Software:}\href{http://oxygenxml.com/}{Oxygen}\subsection*{Verweise:}\href{https://gams.uni-graz.at/o:konde.17}{Annotation (Literaturwissenschaft:
                           grundsätzlich)}, \href{https://gams.uni-graz.at/o:konde.18}{Benutzerschnittstelle (Fokus:
                           Literaturwissenschaft - Bsp. Musil)}, \href{https://gams.uni-graz.at/o:konde.27}{Text/Dokument (Fokus:
                           Literaturwissenschaft - Bsp. Musil)}\subsection*{Projekte:}\href{http://musilonline.at}{Musil Online}\subsection*{Themen:}Digitale Editionswissenschaft\subsection*{Zitiervorschlag:}Boelderl, Artur R.; Fanta, Walter. 2021. Hybridedition (Lesetext). In: KONDE Weißbuch. Hrsg. v. Helmut W. Klug unter Mitarbeit von Selina Galka und Elisabeth Steiner im HRSM Projekt "Kompetenznetzwerk Digitale Edition". URL: https://gams.uni-graz.at/o:konde.22\newpage\section*{Hybridedition: Definition und Motivation} \emph{Wallnig, Thomas; thomas.wallnig@univie.ac.at }\\
        
    Als Hybridedition wird eine editorische Mischform bezeichnet, die sich in zwei grundlegenden Formen artikulieren kann: als Ergänzung analoger durch digitale Materialien und als Kombination derselben. (Sahle 2013, S. 61–62)\\
            
        Der erste Fall meint die Ergänzung einer Publikation in Buchform durch umfangreicheres digital verfügbar gemachtes Material – etwa Bilder, ergänzende \href{http://gams.uni-graz.at/o:konde.34}{Kommentare}, Volltexte und alles dasjenige, was nicht in die Druckversion aufgenommen wird. Der zweite Fall meint eine Kombination aus Druckausgabe und inhaltlich exakt entsprechender digitaler Publikation.\\
            
        Die Motivationen und Anwendungsfälle sind unterschiedlich. Szenarien, die eine Ergänzung analoger durch digitale Materialien nahelegen, sind etwa:\\
            
        \begin{itemize}\item {Editionsvorhaben besonders im literaturwissenschaftlichen Bereich (wie Werkausgaben), bei denen einer digital konzipierten \href{http://gams.uni-graz.at/o:konde.93}{kritischen Edition} für wissenschaftliche Zwecke eine \href{http://gams.uni-graz.at/o:konde.116}{leserinnen- und leserfreundliche Druckausgabe} zur Seite gestellt werden soll; }\item {Editionsvorhaben besonders im historischen Bereich, bei denen die Gesamtfülle an bearbeitetem Material und/oder dessen Beschaffenheit (etwa Bilder) gegen eine vollständige Wiedergabe im Druck sprechen;}\item {Editionsvorhaben, bei denen aus inhaltlichen Gründen eine Auswahl relevanter Abschnitte für die Drucklegung getroffen werden soll;}\item {Editionsvorhaben, bei denen Teile \href{http://gams.uni-graz.at/o:konde.44}{urheberrechtlich} geschützt und daher nicht digital reproduzierbar sind.}\end{itemize}Der zweite Anwendungsfall der Kombination hingegen kann in spezifischen institutionellen Konstellationen plausibel werden, etwa wenn eine Edition in einer Reihe erscheinen soll, die weiterhin auch eine Ausgabe in Druckform anstrebt.\\
            
        In beiden Fällen ist auf das Zusammenwirken der Editorinnen und Editoren mit Repositorium, Verlag und gegebenenfalls Reihenherausgeber und Fördergeber zu achten.\\
            
        \subsection*{Literatur:}\begin{itemize}\item . In: Editionen in der Wissenschaftsgeschichte: A. von Humboldt – G.W. Leibniz: 2016.\item Kocher, Ursula: Vom Nutzen der Hybridedition. In: editio 33: 2019, S. 82–93.\item Sahle, Patrick: Digitale Editionsformen. Zum Umgang mit der Überlieferung unter den Bedingungen des Medienwandels. Teil 2: Befunde, Theorie und Methodik. Norderstedt: 2013.\end{itemize}\subsection*{Verweise:}\href{https://gams.uni-graz.at/o:konde.59}{Digitale Edition}, \href{https://gams.uni-graz.at/o:konde.117}{Liste der Hybrideditionen}, \href{https://gams.uni-graz.at/o:konde.208}{Verlage, die Hybrideditionen unterstützen}, \href{https://gams.uni-graz.at/o:konde.22}{Lesetext Hybridedition}\subsection*{Themen:}Einführung, Digitale Editionswissenschaft\subsection*{Zitiervorschlag:}Wallnig, Thomas. 2021. Hybridedition: Definition und Motivation. In: KONDE Weißbuch. Hrsg. v. Helmut W. Klug unter Mitarbeit von Selina Galka und Elisabeth Steiner im HRSM Projekt "Kompetenznetzwerk Digitale Edition". URL: https://gams.uni-graz.at/o:konde.138\newpage\section*{Hybridedition: Verlagspartizipation und Förderung} \emph{Wallnig, Thomas; thomas.wallnig@univie.ac.at }\\
        
    Grundsätzlich stellt der Druck eines am Computer bearbeiteten Textes stets eine Re-Analogisierung dar, weil bereits maschinenlesbar gemachte Inhalte ihrer diesbezüglichen Funktionalitäten wieder benommen werden. (Sahle 2013, S. 61)\\
            
        Der Abschnitt des Workflows zwischen inhaltlicher Fertigstellung eines Files (oft anachronistisch ‘Manuskript’ genannt) und der Publikation ist oft unterdefiniert. Verlage haben in den vergangenen Jahrzehnten im Bereich Lektorat, Einrichtung und Satz tendenziell Verantwortung und Arbeit abgegeben, gleichzeitig ist diese Arbeit, die oft von den Forscherinnen und Forschern selbst geleistet wird, in den öffentlichen Publikationsförderungen mit eingepreist.\\
            
        Bislang beschränkt sich bei öffentlicher Förderung die Vorgabe darauf, dass die Publikation \href{http://gams.uni-graz.at/o:konde.152}{open access}-zugänglich sein muss, etwa in der \emph{e-book library} des FWF. Es ist absehbar, dass Ähnliches in Zukunft auch die Forschungsdaten selbst betreffen wird, möglicherweise ebenso die mit öffentlichen Mitteln geschriebene Forschungssoftware. Diese Bereiche sind voneinander im Hinblick auf Verantwortung, Bepreisung und \href{http://gams.uni-graz.at/o:konde.119}{Lizenzierung} klar zu scheiden, wenn im Rahmen eines Editionsvorhabens gemeinsam mit einem Verlag bzw. mit einer Fördereinrichtung Konzepte erstellt werden.\\
            
        Die \href{http://gams.uni-graz.at/o:konde.117}{Liste der Hybrideditionen} zeigt, dass mehrere Verlage in diesem Bereich engagiert und tätig sind. Die Prioritäten liegen hier auf der Wahrung kommerzieller Interessen unter Berücksichtigung der Vorgaben öffentlicher Forschungsförderung (vgl. z. B. Repository Policy von de Gruyter). Wesentlich ist, ob der Verlag über eine adäquate Präsentationsplattform für \href{http://gams.uni-graz.at/o:konde.59}{Digitale Editionen} verfügt; ebenso wichtig ist eine adäquate Strategie für die \href{http://gams.uni-graz.at/o:konde.6}{Langzeitarchivierung} der Daten. Viele dieser Fragen sind zu Beginn des Jahres 2020 erst Gegenstand von Diskussion. (vgl. Weiß 2020)\\
            
        Publikationsförderung erfolgt in Österreich zumeist durch öffentliche Einrichtungen, an erster Stelle durch den Fonds zur Förderung der Wissenschaftlichen Forschung (FWF), der auch eine eigene Förderschiene für „neue digitale Publikationsformate“ betreibt. Häufig beteiligen sich auch akademische Einrichtungen (etwa unterschiedliche Ebenen der universitären Verwaltung) an der Förderung akademischer Publikationen, daneben können auch bei den Landesregierungen der österreichischen Bundesländer oder bei der Österreichischen Forschungsgemeinschaft Mittel eingeworben werden. \\
            
        \subsection*{Literatur:}\begin{itemize}\item Repository Policy. URL: \url{https://www.degruyter.com/page/repository-policy}\item Schopper, Daniel; Wallnig, Thomas; Wang, Victor: “Don’t worry, we are also doing a book!” – A Hybrid Edition of the
                              Correspondence of Bernhard and Hieronymus Pez OSB [submitted to peer review for a
                              volume on digital scholarly editing in Austria, ed. Helmut Klug].\item Sahle, Patrick: Digitale Editionsformen. Zum Umgang mit der Überlieferung unter den Bedingungen des Medienwandels. Teil 2: Befunde, Theorie und Methodik. Norderstedt: 2013.\item Weiß, Philipp: Tagungsbericht: Digitales Edieren in der Klassischen Philologie, 25.09.2019 – 27.09.2019. In: H-Soz-Kult: 2020.\end{itemize}\subsection*{Verweise:}\href{https://gams.uni-graz.at/o:konde.96}{Hybridedition}, \href{https://gams.uni-graz.at/o:konde.59}{Digitale Edition}, \href{https://gams.uni-graz.at/o:konde.117}{Liste der Hybrideditionen}, \href{https://gams.uni-graz.at/o:konde.138}{Motivation zur Hybridedition}, \href{https://gams.uni-graz.at/o:konde.22}{Lesetext Hybridedition}\subsection*{Projekte:}\href{https://www.fwf.ac.at/de/forschungsfoerderung/fwf-programme/selbststaendige-publikationen}{FWF Publikationsförderung}\subsection*{Themen:}Digitale Editionswissenschaft\subsection*{Zitiervorschlag:}Wallnig, Thomas. 2021. Hybridedition: Verlagspartizipation und Förderung. In: KONDE Weißbuch. Hrsg. v. Helmut W. Klug unter Mitarbeit von Selina Galka und Elisabeth Steiner im HRSM Projekt "Kompetenznetzwerk Digitale Edition". URL: https://gams.uni-graz.at/o:konde.208\newpage\section*{Hörspieledition} \emph{Raunig, Elisabeth; elisabeth.raunig@uni-graz.at}\\
        
    Audioeditionen bzw. Hörpieleditionen sind bis dato in der Editionsphilologie
                  unterrepräsentiert (vgl. Krebs 2018, S. 220 und Nutt-Kofoth 2019, S.
                     184–185). haben laut Bernhart die Aufgabe, für die “Sichtung,
                  Erschließung, Kommentierung, Herausgabe und Zugänglichmachung auditiver Quellen”
                     (Bernhart 2013, S. 122) zu sorgen. Diese Aufgaben sind
                  unzweifelhaft auch Teil der digitale Hörspiel-/Audioedition; im Zentrum steht
                  dabei die Gegenüberstellung von Text und Ton, die sich vor allem auch in den text-
                  bzw. tonzentierten Forschungsansätzen widerspiegelt (vgl. Huwiler
                     2005). \\
            
        Hörspieltext kann in zwei Varianten vorliegen: als Typoskript oder Renotat. Fast
                  jedem Hörspiel liegt ein Typoskript zugrunde; wenn das nicht der Fall ist, wird in
                  der Forschung von \emph{born-aurels} gesprochen (Bernhart
                     2013, S. 122). In diesem Fall sollte Döhls Forderung nach einer
                  Renotation gefolgt werden, also einem Transkript, das dem genauen Wortlaut des
                  gesendeten Hörspiels folgt. Ein anderer Grund für die Erstellung eines Renotats
                  kann natürlich der Verlust eines Typoskripts sein. Der Ton sollte im Rahmen von
                  digitalen Hörspieleditionen als qualitativ hochwertiges, archivfähiges Digitalisat
                  (vgl. \href{http://gams.uni-graz.at/o:konde.121}{Audio- bzw. Musik- und
                     Videoformate}) des Hörspieltonträgers vorliegen, sodass auch Analysen des
                  Audiosignals vorgenommen werden können.\\
            
        Für die Gestaltung der Benutzeroberfläche einer digitalen Audioedition müssen die
                  Forschungsziele sowie die vorhandenen Daten berücksichtigt werden, generell sollte
                  eine digitale Audioedition aber immer die Audiodatei der Rundfunkproduktionen (den
                  Ton), das Renotat und / oder das Typoskript des Hörspiels beinhalten.
                  Unterschiedliche konzeptionelle und grafische Darstellungen scheinen, abgestimmt
                  auf die Daten, sinnvoll:\\
            
        \begin{itemize}\item {synoptische Darstellung von Renotat und Audio;}\item {synoptische Vergleichsdarstellung von Typoskript und Renotat;}\item {weitere Forschungsdaten können zusätzlich noch in alle synoptischen
                     Darstellungen; integriert werden}\item {Audioplayeroberfläche für die Rezeption des Audios.}\end{itemize}Bei der Darstellung sollte auch darauf geachtet werden, wie die unterschiedlichen
                  Medien verbunden werden können, z.B. verläuft Text im Medium Internet generell
                  vertikal, wohingegen eine Audiospur im digitalen Medium immer horizontal
                  dargestellt wird. \\
            
        Die Verknüpfung der Audiospur mit dem Text sollte über entsprechende Zeitstempel
                  des Audios, die in eine TEI-Repräsentation des Textes eingearbeitet werden,
                  realisiert werden. Als Basis für dieses TEI-XML-Dokument kann das TEI-Modul 7
                  „Performance Texts“ dienen. Ein Dokument kann dabei sowohl Typoskript als auch
                  Renotat abbilden, die z. B. in je einem <div>-Element
                  modelliert werden. Das Typoskript kann dabei strukturell wie ein Dramentext
                  behandelt werden und mit Hilfe der Elemente <stage>
                  (Bühnenanweisung), <sound> (Toneffekte/Musik) und
                     <sp> und den Kindelementen von <sp>
                  (Sprechakt), <speaker> (Sprecher/Rolle),
                     <stage> und <p> (Sprechtext), umgesetzt
                  werden. Bei der Erstellung des Renotats, das die Sprechtexte wiedergibt sowie alle
                  anderen Audioereignisse beschreibt, werden in zuvor definierten Tonabschnitte mit
                  Zeitstempel Sprechertext als \emph{utterances} mit dem Element
                     <u> modelliert bzw. alle Ton-Vorkommnisse, die nicht als
                  Sprache identifiziert werden können, mit dem Element
                  <incident>. Sowohl <u> also auch
                     <incident> können mit den Attributen @start und @end für die Zeitstempel, sowie @xml:id versehen werden. Diese IDs dienen zur Alignierung von
                  Typoskript und Renotat; dafür können auch zusätzlich
                  <seg>-Elemente verwendet werden, wie in folgendem Beispiel
                  dargestellt:\\
            
        \begin{verbatim}<div type="typoscript">
    <sp who="#Sprecherin">
        <speaker>Sprecherin</speaker>
        <p>
            <seg corresp="#TS_02_d1e46">Der dramatisierte Schundroman.</seg>
        </p>
    </sp>
</div>\end{verbatim}\begin{verbatim}<div type="renotat">
    <u end="3.2575641514454574" start="0.928676185396729" who="#sprecherin"
    xml:id="TS_02_d1e46">Der dramatisierte Schundroman.</u>
</div>\end{verbatim}\subsection*{Literatur:}\begin{itemize}\item Bernhart, Toni: Audioedition. Auf dem Weg zu einer Theorie. In: Medienwandel/Medienwechsel in der
                              Editionswissenschaft: 2013, S. 121–128.\item Döhl, Reinhard: Hörspielphilologie? In: Jahrbuch der Dt. Schillergesellschaft 26: 1982, S. 489–511.\item Huwiler, Elke: Storytelling by Sound. A Theoretical Frame for Radio
                              Drama Analysis. In: The Radio Journal International Studies in Broadcast and
                              Audio Media 3: 2005, S. 45–59.\item Krebs, Sophia Victoria: Kritische Audio-Edition: Interdisziplinäre Fachtagung an
                              der Bergischen Universität Wuppertal 12.-14. Juli 2018. In: editio 32: 2018, S. 220–223.\item Nutt-Kofoth, Rüdiger: Plurimedialität, Intermedialität, Transmedialität.
                              Theoretische, methodische und praktische Implikationen einer Text-Ton-
                              Film-Edition von Alfred Döblins Berlin-Alexanderplatz-Werkkomplex
                              (1929–1931). In: Aufführung und Edition. Berlin, Boston: 2019.\item TEI. 7 Performace Texts. URL: \url{https://tei-c.org/release/doc/tei-p5-doc/en/html/DR.html}\end{itemize}\subsection*{Software:}\href{http://www.fon.hum.uva.nl/praat/}{Praat}\subsection*{Verweise:}\href{https://gams.uni-graz.at/o:konde.121}{Audio- bzw. Musik- und
                           Videoformate}, \href{https://gams.uni-graz.at/o:konde.59}{Digitale Edition}, \href{https://gams.uni-graz.at/o:konde.174}{Synopse}\subsection*{Projekte:}\subsection*{Themen:}Annotation und Modellierung, Digitale Editionswissenschaft\subsection*{Zitiervorschlag:}Raunig, Elisabeth. 2021. Hörspieledition. In: KONDE Weißbuch. Hrsg. v. Helmut W. Klug unter Mitarbeit von Selina Galka und Elisabeth Steiner im HRSM Projekt "Kompetenznetzwerk Digitale Edition". URL: https://gams.uni-graz.at/o:konde.95\newpage\section*{IPTC-IIM-Standard} \emph{Klug, Helmut W.; helmut.klug@uni-graz.at }\\
        
    IPTC-IIM (\emph{International Press Telecommunications Council Information Interchange Model}) ist ein Standard zur Beschreibung von bestimmten ‘Daten-Objekten’ bzw. Datencontainern (z. B. \href{http://gams.uni-graz.at/o:konde.122}{Bildformate} wie JPG, TIFF, BMP, Audio- und Videodateien wie AIFF, MP3, AVI, Textdateien wie HTML, oder Containerdateien wie ZIP, TAR) mit Metadaten, bei dem diese Metadaten direkt in den Dateien mitgespeichert werden. Dabei werden sowohl bestimmte Beschreibungsfelder als auch die technische Lösung zur Verspeicherung vorgegeben; die Metadaten können aber auch im XMP-Format gespeichert werden. \href{http://gams.uni-graz.at/o:konde.81}{XMP} wurde von \emph{Adobe} entwickelt: Dabei werden dieselben Metadaten im \href{http://gams.uni-graz.at/o:konde.131}{RDF}-Standard in die Binärdateien geschrieben.\\
            
        Die Beschlagwortung zielt inhaltlich vorrangig auf die Beschreibung von Fotografien (Urheber, Verwertungsrechte, Überschrift, Schlagwörter) ab. IPTC wird vorwiegend im professionellen Bildjournalismus verwendet, sehr viele Bildbearbeitungsprogramme und Social Media-Dienste verarbeiten diese Metadaten. Der Vorteil der Verspeicherung von Metadaten auf diese Weise ist, dass die Metadaten direkt in den Daten vorhanden sind, der Nachteil, dass bei Zerstörung der Daten, gleichzeitig auch die Metadaten verloren sind.\\
            
        \subsection*{Literatur:}\begin{itemize}\item IPTC: IPTC - NAA Information Interchange Model Version 4: 2014. URL: \url{https://www.iptc.org/std/IIM/4.2/specification/IIMV4.2.pdf}.\item IPCT: IPTC Standard Information Interchange Model (IIM. )IIM Schema for XMP. Specification Version 1.0: 2008. URL: \url{https://www.iptc.org/std/IIM/4.1/specification/IPTC-IIM-Schema4XMP-1.0-spec_1.pdf}.\end{itemize}\subsection*{Verweise:}\href{https://gams.uni-graz.at/o:konde.81}{EXIF/XMP}, \href{https://gams.uni-graz.at/o:konde.131}{RDF}, \href{https://gams.uni-graz.at/o:konde.122}{Bildformate}, \href{https://gams.uni-graz.at/o:konde.124}{Metadatenformate für Bilddateien}\subsection*{Themen:}Digitalisierung, Archivierung, Annotation und Modellierung\subsection*{Zitiervorschlag:}Klug, Helmut W. 2021. IPTC-IIM-Standard. In: KONDE Weißbuch. Hrsg. v. Helmut W. Klug unter Mitarbeit von Selina Galka und Elisabeth Steiner im HRSM Projekt "Kompetenznetzwerk Digitale Edition". URL: https://gams.uni-graz.at/o:konde.101\newpage\section*{Informationsarchitektur} \emph{Bürgermeister, Martina; martina.buergermeister@uni-graz.at }\\
        
    Eine Informationsarchitektur ist dafür verantwortlich, dass das dargestellte
                  Wissen \href{http://gams.uni-graz.at/o:konde.59}{Digitaler Editionen} von
                  Benutzerinnen und Benutzern verstanden werden kann. Jedes digitale Editionsprojekt
                  kommt nach einem iterativen \href{http://gams.uni-graz.at/o:konde.56}{Designprozess} zu einer Informationsarchitektur. Deren Informations- und
                  Funktionsmuster sind ausschlaggebend für eine erfolgreiche \emph{\href{http://gams.uni-graz.at/o:konde.207}{User Experience}}.\\
            
        Informationsarchitektur als Forschungsbereich beschäftigt sich mit der
                  benutzerorientierten Aufbereitung von Inhalten. Sie hat sich aus einer Vielzahl
                  von wissenschaftlichen Disziplinen entwickelt, wie Bibliothekswesen, Journalismus
                  und Kommunikationswissenschaften. Zentral ist auch in diesen Disziplinen das
                  Ordnen, Gruppieren, Sortieren und Präsentieren von Inhalten. In der
                  Informationsarchitektur passiert das mit dem Ziel, zu einem besseren Verständnis
                  des menschlichen Handelns und Denkens zu gelangen. Werden diese Einsichten auf die
                  Struktur der Informationsressource übertragen und über das \href{http://gams.uni-graz.at/o:konde.98}{Interface} kommuniziert, steht einer positiven User
                  Experience nichts im Weg. (Garrett 2011, S. 78–105; Burkhard 2008; Ding/Lin
                     2010, S. 1-7)\\
            
        Der Terminus ‘Informationsarchitektur’ wurde von dem Architekten Richard Saul
                  Wurman (1997) Mitte der 1970er-Jahre eingeführt. Damals konzentrierte
                  man sich der Bereich auf Informationsvisualisierung, erst Mitte der 1990er wurden
                  Ordnungssysteme und -strukturen zum Schwerpunkt der Informationsarchitektur.
                     (Resmini/Rosati 2012, 33–46) Eine breite Definition dieser
                  Disziplin liefern Wei Ding und Xia Lin:\\
            
        Information architecture is about organizing and simplifying information,
                     designing, integrating and aggregating information spaces/systems; creating
                     ways for people to find, understand, exchange and manage information; and,
                     therefore, stay on top of information and make right decisions. […] Finally,
                     the goal of IA design is not only to support people to find information but to
                     manage and use information.(Ding/Lin 2010, S. 2)\\
            
        Der Entwurf einer Informationsarchitektur für eine Informationsressource ist ein
                  mehrstufiger, iterativer Prozess. Am Anfang stehen Nachforschungen zum Inhalt, zu
                  Stakeholdern und zur Zielgruppe. Die Ergebnisse der Recherche führen zu einer
                  Projektstrategie, die die Designphase einleitet. In der Designphase werden
                  Architekturdiagramme und \emph{wireframes} erstellt, die einen
                  ersten Entwurf des User-\href{http://gams.uni-graz.at/o:konde.99}{Interface}
                  und notwendiger \href{http://gams.uni-graz.at/o:konde.25}{Metadaten} sowie
                  kontrollierter Vokabularien beinhalten. Danach werden die Entwürfe evaluiert. Die
                  Evaluation sorgt für ein besseres Verstehen der Benutzerbedürfnisse und führt
                  wiederum zum Redesign der Anwendung. (Ding/Lin 2010, S. 23–40)
                  Gängige Techniken der Evaluation sind Clickstream-Analysen, die Entwicklung von
                  Personas und Usabilitytests.\\
            
        \subsection*{Literatur:}\begin{itemize}\item Burkhard, Remo Aslak: Informationsarchitektur. In: Kompendium Informationsdesign. Berlin: 2008, S. 303–320.\item Ding, Wei; Lin, Xia: Information Architecture. The Design and Integration of
                              Information Spaces: 2010.\item Garrett, Jesse James: The Elements of User Experience: user-centered design
                              for the Web and beyond: 2011.\item Resmini, Andrea; Rosati, Luca: A Brief History of Information Architecture. In: Journal of Information Architecture 2: 2012, S. 33–46.\item Rosenfeld, Louis; Morville, Peter: Information Architecture for the World Wide Web: 2002.\item Wurman, Richard Saul: Information Architects: 1997.\end{itemize}\subsection*{Verweise:}\href{https://gams.uni-graz.at/o:konde.206}{User Testing}, \href{https://gams.uni-graz.at/o:konde.207}{User-centered Design}, \href{https://gams.uni-graz.at/o:konde.98}{Interface}, \href{https://gams.uni-graz.at/o:konde.99}{Interface-Design-Cycle}, \href{https://gams.uni-graz.at/o:konde.164}{Responsive Design}, \href{https://gams.uni-graz.at/o:konde.35}{Barrierefreies Design}, \href{https://gams.uni-graz.at/o:konde.56}{Design}, \href{https://gams.uni-graz.at/o:konde.205}{Usability}\subsection*{Themen:}Interfaces, Digitale Editionswissenschaft\subsection*{Zitiervorschlag:}Bürgermeister, Martina. 2021. Informationsarchitektur. In: KONDE Weißbuch. Hrsg. v. Helmut W. Klug unter Mitarbeit von Selina Galka und Elisabeth Steiner im HRSM Projekt "Kompetenznetzwerk Digitale Edition". URL: https://gams.uni-graz.at/o:konde.97\newpage\section*{Interdiskursivität (Fokus: Literaturwissenschaft – Bsp. Musil)} \emph{Boelderl, Artur R.; artur.boelderl@aau.at / Fanta, Walter; walter.fanta@aau.at
                  / Mader, Franziska; franziska.mader@aau.at}\\
        
    Engen traditionelle Formen der Kommentierung das jeweilige Textkorpus in seiner
                  Bedeutungsvielfalt eher ein, so zielt die Annotation von Interdiskursivität
                  darauf, dessen vielschichtige Bedeutungsebenen in einer nicht einschränkenden,
                  nicht urteilenden Weise im Rahmen der Kommentierung an die User einer Digitalen
                  Edition zu vermitteln. Dazu werden die in der literaturwissenschaftlichen
                  Forschung bereits nachgewiesenen Intertextualitätszusammenhänge in eine
                  Kommentarstruktur gebracht, die es erlaubt, Antworten auf die Fragen zu geben,
                  welche sich bei der Textlektüre stellen. \\
            
        Interdiskursivität bezeichnet dabei (im Anschluss an Foucaults Diskursanalyse
                     (1973 \& 1991) und deren Rezeption bei Jürgen Link
                     (1997)) jene zwischen einzelnen (Spezial-)Diskursen vermittelnde
                  Dimension des Kommentars, die den edierten Texten, insbesondere solchen der
                  literarischen Moderne, auch selbst eignet und an die der \href{http://gams.uni-graz.at/o:konde.34}{Kommentar} unmittelbar anschließen kann, denn: „In
                  den literarischen [Inter-]Diskurs Musils und Brochs gehen andere Diskurse […] \emph{auf sehr reflektierte Weise} ein [...].“ (Martens 2006,
                     S. 46, kursiv i. O.) Diese anderen Diskurse werden als Referenztexte
                  (jedenfalls als bibliographischer Nachweis, nach technischen und rechtlichen
                  Möglichkeiten darüber hinaus als Textauszüge bzw. Volltexte) in die \href{http://gams.uni-graz.at/o:konde.59}{Digitale Edition} eingebunden,
                  wodurch ein Netz von Verweisungszusammenhängen entsteht, welches User entsprechend
                  ihren eigenen Lektüreverläufen und Erkenntnisinteressen durchsuchen und abrufen
                  können. \\
            
        Abgelegt werden die Verweisungszusammenhänge im analog zum textgenetischen Dossier
                  (TGD) benannten interdiskursiven Dossier (IDD). Dort werden unter
                     <profileDesc> als diskursive Knotenpunkte Begriffe
                     (<term>) definiert und mit XML-IDs versehen, die sich nach
                  Bezugsgröße in Themen (\emph{subjects}) und Diskurse (\emph{discourses}) differenzieren und, wo möglich und sinnvoll, an
                  Normdatenbanken angegliedert werden (Bsp. 1). Auf diese wird sodann im
                     <text> unter Zuweisung zum jeweiligen Referenztext verwiesen
                  (Bsp. 2).\\
            
        Bsp. 1:\\
            
        \begin{verbatim}<profileDesc>
    <textClass>
        <classCode scheme="#GND">
            <term xml:id="s001" type="subject">Edition</term>
        </classCode>
    </textClass>
</profileDesc>\end{verbatim}Bsp. 2:\\
            
        \begin{verbatim}<text>
    <body>
        <div>
            <head>"über Musil"</head>
            <div resp="#FM" corresp="#s001">
                <head>Wilhelm Bausinger: Studien zu einer historisch-kritischen Ausgabe
                von Robert Musils Roman "Der Mann ohne Eigenschaften"</head>
                <p><quote sameAs="bib.xml#bib_bausinger_studien_1964"
                source="bausinger_studie_1964#pb_002">"Im fiktionalen Zusammenhang des 
                Romans hat die Wendung, die in der "Hasenkatastrophe" eher Metapher, 
                eigentlicher Ausdruck eines Gefühls des Feuilletonisten war, für den 
                Frauenmörder Moosbrugger direkte und tatsächliche, eigentliche 
                Bedeutung gewonnen. Auf diese Weise ist ihr bedrohlicher Gehalt 
                manifest geworden; und Musil kann sie (ob in bewußtem Gedanken an 
                die Stelle in seinem Roman, darf hingestellt bleiben) in dem 
                Feuilleton nicht stehen lassen."</quote> 
                <ref corresp="musil_nzl_1935.xml#nzl_bil_hasmusil_moe"/></p>
            </div>
        </div>
    </body>
</text>\end{verbatim}\subsection*{Literatur:}\begin{itemize}\item Foucault, Michel: Archäologie des Wissens. Frankfurt am Main: 1973.\item Foucault, Michel: Die Ordnung des Diskurses. Hrsg. von  und Ralf Konersmann. Frankfurt am Main: 1991.\item Link, Jürgen: Literaturwissenschaftliche Grundbegriffe: eine
                              programmierte Einführung auf strukturalistischer Basis Literaturwissenschaftliche Grundbegriffe. München: 1997.\item Martens, Gunther: Beobachtungen der Moderne in Hermann Brochs Die
                              Schlafwandler und Robert Musils Der Mann ohne Eigenschaften:
                              rhetorische und narratologische Aspekte von Interdiskursivität Beobachtungen der Moderne in Hermann Brochs Die
                              Schlafwandler und Robert Musils Der Mann ohne Eigenschaften. München: 2006.\end{itemize}Dieser Beitrag wurden im Kontext des FWF-Projekts "MUSIL ONLINE – interdiskursiver Kommentar" 
                  (P 30028-G24) verfasst.\subsection*{Software:}\href{http://oxygenxml.com/}{Oxygen}\subsection*{Verweise:}\href{https://gams.uni-graz.at/o:konde.17}{Annotation (Literaturwissenschaft:
                           grundsätzlich)}, \href{https://gams.uni-graz.at/o:konde.28}{Textgenese}, \href{https://gams.uni-graz.at/o:konde.23}{Makrogenese (Fokus:
                           Literaturwissenschaft - Bsp. Musil)}, \href{https://gams.uni-graz.at/o:konde.24}{Mesogenese (Fokus:
                           Literaturwissenschaft - Bsp. Musil)}, \href{https://gams.uni-graz.at/o:konde.26}{Mikrogenese (Fokus:
                           Literaturwissenschaft - Bsp. Musil)}, \href{https://gams.uni-graz.at/o:konde.96}{Hybridedition}\subsection*{Projekte:}\href{http://musilonline.at}{Musil Online}\subsection*{Themen:}Annotation und Modellierung, Digitale Editionswissenschaft\subsection*{Zitiervorschlag:}Boelderl, Artur R.; Fanta, Walter; Mader, Franziska. 2021. Interdiskursivität (Fokus: Literaturwissenschaft – Bsp.
               Musil). In: KONDE Weißbuch. Hrsg. v. Helmut W. Klug unter Mitarbeit von Selina Galka und Elisabeth Steiner im HRSM Projekt "Kompetenznetzwerk Digitale Edition". URL: https://gams.uni-graz.at/o:konde.19\newpage\section*{Interface} \emph{Rieger, Lisa; lrieger@edu.aau.at }\\
        
    Gemäß allgemeiner Definition ist ein Interface „the point where two subjects,
                  systems, etc. meet and affect each other” (Oxford Learner’s
                     Dictionaries) oder „the point at which two people, things, or events
                  meet or affect each other” (Cambridge Dictionary). \\
            
        Hinsichtlich eines Computers bedeutet dies Folgendes: Ein IT-System besteht aus
                  unterschiedlichen Komponenten mit jeweils eigenen Funktionen. Damit diese
                  Komponenten miteinander kommunizieren können, gibt es Schnittstellen (\emph{Interfaces}), über die Daten ausgetauscht und verarbeitet
                  werden. Sie können in drei große Gruppen unterteilt werden:
                  Hardwareschnittstellen, Softwareschnittstellen und Benutzerschnittstellen.
                  Hardwareschnittstellen sorgen für die Kompatibilität einzelner Hardwareteile
                  (Tastatur, Maus, Bildschirm etc.) untereinander. An den Softwareschnittstellen
                  tauschen einzelne Software-Komponenten Daten und Befehle aus.
                  Benutzerschnittstellen wiederum ermöglichen einem Menschen die Bedienung des
                  Computers. (SoftSelect)\\
            
        Gerade bei \href{http://gams.uni-graz.at/o:konde.59}{Digitalen Editionen} hat
                  die Benutzerschnittstelle (\emph{User Interface}) einen
                  wesentlichen Einfluss auf die subjektive Wahrnehmung der Qualität einer Anwendung,
                  da sie durch ihre Gestaltung und ihren Aufbau die Bedienung erheblich erleichtern
                  oder erschweren kann. Daher ist die graphische Benutzeroberfläche (\emph{Graphical User Interface}, GUI), mit der der User über
                  Fenster und Pop-Up-Menüs mit der Software kommuniziert, heutzutage Standard. Es
                  gibt aber auch die Möglichkeit, über Texteingabe (\emph{Text
                     Interface}) oder Sprachbefehle (\emph{Voice User
                  Interface}) Daten oder Befehle mit einem Programm auszutauschen.
                     (Gründerszene) Für Digitale Editionen sind außerdem APIs (\emph{Application Programming Interfaces}) von großer Bedeutung, da
                  so Daten auch automatisiert abgefragt und ausgetauscht werden können.\\
            
        \subsection*{Literatur:}\begin{itemize}\item Interface. In: Cambridge Dictionary.\item Interface. In: Duden.\item User Interface. URL: \url{https://www.gruenderszene.de/lexikon/begriffe/user-interface}\item Mauger, Vincent: Interface. In: The Routledge Companion to Video Game Studies: 2014, S. 32–40.\item Oxford University Press: interface. In: Oxford Learner's Dictionaries.\item Interface SoftSelect. URL: \url{http://www.softselect.de/business-software-glossar/interface}\item Digital Scholarly Editions as Interfaces. Hrsg. von Roman Bleier, Martina Bürgermeister, Helmut W. Klug, Frederike Neuber und Gerlinde Schneider. Norderstedt: 2018, URL: \url{https://www.i-d-e.de/publikationen/schriften/bd-12-interfaces/}.\item Rosselli Del Turco, Roberto: After the Editing is Done. Designing a Graphic User
                              Interface for Digital Editions. In: Digital Medievalist 7: 2011.\item Thesmann, Stephan: Interface Design: Usability, User Experience und
                              Accessibility im Web gestalten: 2016.\item Bulatovic, Natasa; Gnadt, Timo; Romanello, Matteo; Stiller, Juliane; Thoden, Klaus: Usability in Digital Humanities - Evaluating User
                              Interfaces, Infrastructural Components and the Use of Mobile Devices
                              During Research Proces. Hannover: 2016.\end{itemize}\subsection*{Verweise:}\href{https://gams.uni-graz.at/o:konde.59}{Digitale Edition}, \href{https://gams.uni-graz.at/o:konde.6}{Digitale Nachhaltigkeit}, \href{https://gams.uni-graz.at/o:konde.35}{Barrierefreies Design}, \href{https://gams.uni-graz.at/o:konde.56}{Design}, \href{https://gams.uni-graz.at/o:konde.84}{Farbdesign}, \href{https://gams.uni-graz.at/o:konde.99}{Inferface Design Cycles}, \href{https://gams.uni-graz.at/o:konde.205}{Usability}, \href{https://gams.uni-graz.at/o:konde.207}{User-centred Design}, \href{https://gams.uni-graz.at/o:konde.97}{Informationsarchitektur}\subsection*{Themen:}Interfaces\subsection*{Zitiervorschlag:}Rieger, Lisa. 2021. Interface. In: KONDE Weißbuch. Hrsg. v. Helmut W. Klug unter Mitarbeit von Selina Galka und Elisabeth Steiner im HRSM Projekt "Kompetenznetzwerk Digitale Edition". URL: https://gams.uni-graz.at/o:konde.98\newpage\section*{Interface Design Cycle} \emph{Roman Bleier; roman.bleier@uni-graz.at }\\
        
    \emph{User Interfaces} (Benutzerschnittstellen) werden heute in
                  iterativen Entwicklungszyklen gestaltet, da es auch für \emph{\href{http://gams.uni-graz.at/o:konde.205}{Usability}}-Experten unmöglich ist, eine Benutzerschnittstelle zu entwerfen, die von
                  Anfang an keine Usability-Probleme aufweist. Ein einfacher \emph{User
                     Interface}-Designzyklus, bestehend aus den Schritten Planung,
                  Implementierung, Testen und Evaluierung, sollte in einer iterativen Weise
                  aufgebaut sein und ausgeführt werden. Gould und Lewis beschreiben bereits Mitte
                  der 1980er-Jahre die drei Grundprinzipien von iterativem Design (Gould/Lewis
                     1985): \\
            
        \begin{itemize}\item {früher Fokus auf die Nutzerinnen und Nutzer und die Userinteraktion}\item {Sammlung von empirischen Userdaten durch Testen}\item {Flexibilität zur Änderung und Verbesserung des Systems durch iteratives
                     Design }\end{itemize}Die Nutzerinnen und Nutzer und ihre Interaktion mit dem System stehen klar im
                  Zentrum der iterativen \href{http://gams.uni-graz.at/o:konde.56}{Designstrategie} (\emph{\href{http://gams.uni-graz.at/o:konde.207}{User-centered Design}}). Beim Testen des Systems (\emph{\href{http://gams.uni-graz.at/o:konde.206}{User-testing}}) und dem Sammeln von Userdaten spielen Prototypen eine wichtige Rolle. Beim
                     \emph{
                     Prototyping
                  } geht es darum, ein frühes Modell eines Systems oder Produkts zu entwickeln,
                  welches dann für Tests, User- und Kundenfeedback genutzt werden kann. Der
                  iterative Designzyklus mit Fokus auf den User ist auch der Kern der
                  weitverbreiteten agilen Softwareentwicklung (z. B. \emph{Scrum}),
                  welche auch im \emph{User Interface}-Design Anwendung findet.
                  Agile Entwicklung zeichnet sich zusätzlich durch kurze Entwicklungsiterationen (um
                  auf Veränderungen besser reagieren zu können), Minimierung von Bürokratie und
                  erhöhte Selbstorganisation des Entwicklungsteams aus. \\
            
        Obwohl agile Entwicklung ein aktueller Standard im Software- und Webdesign ist,
                  ist diese Methode bei der Entwicklung von \href{http://gams.uni-graz.at/o:konde.59}{Digitalen Editionen} noch nicht weit verbreitet.
                  Iterative Prozesse und eine detaillierte Userstudie und \emph{\href{http://gams.uni-graz.at/o:konde.206}{User-testing}} wie auch \emph{\href{http://gams.uni-graz.at/o:konde.78}{Editor-testing}} sind für die Interfaces von Digitalen Editionen aber natürlich auch von
                  zentraler Bedeutung. Gegenwärtig dominiert noch die Meinung der Editorin bzw. des
                  Editors (oder des editierenden Teams) das Design. Mit zunehmender Professionalität
                  werden jedoch auch in der Entwicklung von Digitalen Editionen iterative und agile
                  Designmethoden an Bedeutung gewinnen. (Ferraro/Sichani 2018)\\
            
        \subsection*{Literatur:}\begin{itemize}\item Ferraro, Ginestra; Sichani, Anna-Maria: Design as Part of the Plan: Introducing Agile
                              Methodology in Digital Editing Projects. In: Digital Scholarly Editions as Interfaces. Norderstedt: 2018, S. 83–105.\item Gould, John D.; Lewis, Clayton: Designing for Usability: Key Principles and What
                              Designers Think. In: Commun. ACM 28: 1985, S. 300–311.\item Heuwing, Ben; Womser-Hacker, Christa: Zwischen Beobachtung und Partizipation –
                              nutzerzentrierte Methoden für eine Bedarfsanalyse in der digitalen
                              Geschichtswissenschaft. In: Information. Wissenschaft & Praxis 66: 2015, S. 335–344.\item Rosselli Del Turco, Roberto: After the Editing is Done. Designing a Graphic User
                              Interface for Digital Editions. In: Digital Medievalist 7: 2011.\item Thoden, Klaus; Stiller, Juliane; Bulatovic, Natasa; Meiners, Hanna-Lena; Boukhelifa, Nadia: User-Centered Design Practices in Digital Humanities –
                              Experiences from DARIAH and CENDARI. In: ABI Technik 37: 2017, S. 2-11.\end{itemize}\subsection*{Verweise:}\href{https://gams.uni-graz.at/o:konde.205}{Usability}, \href{https://gams.uni-graz.at/o:konde.206}{User-testing}, \href{https://gams.uni-graz.at/o:konde.78}{Editor-testing}, \href{https://gams.uni-graz.at/o:konde.207}{User-centered design}, \href{https://gams.uni-graz.at/o:konde.98}{Interface}\subsection*{Themen:}Interfaces, Digitale Editionswissenschaft\subsection*{Zitiervorschlag:}Bleier, Roman. 2021. Interface Design Cycle. In: KONDE Weißbuch. Hrsg. v. Helmut W. Klug unter Mitarbeit von Selina Galka und Elisabeth Steiner im HRSM Projekt "Kompetenznetzwerk Digitale Edition". URL: https://gams.uni-graz.at/o:konde.99\newpage\section*{International Image Interoperability Framework (IIIF)} \emph{Stigler, Johannes; johannes.stigler@uni-graz.at}\\
        
    Das \emph{International Image Interoperability Framework} (IIIF;
                  gesprochen: Triple-Ei-F) entstand 2011 mit Unterstützung der \emph{Mellon Foundation} aus einer gemeinsamen Initiative renommierter
                  Gedächtnisorganisationen und Forschungseinrichtungen. Die IIIF-Community hat neue
                  Standards zur Bereitstellung von digitalen Bildern und für die Datenpräsentation
                  im Internet geschaffen. Die Standards dienen der Vereinheitlichung der technischen
                  Bilddatenbereitstellung und der Verbesserung der Nutzerfreundlichkeit, sie
                  vereinfachen zudem den Datenaustausch und die internationale Zusammenarbeit in der
                  Forschung. Sie zielen auf eine Standardisierung in Bezug auf die Adressierung von
                  Bildern und Bildausschnitten ab sowie auf die Darstellung von Bildserien (z. B.
                  alle Faksimiles der Seiten einer Handschrift) in Viewern und sonstigen
                  Präsentationslösungen im World Wide Web und unterstützen die dynamische
                  Bereitstellung von Bildern oder Bildausschnitten in verschiedenen Größen,
                  Auflösungsvarianten und Formaten über sogenannte Image-Server.\\
            
        Auf dieser Basis können Tools entwickelt werden, die vielfältig verwendbar und
                  nachnutzbar sind. \emph{Mirador} z. B. ist nicht nur ein Viewer,
                  sondern auch ein Online-Forschungstool für Handschriften. Der Viewer, eine
                  Entwicklung der Universitäten Harvard und Stanford im Rahmen der
                  IIIF-Gemeinschaft, wird bereits in vielen einschlägigen Projekten eingesetzt. Er
                  ermöglicht u. a. ein stufenloses Zoomen in hochaufgelöste Bilder. Der
                  Viewer-Arbeitsbereich lässt sich flexibel konfigurieren und erlaubt das
                  Betrachten, Durchblättern, Annotieren und Vergleichen von digitalen Objekten. Die
                  Objekte können weltweit aus unterschiedlichen Repositorien stammen, sofern diese
                  den IIIF-Standard erfüllen und über \href{http://gams.uni-graz.at/o:konde.12}{persistente Identifikatoren} adressierbar sind. \\
            
        \subsection*{Projekte:}\href{https://showcase.iiif.io/}{IIIF Showcases}\subsection*{Software:}\href{https://iiif.io/}{iiif}, \href{http://projectmirador.org/}{Mirador}\subsection*{Verweise:}\href{https://gams.uni-graz.at/o:konde.36}{Bereitstellung von
                           Digitalisaten}, \href{https://gams.uni-graz.at/o:konde.122}{Bildformate}, \href{https://gams.uni-graz.at/o:konde.63}{Digitalisierungsrichtlinien}, \href{https://gams.uni-graz.at/o:konde.37}{Bilddigitalisierungstechniken}\subsection*{Themen:}Digitalisierung, Archivierung\subsection*{Zitiervorschlag:}Stigler, Johannes. 2021. International Image Interoperability Framework (IIIF). In: KONDE Weißbuch. Hrsg. v. Helmut W. Klug unter Mitarbeit von Selina Galka und Elisabeth Steiner im HRSM Projekt "Kompetenznetzwerk Digitale Edition". URL: https://gams.uni-graz.at/o:konde.123\newpage\section*{Interpretation} \emph{Galka, Selina; selina.galka@uni-graz.at }\\
        
    Interpretation bedeutet das Verstehen und Deuten eines Gegenstandes; dieser Gegenstand kann in unterschiedlichster Form vorliegen (z. B. Text, Musik oder Bild). Interpretation bezeichnet dabei sowohl den Analysevorgang selbst als auch das Ergebnis dieses Vorganges. \\
            
        Ein interpretativer Ansatz in den Textwissenschaften ist die Hermeneutik: Im Zuge eines hermeneutischen Prozesses wird versucht, den Text aus seiner Zeit unter Berücksichtigung der Situation, Motivation und Intention der Verfasserin oder des Verfassers zu verstehen, den Sinn des Textes herauszuarbeiten und gewisse Grundbegriffe zu klären. Unterschiedliche Personen können zu unterschiedlichen Interpretationen gelangen, trotzdem sind Interpretationen aber nicht beliebig, sondern müssen logisch erklärbar sein. \\
            
        Unbestritten ist heute, dass Edition auch Interpretation ist. (Martens 1994, S. 73; Sahle 2013, S. 210) Lange verfolgte man das Ziel einer ‘objektiven’ Edition, aber die Eingriffe der Herausgeberinnen und Herausgeber einer Edition beruhen immer auf Deutung und Wertung, und diese Deutungen sind stets historisch und subjektiv bestimmt. (Sahle 2013, S. 208) Wenn dieser Grundsatz akzeptiert wird, “dann kommt man zur Unterscheidung von Dokumentation und Interpretation, zur Trennung von Befund und Deutung” (Sahle 2013, S. 212). So könnte man wenigstens für einen Teil der Edition den Aspekt der Objektivität bewahren, methodisch korrekt müsste dies bei den jeweiligen Teilen der Edition aber explizit vermerkt werden. (Sahle 2013, S. 213)\\
            
        Auf diese Art und Weise spielt Interpretation natürlich auch bei \href{http://gams.uni-graz.at/o:konde.59}{Digitalen Editionen} eine sehr große Rolle. Sie wirkt sich zum Beispiel auf die \href{http://gams.uni-graz.at/o:konde.178}{TEI}-\href{http://gams.uni-graz.at/o:konde.137}{Modellierung} eines Textes aus, da das Einfügen von \href{http://gams.uni-graz.at/o:konde.126}{Markup} eine detaillierte Analyse des vorliegenden Gegenstandes voraussetzt und interpretative Entscheidungen zu treffen sind.\\
            
        \subsection*{Verweise:}\href{https://gams.uni-graz.at/o:konde.178}{TEI}, \href{https://gams.uni-graz.at/o:konde.126}{Markup}, \href{https://gams.uni-graz.at/o:konde.137}{Modellierung}, \href{https://gams.uni-graz.at/o:konde.195}{Textmodellierung}\subsection*{Literatur:}\begin{itemize}\item Sahle, Patrick: Digitale Editionsformen. Zum Umgang mit der Überlieferung unter den Bedingungen des Medienwandels. Teil 1: Das typografische Erbe. Norderstedt: 2013.\item Martens, Gunter: Neuere Tendenzen in der germanistischen Edition. In: Philosophische Editionen. Erwartungen an sie - Wirkungen durch sie 6: 1994, S. 71–82.\end{itemize}\subsection*{Themen:}Einführung\subsection*{Lexika}\begin{itemize}\item \href{https://lexiconse.uantwerpen.be/index.php/lexicon/editor-interpretation/}{Lexicon of Scholarly Editing}\end{itemize}\subsection*{Zitiervorschlag:}Galka, Selina. 2021. Interpretation. In: KONDE Weißbuch. Hrsg. v. Helmut W. Klug unter Mitarbeit von Selina Galka und Elisabeth Steiner im HRSM Projekt "Kompetenznetzwerk Digitale Edition". URL: https://gams.uni-graz.at/o:konde.100\newpage\section*{Intertextualität (Fokus: Literaturwissenschaft)} \emph{Mader, Franziska; franziska.mader@aau.at / Boelderl, Artur R.;
                  artur.boelderl@aau.at}\\
        
    Theoretisch lassen sich im Diskurs zur Intertextualität zwei Ansätze
                  unterscheiden: ein textdeskriptiver und ein texttheoretischer. Ersterer fokussiert
                  „nachweisbare[ ], spezifische[ ] Textbeziehungen“ (Komfort-Hein 2012, S. 187), der zweite geht von einem entgrenzten
                  Textbegriff aus und versteht Textualität grundsätzlich als dynamisches Netzwerk,
                  in dem die Zeichen miteinander kommunizieren. In einer \href{http://gams.uni-graz.at/o:konde.59}{Digitalen Edition} lassen sich textdeskriptive
                  intertextuelle Bezüge in unterschiedlichem Ausmaß annotieren: einerseits als
                  markiertes Zitat (Bsp. 1). Attributiv können dann der Grad der Markierung – ob es
                  sich um ein markiertes Zitat oder eine Anspielung handelt – und der zitierte Text
                  angeführt werden. Andererseits kann in Fällen, in denen sich ein Text wiederholt
                  und explizit mit einem anderen Text auseinandersetzt, der gesamte Referenztext
                  angeführt werden (Bsp. 2). Die \href{http://gams.uni-graz.at/o:konde.17}{Annotation} kommt ohne Blacktext aus, kodiert aber die Verknüpfung zum
                  Referenztext attributiv als @source (Quelle) und führt dessen Basisinformationen im <head>-Element an. Im Unterschied zum konkreten
                  Zitat verweist die Annotation in diesen Fällen auf den zitierenden Text mit Hilfe
                  des <ref>-Elements (\emph{reference}). Verbleibt die textuelle Beziehung im Œuvre einer Autorin oder
                  eines Autors, spricht man von intratextuellen Bezügen; diese Bezüge können z. B.
                  Text- bzw. Entstehungsvarianten oder motivische Mehrfachverwertungen sein (s.
                  Stichwort \href{http://gams.uni-graz.at/o:konde.21}{Intratextualität}).\\
            
        Bsp. 1:\\
            
        \begin{verbatim}<div resp="#FM" corresp="#s002">
    <head>Charakter</head>
    <p>
        <quote type="allusion" source="nzl.xml#pb_153"
        corresp="#bib_nietzsche_wissenschaft_1887">Man muß heute Charaktere wohl 
        mit der Laterne suchen gehn; wahrscheinlich macht man sich noch dazu 
        lächerlich, wenn man bei Tag mit einem brennenden Licht umhergeht.</quote>
        <ref target="http://www.nietzschesource.org/#eKGWB/FW-125"/>
    </p>
</div>\end{verbatim}Bsp. 2:\\
            
        \begin{verbatim}<div resp="#FM" corresp="#s003">
    <head>Mach: Analyse der Empfindungen</head>
    <p>
        <quote source="mach_ade_1902.xml" sameAs="bib.xml#bib_mach_ade_1902"/>
        <ref corresp="musil_diss_1908.xml"/>
    </p>
</div>\end{verbatim}Dieser Beitrag wurden im Kontext des FWF-Projekts "MUSIL ONLINE – interdiskursiver Kommentar" 
                  (P 30028-G24) verfasst.\subsection*{Literatur:}\begin{itemize}\item Genette, Gérard; Bayer, Wolfram; Hornig, Dieter: Palimpseste: die Literatur auf zweiter Stufe Palimpseste. Frankfurt am Main: 2015.\item Komfort-Hein, Susanne: Intertextualität. In: Germanistik. Sprachwissenschaft – Literaturwissenschaft
                              – Schlüsselkompetenzen. Stuttgart, Weimar: 2012, S. 86–190.\item Kristeva, Julia: Bachtin, das Wort, der Dialog und der Roman. In: Literaturwissenschaft und Linguistik. Ergebnisse und
                              Perspektiven. Band 3. Frankfurt am Main: 1972, S. 345–375.\end{itemize}\subsection*{Software:}\href{http://oxygenxml.com/}{Oxygen}\subsection*{Verweise:}\href{https://gams.uni-graz.at/o:konde.17}{Annotation (Literaturwissenschaft:
                           grundsätzlich)}, \href{https://gams.uni-graz.at/o:konde.21}{Intratextualität (Fokus:
                           Literaturwissenschaft - Bsp. Musil)t}, \href{https://gams.uni-graz.at/o:konde.28}{Textgenese}, \href{https://gams.uni-graz.at/o:konde.23}{Makrogenese (Fokus:
                           Literaturwissenschaft - Bsp. Musil)}, \href{https://gams.uni-graz.at/o:konde.24}{Mesogenese (Fokus:
                           Literaturwissenschaft - Bsp. Musil)}, \href{https://gams.uni-graz.at/o:konde.26}{Mikrogenese (Fokus:
                           Literaturwissenschaft - Bsp. Musil)}, \href{https://gams.uni-graz.at/o:konde.96}{Hybridedition}\subsection*{Projekte:}\href{http://musilonline.at}{Musil Online}\subsection*{Themen:}Annotation und Modellierung, Digitale Editionswissenschaft\subsection*{Zitiervorschlag:}Mader, Franziska; Boelderl, Artur R. 2021. Intertextualität (Fokus: Literaturwissenschaft). In: KONDE Weißbuch. Hrsg. v. Helmut W. Klug unter Mitarbeit von Selina Galka und Elisabeth Steiner im HRSM Projekt "Kompetenznetzwerk Digitale Edition". URL: https://gams.uni-graz.at/o:konde.20\newpage\section*{Intratextualität (Fokus: Literaturwissenschaft – Bsp. Musil)} \emph{Fanta, Walter; walter.fanta@aau.at / Boelderl, Artur R.;
                  artur.boelderl@aau.at}\\
        
    In der Perspektive digitaler Online-Edition manifestiert sich Intratextualität
                  (jenseits der anhaltenden fachwissenschaftlichen Diskussion um den Begriff)
                  heuristisch als jene Form der intertextuellen Beziehung (\href{http://gams.uni-graz.at/o:konde.20}{Intertextualität}), die innerhalb des Œuvres einer
                  Autorin oder eines Autors besteht, wobei sowohl der Grund dieses Bestehens (z. B.
                  entstehungsbedingte Varianz) als auch seine Qualität (z. B. wiederholte Abhandlung
                  eines bestimmten Motivs) bzw. sein Umfang (z. B. bei publizistischen
                  Mehrfachverwertungen) unterschiedlich sein können. Die \href{http://gams.uni-graz.at/o:konde.17}{Annotation} hat in jedem Fall der Eigenart des
                  spezifischen intratextuellen Bezugs Rechnung zu tragen.\\
            
        Am Beispiel Musils gesagt, ist der Umgang dieses Autors mit werkinternen
                  textuellen Bezugnahmen so markant, dass er seitens der UNESCO sogar als ein
                  Hauptcharakteristikum seines Schreibens gewürdigt wurde und Eingang in die
                  Begründung für die Erklärung des Nachlasses zum Dokumentenerbe fand: „mittels
                  [s]eines Siglensystems“ sei das Wissen seiner Zeit „zu einem Ganzen verwoben“
                     (UNESCO 2020). Die Annotation dieses höchst komplexen
                  Siglensystems nach \href{http://gams.uni-graz.at/o:konde.178}{TEI} gelingt in
                  einer zwar nicht ganz einfachen Weise, jedoch wird mit dem verwendeten aufwändigen
                  Referenzsystem erreicht, dass Musils prä-digitale hypertextuelle Schreibapparatur
                  komplett im digitalen Medium abgebildet wird und damit als Hypertextsystem
                  präsentiert werden kann. (Sahle 2013, S. 87) In der hier
                  vorgestellten \href{http://gams.uni-graz.at/o:konde.215}{XML}/\href{http://gams.uni-graz.at/o:konde.178}{TEI}-Architektur sind Siglen an fünf
                  Orten als Textknoten und als abstrahierter Attributwert verzeichnet: \\
            
        \begin{itemize}\item {Seitensiglen im <teiHeader> der
                     Transkription in <msPart> im Element <altIdentifier> in einer typisierten
                     Form;}\item {Seitensiglen im <body><text> der
                     Transkription im Element <fw> in der
                     transkribierten Form;}\item {Verweissiglen im <body><text>
                     der Transkription im Element <ref> in der
                     transkribierten Form;}\item {die Seitensiglen in ihrer makrogenetischen Funktionalität in den Tabellen
                     des Dokuments \emph{tgd.xml} (i. e. textgenetisches
                     Dossier);}\item {Gesamtdokumentation sämtlicher Siglen mit allen Repräsentanten im Dokument
                        \emph{tutorial.xml}.}\end{itemize}Zu Musils intratextueller Verweispraxis zählt neben dem Gebrauch des Siglensystems
                  noch die Verwendung spezieller Chiffren für Textrevision, Verweise und
                  Abkürzungen. Die Chiffren werden generell mit dem Element <metamark> ausgezeichnet, im Attributwert von @function erfolgt die nähere Bestimmung des Schreibakts, der hier seine
                  Spur hinterlassen hat, z. B. <metamark
                     function="deletion"/> für das Deleatur-Zeichen; analog wird bei allen
                  nicht alphanumerischen Zeichen vorgegangen. Wenn sich die Funktion des graphischen
                  Elements nicht bestimmen lässt, kommt <metamark
                     function="unspecified"/> zum Einsatz. Graphische Darstellungen,
                  Skizzen oder Zeichnungen Musils bei seiner Schreibarbeit werden durch <figure> repräsentiert, die häufigen
                  Verweischiffren (Pfeile, Linien und andere Zeichen mit Verweischarakter) durch
                     <metamark function="reference"/>. Chiffren
                  in Abkürzungsfunktion werden mit dem Element <choice> in Verbindung mit <abbr> und <expan> aufgelöst
                  (Bsp. 1: „Parallelaktion“ im \emph{Mann ohne Eigenschaften}). Die
                  von Musil schon früh erfundenen Figurenchiffren, die er über zwanzig Jahre in
                  allen seinen Studien- und Schmierblättern zum \emph{Mann ohne
                     Eigenschaften} verwendet, werden mit dem Element <rs> einem Register zugewiesen, das die Funktion eines
                  textgenetischen Figurenkommentars erfüllt und insofern die Auszeichnungspraxis der
                  Annotation an die Grenze zur Kommentierung führt (Bsp. 2). \\
            
        Inwiefern diese Grenze zugleich mit dem fließenden Übergang zwischen Intra- und
                  Intertextualität korrespondiert, verdeutlicht die Annotation von Personennamen im
                  Textkorpus des Musil-Nachlasses, da Personen fast immer Autoren sind (Bsp. 3: die
                  etliche Male vorkommende Namenschiffre „Th M“). Ausgebaut wird diese
                  Annotationspraxis, wenn ein konkretes Werk des Autors benannt und aus ihm zitiert
                  wird, etwa in den zahlreichen Exzerpten im Nachlass Musils. Als einfaches Beispiel
                  für den weitreichenden Sachverhalt möge die Anmerkung Musils auf einem
                  Studienblatt dienen, Agathes „Gedächtnis hat Ähnlichkeit mit dem der Imbezillen
                  Bleuler 463“ (Bsp. 4 u. 5). Es besteht die Hoffnung, auf diese relativ einfache
                  Weise das gesamte Netz von Intertextualität, das Musils Nachlass überzieht, für
                  die digitale Repräsentation und Nachnutzung einzufangen.\\
            
        Bsp. 1:\\
            
        \begin{verbatim}<choice>
<abbr>//</abbr> <expan>Parallelaktion, Parallele,
parallel</expan></choice>\end{verbatim}Bsp. 2:\\
            
        \begin{verbatim}<rs type="figure" ref="Tuzzi"><hi rend="underline">SCh T</hi></rs>\end{verbatim}Der Attributwert "Tuzzi" ist Lemma im Register \emph{figuren.xml},
                  wo u. a. die Genese der Figur beschrieben wird.\\
            
        Bsp. 3:\\
            
        \begin{verbatim}<rs type="person" ref="Mann_Thomas">Th M</rs>\end{verbatim}Der Attributwert von @ref verweist auf den Eintrag zu Thomas Mann in einer Normdatenbank,
                  die noch zu bestimmen sein wird.\\
            
        Bsp. 4:\\
            
        Gedächtnis hat Ähnlichkeit mit dem der Imbezillen\\
            
        \begin{verbatim}<cit>
<q>Bleuler 463</q>
<bibl><author>Bleuler, (Paul) Eugen</author>
<title>Lehrbuch der Psychiatrie</title></bibl>
</cit>\end{verbatim}Mit dem Attributwert von @corresp in dem Element <bibl>
                  wird auf das Dokument \emph{bibliographie.xml} verwiesen, wo sich
                  das vervollständigte bibliographische Zitat befindet.\\
            
        Bsp. 5:\\
            
        Zusätzlich besteht die Möglichkeit, im Fall der Exzerpte die Quelle im <teiHeader><msPart> einer bestimmten
                  Manuskriptseite zuzuweisen:\\
            
        \begin{verbatim}<surrogates> Quelle:
<cit><q>Bleuler</q>
<bibl><author>Bleuler, (Paul) Eugen</author>
<title>Lehrbuch der Psychiatrie</title></bibl>
</cit></surrogates>\end{verbatim}\subsection*{Literatur:}\begin{itemize}\item Fanta, Walter: Musil online total. In: Forschungsdesign 4.0. Datengenerierung und
                              Wissenstransfer in interdisziplinärer Perspektive. Dresden: 2019, S. 149–179.\item Sahle, Patrick: Digitale Editionsformen. Zum Umgang mit der
                              Überlieferung unter den Bedingungen des Medienwandels. Teil 3:
                              Textbegriffe und Recodierung. Norderstedt: 2013.\item Nachlass Robert Musil. „Memory of the World“/„Gedächtnis
                              der Menschheit“. URL: \url{https://www.unesco.at/kommunikation/dokumentenerbe/memory-of-austria/verzeichnis/detail/article/nachlass-robert-musil/}\end{itemize}Dieser Beitrag wurden im Kontext des FWF-Projekts "MUSIL ONLINE – interdiskursiver Kommentar" 
                  (P 30028-G24) verfasst.\subsection*{Software:}\href{http://oxygenxml.com/}{Oxygen}\subsection*{Verweise:}\href{https://gams.uni-graz.at/o:konde.17}{Annotation (Literaturwissenschaft:
                           grundsätzlich)}, \href{https://gams.uni-graz.at/o:konde.20}{Intertextualität (Fokus:
                           Literaturwissenschaft)}, \href{https://gams.uni-graz.at/o:konde.28}{Textgenese}, \href{https://gams.uni-graz.at/o:konde.23}{Makrogenese (Fokus:
                           Literaturwissenschaft - Bsp. Musil)}, \href{https://gams.uni-graz.at/o:konde.24}{Mesogenese (Fokus:
                           Literaturwissenschaft - Bsp. Musil)}, \href{https://gams.uni-graz.at/o:konde.26}{Mikrogenese (Fokus:
                           Literaturwissenschaft - Bsp. Musil)}, \href{https://gams.uni-graz.at/o:konde.96}{Hybridedition}\subsection*{Projekte:}\href{http://musilonline.at}{Musil Online}\subsection*{Themen:}Annotation und Modellierung, Digitale Editionswissenschaft\subsection*{Zitiervorschlag:}Fanta, Walter; Boelderl, Artur R. 2021. Intratextualität (Fokus: Literaturwissenschaft – Bsp. Musil). In: KONDE Weißbuch. Hrsg. v. Helmut W. Klug unter Mitarbeit von Selina Galka und Elisabeth Steiner im HRSM Projekt "Kompetenznetzwerk Digitale Edition". URL: https://gams.uni-graz.at/o:konde.21\newpage\section*{Kataloge digitaler Editionen} \emph{Kurz, Stephan; stephan.kurz@oeaw.ac.at }\\
        
    Nachdem \href{http://gams.uni-graz.at/o:konde.59}{Digitale Editionen} nur in
                     seltenen Fällen (z. B. Editionsprojekte auf der \href{http://gams.uni-graz.at/o:konde.70}{GAMS}) in Bibliothekskataloge aufgenommen wurden und
                     sonstige zentrale Findemittel bislang nicht etabliert sind, haben einzelne die
                     Aufgabe übernommen, \href{http://gams.uni-graz.at/o:konde.25}{Metadaten} zu
                     Digitalen Editionen zu aggregieren und damit zur Wiederauffindbarkeit publizierter
                     Editionen beitragen.\\
            
        Bei der Planung von neuen digitalen Editionsprojekten sind die solcherart
                     entstandenen Kataloge eine wichtige Hilfestellung, insofern sie auf bestehende –
                     bei freier \href{http://gams.uni-graz.at/o:konde.119}{Lizenzierung}
                     nachnutzbare – Ressourcen sowohl im Bereich der (z. B. Auxiliar-)Daten (z. B.
                     Registerdatenbestände), aber auch im Bereich der Interfaceentwicklung hinzuweisen
                     vermögen. In der Folge werden die wichtigsten derartigen Kataloge mit einigen
                     Stichworten aufgenommen. Überlappungen im Datenbestand sind zu erwarten, die Liste
                     der Kataloge ist sicherlich nicht vollständig, bietet aber eine solide
                     Hilfestellung für den Anwendungsfall, bestehende Ressourcen im Umfeld eines neu zu
                     entwickelnden Editionsprojekts ausfindig zu machen.\\
            
        \begin{itemize}\item {Patrick Sahle, \emph{Scholarly Digital Editions. An annotated
                        List.} Der für den deutschsprachigen Raum aktuell mit über 700 Einträgen
                        umfassendste Katalog ordnet nach Titel, Fachgebiet, Materialtyp, Sprache und
                        Epoche und hebt eine kleine Shortlist von dzt. 30 Projekten als besondere
                        Empfehlungen hervor.
                     }\item {Greta Franzini, \emph{Catalogue of Digital Editions.}
                        Der von Greta Franzini seit 2012 betreute Katalog hat die Besonderheit, dass er
                        auch in DBIS gelistet ist. Zum Zeitpunkt der letzten Überprüfung werden
                        Neueinreichungen über GitHub-Issues hinzugefügt. }\item {Roland S. Kamzelak und Lydia Michel, \emph{Marbacher Editionendatenbank.} Die Datenbank wird seit 2012
                        betrieben und verzeichnet deutschsprachige Editionen. Der Datenzugang ist nur
                        nach Anmeldung möglich.}\item { TEI, \emph{Projects Using the TEI.}Auch das
                        TEI-Konsortium betreibt einen eigenen Katalog von Editionen, die in diesem
                        XML-Format codiert sind. }\item {Liste der Hybrideditionen Im Rahmen der KONDE-AG ‘Hybridedition’ ist
                        eine \href{http://gams.uni-graz.at/o:konde.117}{Liste der
                           Hybrid-Editionen} entstanden.}\end{itemize}\subsection*{Verweise:}\href{https://gams.uni-graz.at/o:konde.59}{Digitale Edition}, \href{https://gams.uni-graz.at/o:konde.25}{Metadaten}, \href{https://gams.uni-graz.at/o:konde.119}{Lizenzierung}, \href{https://gams.uni-graz.at/o:konde.117}{Liste der Hybrideditionen}\subsection*{Projekte:}\href{http://digitale-edition.de/}{Scholarly Digital
                              Editions. An annotated
                              List}, \href{https://dig-ed-cat.acdh.oeaw.ac.at/}{Catalogue
                              of Digital Editions}, \href{https://www.dla-marbach.de/digital-humanities/editionen-db/}{Marbacher Editionendatenbank}, \href{https://tei-c.org/activities/projects/}{Projects using the TEI}, \href{http://dbis.uni-regensburg.de/}{DBIS -
                              Datenbankinfosystem}, \href{https://github.com/gfranzini/digEds_cat/blob/master/CONTRIBUTING.md}{Contribute to digEds_cat}\subsection*{Themen:}Einführung, Digitale Editionswissenschaft\subsection*{Zitiervorschlag:}Kurz, Stephan. 2021. Kataloge digitaler Editionen. In: KONDE Weißbuch. Hrsg. v. Helmut W. Klug unter Mitarbeit von Selina Galka und Elisabeth Steiner im HRSM Projekt "Kompetenznetzwerk Digitale Edition". URL: https://gams.uni-graz.at/o:konde.102\newpage\section*{Kodikologie} \emph{Rieger, Lisa; lrieger@edu.aau.at }\\
        
    Die Kodikologie beschäftigt sich mit mittelalterlichen Codices und deckt sich dabei weitgehend mit der historischen Handschriftenkunde. (Schneider 2014, S. 103) Ihre materialwissenschaftliche Ausrichtung, die im Gegensatz zur inhaltlichen Orientierung anderer geisteswissenschaftlicher Disziplinen steht (Schneider 2016), hilft dabei, ein Verständnis für die Traditionen und Hintergründe von Dokumenten aus jener Zeit zu entwickeln, das im Weiteren für eine sinnvolle Auswertung der Daten nötig ist. (Hodel/Nadig 2019, S. 142) Zu den wichtigsten Forschungsgegenständen der Kodikologie zählen das Verfahren der Herstellung, die Herkunft und Geschichte bzw. Provenienz einer Handschrift, die Untersuchung von Textgemeinschaften sowie die Suche nach möglichen Fragmenten, die nicht zum eigentlichen bzw. ursprünglichen Text gehören. Dabei steht die Kodikologie in engem Austausch mit anderen wissenschaftlichen Disziplinen, wie der \href{http://gams.uni-graz.at/o:konde.155}{Paläographie}, der Inkunabelkunde, der Wasserzeichenkunde sowie naturwissenschaftlichen Methoden zur Analyse des Materials. (Schneider 2016)\\
            
        Die ersten Editoren, die sich eingehend mit diesem Gebiet beschäftigten, waren F. H. von der Hagen, K. Lachmann und die Brüder Grimm. (Schneider 2014, S. 3) Stärker als im deutschsprachigen Gebiet wurde die Forschung auf dem Gebiet der Kodikologie in den letzten Jahrzehnten in Frankreich, Belgien und den Niederlanden betrieben. (Schneider 2014, S. 104)\\
            
        In Zeiten der \href{http://gams.uni-graz.at/o:konde.61}{Digitalisierung} kommen auch für Kodikologinnen und Kodikologen neue Aufgabenbereiche hinzu: sämtliche digitalisierte und online-gestellte Kataloge müssen auch stets den modernen Standards entsprechen, welche eine „kodikologische Beschreibung auf Basis des materiellen Befundes der Handschrift“ (Kranich-Hofbauer 2010, S. 320) und auch neue Fragestellungen berücksichtigen müssen. (Kranich-Hofbauer 2010, S. 320) Durch die Erfassung von Handschriftenbeständen in Datenbanken wächst auch die Bedeutung der Kodikologie, da durch genaue Beschreibungen und digitale Reproduktionen Zusammenhänge auch ohne Einsicht in die tatsächliche Handschrift hergestellt werden können. (Wagner 2009, S. 6)\\
            
        Techniken der \href{http://gams.uni-graz.at/o:konde.59}{Digitalen Edition} können die kodikologische Forschung z. B. durch \href{http://gams.uni-graz.at/o:konde.125}{Metadaten}, die Visualisierung von Handschriften oder in der \href{http://gams.uni-graz.at/o:konde.123}{Zurverfügungstellung von digitalen Bildern} unterstützen.\\
            
        \subsection*{Literatur:}\begin{itemize}\item Hodel, Tobias; Nadig, Michael: Grundlagen der Mediävistik digital vermitteln: 'Ad fontes', aber wie? Grundlagen der Mediävistik digital vermitteln In: Das Mittelalter 24: 2019.\item Kranich-Hofbauer, Karin: Zusammengesetzte Handschriften - Sammelhandschriften. Materialität - Kodikologie - Editorik Zusammengesetzte Handschriften - Sammelhandschriften. In: Materialität in der Editionswissenschaft: 2010, S. 309–321.\item Station 1: Einführung in die Kodikologie Einführung in die Kodikologie. URL: \url{http://dhmuseum.uni-trier.de/node/329}\item Schneider, Karin: Paläographie und Handschriftenkunde für Germanisten. Eine Einführung Paläographie und Handschriftenkunde für Germanisten. Berlin, Boston: 2014, URL: \url{https://www.degruyter.com/view/title/304681?tab_body=toc}.\item Wagner, Bettina: Handschriftenerschließung in Deutschland. Vom gedruckten Katalog zum Informationssystem Handschriftenerschließung in Deutschland. In: Das Mittelalter 14: 2009, S. 3-7.\item Fischer, Franz; Fritze, Christiane [Hrsg; Vogeler, Georg [Hrsg: Kodikologie und Paläographie im digitalen Zeitalter 2 - Codicology and Palaeography in the Digital Age 2. Hrsg. von Franz Fischer, Christiane Fritze und Georg Vogeler. Norderstedt: 2011, URL: \url{http://kups.ub.uni-koeln.de/4337/}.\item Duntze, Oliver; Schaßan, Torsten; Vogeler, Georg: Kodikologie und Paläographie im digitalen Zeitalter 3. Norderstedt: 2015.\item Kodikologie und Paläographie im digitalen Zeitalter 4. Norderstedt: 2017.\end{itemize}\subsection*{Verweise:}\href{https://gams.uni-graz.at/o:konde.155}{Paläographie}, \href{https://gams.uni-graz.at/o:konde.60}{Digitalisierung}, \href{https://gams.uni-graz.at/o:konde.28}{Textgenese/-änderung}, \href{https://gams.uni-graz.at/o:konde.37}{Bilddigitalisierungstechniken}, \href{https://gams.uni-graz.at/o:konde.36}{Bereitstellung von Digitalisaten}, \href{https://gams.uni-graz.at/o:konde.92}{Handschriftenbeschreibung}\subsection*{Themen:}Einführung\subsection*{Software:}\href{https://iiif.io/}{iiif}, \href{https://github.com/leoba/VisColl}{Viscoll}\subsection*{Projekte:}\href{https://www.fragmentarium.ms}{Fragmentarium - Digital Research Laboratory for Medieval Manuscript Fragments}\subsection*{Lexika}\begin{itemize}\item \href{https://edlex.de/index.php?title=Kodikologie}{Edlex: Editionslexikon}\item \href{https://wiki.helsinki.fi/display/stemmatology/Codicology}{Parvum Lexicon Stemmatologicum}\item \href{https://lexiconse.uantwerpen.be/index.php/lexicon/codicology/}{Lexicon of Scholarly Editing}\end{itemize}\subsection*{Zitiervorschlag:}Rieger, Lisa. 2021. Kodikologie. In: KONDE Weißbuch. Hrsg. v. Helmut W. Klug unter Mitarbeit von Selina Galka und Elisabeth Steiner im HRSM Projekt "Kompetenznetzwerk Digitale Edition". URL: https://gams.uni-graz.at/o:konde.103\newpage\section*{Kollaboration} \emph{Bürgermeister, Martina; martina.buergermeister@uni-graz.at / Klug, Helmut W.; helmut.klug@uni-graz.at }\\
        
    Geisteswissenschaftliche Fächer zählen traditionell zu jenen Disziplinen, in denen überwiegend Einzelforschung praktiziert wird. Anders ist es in den digitalen Geisteswissenschaften, in denen es eine hohe Bereitschaft zur Kollaboration gibt. Das zeigen die vielen Beiträge kollaborativer Projekte bei den jährlichen Konferenzen zu den Digital Humanities im deutschsprachigen Raum (DHD). Kollaborativ geforscht wird, wenn die Aufgaben inhaltlich zu umfangreich sind oder es für eine Einzelperson unmöglich ist, diese innerhalb eines vernünftigen Zeitrahmens zu lösen. Kollaborationen bieten sich besonders für das Erstellen von \href{http://gams.uni-graz.at/o:konde.59}{Digitalen Editionen} an, wenn die Quellentexte sehr umfangreich oder verstreut vorliegen oder wenn unterschiedliche Blickwinkel auf das Material aufgearbeitet werden sollen und wenn diese Vielfalt in der Präsentation dargestellt werden soll.\\
            
        Laut Chirag Shah (2010) besteht Kollaboration aus mehreren Komponenten: \emph{communication, contribution, coordination, cooperation}. Demnach ist die Kooperation eine Vorstufe zur Kollaboration. In einer Kooperation finden sich verschiedene Personen mit ähnlichen Interessen zusammen und beteiligen sich an der gemeinsamen Nutzung von Ressourcen, um ein gemeinsames Ziel zu erreichen. (Shah 2010, S. 5-6) Ein Beispiel für kooperatives Arbeiten ist die \href{http://gams.uni-graz.at/o:konde.17}{Annotation} von Texten, die von  mehreren Forschenden gemeinsam umgesetzt wird. Das Annotieren kann aber auch kollaborativ ausgeführt werden (Jacke 2018; Meister 2012), was die Editionsform der \emph{\href{http://gams.uni-graz.at/o:konde.169}{Social Edition}} zeigt. Denn Kollaboration ist jener Prozess, durch den Parteien verschiedene Aspekte eines Problems kennenlernen, indem bewusst Betrachtungsunterschiede zugelassen werden, aber gemeinsam nach Lösungen gesucht wird, die die Grenzen der Einzelvorstellungen überschreiten. (Chrislip/Larson 1994, S. 5; Gray 1989) Eine kollaborative Lösung liegt dann vor, wenn das Ganze mehr ist als die Summe aller Beiträge. (Shah 2010, S. 6)\\
            
        Eine wichtige Rolle bei der Umsetzung von Digitalen Editionen spielen kollaborative Werkzeuge, wie zum Beispiel Metadatenstandards, kontrollierte Vokabularien und \href{http://gams.uni-graz.at/o:konde.151}{Ontologien}, weil sie den Austausch (\emph{contribution, cooperation}) durch ihre Standardisierungen Unmissverständlichkeit herstellen, was die Grundlage eines problemlosen kollaborativen Austausches ist. Die \href{http://gams.uni-graz.at/o:konde.178}{TEI} bietet, in ihrer besonderen Stellung innerhalb von digitalen Editionsprojekten, durch die Möglichkeit der \emph{\href{http://gams.uni-graz.at/o:konde.180}{TEI-Customization}} ein weiteres kooperatives Werkzeug. (Flanders 2012) Mehrere Aspekte der Kollaboration lassen sich durch Filehosting- und Versionskontrollsysteme (z. B.\emph{ GoogleDocs, Github}) abdecken. Sie ermöglichen nicht nur das Koordinieren von Arbeitsprozessen, sondern unterstützen aktiv kollaboratives Schreiben.\\
            
        \subsection*{Literatur:}\begin{itemize}\item Vogeler, Georg: Das Verhältnis von Archiven und Diplomatik im Netz. Von der archivischen zur kollaborativen Erschließung. In: Digitale Urkundenpräsentationen. Beiträge zum Workshop in München, 16. Juni 2010. Norderstedt: 2011, S. 61–82.\item Burr, Elisabeth: DHD 2016. Modellierung, Vernetzung, Visualisierung. Konferenzabstracts., URL: \url{http://dhd2016.de/}.\item DHd2015. Von Daten zu Erkenntnissen 23. bis 27. Feburar 2015, Graz: 2015. URL: \url{http://gams.uni-graz.at/o:dhd2015.abstracts-gesamt}.\item Konferenzabstracts. DHd2017 Bern. Digitale Nachhaltigkeit. 13.-18. Februar 2017: 2017. URL: \url{http://www.dhd2017.ch/wp-content/uploads/2017/03/Abstractband_def3_März.pdf}.\item Digital Humanities. Eine Einführung. Hrsg. von Fotis Jannidis, Hubertus Kohle und Malte Rehbein. Stuttgart: 2017, URL: \url{https://doi.org/10.1007%2f978-3-476-05446-3}.\item Deegan, Marilyn; McCarty, Willard: Collaborative research in the digital humanities. A volume in honour of Harold Short, on the occasion of his 65th birthday and his retirement, September 2010. Farnham, Surrey, England; Burlington, VT, USA: 2011.\item Nowviskie, Bethany: Where Credit Is Due: Preconditions for the Evaluation of Collaborative Digital Scholarship Where Credit Is Due. In: Profession 2011: 2011, S. 169–181.\item Rehbein, Malte: The transition from classical to digital thinking. Reflections on Tim McLoughlin, James Barry and collaborative work. In: Jahrbuch für Computerphilologie 10: 2008, S. 55–67.\item Robinson, Peter: Some principles for the making of collaborative scholarly editions in digital form. In: Digital Humanities Quarterly 11: 2017.\item Pierazzo, Elena: Digital scholarly editing: theories, models and methods. Farnham: 2015, URL: \url{http://hal.univ-grenoble-alpes.fr/hal-01182162}.\item Chrislip, David D.; Larson, Carl E.: Collaborative Leadership: How Citizens and Civic Leaders Can Make a Difference: 1994.\item Gray, Barbara: Collaborating: Finding Common Ground for Multiparty Problems. San Francisco: 1989.\item Flanders, Julia: Collaboration and Dissent: Challenges of Collaborative Standards for Digital Humanities. In: Collaborative Research in the Digital Humanities. Farnham, Surrey, England: 2012, S. 67–80.\item Kollaboratives literaturwissenschaftliches Annotieren. URL: \url{https://fortext.net/routinen/methoden/kollaboratives-literaturwissenschaftliches-annotieren}\item Meister, Jan-Christoph: Crowd Sourcing ‘True Meaning’: A Collaborative Markup Approach to Textual Interpretation. In: Collaborative Research in the Digital Humanities. Farnham, Surrey, England: 2012, S. 105–122.\item Shah, Chirag: Collaborative Information Seeking. In: Advances in Librarianship 32: 2010, S. 3-33.\end{itemize}\subsection*{Software:}\href{http://transcribe-bentham.ucl.ac.uk/td/Transcribe_Bentham}{Bentham Transcription Desk}, \href{https://diyhistory.lib.uiowa.edu}{Civil War Diaries & Letters Transcription Project}, \href{https://github.com/gsbodine/crowd-ed}{Crowd-Ed}, \href{http://www.ala.org.au/get-involved/citizen-science/fielddata-software/}{FieldData}, \href{https://fromthepage.com/}{FromThePage}, \href{http://edgerton-digital-collections.org/notebooks}{Harold "Doc" Edgerton Project}, \href{http://www.digiverso.com/de/products/viewer}{Gobi viewer}, \href{https://islandora.ca/}{Citizen Science, Collaboration}, \href{http://www.mom-wiki.uni-koeln.de/}{Itineranova-Editor}, \href{http://pybossa.com/}{PyBOSSA}, \href{http://github.com/zooniverse/Scribe}{Scribe}, \href{http://scripto.org/}{scripto}, \href{https://textgrid.de/}{TextGrid}, \href{https://textualcommunities.org/app/}{Textual Communities}, \href{https://transkribus.eu/Transkribus/}{Transkribus}, \href{http://bencrowder.net/coding/unbindery/}{Unbindery}, \href{http://menus.nypl.org/}{What's On the Menu?}, \href{http://en.wikisource.org/wiki/Main_Page}{Wikisource}, \href{https://www.zooniverse.org/}{zooniverse}\subsection*{Verweise:}\href{https://gams.uni-graz.at/o:konde.169}{Social Edition}, \href{https://gams.uni-graz.at/o:konde.7}{FAIR-Prinzipien}, \href{https://gams.uni-graz.at/o:konde.152}{Open Access}, \href{https://gams.uni-graz.at/o:konde.47}{Crowdsourcing}, \href{https://gams.uni-graz.at/o:konde.14}{Versionierung}\subsection*{Projekte:}\href{https://en.wikibooks.org/wiki/The_Devonshire_Manuscript}{A Social Edition of the Devonshire MS (BL Add. MS 17492)}\subsection*{Themen:}Einführung, Digitale Editionswissenschaft\subsection*{Lexika}\begin{itemize}\item \href{https://edlex.de/index.php?title=Kollation}{Edlex: Editionslexikon}\item \href{https://lexiconse.uantwerpen.be/index.php/lexicon/collaboration/}{Lexicon of Scholarly Editing}\end{itemize}\subsection*{Zitiervorschlag:}Bürgermeister, Martina; Klug, Helmut W. 2021. Kollaboration. In: KONDE Weißbuch. Hrsg. v. Helmut W. Klug unter Mitarbeit von Selina Galka und Elisabeth Steiner im HRSM Projekt "Kompetenznetzwerk Digitale Edition". URL: https://gams.uni-graz.at/o:konde.104\newpage\section*{Kollation} \emph{Andrews, Tara; tara.andrews@univie.ac.at }\\
        
    Unter Kollation versteht man den Vergleich unterschiedlicher Versionen (oder
                  Formen) eines Textes. Die Kollation stellt einen der zentralen Arbeitsschritte bei
                  der Edition eines Textes dar, welcher in mehreren voneinander abweichenden
                  Überlieferungen vorliegt. Man kann Kollation freilich auch als Objekt auffassen;
                  dann handelt es sich um ein Dokument, das die Ergebnisse eines solchen Vergleiches
                  enthält.\\
            
        Eine Kollation kann verschiedene Ausprägungen annehmen: Unter nicht-digitalen
                  Formen können etwa Marginalnotizen am Rand einer gedruckten Textseite subsumiert
                  werden, ebenso kann eine synoptische Tabellenform gestaltet sein, in welcher jeder
                  Textzeuge eine Spalte einnimmt und die Zeilen so aufeinander abgestimmt sind, dass
                  die Lesarten der verschiedenen Versionen miteinander korrespondieren.\\
            
        Im digitalen Bereich kann solch eine Form einer Kollation leicht mithilfe eines
                  Tabellen-Programms erzeugt und dann beispielsweise als HTML-Tabelle auf einer
                  Webseite veröffentlicht werden. Freilich existieren auch andere, ausschließlich
                  digitale Formate. Die Richtlinien der \href{http://gams.uni-graz.at/o:konde.178}{TEI} nennen drei verschiedene Arten, eine Kollation in einem
                  TEI-kompatiblen \href{http://gams.uni-graz.at/o:konde.215}{XML}-Format
                  auszudrücken (Kap. 12.2):\\
            
        \begin{itemize}\item {Location-referenced Method (empfohlen für die Retrodigitalisierung)}\item {Double end-point attachment method (orientiert sich an einem Grundtext, kann
                     mit überlappenden Strukturen umgehen)}\item {Parallel segmentation method (benötigt keinen Grundtext, kann nicht mit
                     überlappenden Strukturen umgehen)}\end{itemize}Eine weitere, zunehmend beliebte Form ist jene des \emph{variant
                     graph} (oder \emph{collation graph}), der keinen Grundtext
                  benötigt und mit überlappenden Strukturen umgehen kann. Dieser kann zwar in
                  mehreren Datenformaten gespeichert werden, bedarf aber in der Regel einer
                  Spezialsoftware wie \emph{TRAViz} oder \emph{Stemmaweb}.\\
            
        Während die Kollation traditionell durch Forscherinnen und Forscher manuell
                  vorgenommen wird, existieren nun auch einige Software-Programme, durch welche die
                  Arbeit automatisiert oder wenigstens erleichtert wird. Die heute meist genutzten
                  Programme sind \emph{TUSTEP} (siehe auch sein XML-kompatibler
                  Nachfolger \emph{TXSTEP}), \emph{CollateX} und \emph{JuXta}. Jedes dieser Programme erlaubt den Export der
                  Kollations-Daten in eines oder mehrere der oben genannten Formate.\\
            
        \subsection*{Literatur:}\begin{itemize}\item Andrews, Tara: What We Talk About When We Talk About Collation. In: Advances in Digital Scholarly Editing. Leiden: 2017, S. 231–234.\item Collation. URL: \url{https://lexiconse.uantwerpen.be/index.php/lexicon/transcription/}\item P5: Guidelines for Electronic Text Encoding and
                              Interchange. Ch. 12 Critical Apparatus. URL: \url{https://www.tei-c.org/release/doc/tei-p5-doc/en/html/TC.html}\end{itemize}\subsection*{Software:}\href{https://collatex.net}{CollateX}, \href{http://juxtacommons.org}{Juxta-Commons}, \href{http://www.tustep.uni-tuebingen.de/}{TUSTEP}, \href{https://stemmaweb.net/}{The Stemmaweb
                           Project}, \href{http://www.traviz.vizcovery.org}{TRAViz}, \href{http://www.txstep.de}{TXSTEP}\subsection*{Verweise:}\href{https://gams.uni-graz.at/o:konde.210}{Visualisierungstools}, \href{https://gams.uni-graz.at/o:konde.192}{Textkritik in digitalen
                           Editionen}, \href{https://gams.uni-graz.at/o:konde.174}{Synopse}\subsection*{Themen:}Einführung, Digitale Editionswissenschaft\subsection*{Lexika}\begin{itemize}\item \href{https://wiki.helsinki.fi/display/stemmatology/Collation}{Parvum Lexicon Stemmatologicum}\item \href{https://lexiconse.uantwerpen.be/index.php/lexicon/collation/}{Lexicon of Scholarly Editing}\end{itemize}\subsection*{Zitiervorschlag:}Andrews, Tara. 2021. Kollation. In: KONDE Weißbuch. Hrsg. v. Helmut W. Klug unter Mitarbeit von Selina Galka und Elisabeth Steiner im HRSM Projekt "Kompetenznetzwerk Digitale Edition". URL: https://gams.uni-graz.at/o:konde.105\newpage\section*{Kommentar in digitalen Editionen} \emph{Straub, Wolfgang; wolfgang.straub@aau.at }\\
        
    Die Zeiten, in denen etwa \href{http://gams.uni-graz.at/o:konde.93}{historisch-kritische Ausgaben} ohne Kommentar erschienen, sind längst
                  vorbei; Kommentare gehören zum editorischen Inventar. Wie sich ein Kommentar
                  definiert, was er beinhalten und wie umfassend er sein soll, darüber gibt es,
                  trotz der langen Geschichte der Textsorte, keinen editionswissenschaftlichen
                  Konsens. Einigkeit könnte wohl bei einigen allgemeinen Grundsätzen erreicht
                  werden, etwa dass beim Kommentar „[d]ie Beschränkung auf das im editorischen
                  Kontext Notwendige [...] als Richtlinie gelten“ solle (Plachta 1997, S.
                     118) oder man von einem „Brückenschlag zwischen Text und Leser“
                     (Martens 1993, S. IX) sprechen könne. Die Bandbreite der
                  Kommentierungsmöglichkeiten reicht von einem asketischen bis zum Vollkommentar;
                  die beiden geläufigsten Formen sind Überblickskommentar sowie Stellen- und
                  Sachkommentar. Beide dienen in erster Linie dazu, Textanordnung und -konstitution
                  zu begründen, den Entstehungszusammenhang zu vergegenwärtigen und
                  Sacherläuterungen oder Worterklärungen zu liefern. (Woesler 1997, S.
                     23)\\
            
        Nur ein kleiner Teil \href{http://gams.uni-graz.at/o:konde.59}{Digitaler
                     Editionen} weist Inhalte auf, die \emph{expressis verbis}
                  eine Kommentarfunktion ausweisen; und der Kommentar in Digitalen Editionen
                  orientiert sich meist an den aus gedruckten Editionen bekannten und gebräuchlichen
                  Formen – beides haben vergleichende Studien ergeben. (Bleier/Klug 2020, 103;
                     Zihlmann-Märki 2020, S. 167) Kommentierung erfolgt in Editionen nicht
                  selten in Formen, die nicht explizit als Kommentar deklariert werden. Mit Rolf
                  Bräuers weitem Kommentarbegriff ließen sich auch explizit digitale Inhalte wie
                  Hyperlink-Materialien, Kontaktmöglichkeiten oder \emph{Social Input
                  } dem Kommentar zurechnen. (Bräuer 1993) Hinzu kommt, dass
                  Editionen im „digitalen Paradigma“ (Sahle 2013, S. 149f.) bereits mit
                  ihrer Datenmodellierung eine Kommentierung vornehmen; Roman Bleier und Helmut W.
                  Klug kritisieren, dass dies in vielen Editionen zu wenig explizit gemacht werde.
                     (Bleier/Klug 2020, S. 103)\\
            
        Kommentar in Digitalen Editionen bietet die Möglichkeit, Hypertexte innerhalb und
                  außerhalb des Internetauftritts der Edition zu verlinken; digitaler Kommentar kann
                  sich in Richtung eines ‘sozialen Kommentars’ bewegen, indem er Benutzer-Input
                  (angeleitet und redigiert) integriert; und er kann den Boden für weitergehende
                  Analysen bereiten, indem er etwa Forschungsdaten zum Download oder semantisches
                  Markup anbietet. Im Interface kann der Kommentar aus dem Anmerkungsapparat in die
                  Leseansicht der Edition wandern, man kann der Userin, dem User etwa anbieten, den
                  Kommentar in einer synoptische Ansicht oder als Popup nach Bedarf ein- und
                  auszublenden. Zu den Vorteilen eines digitalen gegenüber eines gedruckten
                  Kommentars gehören zudem die Möglichkeit, Kommentare in einer Datenbank zu sammeln
                  und bei Einzelstellen darauf zu verweisen (statt, wie bei Druckeditionen üblich,
                  beim ersten Vorkommen des Lemmas), sowie die Korrigier- und Aktualisierbarkeit,
                  womit – kontinuierliche Redaktion vorausgesetzt – dem Altern der Kommentareinträge
                  entgegengewirkt werden kann.\\
            
         Digitaler Kommentar wird sinnvollerweise mit \href{http://gams.uni-graz.at/o:konde.147}{Norm-} und \href{http://gams.uni-graz.at/o:konde.25}{Metadaten} (etwa \href{http://gams.uni-graz.at/o:konde.109}{GND})
                  angereichert und mit anderen digitalen Ressourcen vernetzt. Bei Verlinkungen ist
                  in Anbetracht der Ubiquität verfügbarer Information allerdings zu bedenken, dass
                  der Qualitätsanspruch der jeweiligen Quelle geklärt werden muss. Patricia
                  Zihlmann-Märki weist zu Recht auf die Gefahr einer „Abkehr von einem Standard“
                  hin, die mit der beobachtbaren Tendenz zur Ablösung der Stellenkommentare durch
                  Erschließungen und Visualisierungen einhergehe – einer Tendenz, die auch auf gerne
                  in Kauf genommene Einsparungen (finanziell) aufwändiger Kommentararbeit hinweisen
                  könnte (Zihlmann-Märki 2020, S. 174)\\
            
        \subsection*{Literatur:}\begin{itemize}\item Bleier, Roman; Klug, Helmut W.: Funktion und Umfang des Kommentars in Digitalen
                              Editionen mittelalterlicher Texte: Eine Bestandsaufnahme. In: Annotieren, Kommentieren, Erläutern. Aspekte des
                              Medienwandels. Berlin/Boston: 2020, S. 97–112.\item Bräuer, Rolf: Unterschiedliche Kommentierungstypen in Ausgaben
                              mittelalterlicher Texte 7: 1993, S. 135–143.\item Martens, Gunter: Vorwort des Herausgebers. In: Kommentierungsverfahren und Kommentarformen. Tübingen: 1993, S. IX–X.\item Plachta, Bodo: Editionswissenschaft. Eine Einführung in Methode und
                              Praxis der Edition neuerer Texte Editionswissenschaft: 1997.\item Sahle, Patrick: Digitale Editionsformen. Zum Umgang mit der
                              Überlieferung unter den Bedingungen des Medienwandels. Teil 2:
                              Befunde, Theorie und Methodik. Norderstedt: 2013.\item Woesler, Winfried: Zu den Aufgaben des heutigen Kommentars. In: editio 7: 1993, S. 18–35.\item Zihlmann-Märki, Patricia: Kommentierung in gedruckten und digitalen
                              Briefausgaben. In: Annotieren, kommentieren, erläutern. Aspekte des
                              Medienwandels. Berlin/Boston: 2020, S. 159–174.\end{itemize}\subsection*{Verweise:}\href{https://gams.uni-graz.at/o:konde.93}{Historisch-kritische Ausgabe}, \href{https://gams.uni-graz.at/o:konde.59}{Digitale Edition}, \href{https://gams.uni-graz.at/o:konde.109}{GND}, \href{https://gams.uni-graz.at/o:konde.147}{Normdaten}, \href{https://gams.uni-graz.at/o:konde.25}{Metadaten}\subsection*{Projekte:}\href{https://gams.uni-graz.at/context:kofler}{Werner
                           Kofler: Kommentar zur Werkausgabe}, \href{https://fontane-nb.dariah.eu/index.html}{Theodor Fontane: Notizbücher. Digitale genetisch-kritische und
                           kommentierte Edition}, \href{https://edition.onb.ac.at/okopenko/context:okopenko/methods/sdef:Context/get}{Tagebücher Andreas Okopenko}\subsection*{Themen:}Digitale Editionswissenschaft\subsection*{Lexika}\begin{itemize}\item \href{https://edlex.de/index.php?title=Kommentar}{Edlex: Editionslexikon}\end{itemize}\subsection*{Zitiervorschlag:}Straub, Wolfgang. 2021. Kommentar in digitalen Editionen. In: KONDE Weißbuch. Hrsg. v. Helmut W. Klug unter Mitarbeit von Selina Galka und Elisabeth Steiner im HRSM Projekt "Kompetenznetzwerk Digitale Edition". URL: https://gams.uni-graz.at/o:konde.34\newpage\section*{Konkordanz} \emph{Klug, Helmut W.; helmut.klug@uni-graz.at }\\
        
    Unter dem Begriff ‘Konkordanz’ werden im Bereich der geisteswissenschaftlichen
                  Forschung unterschiedliche Umsetzungen verstanden: In den Bibel- und
                  Textwissenschaften kann das eine Verknüpfung von Themen, Phrasen und Wörtern eines
                  Textes zu einer normierten Wortliste (\href{http://gams.uni-graz.at/o:konde.115}{Lemmatisierung}, Index, Register) sein, in der Altphilologie und
                  mediävistischen Germanistik eine Verweisliste zur Überlieferung von Textpassagen,
                  Phrasen und Wörtern und in der Sprachwissenschaft z. B. die Auflistung von \emph{Key Words in Context} (KWIC) usw. Bei der Erstellung des \emph{Index Thomasticus} kamen erstmals computergestützten
                  Konkordanzmethoden zum Einsatz. \\
            
        Diese Art der Texterschließung ist mit großem Arbeitsaufwand verbunden und kann
                  daher mit Hilfe digitaler Tools leichter und in größerem Umfang umgesetzt werden;
                     \href{http://gams.uni-graz.at/o:konde.168}{Semantic Web Technologien}
                  erweitern die Vernetzungs- und Recherchemöglichkeiten (Haugen/Apollon 2014,
                     S. 53) durch Methodenkombination. In Kapitel 16 der TEI \emph{Guidelines}(Linking, Segmentation, and Alignment: 16.5.1 Correspondence) wird
                  z. B. veranschaulicht, wie die \href{http://gams.uni-graz.at/o:konde.178}{TEI} zur \href{http://gams.uni-graz.at/o:konde.137}{Modellierung}
                  von Konkordanzen eingesetzt werden kann.\\
            
        \subsection*{Literatur:}\begin{itemize}\item Eide, Øyvind; Olstad, Vemund: TEI for Interactive Concordances: The New Menota Search
                              System: 2013. URL: \url{https://www.tei-c.org/Vault/MembersMeetings/2013/wp-content/uploads/2013/09/Eide+Olstad.pdf}.\item Rosselli Del Turco, Roberto: After the Editing is Done. Designing a Graphic User
                              Interface for Digital Editions. In: Digital Medievalist 7: 2011.\item Haugen, Odd Einar; Apollon, Daniel: The Digital Turn in Textual Scholarship. Historical and
                              Typological Perspectives. In: Digital critical editions. Urbana, Chicago, Springfield: 2014, S. 35–57.\item Sinclair, Stéfan; Ruecker, Stan; Radzikowska, Milena: Information Visualization for Humanities
                              Scholars. In: Literary Studies in the Digital Age: 2013.\item Sinclair, Stéfan; Rockwell, Geoffrey: Text Analysis and Visualization: Making Meaning
                              Count. In: A New Companion to Digital Humanities. Chichester: 2016, S. 274–290.\item 16 Linking, Segmentation, and Alignment. URL: \url{https://tei-c.org/release/doc/tei-p5-doc/en/html/SA.html}\end{itemize}\subsection*{Software:}\href{http://www.tustep.uni-tuebingen.de/}{TUSTEP}, \href{https://voyant-tools.org/}{Voyant}\subsection*{Verweise:}\href{https://gams.uni-graz.at/o:konde.115}{Lemmatisierung}, \href{https://gams.uni-graz.at/o:konde.145}{NLP}, \href{https://gams.uni-graz.at/o:konde.146}{Normalisierung}, \href{https://gams.uni-graz.at/o:konde.174}{Synopse}\subsection*{Projekte:}\href{https://www.corpusthomisticum.org/wintrode.html}{Corpus
                           Thomasticum}, \href{http://mhdbdb.sbg.ac.at/}{Mittelhochdeutsche
                           Begriffsdatenbank (MHDBDB)}\subsection*{Themen:}Einführung, Digitale Editionswissenschaft\subsection*{Lexika}\begin{itemize}\item \href{https://edlex.de/index.php?title=Konkordanz}{Edlex: Editionslexikon}\item \href{https://lexiconse.uantwerpen.be/index.php/lexicon/concordance/}{Lexicon of Scholarly Editing}\end{itemize}\subsection*{Zitiervorschlag:}Klug, Helmut W. 2021. Konkordanz. In: KONDE Weißbuch. Hrsg. v. Helmut W. Klug unter Mitarbeit von Selina Galka und Elisabeth Steiner im HRSM Projekt "Kompetenznetzwerk Digitale Edition". URL: https://gams.uni-graz.at/o:konde.106\newpage\section*{Kontrollierte Vokabularien: GND} \emph{Galka, Selina; selina.galka@uni-graz.at}\\
        
    Die Gemeinsame Normdatei (GND) ermöglicht die Nutzung und Verwaltung von Datensätzen bzw. Normdaten zu Personen, Körperschaften, Geografika, Sachbegriffen und Werken, wobei diese Datensätze nach dem Regelwerk \emph{Resource Description and Access} (RDA) erfasst werden. (Trunk 2020)\\
            
        Die GND wird seit 2012 kooperativ von der Deutschen Nationalbibliothek (DNB), deutschsprachigen Bibliothekenverbünden und der Zeitschriftendatenbank (ZDB) betrieben und fasst seit 2012 die bis dahin getrennt existierenden Personennamendatei (PND), Gemeinsame Körperschaftsdatei (GKD), Schlagwortnormdatei (SWD) und die Einheitssachtiteldatei des Deutschen Musikarchivs (DMA-EST) zusammen. (Behrens-Neumann 2012, S. 25)\\
            
        Jede Entität in der GND weist einen eindeutigen und stabilen Identifikator auf, die GND-ID, mit welcher die Normdaten untereinander, aber auch mit externen Datensätzen oder Webressourcen verknüpft werden können – dadurch entsteht ein Netzwerk aus Daten, welches maschinell auswertbar wird (vgl. \emph{\href{http://gams.uni-graz.at/o:konde.167}{Semantic Web}}). (DNB: GND 2019)\\
            
        Die GND wurde zunächst vor allem im Bibliothekswesen genutzt; mittlerweile spielt sie aber auch in Archiven, Museen und im wissenschaftlichen Bereich – hier eben auch in Bezug auf Digitale Editionen – eine wichtige Rolle. Vor allem Editionen von nicht-fiktionalen Texten wie \href{http://gams.uni-graz.at/o:konde.39}{Brief-} oder \href{http://gams.uni-graz.at/o:konde.175}{Tagebucheditionen} können von derartigen Normdateien profitieren, da man hier auch stark an den historischen Aspekten, den Netzwerken und den Diskursen der Texte interessiert ist. (Stadler 2012, S. 176) So können z. B. Personen in den Texten mittels \href{http://gams.uni-graz.at/o:konde.178}{TEI} ausgezeichnet und identifiziert werden:\\
            
        \begin{verbatim}<persName ref=”http://d-nb.info/gnd/118540238”>Johann Wolfgang von Goethe</persName>\end{verbatim}Ähnlich kann mit Orten verfahren werden, vgl. hierzu den Artikel zu \emph{\href{http://gams.uni-graz.at/o:konde.107}{GeoNames}}.\\
            
        \subsection*{Literatur:}\begin{itemize}\item Behrens-Neumann, Renate: Die Gemeinsame Normdatei (GND). Ein Projekt kommt zum Abschluss. In: Dialog mit Bibliotheken 24: 2012, S. 25–28.\item Gemeinsame Normdatei. URL: \url{https://www.dnb.de/gnd}\item Stadler, Peter: Normdateien in Editionen. In: editio 26: 2012, S. 174–183.\item Informationsseite zur GND. URL: \url{https://wiki.dnb.de/display/ILTIS/Informationsseite+zur+GND}\end{itemize}\subsection*{Verweise:}\href{https://gams.uni-graz.at/o:konde.107}{Geonames}, \href{https://gams.uni-graz.at/o:konde.111}{VIAF}, \href{https://gams.uni-graz.at/o:konde.108}{Getty}, \href{https://gams.uni-graz.at/o:konde.112}{Wikidata}, \href{https://gams.uni-graz.at/o:konde.167}{Semantic Web}, \href{https://gams.uni-graz.at/o:konde.165}{RNAB}, \href{https://gams.uni-graz.at/o:konde.147}{Normdaten}\subsection*{Themen:}Einführung, Metadaten\subsection*{Lexika}\begin{itemize}\item \href{https://edlex.de/index.php?title=Gemeinsame_Normdatei_(GND)}{Edlex: Editionslexikon}\end{itemize}\subsection*{Zitiervorschlag:}Galka, Selina. 2021. Kontrollierte Vokabularien: GND. In: KONDE Weißbuch. Hrsg. v. Helmut W. Klug unter Mitarbeit von Selina Galka und Elisabeth Steiner im HRSM Projekt "Kompetenznetzwerk Digitale Edition". URL: https://gams.uni-graz.at/o:konde.109\newpage\section*{LZA-Datenformate: Vektorgrafikformate} \emph{Lang, Sarah; sarah.lang@uni-graz.at }\\
        
    Vektorgrafiken sind Grafiken, die im Gegensatz zu Pixelgrafiken (auch ‘Rastergrafiken’ genannt) durch sogenannte ‘grafische Primitive’ wie Linien, Punkte, Pfade oder einfache geometrische Formen und Zusatzparameter zur Festlegung ihres Aussehens beschrieben werden. Diese mathematisch-geometrischen Eigenschaften erlauben, dass Vektorgrafiken ohne Qualitätsverlust beliebig skaliert werden können. Trotz dieser Möglichkeit, Vektorgrafiken beispielsweise beliebig und ohne Qualitätsverlust zu vergrößern, wird weniger Speicherplatz benötigt. Der Speicherbedarf wird durch die Vergrößerung nicht verändert, wohingegen Rastergrafiken auf eine Größe festgelegt sind. Sobald im Falle der Rastergrafiken die Pixelzahl für die Vergrößerung zu gering ist, werden einzelne Bildpunkte sichtbar: das Bild ist ‘verpixelt’. Beim Verkleinern hingegen müssen Informationspunkte verschmolzen werden, wodurch die Informationsdichte sinkt. Vektorgrafiken sind beliebig skalierbar,  eignen sich allerdings weniger für sehr kleinteilige Bilder, wie dies etwa bei Fotografien der Fall ist. Doch erlauben sie die exakte Beschreibung der aus geometrischen Elementen bestehenden Grafiken, wohingegen in Rastergrafiken Formen lediglich mithilfe von Pixeln approximiert werden. Die Zusatzparameter zur Darstellung der Grafik, wie etwa die Linienfarbe, -stärke oder Füllfarbe können unabhängig von den eigentlichen Formen verändert werden, die mathematisch-geometrisch beschrieben sind. Auch erlauben Vektorgrafiken das Strukturieren und Gruppieren von Bildinformationen durch Ebenen (\emph{Layer}).\\
            
        2001 entwickelte das W3C für zweidimensionale Vektorgrafiken das \href{http://gams.uni-graz.at/o:konde.215}{XML}-basierte \emph{Scalable Vector Graphics}(SVG)-Format, das für die \href{http://gams.uni-graz.at/o:konde.6}{Langzeitarchivierung (LZA)} empfohlen wird. Es ist \href{http://gams.uni-graz.at/o:konde.152}{offen} und weit verbreitet. Neben geometrischen Formen, \href{http://gams.uni-graz.at/o:konde.25}{Metadaten}, Text, grafischen Primitiven oder auch eingebetteten Rastergrafiken können zudem Animationen oder Skripte in SVG-Daten enthalten sein. Auf Skripte sollte allerdings im Kontext der Langzeitarchivierung verzichtet werden. \\
            
        Da Vektorgrafiken größtenteils aus PDF/A-Dateien extrahiert werden können, wird mitunter auch empfohlen, diese zusätzlich als solche abzuspeichern. Dabei geht zwar die Bearbeitbarkeit verloren, doch ein Eindruck des intendierten Aussehens bleibt erhalten. Das \emph{Portable Document Format} (PDF) wurde 1993 von \emph{Adobe} entwickelt. Als offenes, plattformunabhängiges Dateiformat wurde es 2008 als ISO-Standard zertifiziert. Für die LZA wurde das PDF/A-Format entwickelt. Vektorgrafiken können in PDF-Dokumenten enthalten sein, wo sie durch Pfade beschrieben werden, doch nicht alle Daten in PDFs sind automatisch Vektorgrafiken. PDF/A wird zur Langzeitarchivierung empfohlen. Die proprietären Formate wie INDD von \emph{Adobe InDesign} oder AI von \emph{Adobe Illustrator} sind zur LZA nicht geeignet. \\
            
        \subsection*{Literatur:}\begin{itemize}\item Bilder – Vektorgrafiken und CAD-Daten. URL: \url{https://www.ianus-fdz.de/it-empfehlungen/vektorgrafiken}\item SVG - Implementations. URL: \url{https://www.w3.org/Graphics/SVG/WG/wiki/Implementations}\item Vector graphics. URL: \url{https://en.wikipedia.org/wiki/Vector_graphics}\item Computer Graphics Metafile. URL: \url{https://en.wikipedia.org/wiki/Computer_Graphics_Metafile}\item Scalable Vector Graphics. URL: \url{https://de.wikipedia.org/wiki/Scalable_Vector_Graphics}\item PDF, Kap. 3: Technical Overview. URL: \url{https://en.wikipedia.org/wiki/PDF#Vector_graphics}\item About SVG. 2d Graphics in XML. URL: \url{https://www.w3.org/Graphics/SVG/About.html}\end{itemize}\subsection*{Software:}\href{https://inkscape.org/en/}{Inkscape}, \href{https://vectr.com/}{Vectr}\subsection*{Verweise:}\href{https://gams.uni-graz.at/o:konde.6}{Digitale Nachhaltigkeit}, \href{https://gams.uni-graz.at/o:konde.152}{Open Access}, \href{https://gams.uni-graz.at/o:konde.124}{Metadatenformate für Bilddateien}, \href{https://gams.uni-graz.at/o:konde.122}{Bildformate}, \href{https://gams.uni-graz.at/o:konde.60}{Digitalisierung}\subsection*{Themen:}Archivierung\subsection*{Zitiervorschlag:}Lang, Sarah. 2021. LZA-Datenformate: Vektorgrafikformate. In: KONDE Weißbuch. Hrsg. v. Helmut W. Klug unter Mitarbeit von Selina Galka und Elisabeth Steiner im HRSM Projekt "Kompetenznetzwerk Digitale Edition". URL: https://gams.uni-graz.at/o:konde.125\newpage\section*{Lagenvisualisierung} \emph{Raunig, Elisabeth; elisabeth.raunig@uni-graz.at }\\
        
    Mittelalterliche Codices sind aus Bündeln von Doppelblättern aus Pergament bzw. Papier – den sogenannten ‘Lagen’ – zusammengesetzt. Diese geben Einblick in die Entstehung des Codex und auch in die Geschichte einzelner Texte. Das Ziel einer Lagenvisualisierung ist es, die Anordnung der Doppelblätter in einer Lage und allfällige Unregelmäßigkeiten einzelner Lagen visuell zu verdeutlichen. \\
            
        Die traditionelle \href{http://gams.uni-graz.at/o:konde.103}{Kodikologie} bietet diverse Lagenvisualisierungen: Zum Beispiel kann die Lagenanordnung mit ineinander verschachtelten oben offenen Dreiecken, die diese Bögen visualisieren, dargestellt werden. Bei fehlenden Seiten wird dabei ein Schenkel des offenen Dreiecks verkürzt gezeichnet. Eine textbasierte, platzsparende Möglichkeit der traditionellen Lagenvisualisierung ist die Croust’sche Lagenformel. Hier werden römische und arabische Zahlzeichen verwendet, um den Aufbau eines Codex darzustellen: Die römische Zahl wird für die Lage verwendet, z. B. steht III für eine Ternio, also eine Lage, die aus drei Doppelblättern besteht, daran anschließend steht die Nummer des letzten Blattes der Lage mit arabischen Zahlzeichen. Fehlende oder hinzugefügte Blätter werden mit + oder - und arabischen Zeichen angefügt: z. B. (III-1)42.\\
            
        Diese Möglichkeiten kann auch eine digitale Lagenvisualisierung nutzen. Jedoch muss einer Visualisierung eine Auszeichnung der Lagen zugrunde liegen. Die \href{http://gams.uni-graz.at/o:konde.178}{TEI} bietet dafür keine vorgegebene Lösung. Das Projekt \emph{VisColl} von Dot Porter (Porter et al. 2017, S. 81–100) versucht eine Möglichkeit zu bieten, Lagen und Faksimiles unabhängig von den traditionellen Lagenbeschreibungen und -formeln darzustellen. \emph{VisColl} lehnt sich in der Visualisierung der Lagen an die offene Dreiecksversion an, nutzt jedoch ineinander geschachtelte und nach rechts geöffnete Halbkreise. \emph{VisColl} verwendet XML für die Anordnung der Lagen (\emph{Collation Modeller}) und ein Excel-File für die Faksimiles, das in \href{http://gams.uni-graz.at/o:konde.215}{XML} umgewandelt wird. Mit Hilfe dieser zwei Dateien und einem zur Verfügung gestellten \emph{Collation Visualisation Tool} werden diverse Visualisierungen in HTML zurückgeliefert.\\
            
        \subsection*{Literatur:}\begin{itemize}\item Porter, Dot: VisColl: A New Collation Tool for Manuscript Studies. In: Kodikoligie und Paläographie im Digitalen Zeitalter 4. Norderstedt: 2017, S. 81–100.\end{itemize}\subsection*{Software:}\href{https://github.com/leoba/VisColl}{Viscoll}\subsection*{Projekte:}\href{https://viscoll.org}{VisColl}, \href{https://github.com/KislakCenter/VisColl/}{Viscoll: Codebase}\subsection*{Verweise:}\href{https://gams.uni-graz.at/o:konde.178}{TEI}, \href{https://gams.uni-graz.at/o:konde.92}{Handschriftenbeschreibung}, \href{https://gams.uni-graz.at/o:konde.103}{Kodikologie}, \href{https://gams.uni-graz.at/o:konde.215}{XML}, \href{https://gams.uni-graz.at/o:konde.54}{Datenvisualisierung}\subsection*{Themen:}Annotation und Modellierung, Digitale Editionswissenschaft, Software und Softwareentwicklung\subsection*{Zitiervorschlag:}Raunig, Elisabeth. 2021. Lagenvisualisierung. In: KONDE Weißbuch. Hrsg. v. Helmut W. Klug unter Mitarbeit von Selina Galka und Elisabeth Steiner im HRSM Projekt "Kompetenznetzwerk Digitale Edition". URL: https://gams.uni-graz.at/o:konde.113\newpage\section*{Langzeitarchivierung} \emph{Klug, Helmut W.; helmut.klug@uni-graz.at}\\
        
    siehe \href{http://gams.uni-graz.at/o:konde.6}{Digitale Nachhaltigkeit}\\
            
        \subsection*{Verweise:}\href{https://gams.uni-graz.at/o:konde.6}{Digitale Nachhaltigkeit}\subsection*{Zitiervorschlag:}Klug, Helmut W. 2021. Langzeitarchivierung. In: KONDE Weißbuch. Hrsg. v. Helmut W. Klug unter Mitarbeit von Selina Galka und Elisabeth Steiner im HRSM Projekt "Kompetenznetzwerk Digitale Edition". URL: https://gams.uni-graz.at/o:konde.114\newpage\section*{Lemmatisierung} \emph{Resch, Claudia; claudia.resch@oeaw.ac.at }\\
        
    In Zusammenhang mit der Erschließung von digitalen Textdaten meint Lemmatisierung die Rückführung eines vorkommenden Wortes – einer Vollform – auf seine Grundform (auch: Lemma, Nennform, Basisform oder kanonische Form), die stellvertretend für das gesamte Flexionsparadigma eines Wortes steht. So werden etwa die Wortformen \emph{helfe}, \emph{hilfst}, \emph{hilft}, \emph{helft}, \emph{geholfen} oder \emph{hilf} auf ein gemeinsames Lemma \emph{helfen} zusammengeführt. Durch diesen Arbeitsschritt kann die Suche erheblich erleichtert werden: Anstatt alle Formen eines Wortes abfragen zu müssen, erhalten Benutzerinnen und Benutzer durch die Eingabe einer Grundform alle ihr zugeordneten Wortformen. Besondere Bedeutung hat die Lemmatisierung für \href{http://gams.uni-graz.at/o:konde.94}{historische Korpora} mit höherer grafischer und formaler Varianz bzw. für regionale Sprachvarietäten oder Daten gesprochener Sprache. Durch die Rückführung der Non-Standard-Daten auf eine einheitliche Grundform können diese Varianten ebenfalls mit einem einzigen Suchbefehl gefunden werden.\\
            
        Die Ansetzung des Lemmas erfolgt nach bestimmten Richtlinien und kann auch mit Hilfe von Tools – sogenannter ‘Lemmatisierer’ (\emph{lemmatizer}) – durchgeführt werden. Diese versuchen verschiedene Wortformen mit ihrer jeweiligen Grundform zu verbinden und sind dazu mit anderen Ressourcen, etwa mit maschinenlesbaren Lexika, ausgestattet, in denen hinterlegt ist, welcher Flexionssystematik bestimmte Worte folgen. In jedem Fall muss aber für Benutzerinnen und Benutzer nachvollziehbar dokumentiert sein, nach welchen Regeln lemmatisiert worden ist.\\
            
        Die Lemmatisierung ist – gemeinsam mit der Tokenisierung und der Wortartenzuordnung (\emph{\href{http://gams.uni-graz.at/o:konde.156}{Part-of-Speech-Tagging}}) – Teil der linguistischen Annotation.\\
            
        \subsection*{Literatur:}\begin{itemize}\item Perkuhn, Rainer; Keibel, Holger; Kupietz, Marc: Korpuslinguistik. Paderborn: 2012.\item Harras, Gisela; Proost, Kristel: Strategien der Lemmatisierung von Idiomen. In: Deutsche Sprache 30: 2002, S. 167–183.\item Hirschmann, Hagen: Korpuslinguistik. Eine Einführung. Mit Abbildungen und Grafiken Korpuslinguistik. Berlin: 2019, URL: \url{https://link.springer.com/book/10.1007%2F978-3-476-05493-7}.\item Lemnitzer, Lothar; Zinsmeister, Heike: Korpuslinguistik. Eine Einführung. Tübingen: 2010.\item Manning, Christopher D.; Raghavan, Prabhakara; Schütze, Hinrich: Introduction to information retrieval. Cambridge Univ. Press. 2008. - Google Suche. URL: \url{https://www.google.com/search?client=safari&rls=en&q=Manning,+Christopher+D.;+Raghavan,+Prabhakara;+Sch%C3%BCtze,+Hinrich:+Introduction+to+information+retrieval.+Cambridge+Univ.+Press.+2008.&ie=UTF-8&oe=UTF-8}\end{itemize}\subsection*{Software:}\href{https://weblicht.sfs.uni-tuebingen.de/weblicht/}{weblicht}, \href{https://www.nltk.org/}{Natural Language Toolkit (nltk)}, \href{https://spacy.io/}{spacy }, \href{http://lemmatise.ijs.si/}{LemmaGen}, \href{http://alumni.media.mit.edu/~hugo/montylingua/index.html}{MontyLingua}, \href{https://sites.google.com/site/morfetteweb/}{Morfette}, \href{https://cst.dk/online/lemmatiser/uk/}{CST's Lemmatiser}\subsection*{Verweise:}\href{https://gams.uni-graz.at/o:konde.17}{Textannotation}, \href{https://gams.uni-graz.at/o:konde.156}{Part-of-Speech-Tagging}, \href{https://gams.uni-graz.at/o:konde.212}{Weblicht}, \href{https://gams.uni-graz.at/o:konde.216}{xsl-Tokenizer}, \href{https://gams.uni-graz.at/o:konde.145}{NLP}, \href{https://gams.uni-graz.at/o:konde.141}{NER}\subsection*{Themen:}Einführung, Natural Language Processing\subsection*{Projekte:}\href{http://www.deutschestextarchiv.de/}{Deutsches Textarchiv}, \href{https://acdh.oeaw.ac.at/abacus/}{Austrian Baroque Corpus (ABaC:us)}, \href{https://traveldigital.acdh.oeaw.ac.at/}{travel!digital}\subsection*{Lexika}\begin{itemize}\item \href{https://edlex.de/index.php?title=Lemmatisierung}{Edlex: Editionslexikon}\end{itemize}\subsection*{Zitiervorschlag:}Resch, Claudia. 2021. Lemmatisierung. In: KONDE Weißbuch. Hrsg. v. Helmut W. Klug unter Mitarbeit von Selina Galka und Elisabeth Steiner im HRSM Projekt "Kompetenznetzwerk Digitale Edition". URL: https://gams.uni-graz.at/o:konde.115\newpage\section*{Leseausgabe} \emph{Moser, Doris; doris.moser@aau.at }\\
        
    Der Begriff ‘Leseausgabe’ bezeichnet die niederschwelligste Form einer auf wissenschaftlicher Grundlage erarbeiteten Edition eines (vorwiegend) literarischen Werkes in Buchform. Das Zielpublikum ist eine interessierte Leserschaft ohne einschlägige wissenschaftliche Vorbildung (\emph{science to public}), die in der Leseausgabe einen authentischen Text vorfinden soll. Eine Leseausgabe enthält den edierten Text und ein Nachwort mit Hinweisen auf die verwendete Textgrundlage. Das Nachwort (oder Vorwort) kann knapp ausfallen oder ausführlich sein und über Entstehungs- und Rezeptionsgeschichte, gelegentlich auch über die Überlieferungsgeschichte informieren. Mitunter enthält sie einen Abschnitt, in dem grundlegende editorische Entscheidungen angegeben werden, sowie rudimentäre Stellenkommentare. Ein kritischer Apparat und ein Variantenverzeichnis bleiben der kritischen bzw. \href{http://gams.uni-graz.at/o:konde.93}{historisch-kritischen Ausgabe} vorbehalten.\\
            
        Leseausgaben existieren als Edition eines Einzelwerks, einer Werkauswahl (\href{http://gams.uni-graz.at/o:konde.213}{Werkausgabe}) oder des Gesamtwerks (\href{http://gams.uni-graz.at/o:konde.91}{Gesamtausgabe}) einer Autorin, eines Autors. Allgemein verbindliche editorische Standards für Leseausgaben gibt es nicht, zu unterschiedlich sind Ausgangslagen (z. B. Neuausgabe vs. \href{http://gams.uni-graz.at/o:konde.140}{Nachlassedition}) und Zielvorstellungen (z. B. wissenschaftliche vs. kommerzielle Aspekte). Auch beschäftigt sich die Editionswissenschaft nicht systematisch mit dem Typus Leseausgabe, er wird nur in Abgrenzung zu \emph{science-to-science}-Typen der Edition, wie der kritischen (kommentierten) Edition oder der historisch-kritischen Edition, behandelt. Der Übergang von einer \href{http://gams.uni-graz.at/o:konde.173}{Studienausgabe}, der meist preisgünstigeren kleinen Schwester der kritischen bzw. historisch-kritischen Edition, zur Leseausgabe ist fließend, gelegentlich werden die Begriffe inzwischen synonym bzw. gemeinsam verwendet. (Die \emph{Johann Peter Hebel-Ausgabe} von 2019 bei \emph{Wallstein} ist eine ‚Kommentierte Lese- und Studienausgabe‘). \\
            
        Im 19. Jahrhundert wurde die Leseausgabe als ‘Volksausgabe’ bezeichnet, ein Begriff, der gelegentlich für ideologisch motivierte Massenausgaben ohne verlässliche Textgrundlage verwendet wurde. Später haben Buchgemeinschaften preisgünstige Klassiker-Ausgaben als Volksausgaben vermarktet. Die meisten Leseausgaben entstehen als Sekundärverwertung einer kritischen bzw. historisch-kritischen Buchausgabe. Gegenwärtig findet man Leseausgaben auch als Auskoppelung aus \href{http://gams.uni-graz.at/o:konde.96}{Hybridausgaben}, die den Lesetext in Buchform und den editorischen \href{http://gams.uni-graz.at/o:konde.34}{Kommentar} und \href{http://gams.uni-graz.at/o:konde.32}{Apparat} in digitaler Form bieten. Es gibt aber auch Leseausgaben als eigenständig erarbeitete Editionen, die mit der Erschließung des Werks zugleich seine Vermittlung an ein breiteres Lesepublikum beabsichtigen. Gerade dieser Aspekt der Textvermittlung könnte auch im Rahmen von Digitalen Editionen weiterverfolgt werden, mit der Fragestellung, ob und wie Lesefassungen edierter Texte in eine Onlinepräsenz eingebaut werden könnten.\\
            
        \subsection*{Literatur:}\begin{itemize}\item Im Dickicht der Texte. Editionswissenschaft als interdisziplinäre Grundlagenforschung. Hrsg. von Gesa Dane, Jörg Jungmayr und Marcus Schotte: 2013.\item Nutt-Kofoth, Rüdiger: Editionswissenschaft. In: Methodengeschichte der Germanistik: 2009, S. 109–132.\item Plachta, Bodo: Editionswissenschaft. Eine Einführung in Methode und Praxis der Edition neuerer Texte. Stuttgart: 2013.\item Christine Lavant: Werke in vier Bänden. Hrsg. von Klaus Amann und Doris Moser. Göttingen: 2015.\end{itemize}\subsection*{Verweise:}\href{https://gams.uni-graz.at/o:konde.213}{Werkausgabe}, \href{https://gams.uni-graz.at/o:konde.91}{Gesamtausgabe}, \href{https://gams.uni-graz.at/o:konde.34}{Kommentar}, \href{https://gams.uni-graz.at/o:konde.96}{Hybridedition}, \href{https://gams.uni-graz.at/o:konde.93}{Historisch-kritische Edition}, \href{https://gams.uni-graz.at/o:konde.140}{Nachlassedition}\subsection*{Themen:}Einführung, Digitale Editionswissenschaft\subsection*{Lexika}\begin{itemize}\item \href{https://edlex.de/index.php?title=Leseausgabe}{Edlex: Editionslexikon}\end{itemize}\subsection*{Zitiervorschlag:}Moser, Doris. 2021. Leseausgabe. In: KONDE Weißbuch. Hrsg. v. Helmut W. Klug unter Mitarbeit von Selina Galka und Elisabeth Steiner im HRSM Projekt "Kompetenznetzwerk Digitale Edition". URL: https://gams.uni-graz.at/o:konde.116\newpage\section*{Linked Open Data  (LOD)} \emph{Steiner, Christian; christian.steiner@uni-graz.at}\\
        
    \emph{Linked (Open) Data} (LD oder LOD) sind Daten, die sowohl miteinander verknüpft als auch online öffentlich mit möglichst niedrigen rechtlichen Schranken (\href{http://gams.uni-graz.at/o:konde.7}{FAIR-Prinzipien}) zugänglich sind. Mit Hilfe von LOD ist es möglich, Verbindungen zwischen Ressourcen zu finden, die vorher nicht bekannt waren. Das Community-Projekt \emph{Linking Open Data} spielt eine große Rolle beim Aufbau dieses Netzwerks. Dazu werden Datensätze, die aus vielen verschiedenen Projekten stammen, miteinander verbunden. Die LOD-Cloud wächst weiterhin exponentiell.\\
            
        Der Begriff \emph{Linked Data} bezeichnet eine Reihe von bewährten Verfahren für die Veröffentlichung und Verknüpfung von \href{http://gams.uni-graz.at/o:konde.131}{RDF}-Daten im Web. Tim Berners-Lee (2006) definiert vier Regeln, welche die \emph{Linked-Data}-Prinzipien bilden: \\
            
        \begin{itemize}\item {URIs als Namen für Dinge können alles sein, nicht nur Dokumente. Das Konzept ‘Ich’ ist nicht dasselbe wie das Konzept für meine Website. Beides, also Informationsressourcen und Nicht-Informationsressourcen, können hinter URIs platziert werden. }\item {HTTP-URIs haben den großen Vorteil, dass sie global eindeutige Namen sind und verteilte Eigentumsverhältnisse haben. Vor allem aber erlauben sie es, nach diesen Namen zu suchen und die URIs zu dereferenzieren. }\item {Nützliche Informationen sollten als \href{http://gams.uni-graz.at/o:konde.131}{RDF} bereitgestellt werden, wenn jemand eine URL eingibt. Die Verwendung von Standards erhöht die Interoperabilität verschiedener Datenquellen erheblich. }\item {Um zugehörige Informationen zu finden, sollten RDF-Links zu anderen URIs immer zur Verfügung gestellt werden. Dieses Prinzip wird manchmal auch als \emph{follow-your-nose}-Ansatz bezeichnet. Denn einerseits ermöglicht es den Benutzerinnen und Benutzern, zusätzliche Informationen zu finden, indem sie einfach den Links folgen. Auf der anderen Seite hilft die Verwendung von manschinenlesenbaren Links anstelle von Standard-Weblinks (@href) Maschinen dabei, die Links automatisch zu verarbeiten. }\end{itemize}\subsection*{Literatur:}\begin{itemize}\item Linked Data - Design Issues. URL: \url{http://www.w3.org/DesignIssues/LinkedData.html}\item Berners-Lee, Tim; Bizer, Christian; Heath, Tom: Linked data-the story so far. In: International Journal on Semantic Web and Information Systems 5: 2009, S. 1–22.\item Halb, Wolfgang: Creating, Interrelating and Consuming Linked Data on the Web. Graz: 2012.\item Hogan, Aidan: Linked Data & the Semantic Web Standards. In: Linked Data Management: Principles and Techniques in the Series on Emerging Directions in Database Systems and Applications, CRC Press, Boca Raton, FL: 2013.\end{itemize}\subsection*{Software:}\href{https://jena.apache.org/}{Apache Jena}, \href{https://www.blazegraph.com/}{BlazeGraph}, \href{https://enrich.acdh.oeaw.ac.at}{enrich/stanbol (ACDH-OeAW)}, \href{geonames.org}{Geonames}, \href{geonames.org}{Geonames}, \href{https://www.w3.org/RDF/}{RDF}, \href{https://www.wikidata.org/wiki/Wikidata:Main_Page}{Wikidata}, \href{https://www.ontotext.com/graphdb‎}{GraphDB}\subsection*{Verweise:}\href{https://gams.uni-graz.at/o:konde.168}{Semantic Web}, \href{https://gams.uni-graz.at/o:konde.131}{RDF}\subsection*{Projekte:}\href{https://medea.hypotheses.org}{MEDEA. Modelling semantically Enriched Digital Edition of Accounts}, \href{https://correspsearch.net}{correspSearch}, \href{https://gams.uni-graz.at/corema}{CoReMA - Cooking Recipes of the Middle Ages}\subsection*{Themen:}Einführung, Archivierung, Digitale Editionswissenschaft\subsection*{Lexika}\begin{itemize}\item \href{https://edlex.de/index.php?title=Linked_Open_Data_(LOD)}{Edlex: Editionslexikon}\end{itemize}\subsection*{Zitiervorschlag:}Steiner, Christian. 2021. Linked Open Data  (LOD). In: KONDE Weißbuch. Hrsg. v. Helmut W. Klug unter Mitarbeit von Selina Galka und Elisabeth Steiner im HRSM Projekt "Kompetenznetzwerk Digitale Edition". URL: https://gams.uni-graz.at/o:konde.8\newpage\section*{Liste der Hybrid-Editionen} \emph{Klug, Helmut W.; helmut.klug@uni-graz.at / Galka, Selina; selina.galka@uni-graz.at }\\
        
    Ein Ergebnis der KONDE-Arbeitsgruppe ‘Hybridedition’ unter der Leitung von Christiane Fritze war eine Sammlung der bis dato bekannten Hybrideditionen. Die Idee war es, eine Ressource anzubieten, mithilfe derer Ideen für die Planung einer eigenen \href{http://gams.uni-graz.at/o:konde.96}{Hybridedition} generiert werden können. Aus dieser Sammlung ist die nachstehende Liste der aktuell im Netz und im Buchhandel verfügbaren Hybrideditionen entstanden.\\
            
        \begin{itemize}\item {ATÖdön von Horváth Historisch-kritische Ausgabe - Digitale Edition1901-1938Ödön von Horváth: Wiener Ausgabe sämtlicher Werke. Historisch-kritische EditionKlausKastbergerDe Gruyter2009ff}\item {ATHugo von Montfort - Das poetische Werk - Augenfassung1357-1423Hugo von Montfort: Das poetische Werk [Texte, Melodien, Einführung]3-11-017604-1109;13888442 KBVWernfriedHofmeisterBerlin [u.a.]Walter de Gruyter2005De-Gruyter-Texte}\item {ATGründungsurkunde und Insignien der Karl-Franzens-Universität Graz1.1.1585Gründungsurkunde und Insignien der Karl-Franzens-Universität Graz978-3-7011-0295-22039103-EWalterHöflechnerAloisKernbauerGernotKrapingerAloisKernbauerGrazLeykam20142020-12-09T19:21:07Zhttp://d-nb.info/1051042011/04}\item {ATAufbruch in die Moderne - Gebrauchsgrafik19.-20.Jhdt.Plakate: Aufbruch in die Moderne, am Beispiel der UNESCO City of Design GrazPlakate978-3-7011-0326-32047088-C, 2047087-CEvaKleinGrazLeykam2014}\item {ATHermann Bahr: Arthur Schnitzler - Briefwechsel, Aufzeichnungen, Dokumente 1891-19311891-1931Briefwechsel, Aufzeichnungen, Dokumente 1891-1931 - Hermann Bahr, Arthur Schnitzler | Wallstein Verlag2020-12-09T19:26:20Zhttps://www.wallstein-verlag.de/9783835332287-hermann-bahr-arthur-schnitzler-briefwechsel-aufzeichnungen-dokumente-1891-1931.html}\item {ATDigitale Werkausgabe: Wenzel Raimund Johann Birck1718-1763Wenzel Birck (Pürk) 1718 - 1763: Leben und Werk eines Wiener Hofmusikers an der Wende vom Barock zur KlassikWenzel Birck (Pürk) 1718 - 17631193888-C, D-23505, E-445MichaelStephanides1982}\item {ATJ.J. Fux Online1660-1741WerkeJohann JosephFuxcontributorGernotGrubercontributorHerbertSeifertcontributorInstitut für kunst- und musikhistorische ForschungenWienHollitzer Verlag-2016}\item {ATkommentierte Werkausgabe Werner Kofler - Hybridedition1947-2011Kommentierte WerkausgabeWernerKoflercontributorClaudiaDürrcontributorJohannSonnleitnercontributorWolfgangStraubSonderzahl, 20182018}\item {ATAndreas Okopenko: Tagebücher aus dem Nachlass (Hybridedition)1949-1954Tagebücher aus dem Nachlass: (1945 bis 1955)Tagebücher aus dem Nachlass978-3-903110-64-92171742-B, 2171141-B, L XX Oko. T/1, I-1637390, A-363938AndreasOkopenkocontributorRolandInnerhofercontributorBernhardFetzcontributorChristianZollercontributorLauraTezarekcontributorArnoHerberthcontributorDesireeHebenstreitcontributorHolgerEnglerthWienKlever2020KLEVER - Essay}\item {ATMusil Online 1880-1942GesamtausgabeRobertMusilcontributorWalterFantaWien2016}\item {ATDie Edition der Pez-Korrespondenz1683-1762Die gelehrte Korrespondenz der Brüder Pez Text, Regesten, Kommentare.ThomasWallnigThomasStockingerWien1Böhlau2010https://e-book.fwf.ac.at/view/o:370Quelleneditionen des Instituts für Österreichische Geschichtsforschung2/1}\item {ATBriefwechsel Sauer-Seuffert1880-1926Der Briefwechsel zwischen August Sauer und Bernhard Seuffert 1880 bis 1926978-3-205-23279-72156736-C, 2156734-CAugustSauerBernhardSeuffertcontributorBernhardFetzcontributorHans-HaraldMüllercontributorMirkoNottscheidcontributorMarcelIlletschkocontributorDesireeHebenstreitWien Köln WeimarBöhlau Verlag20202020-12-17T15:38:55Zhttps://d-nb.info/1179950941/04}\item {ATKritische Gesamtausgabe des Nachlasses von Ferdinand Ebner1882-1931Gesammelte WerkeIA, P78456, BF 5600FerdinandEbnercontributorMichaelPflieglercontributorLudwigHänselThomas-Morus-Presse im Verlag Herder-19581958}\item {ATArthur Schnitzler - Tagebuch1879 – 1931arthur\_schnitzler\_tagebuch2020-12-18T07:50:00Zhttps://www.austriaca.at/arthur\_schnitzler\_tagebuch}\item {ATArthur Schnitzler - Literary Correspondences1889–1931}\item {ATDie Ministerratsprotokolle 1848–1918 online1848–1918Die Ministerratsprotokolle Österreichs und der Österreichisch-Ungarischen Monarchie 1848 - 19187170, B26474MinisterratÖsterreichMinisterrat für Gemeinsame Angelegenheiten der Österreichisch-Ungarischen MonarchieÖsterreich-UngarnÖsterrBundesverl, Anfangs, teils, teils, Wien2021-02-15T12:09:18Zhttps://mrp.oeaw.ac.at/pages/index.html}\end{itemize}\\
            
        \subsection*{Verweise:}\href{https://gams.uni-graz.at/o:konde.138}{Motivation zur Hybridedition}, \href{https://gams.uni-graz.at/o:konde.22}{Lesetext (Hybridedition)}, \href{https://gams.uni-graz.at/o:konde.96}{Hybridedition}, \href{https://gams.uni-graz.at/o:konde.208}{Verlage, die Hybrideditionen unterstützen}\subsection*{Themen:}Einführung, Interfaces, Digitale Editionswissenschaft\subsection*{Zitiervorschlag:}Klug, Helmut W.; Galka, Selina. 2021. Liste der Hybrid-Editionen. In: KONDE Weißbuch. Hrsg. v. Helmut W. Klug unter Mitarbeit von Selina Galka und Elisabeth Steiner im HRSM Projekt "Kompetenznetzwerk Digitale Edition". URL: https://gams.uni-graz.at/o:konde.117\newpage\section*{Lizenzierung} \emph{Klug, Helmut W.; helmut.klug@uni-graz.at }\\
        
    Mit ‘Lizenzierung’ wird der Vorgang bezeichnet, die Verwendung bestimmter Rechtssubjekte (z. B. urheberrechtlich geschützte Werke) durch Dritte zu definieren. Für urheberrechtlich geschützte Werke bestimmt das Urheberrechtsgesetz Nutzungsrechte, die Urhebern vorbehalten sind. Mithilfe sogenannter Lizenzen kann ein Urheber bestimmen, in welchem Umfang und unter welchen Bedingungen bestimmte Nutzungsrechte (Verbreitung, Bearbeitung, kommerzielle Nutzung) durch Dritte erlaubt sind. Dies kann mithilfe einer pauschalen Lizenz (z. B. \href{http://gams.uni-graz.at/o:konde.45}{Creative Commons}-Lizenzen) erfolgen, aber auch durch Lizenzverträge mit Verlagen oder Einzelpersonen.\\
            
        \subsection*{Literatur:}\begin{itemize}\item DFG-Praxisregeln "Digitalisierung", Deutsche Forschungsgemeinschaft: 2016. URL: \url{https://www.dfg.de/formulare/12_151/}.\end{itemize}\subsection*{Software:}\href{Vectr}{CC Lizenzgenerator}\subsection*{Verweise:}\href{https://gams.uni-graz.at/o:konde.152}{Open Access}, \href{https://gams.uni-graz.at/o:konde.9}{Lizenzmodelle}, \href{https://gams.uni-graz.at/o:konde.44}{Urheberrecht}, \href{https://gams.uni-graz.at/o:konde.223}{Digitalisierung: Rechtliches}, \href{https://gams.uni-graz.at/o:konde.222}{Freie Werknutzungen}\subsection*{Themen:}Rechtliche Aspekte\subsection*{Zitiervorschlag:}Klug, Helmut W. 2021. Lizenzierung. In: KONDE Weißbuch. Hrsg. v. Helmut W. Klug unter Mitarbeit von Selina Galka und Elisabeth Steiner im HRSM Projekt "Kompetenznetzwerk Digitale Edition". URL: https://gams.uni-graz.at/o:konde.119\newpage\section*{Lizenzmodelle} \emph{Scholger, Walter; walter.scholger@uni-graz.at }\\
        
    Das \href{http://gams.uni-graz.at/o:konde.44}{Urheberrecht} sieht im Rahmen
                  der im Urheberrechtsgesetz definierten \href{http://gams.uni-graz.at/o:konde.222}{Freien Werknutzungen}
                  eine Reihe von Nachnutzungsszenarien, insbesondere für Forschung und
                  Bildung, vor. Die Kenntnis dieser Ausnahmen darf jedoch nicht vorausgesetzt
                  werden, zumal Urheberrecht in den meisten akademischen Studien und Lehrgängen
                  nicht Bestandteil der Curricula ist. \\
            
        Um dafür Rechtssicherheit zu schaffen und eine rechtmäßige Nachnutzung der eigenen
                  Werke zu gewährleisten, kann die Urheberin bzw. der Urheber
                  Werknutzungsbewilligungen an Dritte vergeben. Der einfachste Weg, eine solche
                  Bewilligung einzuräumen, ist das Erteilen einer Lizenz. \\
            
        Es muss zunächst festgehalten werden, dass \\
            
        \begin{itemize}\item {nur die Rechteinhaberin bzw. der Rechteinhaber eine Lizenz vergeben darf. }\item {im Falle mehrerer Miturheberinnen und Miturheber diese Entscheidung von
                     allen Beteiligten gemeinsam getroffen werden muss. }\item {nur Werke im Sinne des Urheberrechts lizenziert werden können. Wird für eine
                     Schöpfung, die gemäß Urheberrecht nicht als Werk gilt, dennoch eine Lizenz
                     vergeben, ist diese rechtlich ungültig. }\item {gemeinfreie Werke, also jene, deren urheberrechtliche Schutzfrist abgelaufen
                     ist, sowie Rohdaten, die von vornherein mangels hinreichender Eigentümlichkeit
                     nicht als Werk gelten und daher auch keinen Urheberrechtsanspruch begründen
                     können, nicht lizenziert werden können. Eine dennoch vorgenommene Lizenzierung
                     ist ungültig und kann lediglich als ‘Willensbekundung’ verstanden werden.
                  }\end{itemize}Gerade im digitalen Raum, der nationale Grenzen überschreitet, ergibt sich – auch
                  aufgrund der territorialen Beschränkung des Urheberrechts – die Notwendigkeit
                  rechtlicher Regelungen, die einerseits das Urheberrecht der Forscherinnen und
                  Forscher schützen, andererseits jedoch auch die Wiederverwendbarkeit ihrer
                  Arbeiten sicherstellen sollen.\\
            
        Offene Lizenzierungsmodelle bieten hier Lösungen, die im Einklang mit den \href{http://gams.uni-graz.at/o:konde.152}{Open Access}-Bestimmungen nationaler
                  und internationaler Förderungsorganisationen eine Nachnutzung wissenschaftlicher
                  Werke zu den durch die Urheberin bzw. den Urheber definierten Bedingungen
                  ermöglichen. Das mittlerweile gebräuchlichste und gleichsam zum Standard erhobene
                  offene Lizenzierungsmodell sind die \href{http://gams.uni-graz.at/o:konde.45}{Creative Commons}-Lizenzen. \\
            
        Die Lizenzierung wissenschaftlicher Publikationen mittels offener Lizenzmodelle
                  wie Creative Commons ermöglicht der Urheberin bzw. dem Urheber die selbstbestimmte
                  Kommunikation mit der Wissensgesellschaft ohne die Vermittlerrolle eines Verlags,
                  einer Verwertungsgesellschaft oder juristischer Unterstützung.\\
            
        Für den Bereich der Digital Humanities bietet der \emph{LINDAT Public
                     License Selector} ein für Laien geeignetes Werkzeug, das mittels der
                  Beantwortung einer Reihe von Fragen eine Auswahl der passenden Lizenzmodelle
                  anzeigt und sowohl für Publikationen (‘Werke’) als auch für Software verwendbar
                  ist.\\
            
        \subsection*{Literatur:}\begin{itemize}\item Zimmermann, Claudia: Leitfaden für die Erstellung von Open Educational
                              Resources: Leitfaden für die Erstellung von Open Educational
                              Informationen und praktische Übungen für Hochschullehrende, 2.
                              überarb. Aufl: 2018. URL: \url{https://openeducation.at/fileadmin/user_upload/p_oea/OEA-Leitfaden_online_Aufl2.pdf}.\item Amini, Seyavash; Blechl, Guido; Hamdi, Djawaneh; Losehand, Joachim: FAQs zu Creative-Commons-Lizenzen unter besonderer
                              Berücksichtigung der Wissenschaft: 2015. URL: \url{https://phaidra.univie.ac.at/view/o:459183}.\item Scholger, Walter: Urheberrecht und offene Lizenzen im wissenschaftlichen
                              Publikationsprozess. In: Publikationsberatung an Universitäten. Ein
                              Praxisleitfaden zum Aufbau publikationsunterstützender
                              Services. Bielefeld: 2020, S. 123–147.\end{itemize}\subsection*{Software:}\href{https://ufal.github.io/public-license-selector/}{LINDAT Public
                           License Selector}\subsection*{Verweise:}\href{https://gams.uni-graz.at/o:konde.44}{Urheberrecht}, \href{https://gams.uni-graz.at/o:konde.45}{Creative Commons}, \href{https://gams.uni-graz.at/o:konde.119}{Lizenzierung}, \href{https://gams.uni-graz.at/o:konde.222}{Freie Werknutzungen}\subsection*{Themen:}Rechtliche Aspekte\subsection*{Zitiervorschlag:}Scholger, Walter. 2021. Lizenzmodelle. In: KONDE Weißbuch. Hrsg. v. Helmut W. Klug unter Mitarbeit von Selina Galka und Elisabeth Steiner im HRSM Projekt "Kompetenznetzwerk Digitale Edition". URL: https://gams.uni-graz.at/o:konde.9\newpage\section*{Makrogenese (Fokus: Literaturwissenschaft – Bsp. Musil)} \emph{Fanta, Walter; walter.fanta@aau.at / Boelderl, Artur R.;
                  artur.boelderl@aau.at}\\
        
    Die \href{http://gams.uni-graz.at/o:konde.17}{Annotation} der Makrogenese
                  (vgl. \href{http://gams.uni-graz.at/o:konde.28}{Textgenese}) setzt einen
                  Überblick über das Ensemble der werkbezogenen Dokumente in ihrem
                  Entstehungszusammenhang voraus. Im Falle Musils erfordert die Gesamtanlage des
                  Nachlasses eine Übersetzung zwischen der Anordnungslogik der Nachlassdokumente
                  einerseits und der des Werks andererseits, wobei die Datei \emph{tgd.xml} (i. e. textgenetisches Dossier) als Master-Dokument fungiert, in
                  dem alle textgenetischen Relationen zusammenlaufen. Im
                     <teiHeader> der \href{http://gams.uni-graz.at/o:konde.197}{Transkriptionen} ist unter <msPart> jede
                  Manuskriptseite einem Werk oder Werkprojekt (zu Lebzeiten des Autors
                  unveröffentlicht) zugeordnet. Ihre zeitliche Verortung erlaubt die Unterscheidung
                  zwischen meso- (innerhalb ein- und desselben Datierungsabschnitts, s. \href{http://gams.uni-graz.at/o:konde.24}{Annotation: Mesogenese}) und
                  makrogenetischer Ebene (Umschreiben der Entwürfe, Generierung neuer Fassungen). \\
            
        Besagte Annotation der werkbezogenen Hierarchien geschieht im
                     <teiHeader> (Bsp. 1). Die Trennlinie zwischen der meso- und
                  der makrogenetischen Ebene wird durch die zeitliche Verortung gezogen. Schreibakte
                  innerhalb ein- und desselben Datierungsabschnitts gehören zur ersteren, letztere
                  besteht im Umschreiben der Entwürfe, der Generierung neuer Fassungen, sie spiegelt
                  sich in Metamorphosen des Texts über große Zeiträume hinweg. Der \href{http://gams.uni-graz.at/o:konde.178}{TEI}-konformen Annotation des
                  chronologischen Befunds kommt eine Schlüsselrolle zu. Die Verzeichnung der
                  Datierungsabschnitte und der im Manuskript angegebenen bzw. erschlossenen
                  Datumsangaben erfolgt im <teiHeader> <msPart> und in den
                  Tabellen von \emph{tgd.xml} (Bsp. 2).\\
            
        Bsp. 1:\\
            
        \begin{verbatim}<msItem>
    <title type="work">Der Mann ohne Eigenschaften</title>
    <titlePart type="work-part">Zweites Buch. Fortsetzung aus dem 
    Nachlass (1937–1942)</titlePart>
    <titlePart type="chapter-group">Dritte Genfer Ersetzungsreihe
    </titlePart>
    <title type="chapter-project" n="48">Liebe deinen Nächsten wie
    dich selbst</title>
</msItem>\end{verbatim}Bsp. 2:\\
            
        \begin{verbatim}<origDate datingMethod="#dp" notBefore="1940-03-01"
notAfter="1941-04-01">9-3: März 1940 - April 1941</origDate>        
<date datingMethod="#dp" n="9-3" notBefore-iso="1940-03"
notAfter-iso="1941-04">März 1940 - April 1941</date>
<origDate when-iso="1929-10-30"/>\end{verbatim}Dieser Beitrag wurden im Kontext des FWF-Projekts "MUSIL ONLINE – interdiskursiver Kommentar" 
                  (P 30028-G24) verfasst.\subsection*{Software:}\href{http://oxygenxml.com/}{Oxygen}\subsection*{Verweise:}\href{https://gams.uni-graz.at/o:konde.17}{Annotation (Literaturwissenschaft:
                           grundsätzlich)}, \href{https://gams.uni-graz.at/o:konde.19}{Interdiskursivität (Fokus:
                           Literaturwissenschaft - Bsp. Musil)}, \href{https://gams.uni-graz.at/o:konde.20}{Intertextualität (Fokus:
                           Literaturwissenschaft)}, \href{https://gams.uni-graz.at/o:konde.21}{Intratextualität (Fokus:
                           Literaturwissenschaft - Bsp. Musil)}, \href{https://gams.uni-graz.at/o:konde.28}{Textgenese}, \href{https://gams.uni-graz.at/o:konde.24}{Mesogenese (Fokus:
                           Literaturwissenschaft - Bsp. Musil)}, \href{https://gams.uni-graz.at/o:konde.26}{Mikrogenese (Fokus:
                           Literaturwissenschaft - Bsp. Musil)}, \href{https://gams.uni-graz.at/o:konde.96}{Hybridedition}\subsection*{Projekte:}\href{http://musilonline.at}{Musil Online}\subsection*{Themen:}Annotation und Modellierung, Digitale Editionswissenschaft\subsection*{Lexika}\begin{itemize}\item \href{https://lexiconse.uantwerpen.be/index.php/lexicon/macrogenesis/}{Lexicon of Scholarly Editing}\end{itemize}\subsection*{Zitiervorschlag:}Fanta, Walter; Boelderl, Artur R. 2021. Makrogenese (Fokus: Literaturwissenschaft – Bsp. Musil). In: KONDE Weißbuch. Hrsg. v. Helmut W. Klug unter Mitarbeit von Selina Galka und Elisabeth Steiner im HRSM Projekt "Kompetenznetzwerk Digitale Edition". URL: https://gams.uni-graz.at/o:konde.23\newpage\section*{Markup} \emph{Galka, Selina; selina.galka@uni-graz.at }\\
        
    Ein Text wird vom Computer als eine lange Kette von gleichwertigen Zeichen
                  wahrgenommen, die er nicht differenzieren kann. (Jannidis 2017, S.
                     99) Ohne Hilfe kann er keine Strukturen ausmachen; nur wenn explizit der
                  Hinweis eingefügt wird, dass es sich dabei um Strukturmerkmale, Wissen oder
                  Erkenntnisse handelt, können diese Teile des Textes weiterverarbeitet werden.
                  Diese Hinweise können mittels Markup eingefügt werden; es handelt sich dabei um
                  Markierungen in einem Text, die eine semantische Bedeutung (\href{http://gams.uni-graz.at/o:konde.17}{Annotation}) haben und Start- und Endposition des
                  jeweiligen Vorkommens markieren. (Jannidis 2017, S. 99) Es kann
                  zwischen deskriptivem und prozeduralem Markup unterschieden werden. \\
            
        Hier ein Beispiel für deskriptives Markup, bei dem die Auszeichnungen direkt im
                  Text vorgenommen werden:\\
            
        \begin{verbatim}<date>20. November 2019</date>
Lieber <name>Stefan</name>,ich freue mich auf 
unser Treffen in <place>Wien</place>.
Liebe Grüße,
<name>Sebastian</name>\end{verbatim}Die Markierungen bzw. das Markup wurde hier mit spitzen Klammern eingefügt,
                  implizite Strukturen werden also explizit gemacht. Der undifferenzierte Strom an
                  Zeichen wird mittels der eingetragenen Informationen strukturiert und kann mit dem
                  Computer gezielt weiterverarbeitet werden. Bei deskriptivem Markup werden
                  Textstrukturen auf formalisierte Weise benannt, charakterisiert und annotiert und
                  können so nachgenutzt werden – die Wiedergabe und Weiterverarbeitung findet erst
                  in einem nächsten Schritt statt.\\
            
        Die Informationen können sowohl manuell als auch automatisch eingetragen werden,
                  immer aber muss vorher überlegt werden, welche Strukturen man festhalten will
                     (\href{http://gams.uni-graz.at/o:konde.137}{Modellierung}).
                     (Jannidis 2017, S. 99f.) Alternativ ist es möglich, die
                  Annotationen separat abzuspeichern, mit sogenanntem 
                  \href{http://gams.uni-graz.at/o:konde.171}{Stand-off-Markup}.\\
            
        Um Daten austauschbar zu machen und zusammenführen zu können, wird in der Regel
                  kein beliebiges \emph{Markup} verwendet, sondern so genannte \emph{Markup Languages}, wie z. B. \href{http://gams.uni-graz.at/o:konde.215}{XML}, die gewissen Richtlinien unterliegen können,
                  wie z. B. \href{http://gams.uni-graz.at/o:konde.178}{TEI}. \\
            
        \subsection*{Literatur:}\begin{itemize}\item Jannidis, Fotis: Grundlagen der Datenmodellierung. In: Digital Humanities. Eine Einführung. Stuttgart: 2017, S. 99–108.\item Sahle, Patrick: Digitale Editionsformen. Zum Umgang mit der
                              Überlieferung unter den Bedingungen des Medienwandels. Teil 3:
                              Textbegriffe und Recodierung. Norderstedt: 2013.\item Schmidt, Desmond: The role of markup in the digital humanities. In: Historical Social Research 37: 2012, S. 125–146.\end{itemize}\subsection*{Software:}\href{http://oxygenxml.com/}{Oxygen}\subsection*{Verweise:}\href{https://gams.uni-graz.at/o:konde.137}{Modellierung}, \href{https://gams.uni-graz.at/o:konde.195}{Textmodellierung}, \href{https://gams.uni-graz.at/o:konde.29}{Annotationsstandards}, \href{https://gams.uni-graz.at/o:konde.215}{XML}, \href{https://gams.uni-graz.at/o:konde.171}{Stand-Off-Markup}, \href{https://gams.uni-graz.at/o:konde.15}{Alternativen zur Kodierung mit
                           TEI}\subsection*{Themen:}Einführung, Annotation und Modellierung\subsection*{Lexika}\begin{itemize}\item \href{https://lexiconse.uantwerpen.be/index.php/lexicon/markup/}{Lexicon of Scholarly Editing}\end{itemize}\subsection*{Zitiervorschlag:}Galka, Selina. 2021. Markup. In: KONDE Weißbuch. Hrsg. v. Helmut W. Klug unter Mitarbeit von Selina Galka und Elisabeth Steiner im HRSM Projekt "Kompetenznetzwerk Digitale Edition". URL: https://gams.uni-graz.at/o:konde.126\newpage\section*{Materialität} \emph{Bosse, Anke; anke.bosse@aau.at }\\
        
    Ohne Material keine Edition – das gilt seit ihren Anfängen. An den jeweils
                  überlieferten Textzeugen findet jede Edition allererst ihre Berechtigung – aber
                  auch ihre Widerstände, vor allem bei unvollständiger Überlieferung.\\
            
        Doch ein Bewusstsein für diese unhintergehbare Materialität ließ auf sich warten.
                  Die vor allem noch in den einflussreichen Schriften Hegels ausgeprägte Perspektive
                  auf eine akzidentielle und daher nicht-bedeutungstragende Materialität veränderte
                  sich erst in jüngerer Zeit und führte zu editorischen Konsequenzen. Die
                  jahrhundertelange Tradition der \href{http://gams.uni-graz.at/o:konde.192}{Textkritik} – die das Ziel hat, aus dem (fragmentarisch) Überlieferten
                  editorisch den verlorenen Urtext oder einen ‚idealen‘ Text herzustellen –
                  kombinierte sich unter diesem Einfluss bis ins 20. Jahrhundert hinein mit der
                  Textwissenschaft und dem Strukturalismus dergestalt, dass der Text an sich im
                  Fokus stand, unabhängig von seiner materialen und medialen Erscheinungsform. Dies
                  gilt und galt für die seit dem 18. Jahrhundert entstandenen Texte noch deutlich
                  länger als für mittelalterliche und (früh-)neuzeitliche, denn bei diesen setzte
                  das Bewusstsein für die spezifische Materialität der Überlieferungsträger früher
                  ein, bis schließlich ab Ende des 19. Jahrhunderts die verstärkte Arbeit mit
                  Archivmaterialien im Rahmen \href{http://gams.uni-graz.at/o:konde.93}{historisch-kritischer} und \href{http://gams.uni-graz.at/o:konde.90}{textgenetischer Ausgaben}, dann vor allem in \href{http://gams.uni-graz.at/o:konde.83}{Faksimile-Ausgaben} dieses Bewusstsein auch für
                  ‚moderne‘ Texte schuf – verstärkt durch die \emph{\href{http://gams.uni-graz.at/o:konde.46}{critique génétique}}, die Schreibprozessforschung und den \emph{material turn}
                  seit den 1960er- und 1970er-Jahren. (Benne 2015, Gumbrecht/Pfeiffer 1988,
                     Grésillon 1999, Hay 2008, Heibach/Rohde 2015)\\
            
        Zur Beschreibung der konkreten materialen Eigenschaften von Textzeugen sind
                  verschiedene, einander ergänzende Modelle entwickelt worden, die aber je nach \emph{dossier génétique} oder Edition auszudifferenzieren sind.
                     (Henzel 2015, Lukas/Nutt-Kofoth/Podewski 2014, Röcken 2008, Schubert
                     2010) Diese Beschreibung dient nicht nur der eindeutigen Identifizierung
                  von Textzeugen, sondern auch ihrer möglichen Datierung, ihrer Vergleichbarkeit,
                  ihrer Funktion im Schreibprozess und/oder in der Textgenese bis hin zur Erfassung
                  autorspezifischer Arbeitsweisen. Materialität entfaltet so eine (praxeologische)
                  Semantik, die das Sinnangebot des Textes erweitert und diesen mehrfach neu
                  perspektiviert. \\
            
        Faksimile-Ausgaben versuchen, das Original zu simulieren, um ein größeres Publikum
                  teilhaben zu lassen, das keinen Zugang zu den Originalen im Archiv hat [vgl. z. B.
                  Büchner, Kafka]. Spätestens hier musste ein Bewusstsein für die Materialität nicht
                  nur des Überlieferungsträgers, sondern auch der Edition einsetzen: Was ist im
                  Medium Buch darstellbar und vermittelbar, was nicht (mehr)? Müssen einzelne
                  Manuskriptblätter nicht auch als solche in der Edition reproduziert werden? Muss
                  angesichts immer besserer Reproduktionstechniken das Faksimile als solches
                  erkennbar bleiben – und wie? Ist Edition nicht immer auch zwingend mit einem
                  Medienwechsel verbunden? Materialität und Medialität überschneiden sich hier.
                     (Bohnenkamp 2013, Schubert 2010)\\
            
        Weder diese grundlegenden Fragen noch die möglichst funktionelle
                  Textzeugenbeschreibung noch das Bewusstsein für Materialität sind mit der
                  Digitalisierung seit den 1990er-Jahren ‚vom Tisch‘ – sie verschieben sich nur,
                  wenn auch bedeutend. Mit der Entwicklung von \href{http://gams.uni-graz.at/o:konde.59}{Digitalen Editionen} und von \href{http://gams.uni-graz.at/o:konde.96}{Hybrideditionen}, die Buch und digitale Komponente
                  verbinden, ist es möglich, einem weltweiten Publikum mittels \emph{\href{http://gams.uni-graz.at/o:konde.152}{Open Access}} die originalen Textzeugen in hochauflösenden \href{http://gams.uni-graz.at/o:konde.36}{Digitalisaten} als grafische Oberfläche und mit \href{http://gams.uni-graz.at/o:konde.25}{Metadaten} angereichert zugänglich zu
                  machen. Umso dringlicher stellt sich die Frage, wie der Medienwechsel im Digitalen
                  zu präsentieren und zu markieren ist. In Digitalen Editionen und Hybrideditionen
                  geht das, was einmal \href{http://gams.uni-graz.at/o:konde.92}{Textzeugenbeschreibung} war, ein in ein jederzeit aktualisierbares
                  Metadaten-Set und weltweit singuläre \emph{document identifier}.
                  Die Vergleichs-, Datierungs-, Durchsuchungs- und Verlinkungsmöglichkeiten
                  erweitern sich rasant – und mit ihnen die Forschungsperspektiven. Ein offener
                  Prozess – mit interessanten \emph{side effects}. Denn in Reaktion
                  auf die allgegenwärtige \href{http://gams.uni-graz.at/o:konde.60}{Digitalisierung} steigt auffallend das Interesse auch des großen
                  Publikums an dem einen Original, an dessen einzigartiger, materieller Aura.\\
            
        \subsection*{Literatur:}\begin{itemize}\item Benne, Christian: Die Erfindung des Manuskripts: zur Theorie und
                              Geschichte literarischer Gegenständlichkeit Die Erfindung des Manuskripts. Berlin: 2015.\item Bohnenkamp-Renken, Anne: Medienwandel / Medienwechsel in der
                              Editionswissenschaft. Berlin, Boston: 2012.\item Büchner, Georg: Woyzeck: Faksimileausgabe der Handschriften Woyzeck. Hrsg. von  und Gerhard Schmid. Leipzig: 1981.\item Grésillon, Almuth: Literarische Handschriften: Einführung in die "critique
                              génétique" Literarische Handschriften. Bern: 1999.\item Gumbrecht, Hans Ulrich; Elsner, Monika: Materialität der Kommunikation. Frankfurt am Main: 1988.\item Hay, Louis: Materialität und Immaterialität der Handschrift. In: editio 22: 2008, S. 1–21.\item Ästhetik der Materialität. Hrsg. von Christiane Heibach und Carsten Rohde. Paderborn: 2015.\item Henzel, Katrin: Zur Praxis der Handschriftenbeschreibung. Am Beispiel
                              des Modells der historisch-kritischen Edition von Goethes
                              Faust. In: Vom Nutzen der Editionen. Zur Bedeutung moderner
                              Editorik für die Erforschung von Literatur- und
                              Kulturgeschichte. Berlin, Boston: 2015, S. 75–95.\item Kafka, Franz: Historisch-kritische Ausgabe sämtlicher Handschriften,
                              Drucke und Typoskripte: eine Edition des Instituts für
                              Textkritik Historisch-kritische Ausgabe sämtlicher
                              Handschriften, Drucke und Typoskripte, RolandReuß PeterStaengle. Stroemfeld: 1995.\item Text - Material - Medium: zur Relevanz editorischer
                              Dokumentationen für die literaturwissenschaftliche
                              Interpretation Text - Material - Medium. Hrsg. von Wolfgang Lukas, Rüdiger Nutt-Kofoth und Madleen Podewski. Berlin, Boston: 2014.\item Röcken, Per: Was ist – aus editorischer Sicht – Materialität? Versuch
                              einer Explikation des Ausdrucks und einer sachlichen Klärung Was ist – aus editorischer Sicht –
                              Materialität?. In: editio 22: 2008, S. 22–46.\item Materialität in der Editionswissenschaft. Hrsg. von  und Martin J Schubert. Berlin, Boston: 2010.\end{itemize}\subsection*{Verweise:}\href{https://gams.uni-graz.at/o:konde.46}{critique génétique}, \href{https://gams.uni-graz.at/o:konde.59}{Digitale Edition}, \href{https://gams.uni-graz.at/o:konde.60}{Digitalisierung}, \href{https://gams.uni-graz.at/o:konde.83}{Faksimileausgabe/edition}, \href{https://gams.uni-graz.at/o:konde.93}{Historisch-kritische Edition /
                           Ausgabe}, \href{https://gams.uni-graz.at/o:konde.96}{Hybridedition}, \href{https://gams.uni-graz.at/o:konde.152}{Open Access}, \href{https://gams.uni-graz.at/o:konde.192}{Textkritik}, \href{https://gams.uni-graz.at/o:konde.34}{Kommentar}, \href{https://gams.uni-graz.at/o:konde.36}{Bereitstellung von Digitalisaten}, \href{https://gams.uni-graz.at/o:konde.25}{Metadaten}, \href{https://gams.uni-graz.at/o:konde.92}{Handschriftenbeschreibung}, \href{https://gams.uni-graz.at/o:konde.117}{Liste der Hybrideditionen}\subsection*{Themen:}Digitale Editionswissenschaft\subsection*{Lexika}\begin{itemize}\item \href{https://edlex.de/index.php?title=Materialit%C3%A4t}{Edlex: Editionslexikon}\end{itemize}\subsection*{Zitiervorschlag:}Bosse, Anke. 2021. Materialität. In: KONDE Weißbuch. Hrsg. v. Helmut W. Klug unter Mitarbeit von Selina Galka und Elisabeth Steiner im HRSM Projekt "Kompetenznetzwerk Digitale Edition". URL: https://gams.uni-graz.at/o:konde.127\newpage\section*{Mesogenese (Fokus: Literaturwissenschaft – Bsp. Musil)} \emph{Fanta, Walter; walter.fanta@aau.at / Boelderl, Artur R.;
                  artur.boelderl@aau.at}\\
        
    Auf der einzelnen Entwurfsseite manifestieren sich Spuren referenzierter
                  Manuskripte, die es zu identifizieren und als Referenznetz zu beschreiben gilt
                  (vgl. \href{http://gams.uni-graz.at/o:konde.28}{Textgenese}). Die
                  seitenbezogenen \href{http://gams.uni-graz.at/o:konde.25}{Metadaten} für das
                  Zusammenspiel mehrerer Manuskripte im Schreibprozess sind in drei miteinander
                  verknüpften Bereichen abgelegt: \\
            
        \begin{itemize}\item {im <teiHeader> unter <msPart>,}\item {im <body> in der Umgebung des Elements
                        <pb> sowie durch die Auszeichnung der
                     Verweissiglen mit dem Element <ref> und }\item {in einem zusätzlichen Dokument \emph{tgd.xml} (i. e.
                     textgenetisches Dossier) in einer Tabelle, welche die Stufen des
                     Entwurfsprozesses darstellt.}\end{itemize}Ad A: In <msPart> finden sich alle für die
                  Identifizierung der Seiten und des Manuskripttyps relevanten Informationen in \href{http://gams.uni-graz.at/o:konde.178}{TEI}-Kodierung. Am Beispiel des
                  Musilschen Nachlasses illustriert: Für die Seiten sind drei Nomenklaturen
                  vorhanden: die aktuelle ÖNB-Sigle, die herkömmliche sogenannte
                  Kaiser/Wilkins-Sigle, nach dem Philologen-Ehepaar benannt, das in den 1950er- und
                  1960er-Jahren für die Erschließung verantwortlich war, und die Seitensigle des
                  Autors (Bsp. 1: Mappe V/6, S. 14). Die Zuordnung der Seite zu einem Manuskripttyp
                  schließlich befindet sich gemeinsam mit der Angabe der Schreiberhand der
                  Grundschicht am Ende der Einträge unter <msPart> (Bsp. 2).\\
            
        Ad B: Im Textkörper der Transkription wird auf diese Metadaten bei der Annotation
                  des Anfangs der Entwurfsseite verwiesen (Bsp. 3). Im Element <pb> ist mit dem Attributwert von @corresp die ÖNB-Sigle und mit dem Attributwert von @facs der Identifikator der entsprechenden Bilddatei im Repositorium
                  der ÖNB angegeben. Das Element <fw> dient zur
                  Annotation der Seitensiglen, die typisierte Form als Attributwert, die
                  transkribierte Form als Textknoten. Wo immer im Korpus ein Verweis auf die Sigle
                  existiert, ist er mit dem Element <ref>
                  ausgezeichnet (Bsp. 4).\\
            
        Ad C: Das Dokument \emph{tgd.xml} ist dafür eingerichtet,
                  textgenetische Dossiers (TGD) zu den Schreibprojekten – im Falle Musils auf Werk-
                  und Kapitelprojekt-Ebene – durch Verzeichnisse in Tabellenform zu repräsentieren.
                  In den Tabellen laufen wie in einer Relaisstation sämtliche Verknüpfungslinien
                  zwischen den textgenetisch relevanten Informationen zusammen. Die mesogenetische
                  Ebene ist durch die einzelnen Tabellenzeilen vertreten, die Stufen innerhalb des
                  Gesamtprozesses (\href{http://gams.uni-graz.at/o:konde.23}{Makrogenese})
                  angeben. Die Zeilen/Stufen entsprechen der synchronen Ebene des Zusammenspiels von
                  Entwürfen, Studienblättern, Schmierblättern. Die Zellen enthalten je nach Spalte
                  Informationen und Verlinkungen zur Anzahl der Manuskriptseiten, zu Seitensiglen,
                  Manuskripttyp, Datierung, Pagina und Identifikatoren der Faksimiles. Als Konnektor
                  fungiert der die Stufe identifizierende Attributwert im <teiHeader> des Transkriptions-Dokuments (Bsp. 5: Kodierung der
                  entsprechenden Tabellenzeile, in der die ID vergeben wird).\\
            
        Bsp. 1:\\
            
        \begin{verbatim}<msPart xml:id="sn15093-05-06-014-40">
    <msIdentifier>
        <idno type="MO">sn15093-05-06-014-40</idno>
        <altIdentifier>
            <idno xml:id="k2_r48_6" type="page-sigla">(48. Nächsten)
            6</idno>
        </altIdentifier>
        <altIdentifier>
            <idno type="KWS">V/6/14</idno>
        </altIdentifier>
    </msIdentifier>
</msPart>\end{verbatim}Beachtung verdient Z. 5: Der Textknoten gibt die von Musil am Seitenanfang
                  verwendete Siglierung an, den in der Achtung-Attribut verzeichneten
                  Attributwert in typisierter Form. Aufgelöst bedeutet die Sigle: Reinschrift des
                  Kapitelentwurfs Nr. 48 in der Fortsetzung des Zweiten Buchs mit dem Titel \emph{Liebe deinen nächsten wie dich selbst}, Seite 6. Damit hängt
                  ein weiterer, im Element <msItem> gegebener
                  Identifikator zusammen: <filiation type="step"
                     corresp="#moe3_3ge_lie_3">Stufe 3</filiation> Diese Annotation
                  besagt, dass es sich um die dritte Stufe der Entwurfsarbeit Musils an diesem
                  Kapitel handelt. Mit dem Attribut @corresp wird auf die Tabelle im Dokument \emph{tgd.xml
                  } verwiesen, wo sich der Identifikator und alle nötigen Auflösungen und
                  Erläuterungen des Attributwerts befinden.\\
            
        Bsp. 2:\\
            
        \begin{verbatim}<physDesc>
    <objectDesc>
        <layoutDesc>
            <layout style="black_ink">
            <idno type="mst" n="draft_final" >
            Entwurfsreinschrift</idno> schwarze Tinte</layout>
        </layoutDesc>
    </objectDesc>
 </physDesc>\end{verbatim}Bsp. 3:\\
            
        \begin{verbatim}<pb corresp="#sn15093-05-06-014-40" facs="+Z151824105/00000017.jpg"/>
<fw type="ps" n="#k2_r48_6">(<hi rend="underline">48</hi>
. .. Nächsten ..) <seg rend="right">6.</seg></fw>\end{verbatim}Bsp. 4:\\
            
        \begin{verbatim}<ref target="#sn15093-05-06-017-40">
<idno type="ps" n="#k2_r48_9">48 9!</idno></ref>\end{verbatim}Bsp. 5:\\
            
        \begin{verbatim}<row role="data" xml:id="moe3_3ge_lie_3"> 
<cell role="data"><idno type="step" n="3">3</idno>
</cell>\end{verbatim}Dieser Beitrag wurden im Kontext des FWF-Projekts "MUSIL ONLINE – interdiskursiver Kommentar" 
                  (P 30028-G24) verfasst.\subsection*{Software:}\href{http://oxygenxml.com/}{Oxygen}\subsection*{Verweise:}\href{https://gams.uni-graz.at/o:konde.17}{Annotation (Literaturwissenschaft:
                           grundsätzlich)}, \href{https://gams.uni-graz.at/o:konde.19}{Interdiskursivität (Fokus:
                           Literaturwissenschaft - Bsp. Musil)}, \href{https://gams.uni-graz.at/o:konde.20}{Intertextualität (Fokus:
                           Literaturwissenschaft)}, \href{https://gams.uni-graz.at/o:konde.21}{Intratextualität (Fokus:
                           Literaturwissenschaft - Bsp. Musil)}, \href{https://gams.uni-graz.at/o:konde.28}{Textgenese}, \href{https://gams.uni-graz.at/o:konde.23}{Makrogenese (Fokus:
                           Literaturwissenschaft - Bsp. Musil)}, \href{https://gams.uni-graz.at/o:konde.26}{Mikrogenese (Fokus:
                           Literaturwissenschaft - Bsp. Musil)}, \href{https://gams.uni-graz.at/o:konde.96}{Hybridedition}\subsection*{Projekte:}\href{http://musilonline.at}{Musil Online}\subsection*{Themen:}Annotation und Modellierung, Digitale Editionswissenschaft\subsection*{Zitiervorschlag:}Fanta, Walter; Boelderl, Artur R. 2021. Mesogenese (Fokus: Literaturwissenschaft – Bsp. Musil). In: KONDE Weißbuch. Hrsg. v. Helmut W. Klug unter Mitarbeit von Selina Galka und Elisabeth Steiner im HRSM Projekt "Kompetenznetzwerk Digitale Edition". URL: https://gams.uni-graz.at/o:konde.24\newpage\section*{Metadata Harvesting} \emph{Stigler, Johannes; johannes.stigler@uni-graz.at }\\
        
    Seit Mitte der 90er-Jahre des letzten Jahrhunderts entstanden weltweit Archive mit unterschiedlichsten wissenschaftlichen digitalen Inhalten. Daraus entwickelte sich der Wunsch nach einer anbieterübergreifenden, aber dennoch gezielten Suchmöglichkeit. In der Bibliothekswelt galt dafür lange Zeit eine sogenannte Z39.50-Schnittstelle als \emph{State of the Art}. Bei diesem Verfahren handelte es sich um eine \emph{Cross Database Search}, bei der in jeder Suchanfrage einer Nutzerin/eines Nutzers, die Server der einzelnen Informationsanbieter – ob verfügbar oder nicht verfügbar – angefragt wurden. \\
            
        Demgegenüber hat sich in den letzten Jahre das sogenannte \emph{Metadata Harvesting} durchgesetzt. Sammlerdienste (\emph{Service Provider}) besuchen dabei in regelmäßigen Abständen Informationsanbieter (\emph{Data Provider}), die über eine standardisierte Schnittstelle \href{http://gams.uni-graz.at/o:konde.225}{Metadaten} (= kurze Pressemitteilungen) über die digitalen Inhalte (Publikationen, edierte Handschriften, erschlossene Artefakte u. v. m.) in ihren Forschungsdatenrepositorien bereitstellen. In diesen gesammelten Metadaten kann dann auf einem Portal des \emph{Service Providers} anbieterübergreifend gesucht werden. Die eigentlichen Inhalte verbleiben beim \emph{Data Provider}, der mittels eines \emph{\href{http://gams.uni-graz.at/o:konde.12}{persistenten Identifiers}} auf seine Inhalte verweist. Über diese Referenz gelangen die Suchenden dann auch zu den Vollinhalten der gefundenen Einträge. \\
            
        Die \emph{Open Archives Initiative} (OAI) hat im Jahre 2000 erstmalig das auf \href{http://gams.uni-graz.at/o:konde.215}{XML} und REST basierende \emph{OAI Protocol for Metadata Harvesting} (OAI-PMH) veröffentlicht, über das die standardisierte Kommunikation zwischen \emph{Service} und \emph{Data Providern} abgewickelt wird.\\
            
        Unter dem Namen \emph{OAI Object Reuse and Exchange} (OAI-ORE) wurden in der \emph{Open Archives Initiative} ergänzend zu OAI-PMH Verfahren geschaffen, um die Binnenstruktur digitaler Objekte in Repositorien und die Verknüpfungen zwischen ihnen abzubilden. So kann die digitale Repräsentation einer Handschrift in einem Repositorium z. B. aus verschiedenen Versionen und Formaten (Volltext in PDF und HTML, Metadaten in  \href{http://gams.uni-graz.at/o:konde.131}{RDF} etc.) und auch aus verschiedenen Teilen bestehen (\href{http://gams.uni-graz.at/o:konde.178}{TEI}-Transkript, Faksimiles etc.) sowie Verknüpfungen zu anderen Dokumenten besitzen (Übersetzung, Zitation, \href{http://gams.uni-graz.at/o:konde.14}{Versionierung} etc.). Die Grundidee von OAI-ORE besteht darin, diese Binnenstruktur eines Dokumentes maschinenlesbar in einer \emph{Resource Map} abzubilden und für den \emph{Service Provider} verfügbar zu machen. 		\\
            
        Die Verfügbarkeit von Forschungsergebnissen auf solchen Suchportalen trägt unmittelbar zu deren Sichtbarkeit bei und so kann \emph{Metadata Harvesting} auch als wichtiges Instrument der Öffentlichkeitsarbeit in der Scientific Community verstanden werden.	\\
            
        Auswahl einiger für die Digitalen Geisteswissenschaften bedeutender \emph{Service Provider}:\\
            
        \begin{itemize}\item {https://europeana.eu}\item {http://kulturpool.at}\item {https://correspsearch.net/}\item {https://pelagios.org/}\item {http://numismatics.org/ocre/}\end{itemize}\subsection*{Projekte:}\href{http://openarchives.org/}{Open Archives Initiative}, \href{http://openarchives.org/OAI/openarchivesprotocol.html}{OAI Protocol for Metadata Harvesting}, \href{http://openarchives.org/ore}{OAI Object Reuse and Exchange}\subsection*{Verweise:}\href{https://gams.uni-graz.at/o:konde.225}{Metadaten}, \href{https://gams.uni-graz.at/o:konde.12}{Persistent Identifier}, \href{https://gams.uni-graz.at/o:konde.6}{Digitale Nachhaltigkeit}, \href{https://gams.uni-graz.at/o:konde.87}{Bereitstellung von Forschungsdaten}\subsection*{Themen:}Metadaten, Archivierung\subsection*{Zitiervorschlag:}Stigler, Johannes. 2021. Metadata Harvesting. In: KONDE Weißbuch. Hrsg. v. Helmut W. Klug unter Mitarbeit von Selina Galka und Elisabeth Steiner im HRSM Projekt "Kompetenznetzwerk Digitale Edition". URL: https://gams.uni-graz.at/o:konde.10\newpage\section*{Metadaten (Fokus: Literaturwissenschaft – Bsp. Musil)} \emph{Fanta, Walter; walter.fanta@aau.at / Boelderl, Artur R.; artur.boelderl@aau.at }\\
        
    Metadaten sind im Zusammenhang mit der \href{http://gams.uni-graz.at/o:konde.28}{Textgenese}\href{http://gams.uni-graz.at/o:konde.17}{Annotationen}, die\\
            
        \begin{itemize}\item {die digitale Repräsentation des Textes um Information aus dem Dokument und }\item {die digitale Repräsentation des Dokuments um Informationen zum Text
                     anreichern.}\end{itemize}Ad A: Es handelt sich um sogenannte archivalische Informationen. Sie betreffen die
                  Materialität des Dokuments – das Papier (Papierart, Farbe, Format) bzw.
                  gegebenenfalls andere Trägermaterialien –, die Schreiberhand und das
                  Beschriftungsmaterial sowie die räumliche Anordnung des Texts (damit hängen der
                  Textausreifungsgrad, der Manuskripttyp – z. B. Notiz, Entwurf usw. –
                  zusammen).\\
            
        Ad B: Es handelt sich um die sogenannten philologischen Informationen. Sie
                  betreffen die raumzeitliche Verortung der Entstehung des Texts, also die
                  Datierung, die Chronologie, die inhaltliche Zuordnung in einem Œuvre oder in einem
                  anderen Entstehungszusammenhang, also den Status des jeweiligen Textes im
                  textgenetischen Dossier eines Schreibprojekts.\\
            
        Die Metadaten verknüpfen Text und Dokument. Ihre digitale Repräsentation scheint
                  die Struktur einer relationalen Datenbank zu erfordern, da die Informationen als
                  Feldinhalte zu den genannten archivalischen und philologischen Kategorien in
                  Datensätzen zu bestimmten Manuskripten bzw. Manuskriptteilen angeordnet werden
                  können, um das Beziehungsgeflecht textgenetischer Dynamik darzustellen. Die
                  Seitendokumentation der \emph{Klagenfurter Aufgabe} ist mit den
                  Metadaten so verfahren: \\
            
        Bsp. 1: Datensatz zu Musil, Nachlass, Mappe V/4, S. 216\\
            
        Feldname / Feldinhalt\\
            
        \begin{verbatim}Pagina                V/4/216
Sigle                 Ü6-2.1 6
Schreiber             Musil, Robert
Hauptbeschriftung     Tinte schwarz
Nebenbeschriftung     Bleistift Rotstift
Textstufe             Notiz Typ 3
Papier                cremefarben
Art                   Kanzleiblatt
Format                210x340
Datierungsabschnitt   7-6: November 1935 - Juni 1936
Datumsangabe          1936-04-06
Textgruppe            Band 3
Werk/Titel            Der Mann ohne Eigenschaften
Werkteil/MoE          Fortsetzung 1933-1936
Kapitelkomplex        Clarisse
Kapitelprojekt        Frühspaziergang 9\end{verbatim}Es ist möglich, die Metadaten in der linearen Struktur von \href{http://gams.uni-graz.at/o:konde.215}{XML}/\href{http://gams.uni-graz.at/o:konde.178}{TEI} zu verankern, um zu erreichen, dass so die Interoperabilität der
                  Spezifikationen gewahrt bleibt. Dies geschieht in <teiHeader>
                  für jede der ca. 12.000 Manuskriptseiten des Musil-Nachlasses nach dem folgenden
                  Annotationsprinzip: \\
            
        Bsp. 2:\\
            
        \begin{verbatim}<msPart xml:id="sn15091-05-04-216-10">
    <msIdentifier>
        <idno type="MO">sn15091-05-04-216-10</idno>
        <altIdentifier>
            <idno type="KWS">V/4/216</idno>
        </altIdentifier>
        <altIdentifier>
            <idno type="ps" xml:id="Ü6_2.1_6">Ü6-2.1 6</idno>
        </altIdentifier>
    </msIdentifier>
    <head>
        <origDate datingMethod="#dp" n="7-6" notBefore="1935-11"
        notAfter="1936-06">7-6: November 1935 - Juni 1936</origDate>
        <origDate when-iso="1936-04-06"/>
    </head>
    <msContents>
        <msItem>
            <title type="work">Der Mann ohne Eigenschaften</title>
            <titlePart type="work-part">Fortsetzung 1933-1936</titlePart>
            <titlePart type="chapter-group">Clarisse</titlePart>
            <title type="chapter-project">Frühspaziergang</title>
            <filiation type="step" corresp="#moe4_cla_fru_7">Stufe
            9</filiation>
        </msItem>
    </msContents>
    <physDesc>
        <objectDesc>
            <supportDesc>
                <support rend="recte">Kanzleiblatt cremefarben</support>
                <extent>211x341</extent>
            </supportDesc>
            <layoutDesc>
            <layout style="black_ink">
                <idno type="mst" n="draft_fragmentary">Entwurfsfragment</idno>
                Tinte schwarz
            </layout>
            </layoutDesc>
        </objectDesc>
    </physDesc>
</msPart>
<profileDesc>
    <handNotes>
        <handNote medium="pencil" xml:id="hn_1">Bleistift</handNote>
        <handNote medium="pencil" xml:id="hn_3">Rotstift</handNote>
    </handNotes>
</profileDesc>\end{verbatim}Dieser Beitrag wurden im Kontext des FWF-Projekts "MUSIL ONLINE – interdiskursiver Kommentar" 
                  (P 30028-G24) verfasst.\subsection*{Software:}\href{http://oxygenxml.com/}{Oxygen}\subsection*{Verweise:}\href{https://gams.uni-graz.at/o:konde.17}{Annotation (Literaturwissenschaft:
                           grundsätzlich)}, \href{https://gams.uni-graz.at/o:konde.27}{Text/Dokument (Fokus:
                           Literaturwissenschaft - Bsp. Musil)}, \href{https://gams.uni-graz.at/o:konde.28}{Textgenese}, \href{https://gams.uni-graz.at/o:konde.23}{Makrogenese (Fokus:
                           Literaturwissenschaft - Bsp. Musil)}, \href{https://gams.uni-graz.at/o:konde.24}{Mesogenese (Fokus:
                           Literaturwissenschaft - Bsp. Musil)}, \href{https://gams.uni-graz.at/o:konde.26}{Mikrogenese (Fokus:
                           Literaturwissenschaft - Bsp. Musil)}, \href{https://gams.uni-graz.at/o:konde.96}{Hybridedition}, \href{https://gams.uni-graz.at/o:konde.225}{Metadaten (allgemein)}\subsection*{Projekte:}\href{http://musilonline.at}{Musil Online}\subsection*{Themen:}Metadaten, Annotation und Modellierung\subsection*{Zitiervorschlag:}Fanta, Walter; Boelderl, Artur R. 2021. Metadaten (Fokus: Literaturwissenschaft – Bsp. Musil). In: KONDE Weißbuch. Hrsg. v. Helmut W. Klug unter Mitarbeit von Selina Galka und Elisabeth Steiner im HRSM Projekt "Kompetenznetzwerk Digitale Edition". URL: https://gams.uni-graz.at/o:konde.25\newpage\section*{Metadaten (allgemein)} \emph{Steiner, Elisabeth; elisabeth.steiner@uni-graz.at }\\
        
    Metadaten, verstanden als ‘Daten über Daten’, haben in der analogen Welt eine Jahrhunderte lange Geschichte. So ist der klassische Zettelkatalog in der Bibliothek nichts anderes als eine Sammlung von strukturierten Metadaten über die Werke, die man durch die Benutzung desselben finden möchte.\\
            
        In der digitalen Welt dienen Metadaten ebenfalls der Auffindbarkeit, aber auch der Verwaltung, der Wiederverwendbarkeit und der Archivierung von Ressourcen. Der große Unterschied dabei ist, dass sie hier nicht nur menschenlesbar, sondern auch für Maschinen verständlich sein müssen. Metadaten können nach ihrer Funktion unterteilt werden, z. B. in administrative, deskriptive oder technische Metadaten. Weiters können Strukturstandards, Wertstandards, Inhaltsstandards und Formatstandards unterschieden werden.\\
            
        Datenstrukturstandards geben die Struktur und Kategorien vor, mit denen ein digitales Objekt beschrieben werden kann (z. B. \emph{\href{http://gams.uni-graz.at/o:konde.128}{Dublin Core}}, MODS). Datenwertstandards hingegen geben das Aussehen der Information innerhalb dieser Elemente vor, meist in Form von kontrollierten Vokabularien oder Thesauri und \href{http://gams.uni-graz.at/o:konde.147}{Normdaten} (z. B. \href{http://gams.uni-graz.at/o:konde.111}{VIAF}, \emph{\href{http://gams.uni-graz.at/o:konde.107}{GeoNames}}, \emph{\href{http://gams.uni-graz.at/o:konde.108}{Getty Vocabularies}}). Richtlinien für die inhaltliche Erschließung über Katalogisierungs- oder Beschreibungsregeln liefern die Dateninhaltsstandards (z. B. behandelt \href{http://gams.uni-graz.at/o:konde.165}{RNA} die Erschließung von Nachlässen). Datenformatstandards definieren die maschinenlesbare Realisierung, z. B. in Form von \href{http://gams.uni-graz.at/o:konde.166}{Schemadateien}; viele Standards verwenden als Syntaxbasis \href{http://gams.uni-graz.at/o:konde.215}{XML}. Metadaten können von einem Format ins andere gemappt werden, so entsteht ein \emph{Crosswalk}.\\
            
        Mit dem Anwachsen der Menge an digital(isiert)en Metadaten tritt die Interoperabilität immer mehr in den Vordergrund. ‘Gute’ Metadaten sind daher jene, die nicht nur im eigenen Projektkontext funktionieren, sondern auch möglichst einen Austausch und eine Vergleichbarkeit mit anderen Projekten erlauben. Hier kommt der Anreicherung mit standardisierten URIs zur Erzeugung von \emph{\href{http://gams.uni-graz.at/o:konde.8}{Linked Open Data}} eine besondere Bedeutung zu. Die \href{http://gams.uni-graz.at/o:konde.49}{Aggregation} von Metadaten beispielsweise in einen gemeinsamen Suchraum wird oft über die Verwendung von OAI-PMH realisiert: die Metadaten werden ‘geharvestet’ (\emph{\href{http://gams.uni-graz.at/o:konde.10}{Metadata Harvesting}}) und in einem virtuellen Raum gemeinsam repräsentiert. Vernetzungsinitiativen dieser Art können generisch (z. B. \emph{Europeana}), quellenzentriert (z. B. \emph{CorrespSearch} für Briefe, \emph{nomisma.org} für Münzen) oder diszplinenspezifisch (z. B. \emph{Pelagios} für Alte Geschichte und Altertumskunde) sein.\\
            
        \subsection*{Literatur:}\begin{itemize}\item Introduction to metadata. Version 3.0. URL: \url{http://www.getty.edu/publications/intrometadata/}\item Riley, Jenn: Understanding Metadata: 2017. URL: \url{https://groups.niso.org/apps/group_public/download.php/17446/Understanding%20Metadata.pdf}.\end{itemize}\subsection*{Verweise:}\href{https://gams.uni-graz.at/o:konde.124}{Metadatenformate für Bilddateien}, \href{https://gams.uni-graz.at/o:konde.10}{Metadata Harvesting}, \href{https://gams.uni-graz.at/o:konde.128}{DCMI}, \href{https://gams.uni-graz.at/o:konde.8}{Linked Open Data}, \href{https://gams.uni-graz.at/o:konde.147}{Normdaten}, \href{https://gams.uni-graz.at/o:konde.6}{Digitale Nachhaltigkeit}, \href{https://gams.uni-graz.at/o:konde.49}{Datenaggregation}\subsection*{Projekte:}\href{https://www.europeana.eu/portal/de}{Europeana}, \href{https://correspsearch.net}{correspSearch}, \href{http://nomisma.org/}{Nomisma}, \href{https://pelagios.org/}{Pelagios}, \href{http://openarchives.org/OAI/openarchivesprotocol.html}{OAI Protocol for Metadata Harvesting}\subsection*{Themen:}Einführung, Metadaten, Archivierung, Digitale Editionswissenschaft\subsection*{Lexika}\begin{itemize}\item \href{https://edlex.de/index.php?title=Metadaten_(digitale)}{Edlex: Editionslexikon}\end{itemize}\subsection*{Zitiervorschlag:}Steiner, Elisabeth. 2021. Metadaten (allgemein). In: KONDE Weißbuch. Hrsg. v. Helmut W. Klug unter Mitarbeit von Selina Galka und Elisabeth Steiner im HRSM Projekt "Kompetenznetzwerk Digitale Edition". URL: https://gams.uni-graz.at/o:konde.225\newpage\section*{Metadaten-Schemata für LZA: METS} \emph{Steiner, Elisabeth; elisabeth.steiner@uni-graz.at }\\
        
    METS (\emph{Metadata Encoding \& Transmission Standard}) ist eine Spezifikation für die Erfassung von deskriptiven, administrativen und strukturellen Metadaten im \href{http://gams.uni-graz.at/o:konde.215}{XML}-Format. Der Standard wird für die Beschreibung von digitalen Objekten in Repositorien und zum Austausch von \href{http://gams.uni-graz.at/o:konde.25}{Metadaten} und digitalen Objekten zwischen Repositorien verwendet und von der\emph{ Library of Congress} betreut. METS fungiert dabei auch als Containerformat, das andere Metadatenstandards integrieren kann.\\
            
        Ein METS-Dokument besteht aus sieben Abschnitten. Im METS-Header (<metsHdr>) werden Informationen zum METS-Dokument selbst gespeichert. In der Sektion für deskriptive Metadaten (<dmdSec>) kann beispielsweise eine MODS-Beschreibung (bibliografische Metadaten im  \emph{Metadata Object Description}-Schema) eingebaut werden, in der Sektion für administrative Metadaten (<amdSec>) beispielsweise eine \href{http://gams.uni-graz.at/o:konde.130}{PREMIS}-Referenz. Darüber hinaus können alle möglichen Standards (\emph{\href{http://gams.uni-graz.at/o:konde.128}{Dublin Core}}, \href{http://gams.uni-graz.at/o:konde.178}{TEI}-Header etc.) in einem Metadatencontainer verpackt und integriert werden, wobei einige Richtlinien explizit für die Verwendung innerhalb von METS optimiert sind (beispielsweise MODS oder PREMIS). Die administrativen Metadaten gliedern sich weiter in technische Metadaten, Daten zu Rechten, Daten zum analogen Quelldokument und schließlich Archivierungsinformationen. Im <fileSec>-Element werden sämtliche im METS-Dokument verwendeten Dateien referenziert, die danach in der <structMap> in ihrer physikalischen und logischen Struktur angeordnet werden. Im <structLink>-Bereich wird die Verbindung zwischen diesen beiden Bereichen hergestellt, sodass durch das Dokument nach physischen Gesichtspunkten (Seiten durchblättern) wie auch nach logischen Gesichtspunkten (von Kapitel zu Kapitel springen) navigiert werden kann. In der letzten Sektion (<behaviorSec>) schließlich werden Informationen für die Anzeige und Darstellung des METS-Dokumentes gespeichert.\\
            
        METS ist nicht nur ein Standard zur \href{http://gams.uni-graz.at/o:konde.6}{Langzeitarchivierung}, sondern bildet häufig auch die Grundlage für die Darstellung von Daten, üblicherweise im Zusammenhang mit Viewern, die das Durchblättern und Navigieren in einer logischen Einheit von \href{http://gams.uni-graz.at/o:konde.36}{Digitalisaten} erlauben (z. B. retrodigitalisierte Handschriften). Ein bekanntes Beispiel ist der DFG-Viewer.\\
            
        \subsection*{Literatur:}\begin{itemize}\item METS. Metadata Encoding & Transmission Standard. URL: \url{http://www.loc.gov/standards/mets/}\end{itemize}\subsection*{Software:}\href{https://dfg-viewer.de/}{DFG Viewer}\subsection*{Projekte:}\href{http://www.loc.gov/standards/mods/}{Metadata Object Description Schema}\subsection*{Verweise:}\href{https://gams.uni-graz.at/o:konde.130}{PREMIS}, \href{https://gams.uni-graz.at/o:konde.6}{Langzeitarchivierung}, \href{https://gams.uni-graz.at/o:konde.36}{Bereitstellung von Digitalisaten}\subsection*{Themen:}Metadaten, Archivierung\subsection*{Zitiervorschlag:}Steiner, Elisabeth. 2021. Metadaten-Schemata für LZA: METS. In: KONDE Weißbuch. Hrsg. v. Helmut W. Klug unter Mitarbeit von Selina Galka und Elisabeth Steiner im HRSM Projekt "Kompetenznetzwerk Digitale Edition". URL: https://gams.uni-graz.at/o:konde.129\newpage\section*{Metadaten-Schemata für LZA: SKOS} \emph{Bleier, Roman; roman.bleier@uni-graz.at }\\
        
    Das seit 2003 entwickelte \emph{Simple Knowledge Organization System
                  }(SKOS) ist ein W3C-Metadatenstandard für die digitale Organisation von
                  Wissen. Die Grundidee ist, dass kontrollierte Vokabularien, Taxonomien,Thesauri
                  etc. ähnliche, einfache (\emph{simple}) Strukturen aufweisen. SKOS
                  versucht diese einfachen Strukturen abzubilden und für den Austausch und die
                  Verlinkung im \href{http://gams.uni-graz.at/o:konde.167}{Semantischen Web}
                  aufzubereiten. Daher zählt SKOS auch zu den Technologien des Semantischen Web und
                  basiert auf \href{http://gams.uni-graz.at/o:konde.131}{RDF} (\emph{Resource Description Framework}) und \href{http://gams.uni-graz.at/o:konde.131}{RDFS} (RDF-Schema).\\
            
        Das grundlegende SKOS-Modell (\emph{basic} SKOS) besteht aus drei
                  Bausteinen: Konzepte (\emph{Concepts}),
                  Namen/Bezeichnungen/Synonyme (\emph{Labels}) und Beziehungen (\emph{Relations}) (SKOS Primer). In SKOS sind alle
                  Begriffe in einem Klassifizierungssystem abstrakte Konzepte (\emph{skos:Concept}), die über eine URL klar identifiziert werden können.
                  Begriffen können unterschiedliche Namen/Bezeichnungen/Synonyme zugeordnet werden:
                  zum Beispiel eine bevorzugte Bezeichnung (\emph{Preferred Lexical
                     Label, skos:prefLabel}) oder eine alternative Bezeichnung (\emph{Alternative Lexical Label, skos:altLabel}). Weiters können
                  unterschiedliche Verwandtschaftsbeziehungen zwischen den Begriffen modelliert
                  werden: weitere (\emph{Broader, skos:broader}) und engere
                  Beziehungen (\emph{Narrower, skos:narrower}). \\
            
        In den digitalen Geisteswissenschaften kann SKOS überall dort angewandt werden, wo
                  die Verwendung von kontrollierten Vokabularien sinnvoll ist, zum Beispiel um
                  digitale Ressourcen zu beschreiben, zu gruppieren etc. (Zaytseva
                     2020) Auch \href{http://gams.uni-graz.at/o:konde.59}{Digitale
                     Editionen} können von der Verwendung von SKOS profitieren. Zum Beispiel
                  beschreibt Vogeler die Möglichkeiten durch den Einsatz von SKOS bei der \href{http://gams.uni-graz.at/o:konde.137}{Modellierung} von historischen
                  Rechnungsbüchern (Vogeler 2016, S. 32–33) und der Erstellung von
                  Registern von Waren, Dienstleistungen, Währungen, Personen und Orten. Scholger
                  beschreibt den Einsatz von SKOS bei der Entwicklung eines Thesaurus von
                  graphischen Elementen in einer digitalen Edition von Werktagebüchern eines
                  österreichischen Künstlers (Scholger 2019, S. 47–48). \\
            
        Für das Erstellen, Bearbeiten, Validieren und Darstellen von SKOS-Vokabularien
                  können unterschiedliche Tools verwendet werden. Eine Auswahlliste findet sich bei Zaytseva 
                  (Zaytseva 2020) oder auch im SKOS community wiki.\\
            
        \subsection*{Literatur:}\begin{itemize}\item Rehbein, Malte: Ontologien. Stuttgart: 2017, S. 162–176.\item Scholger, Martina: Pieces of a Bigger Puzzle: Tracing the Evolution of
                              Artworks and Conceptual Ideas in Artists' Notebooks. In: Versioning Cultural Objects: Digital Approaches 13. Norderstedt: 2019, S. 27–56.\item SKOS Simple Knowledge Organization System Primer. URL: \url{https://www.w3.org/TR/skos-primer/}\item SKOS. URL: \url{https://www.w3.org/2001/sw/wiki/SKOS}\item Vogeler, Georg: The Content of Accounts and Registers in their Digital
                              Edition. XML/TEI, Spreadsheets, and Semantic Web Technologies. In: Konzeptionelle Überlegungen zur Edition von Rechnungen
                              und Amtsbüchern des späten Mittelalters. Göttingen: 2016, S. 13–41.\item Controlled Vocabularies and SKOS. URL: \url{https://campus.dariah.eu/resource/controlled-vocabularies-and-skos}\end{itemize}\subsection*{Verweise:}\href{https://gams.uni-graz.at/o:konde.131}{Metadaten Schemata für LZA: RDF,
                           RDFS, OWL u.a.}, \href{https://gams.uni-graz.at/o:konde.25}{Metadaten}\subsection*{Themen:}Metadaten\subsection*{Software:}\href{https://protege.stanford.edu/}{protegé}, \href{http://skosmos.org}{Skosmos}\subsection*{Lexika}\begin{itemize}\item \href{https://edlex.de/index.php?title=Simple_Knowledge_Organisation_System_(SKOS)}{Edlex: Editionslexikon}\end{itemize}\subsection*{Zitiervorschlag:}Bleier, Roman. 2021. Metadaten-Schemata für LZA: SKOS. In: KONDE Weißbuch. Hrsg. v. Helmut W. Klug unter Mitarbeit von Selina Galka und Elisabeth Steiner im HRSM Projekt "Kompetenznetzwerk Digitale Edition". URL: https://gams.uni-graz.at/o:konde.132\newpage\section*{Metadatenformate für Bilddateien} \emph{Klug, Helmut W.; helmut.klug@uni-graz.at }\\
        
    In den \emph{DFG-Praxisregeln ‘Digitalisierung’}(S. 30–34) werden folgende Empfehlungen ausgesprochen: \\
            
        a) Die grundlegende Voraussetzung ist ein von Software unabhängiges Format in einer standardkonformen Form, in der Regel also eine \href{http://gams.uni-graz.at/o:konde.215}{XML}-Kodierung. \\
            
        b) „Die Verknüpfung zwischen den Metadaten einerseits und den digitalen Images andererseits zu einem Objekt muss dabei immer auf der Ebene der Metadaten gewährleistet sein. Zusätzlich können Metadaten auch in den Header der digitalen Images eingebettet werden, jedoch werden diese von den Software-Produkten unterschiedlich dargestellt und im schlimmsten Fall sogar korrumpiert, so dass [sic!] die Einbettung in jedem Fall nur eine ergänzende Option ist.“ (S. 30)\\
            
        Die DFG, die hier eher den Aufbau von Repositorien als die \href{http://gams.uni-graz.at/o:konde.6}{Langzeitarchivierung} im Auge hat, empfiehlt objektabhängig: \\
            
        \begin{itemize}\item {\href{http://gams.uni-graz.at/o:konde.129}{METS} für Volldigitalisate von Textwerken mit deskriptiven wie strukturellen \href{http://gams.uni-graz.at/o:konde.25}{Metadaten}, das als Containerformat verschiedene Metadatenstandards vereinen kann.}\item {LIDO für Sammlungen mit Bild-, Audio- oder Videoressourcen (i. e. eine Reihe unterschiedlicher Museumsobjekte), da neben semantischen Beschreibungsmethoden eigene Elemente zur Referenzierung der digitalen Ressourcen vorhanden sind.}\item {EAD: für Archivmaterialien}\item {SAFT: Standard-Austauschformat für archivische Findemittel}\end{itemize}Die \emph{Kleine Enzyklopädie der Langzeitarchivierung} empfiehlt neben METS:\\
            
        \begin{itemize}\item {\href{http://gams.uni-graz.at/o:konde.130}{PREMIS}:  Basis ist \href{http://gams.uni-graz.at/o:konde.11}{OAIS-Model}, der Fokus liegt auf Langzeitarchivierungsmetadaten (u. a. Beschreibung von Objekten und ihrer Kontexte, ihre Beziehungen und Verknüpfungen) (Kapitel 6.3)}\item {LMER: ‘Langzeitarchivierungsmetadaten für elektronische Ressourcen’, aus der bibliothekarischen Praxis  (Kapitel 6.4)}\item {MIX: XML-Schema für technische Metadaten zur Verwaltung digitaler Bildsammlungen  (Kapitel 6.5)}\end{itemize}An Metadaten sollten zumindest aufgenommen werden:\\
            
        \begin{itemize}\item {Besitzer}\item {Rechte}\item {bibliografische Daten}\item {inhaltlich beschreibende Daten}\item {technische Daten zu Dateien (Auflösung, Farbtiefe, Speicherstruktur).}\end{itemize}\subsection*{Literatur:}\begin{itemize}\item DFG-Praxisregeln "Digitalisierung", Deutsche Forschungsgemeinschaft: 2016. URL: \url{https://www.dfg.de/formulare/12_151/}.\item nestor Handbuch. Eine keine Enzyklopädie der digitalen Langzeitarchivierung. Hrsg. von Heike Neuroth, Achim Oßwald, Regine Scheffel, Stefan Strathmann und Mathias Jehn. Glückstadt: 2016, URL: \url{urn:nbn:de:0008-2010071949}.\end{itemize}\subsection*{Verweise:}\href{https://gams.uni-graz.at/o:konde.60}{Digitalisierung}, \href{https://gams.uni-graz.at/o:konde.63}{Digitalisierungsrichtlinien}, \href{https://gams.uni-graz.at/o:konde.36}{Bereitstellung von Digitalisaten}, \href{https://gams.uni-graz.at/o:konde.6}{Digitale Nachhaltigkeit}, \href{https://gams.uni-graz.at/o:konde.129}{Metadaten Schemata für LZA: METS}, \href{https://gams.uni-graz.at/o:konde.130}{Metadaten Schemata für LZA: PREMIS}\subsection*{Themen:}Metadaten, Archivierung\subsection*{Zitiervorschlag:}Klug, Helmut W. 2021. Metadatenformate für Bilddateien. In: KONDE Weißbuch. Hrsg. v. Helmut W. Klug unter Mitarbeit von Selina Galka und Elisabeth Steiner im HRSM Projekt "Kompetenznetzwerk Digitale Edition". URL: https://gams.uni-graz.at/o:konde.124\newpage\section*{Mikrogenese (Fokus: Literaturwissenschaft – Bsp. Musil)} \emph{Fanta, Walter; walter.fanta@aau.at / Boelderl, Artur R.;
                  artur.boelderl@aau.at}\\
        
    Aspekte der Mikrogenese (vgl. \href{http://gams.uni-graz.at/o:konde.28}{Textgenese}) manifestieren sich als Textrevisionen auf der einzelnen
                  Seite – als Streichungen <del> und Einfügungen
                     <add>. Die Hierarchieebenen der Streichungsprozeduren werden
                  mit einem nummerierten Attributwert \emph{level} angegeben, nach
                  dem Muster <del status="level_n">; entsprechend sind auch die
                  Einfügungen hierarchisiert. Für den Fall von Überlappungen ist die Zuhilfenahme
                  der leeren Elemente <delSpan> bzw. <addSpan>
                  vorgesehen. Werden Ersetzungen größerer gestrichener Textblöcke auf weiteren
                  Seiten (Beiblättern) vorgenommen und beispielsweise mit Verweiszeichen kenntlich
                  gemacht, die den ersetzenden Text dem gestrichenen zuordnen, wird das Element
                     <metamark> verwendet, um sowohl die Prozedur selbst
                  anzuzeigen als auch die Verknüpfung zu gewährleisten (Bsp. 1). Bei Umstellung wird
                  das Element <seg> für die Markierung des Textbereiches
                  verwendet und gegebenenfalls das Element <metamark> für die
                  Verzeichnung von Umstellungssignalen (Bsp. 2). Auch die \href{http://gams.uni-graz.at/o:konde.17}{Annotation} der Setzung von Alternativvarianten
                  erfolgt mittels <seg> (Bsp. 3). Für nicht zum Entwurfstext
                  gehörende Randbemerkungen (schreibtechnische Anmerkungen, Kommentare, Reflexionen)
                  wird <note place="margin" resp="author"> verwendet. \\
            
        Für die Annotation der Schreibmaterialverwendung kann nicht einfach das sonst in
                  der \href{http://gams.uni-graz.at/o:konde.178}{TEI} gebräuchliche Element
                     <handShift> herangezogen werden, da das zu annotierende
                  Phänomen nicht immer ein Wechsel der Schreiberhand ist, sondern ihr Hinzukommen in
                  zeitlichem Abstand von der Entstehung der Grundschicht steht, deren Schreiberhand
                  auf einer übergeordneten Ebene – im <teiHeader> unter
                     <msDesc><msPart> bei der Zuordnung der Manuskripttype
                  – verzeichnet ist. Daher gelangt in den Elementen <del>,
                     <add>, <note> und
                     <seg>, welche die Korrekturschicht markieren, der
                  Attributwert @hand zum Einsatz; bei <metamark> ist @hand nicht zulässig, da erfolgt das \href{http://gams.uni-graz.at/o:konde.126}{Markup} mit @rend. Als Attributwert wird ein Kürzel für die jeweilige Schreiberhand
                  zugeordnet, das im <teiHeader> im Bereich
                     <profileDesc><handNotes> aufgelöst wird (Bsp. 4). Für
                  die möglichst exakte Beschreibung des Zeicheninstrumentariums von Revisionen
                  findet das Attribut @rendition Verwendung, die Kürzel der Attributwerte sind im
                     <teiHeader> im Bereich
                     <encodingDesc><tagsDecl> aufgelöst (Bsp. 5).\\
            
        Zur kompakteren Gestaltung des Modells wäre eine Typisierung der Schreibakte
                  denkbar, indem zugehörige Elemente und Attribute in einer hierarchischen Ordnung
                  in Klassen zusammengefasst werden. Die Anregung dazu gibt ein Modell, das
                  Clausen/Klug (2019, S. 144–149) anhand von mittelalterlichen Codices
                  entwickelt haben. Auf das Beispiel Musil übertragen, ließe sich eine vierstufige
                  Hierarchie definieren: Schreibakte / Elemente / Attribute / Attributwerte. Für die
                  Annotationen auf jeder Ebene würde eine vereinfachte Form definiert werden. Aus
                  den Kombinationen innerhalb einer Zeichenkette würden unterschiedliche Typen von
                  Revisionsakten identifiziert werden können, wertvoll für die entsprechende
                  Präsentation an der Benutzerschnittstelle (z. B. in Form eines Web-Interface bzw.
                  im weiteren Sinne die (grafische) Präsentationsplattform, über die Nutzer*innen
                  auf die Repräsentationsdaten zugreifen bzw. diese dargeboten bekommen) und für die
                  maschinelle Nachnutzung zu Analysezwecken. Eine solche Kette wäre z. B.:
                  TRA.NUM.MAR.INK für eine Umstellung, die Musil durch Nummerierung am Rand vornahm. \\
            
        Bsp. 1:\\
            
        \begin{verbatim}<metamark function="reference" xml:id="T_0107025-1"/> …
<metamark function="reference" corresp="#T_0107025-1"/>\end{verbatim}Bsp. 2:\\
            
        \begin{verbatim}<seg rend="before" type="transposition" xml:id="T_xxxxxxx-n"> … </seg> … 
<seg rend="after" type="transposition" corresp="#T_xxxxxxx-n"> … </seg>
<metamark function="reference" place="margin"> … </metamark>\end{verbatim}Bsp. 3:\\
            
        Wenn sich zu einer Phrase xxx im Haupttext am Rand die nicht als Korrektur
realisierten alternativen Phrasen yyy und zzz finden, wird dies so kodiert:\\
            
        \begin{verbatim}<seg type="variant">xxx</seg> … <add place="variant">
<seg type="variant">yyy</seg> <seg type="variant">zzz</seg></add>\end{verbatim}Bsp. 4:\\
            
        \begin{verbatim}<seg hand="#hn_1"/> im <body> mit <handNote
medium="pencil" xml:id="hn_1">Bleistift</handNote> im
<tei-Header>\end{verbatim}Bsp. 5:\\
            
        \begin{verbatim}<seg type="hi" rendition="#r_7"/> als 
<rendition xml:id="r_7">geschweifte Klammer rechts</rendition>\end{verbatim}\subsection*{Literatur:}\begin{itemize}\item Clausen, Hans; Klug, Helmut: Schreiberische Sorgfalt: Der Einsatz digitaler Verfahren
                              für die textgenetische Analyse mittelalterlicher Handschriften. In: Textgenese in der digitalen Edition. Berlin, Boston: 2019, S. 139–151.\end{itemize}Dieser Beitrag wurden im Kontext des FWF-Projekts "MUSIL ONLINE – interdiskursiver Kommentar" 
                  (P 30028-G24) verfasst.\subsection*{Software:}\href{http://oxygenxml.com/}{Oxygen}\subsection*{Verweise:}\href{https://gams.uni-graz.at/o:konde.17}{Annotation (Literaturwissenschaft:
                           grundsätzlich)}, \href{https://gams.uni-graz.at/o:konde.19}{Interdiskursivität (Fokus:
                           Literaturwissenschaft - Bsp. Musil)}, \href{https://gams.uni-graz.at/o:konde.20}{Intertextualität (Fokus:
                           Literaturwissenschaft)}, \href{https://gams.uni-graz.at/o:konde.21}{Intratextualität (Fokus:
                           Literaturwissenschaft - Bsp. Musil)}, \href{https://gams.uni-graz.at/o:konde.28}{Textgenese}, \href{https://gams.uni-graz.at/o:konde.23}{Makrogenese (Fokus:
                           Literaturwissenschaft - Bsp. Musil)}, \href{https://gams.uni-graz.at/o:konde.24}{Mesogenese (Fokus:
                           Literaturwissenschaft - Bsp. Musil)}, \href{https://gams.uni-graz.at/o:konde.96}{Hybridedition}\subsection*{Projekte:}\href{http://musilonline.at}{Musil Online}\subsection*{Themen:}Annotation und Modellierung, Digitale Editionswissenschaft\subsection*{Lexika}\begin{itemize}\item \href{https://lexiconse.uantwerpen.be/index.php/lexicon/microgenesis/}{Lexicon of Scholarly Editing}\end{itemize}\subsection*{Zitiervorschlag:}Fanta, Walter; Boelderl, Artur R. 2021. Mikrogenese (Fokus: Literaturwissenschaft – Bsp. Musil). In: KONDE Weißbuch. Hrsg. v. Helmut W. Klug unter Mitarbeit von Selina Galka und Elisabeth Steiner im HRSM Projekt "Kompetenznetzwerk Digitale Edition". URL: https://gams.uni-graz.at/o:konde.26\newpage\section*{Mittelhochdeutsche Begriffsdatenbank (MHDBDB)} \emph{Hinkelmanns, Peter; peter.hinkelmanns@sbg.ac.at / Zeppezauer-Wachauer,
                  Katharina; katharina.wachauer@sbg.ac.at }\\
        
    Die Mittelhochdeutsche Begriffsdatenbank (MHDBDB)
                     (Zeppezauer-Wachauer/Hinkelmanns/Schmidt 1992), ein Teil des
                  Interdisziplinären Zentrums für Mittelalter und Frühneuzeit (IZMF) der \href{http://gams.uni-graz.at/o:konde.203}{Universität Salzburg}, ist ein seit
                  den 1970er-Jahren betriebenes und beständig erweitertes Recherchetool für das
                  Mittelhochdeutsche, das auf der Grundlage \href{http://gams.uni-graz.at/o:konde.59}{Digitaler Editionen} aufgebaut ist. Kernelemente sind
                  eine komplexe Suchmaschine und ein onomasiologisches Wörterbuch, in dem mittels
                  eines Begriffssystems Bedeutungen von Wortartikeln beschrieben werden, deren
                  Vorkommen im Korpus dann auf diese Lemmata und die entsprechende im Kontext
                  gültige Bedeutung bezogen werden. Die MHDBDB stellt somit ein umfassendes sprach-
                  und literaturwissenschaftliches Werkzeug dar.\\
            
        Für ihren Relaunch (ab 2021) setzt die MHDBDB auf ein Datenmodell basierend auf
                  unterschiedlichen \href{http://gams.uni-graz.at/o:konde.168}{Semantic Web-Technologien}, wie \href{http://gams.uni-graz.at/o:konde.131}{RDF-Vokabularen} und \href{http://gams.uni-graz.at/o:konde.151}{Ontologien}, sowie auf \href{http://gams.uni-graz.at/o:konde.178}{TEI} zur Kodierung der Korpustexte. TEI-Texte können
                  im \href{http://gams.uni-graz.at/o:konde.171}{Stand
                     off-Verfahren} beliebig mit \emph{\href{http://gams.uni-graz.at/o:konde.8}{Linked Open Data}} verknüpft werden. Annotationen wie etwa \href{http://gams.uni-graz.at/o:konde.156}{Part of Speech (PoS)} oder
                  Phrasen- und Satzstrukturen werden direkt auf die Tokens der Texte bezogen. Die
                  Vernetzung zwischen RDF- und TEI-Daten erfolgt mittels \emph{Web
                     Annotation Vocabulary.}(Sanderson/Ciccarese/Young 2017) Das Begriffssystem wird als
                  hierarchischer Thesaurus mit dem \href{http://gams.uni-graz.at/o:konde.132}{Simple Knowledge Organization System (SKOS)}(Miles/Bechhofer 2009) beschrieben, Wortartikel werden nach den
                  Vorgaben des \emph{OntoLex-Lemon-Lexicography}-Modules
                     (Bosque-Gil/Gracia 2019) kodiert. Weitere genutzte Ontologien bzw.
                  Vokabulare sind \emph{BibFrame 2.0}(Bibliographic Framework Initiative 2019) und die GND \emph{Ontology}(Haffner 2019).\\
            
        Normdaten z. B. der \href{http://gams.uni-graz.at/o:konde.109}{Gemeinsamen
                     Normdatei (GND)}(Gemeinsame Normdatei 2019) und \href{http://gams.uni-graz.at/o:konde.112}{Wikidata}(Wikidata 2019) werden in der MHDBDB nachgenutzt, um den Nutzerinnen
                  und Nutzern bessere Zugriffsmöglichkeiten auf das Datenmaterial zu ermöglichen.
                  Die Vernetzung mit Metadatenrepositorien ermöglicht das wechselseitige Anreichern
                  der Daten.\\
            
        Sämtliche MHDBDB-Daten werden ab 2021 unter einer \href{http://gams.uni-graz.at/o:konde.45}{Creative Commons}-Lizenz (voraussichtlich CC BY-NC-SA
                  3.0 AT) mittels des für den Betrieb erforderlichen universitären
                  Infrastruktur-Repositoriums \href{http://gams.uni-graz.at/o:konde.68}{dhPLUS}
                  bereitgestellt. Einen Überblick über die mittelhochdeutsche Lexikographie im \emph{Semantic Web} und eine ausführlichere Beschreibung des
                  Datenmodells bietet Hinkelmanns 2019.\\
            
        \subsection*{Literatur:}\begin{itemize}\item Bibliographic Framework Initiative Bibframe. URL: \url{http://www.loc.gov/bibframe/}\item The OntoLex Lemon Lexicography Module. URL: \url{https://www.w3.org/2019/09/lexicog/}\item Gemeinsame Normdatei. URL: \url{https://www.dnb.de/gnd}\item GND Ontology. URL: \url{https://d-nb.info/standards/elementset/gnd}\item Hinkelmanns, Peter: Mittelhochdeutsche Lexikographie und Semantic Web. Die
                              Anbindung der ‚Mittelhochdeutschen Begriffsdatenbank‘ an Linked Open
                              Data. In: Das Mittelalter 24: 2019, S. 129–141.\item Hinkelmanns, Peter; Zeppezauer-Wachauer, Katharina: ez ist ein wârheit, niht ein spel, daz netze was
                              sinewel. Die MHDBDB im Semantic Web. In: In Bamberger interdisziplinäre Mittelalterstudien. IN
                              PRESS.\item SKOS Simple Knowledge Organization System
                              Reference SKOS. URL: \url{http://www.w3.org/TR/skos-reference}\item Web Annotation Vocabulary. URL: \url{https://www.w3.org/TR/annotation-vocab/}\item Wikidata. URL: \url{http://www.wikidata.org}\item Mittelhochdeutsche Begriffsdatenbank (MHDBDB). URL: \url{http://www.mhdbdb.sbg.ac.at/}\end{itemize}\subsection*{Verweise:}\href{https://gams.uni-graz.at/o:konde.45}{Creative Commons}, \href{https://gams.uni-graz.at/o:konde.68}{Disseminations-Services:
                           DHPLUS}, \href{https://gams.uni-graz.at/o:konde.109}{Kontrollierte Vokabularien:
                           GND}, \href{https://gams.uni-graz.at/o:konde.112}{Kontrollierte Vokabularien:
                           Wikidata}, \href{https://gams.uni-graz.at/o:konde.8}{Linked Open Data}, \href{https://gams.uni-graz.at/o:konde.132}{Metadaten Schemata für LZA:
                           SKOS}, \href{https://gams.uni-graz.at/o:konde.131}{Metadaten Schemata für LZA: RDF,
                           RDFS, OWL u.a.}, \href{https://gams.uni-graz.at/o:konde.147}{Normdaten}, \href{https://gams.uni-graz.at/o:konde.151}{Ontologie}, \href{https://gams.uni-graz.at/o:konde.167}{Semantic Web}, \href{https://gams.uni-graz.at/o:konde.168}{Semantic Web-Technologien}, \href{https://gams.uni-graz.at/o:konde.171}{Stand-off-Markup}, \href{https://gams.uni-graz.at/o:konde.178}{TEI}, \href{https://gams.uni-graz.at/o:konde.203}{Universität Salzburg}\subsection*{Themen:}Institutionen\subsection*{Zitiervorschlag:}Hinkelmanns, Peter; Zeppezauer-Wachauer, Katharina. 2021. Mittelhochdeutsche Begriffsdatenbank (MHDBDB). In: KONDE Weißbuch. Hrsg. v. Helmut W. Klug unter Mitarbeit von Selina Galka und Elisabeth Steiner im HRSM Projekt "Kompetenznetzwerk Digitale Edition". URL: https://gams.uni-graz.at/o:konde.52\newpage\section*{Mockup} \emph{Sonnberger, Jakob; jakob.sonnberger@uni-graz.at }\\
        
    Unter einem \emph{Mockup} im Kontext der Webentwicklung versteht
                  man einen frühen, visuellen und funktionalen Entwurf des geplanten Webauftritts.
                  Im Vordergrund steht dabei die \emph{\href{http://gams.uni-graz.at/o:konde.205}{Usability}} (\emph{Userflow}), also wie sich künftige Benutzerinnen und
                  Benutzer durch die Anwendung navigieren können und sollen. Dementsprechend wird das \href{http://gams.uni-graz.at/o:konde.56}{Design} der Anwendung nur grob
                  skizziert, während das Hauptaugenmerk auf der Platzierung der wesentlichen
                  Inhalte, der Menüführung und der internen Logik des Seitenaufbaus liegt. \\
            
        Neben ihrer Funktion als konzeptionelles Hilfsmittel dienen \emph{Mockups} hauptsächlich als Kommunikationsgrundlage zwischen den einzelnen
                  Projektbeteiligten sowie mit Auftraggeberinnen und Auftraggebern (Was muss die
                  Anwendung ,können’?). \\
            
        Die Spannweite von \emph{Mockups} reicht dabei von händischen über
                  digitale Skizzen bis hin zu interaktiven – mit professioneller \href{http://gams.uni-graz.at/o:konde.136}{Mockup-Software} erstellten – Webanwendungen, wobei die Begriffe \emph{Wireframe}, \emph{Mockup} und 
                     \emph{Prototype}
                   zunehmend verschwimmen
                  bzw. synonym verwendet werden. \\
            
        \subsection*{Literatur:}\begin{itemize}\item Meidl, Oliver: Globales Webdesign: Anforderungen und Herausforderungen
                              an Globale Webseiten. Wiesbaden: 2014.\item Thesmann, Stephan: Interface Design: Usability, User Experience und
                              Accessibility im Web gestalten: 2016.\end{itemize}\subsection*{Verweise:}\href{https://gams.uni-graz.at/o:konde.205}{Usability}, \href{https://gams.uni-graz.at/o:konde.56}{Design}, \href{https://gams.uni-graz.at/o:konde.136}{Mockup-Software}\subsection*{Themen:}Interfaces\subsection*{Zitiervorschlag:}Sonnberger, Jakob. 2021. Mockup. In: KONDE Weißbuch. Hrsg. v. Helmut W. Klug unter Mitarbeit von Selina Galka und Elisabeth Steiner im HRSM Projekt "Kompetenznetzwerk Digitale Edition". URL: https://gams.uni-graz.at/o:konde.135\newpage\section*{Mockup-Software} \emph{Sonnberger, Jakob; jakob.sonnberger@uni-graz.at }\\
        
    Als \emph{Mockup}-Software oder \emph{Mockup}-Tools bezeichnet man Anwendungen, die speziell für die Erstellung eines \emph{\href{http://gams.uni-graz.at/o:konde.135}{Mockups}}, also eines ersten visuellen und funktionalen Entwurfs einer geplanten (Web-)Anwendung konzipiert sind. \\
            
        \emph{Mockup}-Tools bieten Sets an vorgefertigten, umgebungstypischen (App, Website, spezifisches Betriebssystem) Designelementen (Navigationselemente, Inhaltsblöcke), die nach dem Baukasten- und Drag\&Drop-Prinzip angeordnet werden können, wobei die Möglichkeit der Modifikation der einzelnen Elemente je nach Softwarelösung stark variiert. Bei manchen Anbietern gibt es darüber hinaus fertige Entwurfsvorlagen, die schnell und einfach den eigenen Bedürfnissen angepasst werden können. Weiters verfügen einige Lösungen über eine Benutzeransicht, in der Interaktivitäten simuliert werden können, beispielsweise wie sich Benutzerinnen und Benutzer durch verschiedene Bereiche einer Webanwendung navigieren (Click-Dummies).\\
            
        Prinzipiell wird zwischen Desktopanwendungen und browserbasierten Lösungen unterschieden, wobei letztere oft die Möglichkeit des \href{http://gams.uni-graz.at/o:konde.104}{kollaborativen Arbeitens} an einem Entwurf unterstützen. Neben einigen wenigen Open Source-Produkten sind die meisten professionellen \emph{Mockup}-Tools kostenpflichtig, bieten aber zumeist – zeitlich oder funktional beschränkte – Trial-Versionen der Anwendungen an. \\
            
        \subsection*{Literatur:}\begin{itemize}\item Thesmann, Stephan: Interface Design: Usability, User Experience und Accessibility im Web gestalten: 2016.\end{itemize}\subsection*{Verweise:}\href{https://gams.uni-graz.at/o:konde.135}{Mockup}, \href{https://gams.uni-graz.at/o:konde.104}{Kollaboration}\subsection*{Themen:}Interfaces\subsection*{Zitiervorschlag:}Sonnberger, Jakob. 2021. Mockup-Software. In: KONDE Weißbuch. Hrsg. v. Helmut W. Klug unter Mitarbeit von Selina Galka und Elisabeth Steiner im HRSM Projekt "Kompetenznetzwerk Digitale Edition". URL: https://gams.uni-graz.at/o:konde.136\newpage\section*{Modellierung} \emph{Galka, Selina; selina.galka@uni-graz.at }\\
        
    Ein Modell bildet einen Ausschnitt aus der realen Welt ab, berücksichtigt dabei
                  aber nur jene Eigenschaften, die für eine bestimmte Fragestellung relevant sind.
                  Ein Modell hat einen Zweck – man versucht im Zuge der Modellierung in den Digital
                  Humanities Objekte zu identifizieren (z. B. Texteinheiten, visuelle Zeichen,
                  linguistische Merkmale oder Personen), Zusammenhänge festzustellen (z. B.
                  Gruppierungen von Objekten) und Regeln zu formulieren.\\
            
        Modelle im Allgemeinen sind abstrakt und können nicht von Computern verarbeitet
                  werden. Dazu müssen sie eindeutig und explizit vorliegen, also in eine formale
                  Form gebracht werden. Ein Datenmodell ist ein solches formales Modell –
                  Datenmodelle ermöglichen komplexe maschinelle Operationen in Bezug auf die Daten,
                  dienen als Grundlage der Kommunikation über die Daten, sichern eine höhere
                  Qualität der Daten (Formulierung von Bedingungen) und ermöglichen bei der
                  Verwendung von etablierten Standards (z. B. \href{http://gams.uni-graz.at/o:konde.178}{TEI}) den Austausch oder das Zusammenführen von
                  Daten. (Jannidis/Kohle/Rehbein 2017, S. 100)\\
            
        Der erste Schritt in der Modellierung ist das Erstellen eines konzeptionellen
                  Datenmodells. Man identifiziert die für die Fragestellung und den Zweck relevanten
                  Entitäten, Attribute und Beziehungen. (Jannidis/Kohle/Rehbein 2017, S.
                     103) Dafür gibt es unterschiedliche Methoden: das konzeptionelle Modell
                  kann beispielsweise verbalisiert werden, in Diagrammen dargestellt werden (z. B.
                  UML, ER) oder formal notiert werden (z. B. \href{http://gams.uni-graz.at/o:konde.215}{XML}-\href{http://gams.uni-graz.at/o:konde.166}{Schema}, \href{http://gams.uni-graz.at/o:konde.151}{Ontologie} in
                     \href{http://gams.uni-graz.at/o:konde.131}{RDFS/OWL}). Bei der
                  Modellierung eines Briefes mit XML könnten folgende Entitäten und Attribute
                  berücksichtigt werden:\\
            
        \begin{verbatim}<brief><date>20. November 2019</date>
Lieber <name type=”forename” gender=”male”>Stefan</name>, 
ich freue mich auf unser Treffen in <place>Wien</place>.
Liebe Grüße,
<name>Sebastian</name>
</brief>\end{verbatim}In diesem Beispiel wurde das abstrakte Briefmodell, bei dem Entitäten wie der
                  Brief an sich oder die Datumszeile identifiziert wurden, in XML abgebildet; mit
                     \href{http://gams.uni-graz.at/o:konde.126}{Markup} werden implizite
                  Strukturen expliziert. Die modellierten Daten können somit maschinell prozessiert
                  werden. Dabei handelt es sich aber nur um eine mögliche Umsetzung des
                  konzeptionellen Datenmodells – andere mögliche Datenformate wären beispielsweise
                  RDF-Dateien mit definierten Vokabularien oder Datenstrukturen und Daten in einer
                  Datenbank.\\
            
        In der Modellierung wird also unterschieden zwischen der modellierten Instanz
                  selbst, dem Datenmodell ( = das Muster, das auf mehrere Instanzen anwendbar ist)
                  und dem Metamodell (kann auf mehrere Datenmodelle angewendet werden).\\
            
        \subsection*{Literatur:}\begin{itemize}\item Burr, Elisabeth: DHD 2016. Modellierung, Vernetzung, Visualisierung.
                              Konferenzabstracts, URL: \url{http://dhd2016.de/}.\item Jannidis, Fotis; Kohle, Hubertus: Digital Humanities. Eine Einführung. Mit Abbildungen und
                              Grafiken Digital Humanities. Hrsg. von  und Malte Rehbein. Stuttgart: 2017.\item Lukas, Wolfgang: Archiv – Text - Zeit. Überlegungen zur Modellierung und
                              Visualisierung von Textgenese im analogen und digitalen Medium. In: Textgenese in der digitalen Edition. Berlin, Boston: 2019, S. 23–50.\item The shape of data in digital humanities: modeling texts
                              and text-based resources. Hrsg. von Julia Flanders und Fotis Jannidis. London: 2019.\end{itemize}\subsection*{Verweise:}\href{https://gams.uni-graz.at/o:konde.126}{Markup}, \href{https://gams.uni-graz.at/o:konde.79}{Einführung: Was ist XML/TEI?}, \href{https://gams.uni-graz.at/o:konde.195}{Textmodellierung}, \href{https://gams.uni-graz.at/o:konde.50}{Datenmodell "hyperdiplomatische
                           Transkription"}, \href{https://gams.uni-graz.at/o:konde.51}{Datenmodell "Kalender"}, \href{https://gams.uni-graz.at/o:konde.52}{Mittelhochdeutsche
                           Begriffsdatenbank}, \href{https://gams.uni-graz.at/o:konde.53}{Datenmodell “eventSearch”}, \href{https://gams.uni-graz.at/o:konde.151}{Ontologie}, \href{https://gams.uni-graz.at/o:konde.131}{RDF}\subsection*{Themen:}Einführung, Annotation und Modellierung\subsection*{Zitiervorschlag:}Galka, Selina. 2021. Modellierung. In: KONDE Weißbuch. Hrsg. v. Helmut W. Klug unter Mitarbeit von Selina Galka und Elisabeth Steiner im HRSM Projekt "Kompetenznetzwerk Digitale Edition". URL: https://gams.uni-graz.at/o:konde.137\newpage\section*{Music Encoding Initiative (MEI)} \emph{Galka, Selina; selina.galka@uni-graz.at}\\
        
    Die \emph{Music Encoding Initiative} (MEI) ist ein Projekt und
                  Dokumentenformat zur Kodierung, zum Austausch und zur Archivierung von
                  musikalischen Inhalten. (MEI) Das Format wurde von Perry Roland an
                  der \emph{University of Virginia} entwickelt und wird heute an der
                  Akademie der Wissenschaften und der Literatur in Mainz betreut. MEI orientiert
                  sich bezüglich der Zielsetzungen und Organisation an der \href{http://gams.uni-graz.at/o:konde.178}{TEI} (\emph{Text Encoding
                     Initiative}) und bietet sich zur Kodierung von Inhalten im Rahmen von \href{http://gams.uni-graz.at/o:konde.139}{Digitalen Musikeditionen} an.\\
            
        MEI basiert auf dem \href{http://gams.uni-graz.at/o:konde.225}{XML}-Format;
                  die Kodierungsrichtlinien werden in den MEI-\emph{Guidelines}
                  festgehalten (aktueller Stand: MEI 4.0.1). MEI zeichnet sich vor allem durch einen
                  sehr umfangreichen Metadatenbereich aus (<meiHead>), in welchem
                  Informationen zur digitalen Datei, dem Projekt, den Editionsrichtlinien, der
                  Entstehungsgeschichte des Werkes und den einzelnen Quellen festgehalten werden
                  können. (Kepper 2006, S. 11) Im <music>-Bereich
                  eines MEI-Dokuments werden die musikalischen Inhalte kodiert, wobei hier
                  unterschiedliche Module für moderne Notation, Neumen oder Mensuralnotation
                  bereitgestellt werden. Ähnlich wie bei der TEI werden ebenfalls Möglichkeiten zur
                  Kodierung von Varianten, Streichungen, Ergänzungen und anderen Phänomenen
                  angeboten.\\
            
        Von der MEI-Community werden laufend Tools zur Datengenerierung, Konversion und
                  auch Darstellung von MEI-Daten entwickelt, wie z. B. \emph{Verovio} oder \emph{MerMEId}, noch handelt es sich dabei
                  jedoch meist nur um Plugins oder Bibliotheken und nicht um vollständige
                  Softwarelösungen. (MEI: Official Tools)\\
            
        Neben MEI steht auch noch MusicXML als XML-basiertes Format zur Verfügung, welches
                  von Michael Good entworfen wurde. MusicXML ist in der Community sehr weit
                  verbreitet und wird von einigen wichtigen Notationsprogrammen unterstützt; MEI
                  hingegen bietet eine bessere Ausgangsbasis für editionswissenschaftliche Zwecke.
                     (Kepper 2011, S. 379f.)\\
            
        \subsection*{Literatur:}\begin{itemize}\item Devaney, Johanna; Léveillé Gauvin, Hubert: Encoding music performance data in Humdrum and
                              MEI. In: International Journal on Digital Libraries 20: 2019, S. 81–91.\item Kepper, Johannes: Codierungsformen von Musik. In: Kolloquium des Ausschusses für
                              musikwissenschaftliche Editionen der Union der deutschen Akademien der
                              Wissenschaften. Akademie der Wissenschaften Mainz, 16. – 18. November
                              2006.\item Kepper, Johannes: Musikedition im Zeichen neuer Medien. Historische
                              Entwicklung und gegenwärtige Perspektiven musikalischer
                              Gesamtausgaben Musikedition im Zeichen neuer Medien. Norderstedt: 2011, URL: \url{https://kups.ub.uni-koeln.de/6639/}.\item Introduction to the Music Encoding Initiative. URL: \url{https://doi.org/10.21428/65a6243c.9fa9b4f7}\item Music Encoding Initiative. URL: \url{https://music-encoding.org}\item MEI: Official Tools. URL: \url{https://music-encoding.org/resources/tools.html}\item MEI Guidelines 4.0.1. URL: \url{https://music-encoding.org/guidelines/v4/content/}\item Seipelt, Agnes; Gulewycz, Paul; Klugseder, Robert: Digitale Musikanalyse mit den Techniken der Music
                              Encoding Initiative (MEI) am Beispiel von Kompositionsstudien Anton
                              Bruckners. In: Die Musikforschung 71: 2018, S. 366–378.\end{itemize}\subsection*{Software:}\href{http://www5.kb.dk/en/nb/dcm/projekter/mermeid.html}{MerMEId}, \href{https://github.com/music-encoding/sibmei/releases}{SibMEI}, \href{https://github.com/DDMAL/libmei/}{LibMEI}, \href{http://web.mit.edu/music21/}{MEI to Music21
                           Converter}, \href{https://github.com/rettinghaus/MEILER}{MEILER}, \href{http://zolaemil.github.io/meiView/}{meiView}, \href{http://www.partitionnumerique.com/music-sheet-viewer-wordpress-plugin/}{Music Sheet Viewer}, \href{https://www.verovio.org/index.xhtml}{Verovio}\subsection*{Verweise:}\href{https://gams.uni-graz.at/o:konde.178}{TEI}, \href{https://gams.uni-graz.at/o:konde.215}{XML}, \href{https://gams.uni-graz.at/o:konde.139}{Digitale Musikedition}\subsection*{Projekte:}\href{https://music-encoding.org/community/projects-users.html}{MEI:
                           Projects}\subsection*{Themen:}Einführung, Annotation und Modellierung\subsection*{Lexika}\begin{itemize}\item \href{https://edlex.de/index.php?title=Music_Encoding_Initiative_(MEI)}{Edlex: Editionslexikon}\end{itemize}\subsection*{Zitiervorschlag:}Galka, Selina. 2021. Music Encoding Initiative (MEI). In: KONDE Weißbuch. Hrsg. v. Helmut W. Klug unter Mitarbeit von Selina Galka und Elisabeth Steiner im HRSM Projekt "Kompetenznetzwerk Digitale Edition". URL: https://gams.uni-graz.at/o:konde.226\newpage\section*{NLP} \emph{Bleier, Roman; roman.bleier@uni-graz.at}\\
        
    \emph{Natural Language Processing} (NLP) oder auch maschinelle
                  Sprachverarbeitung beschäftigt sich mit der algorithmengestützten Verarbeitung von
                  natürlicher Sprache. Teilaufgaben von NLP beschäftigen sich unter anderem mit
                  Spracherkennung, Tokenisierung von Texten, \emph{\href{http://gams.uni-graz.at/o:konde.156}{Part-Of-Speech-Tagging/PoS}}, \emph{\href{http://gams.uni-graz.at/o:konde.141}{Named Entity Recognition/NER}} und dem automatisierten Erkennen und der Extraktion der Bedeutung von
                  Wörtern im Satzgefüge und den Beziehungen zu anderen Wörtern sowie von Sätzen
                  zueinander. In sogenannten NLP-Pipelines werden mehrere dieser Teilbereiche
                  aufeinanderfolgend ausgeführt und ein Text (\emph{plain text})
                  schrittweise mit Information angereichert. Ein Beispiel dafür ist das
                  Onlineservice \emph{\href{http://gams.uni-graz.at/o:konde.212}{WebLicht}}. Es erlaubt der Nutzerin bzw. dem Nutzer, eine NLP-Pipeline mit
                  austauschbaren Teilaufgaben, die von Drittanbietern zur Verfügung gestellt werden,
                  zusammenzustellen und auf Texte anzuwenden. \\
            
        Im Kontext von \href{http://gams.uni-graz.at/o:konde.59}{Digitalen Editionen}
                  kann NLP bei der Aufbereitung und \href{http://gams.uni-graz.at/o:konde.146}{Normalisierung} von Texten und der automatisierten Anreicherung mit
                  semantischen Informationen zum Einsatz kommen, zum Beispiel beim automatisierten
                  Erkennen von Sätzen und dem Taggen von Personen und Orten. Die Daten von Digitalen
                  Editionen können aber auch als Grundlage für NLP-Analysen dienen. In diesem Falle
                  ist es der fertig edierte, elektronische Text, der mit NLP-Methoden für die
                  Expertenanalyse aufbereitet wird. Ein weiterer Anwendungsfall wäre, dass die
                  Editionsdaten als Trainingsdaten für \emph{Machine Learning}
                  verwendet werden (z. B. Personen- und Ortsdaten für \href{http://gams.uni-graz.at/o:konde.141}{NER}). \\
            
        \subsection*{Literatur:}\begin{itemize}\item Bird, Steven; Klein, Ewan; Loper, Edward: Natural Language Processing with Python. Bejing u.a.: 2009, URL: \url{https://www.nltk.org/book/}.\item Jurafsky, Daniel; Martin, James H.: Speech and Language Processing: An Introduction to
                              Natural Language Processing, Computational Linguistics, and Speech
                              Recognition. Upper Saddle River, New Jersey, United States of
                                 America: 2009.\item Piotrowski, Michael: Natural Language Processing for Historical Texts. Hrsg. von  und Graeme Hirst: 2012, URL: \url{http://doi.org/10.2200/S00436ED1V01Y201207HLT017}.\item WebLicht. Main Page Main page. URL: \url{https://weblicht.sfs.uni-tuebingen.de/weblichtwiki/index.php/Main_Page}\end{itemize}\subsection*{Verweise:}\href{https://gams.uni-graz.at/o:konde.156}{Part-of-Speech-Tagging}, \href{https://gams.uni-graz.at/o:konde.170}{spaCy}, \href{https://gams.uni-graz.at/o:konde.141}{Named Entity Recognition /
                           NER}, \href{https://gams.uni-graz.at/o:konde.212}{Weblicht}, \href{https://gams.uni-graz.at/o:konde.216}{xTokenizer}, \href{https://gams.uni-graz.at/o:konde.176}{Tagger}\subsection*{Software:}\href{https://www.clarin.eu/content/services}{CLARIN-mediated NLP-services}, \href{https://enrich.acdh.oeaw.ac.at}{enrich/stanbol
                           (ACDH-OeAW)}, \href{https://manuscriptdesk.uantwerpen.be/md/Main_Page}{Manuscript
                           Desk}, \href{http://corpus-tools.org/pepper/}{SaltNPepper}, \href{https://code.google.com/archive/p/topic-modeling-tool/}{topic-modelling-tool}, \href{https://weblicht.sfs.uni-tuebingen.de/weblicht/}{weblicht}, \href{http://www.teitok.org/index.php?action=about}{TEITOK}, \href{http://opennlp.apache.org/}{Apache
                           OPENNLP}, \href{http://ucrel.lancs.ac.uk/claws/}{CLAWS
                           POS-Tagger for English}, \href{http://art.uniroma2.it/external/chaosproject/}{Chaos}, \href{http://www.languagecomputer.com/}{CiceroLight}, \href{https://github.com/dbpedia-spotlight/dbpedia-spotlight/wiki}{DBpedia Spotlight}, \href{http://www.talp.upc.edu/}{FreeLing}, \href{https://www.digitisation.eu}{IMPACT Tools and
                           Data}, \href{https://www.nltk.org/}{Natural Language Toolkit
                           (nltk)}, \href{http://cltk.org/}{Classical Language Toolkit
                           (cltk)}, \href{https://spacy.io/}{spacy }, \href{https://github.com/zalandoresearch/flair}{flair}, \href{https://www.ims.uni-stuttgart.de/forschung/ressourcen/werkzeuge/german-ner/}{German NER}, \href{https://github.com/tudarmstadt-lt/GermaNER}{GermaNER}, \href{https://www.cis.uni-muenchen.de/~schmid/tools/TreeTagger/}{TreeTagger}, \href{https://www.cis.uni-muenchen.de/~schmid/tools/RNNTagger/}{RNNTagger}, \href{https://github.com/tsproisl/SoMeWeTa}{SoMeWeTa}, \href{https://github.com/ee-2/SurrogateGeneration}{Surrogate Generation}, \href{http://neuroner.com/}{NeuroNER}, \href{https://robcast.github.io/digilib/}{Digilib}, \href{http://lemmatise.ijs.si/}{LemmaGen}, \href{http://alumni.media.mit.edu/~hugo/montylingua/index.html}{MontyLingua}, \href{https://sites.google.com/site/morfetteweb/}{Morfette}, \href{https://cst.dk/online/lemmatiser/uk/}{CST's
                           Lemmatiser}, \href{https://github.com/acdh-oeaw/acdh-spacytei}{acdh-spacytei}, \href{https://github.com/acdh-oeaw/xsl-tokenizer}{xsl-tokenizer}, \href{http://corpus-tools.org/annis/}{ANNIS}\subsection*{Themen:}Einführung, Natural Language Processing\subsection*{Zitiervorschlag:}Bleier, Roman. 2021. NLP. In: KONDE Weißbuch. Hrsg. v. Helmut W. Klug unter Mitarbeit von Selina Galka und Elisabeth Steiner im HRSM Projekt "Kompetenznetzwerk Digitale Edition". URL: https://gams.uni-graz.at/o:konde.145\newpage\section*{Nachlassedition} \emph{Zangerl, Lina Maria; linamaria.zangerl@sbg.ac.at }\\
        
    Eine Nachlassedition ist eine Edition von Dokumenten aus persönlicher Provenienz.
                  Unter einem Nachlass sind dabei im weitesten Sinn alle Materialien zu verstehen,
                  die über das private und öffentliche Leben einer Person Aufschluss geben können.
                  Das sind zumeist unikale Schriftstücke, aber auch Drucke, Arbeitsbibliotheken,
                  Bilder, audiovisuelle Materialien und Gegenstände sowie auch digitale
                  Nachlassteile, die sich bei Personen wie Politikerinnen und Politikern,
                  Schriftstellerinnen und Schriftstellern, Künstlerinnen und Künstlern und
                  Wissenschafterinnen und Wissenschaftern zu ihren Lebzeiten angesammelt haben.
                     (Bülow 2016, S. 145) Während sich die archivarische und
                  bibliothekarische Erschließung von Nachlässen auf das Ordnen und Verzeichnen der
                  Nachlassbestandteile konzentriert, nimmt eine Nachlassedition die Veröffentlichung
                  vor. Historische Nachlasseditionen waren dabei oftmals Manipulationen der
                  Editorinnen und Editoren ausgesetzt, wie berühmte Beispiele (Nietzsche, Novalis,
                  Kafka) zeigen. (Woesler 2003, S. 62)\\
            
        Der Begriff Nachlassedition wird auch für Editionen einzelner, ausgewählter
                  Quellen aus dem Nachlass verwendet. Meist sind aber vor allem Editionen gemeint,
                  die ganze Nachlässe abzubilden versuchen. Von anderen Editionstypen unterscheidet
                  sich die Nachlassedition dahingehend, dass sie die überlieferten Originale in den
                  Mittelpunkt rückt. Die Edition ganzer Nachlässe hat zum Ziel, wie auch die \href{http://gams.uni-graz.at/o:konde.33}{Archivedition}, eine möglichst
                  objektive Erfassung aller Textzeugen im Sinne einer Grundlagenforschung (auch für
                  weitere, aufbauende Editionen) zu präsentieren. (Sahle 2013, I, 218)
                  Faksimilierung und \href{http://gams.uni-graz.at/o:konde.197}{Transkription}
                  des Nachlasses sind demnach vorrangig, \href{http://gams.uni-graz.at/o:konde.28}{Textkonstitution} und \href{http://gams.uni-graz.at/o:konde.34}{Kommentar} spielen im Vergleich zu anderen Editionstypen eine kleinere
                  Rolle. Somit hat die Nachlassedition, wie die \href{http://gams.uni-graz.at/o:konde.160}{Quellenedition}, zuallererst einen erschließenden
                  Charakter, der keine bestimmte Interpretation der Dokumente prädisponieren will.
                  Mit der \href{http://gams.uni-graz.at/o:konde.93}{historisch-kritischen
                     Edition} hat sie gemein, dass sie möglichst vollständig alle im Nachlass
                  erhaltenen Textvarianten aufnimmt. Beeinflusst von der \emph{\href{http://gams.uni-graz.at/o:konde.46}{critique génétique}} will die Nachlassedition den \href{http://gams.uni-graz.at/o:konde.127}{materiellen Bedingungen} des schöpferischen Prozesses Rechnung tragen und
                  zudem die Komplexität der Beziehungen zwischen einzelnen Dokumenten im Nachlass
                  deutlich machen. \\
            
        In dieser Hinsicht ergeben sich durch digitale Nachlasseditionen neue
                  Möglichkeiten: Die Grenzen zwischen Archiv, textgenetischer Forschung und Edition
                  werden im Digitalen fließend. Als beispielhafte, frühe digitale Nachlassedition
                  kann die 2009 erstmals auf DVD erschienene \emph{Klagenfurter
                     Ausgabe} sämtlicher Werke, Briefe und nachgelassener Schriften Robert
                  Musils gelten. Heute machen vor allem webgestützte Nachlasseditionen die enge
                  Beziehung von archivarisch-bibliothekarischer Erschließung und Edition sinnfällig.
                     
                  \href{http://gams.uni-graz.at/o:konde.168}{Semantic-Web-Technologien} bieten neue
                  Möglichkeiten, die umfangreichen thematischen, biografischen und topografischen
                  Verbindungen von Einzeldokumenten und Nachlassteilen untereinander und nach außen
                  aufzuzeigen sowie das (Kontext-)Wissen der Bearbeiterinnen und Bearbeiter zu den
                  Originalen beispielsweise auf Basis von \href{http://gams.uni-graz.at/o:konde.151}{Ontologien} zu formalisieren. \\
            
        \subsection*{Literatur:}\begin{itemize}\item von Bülow, Ulrich: Nachlässe. In: Handbuch Archiv. Stuttgart: 2016, S. 143–152.\item Sahle, Patrick: Digitale Editionsformen. Zum Umgang mit der
                              Überlieferung unter den Bedingungen des Medienwandels. Teil 1: Das
                              typografische Erbe. Norderstedt: 2013.\item Woesler, Winfried: Der Editor und 'sein' Autor. In: editio 17: 2003, S. 50–66.\end{itemize}\subsection*{Verweise:}\href{https://gams.uni-graz.at/o:konde.33}{Archivausgabe}, \href{https://gams.uni-graz.at/o:konde.83}{Faksimileausgabe}, \href{https://gams.uni-graz.at/o:konde.93}{Historisch-kritische Edition}, \href{https://gams.uni-graz.at/o:konde.151}{Ontologie}, \href{https://gams.uni-graz.at/o:konde.160}{Quellenedition}, \href{https://gams.uni-graz.at/o:konde.167}{Semantic Web}, \href{https://gams.uni-graz.at/o:konde.168}{Semantik Web Technologien}, \href{https://gams.uni-graz.at/o:konde.192}{Textkritik}, \href{https://gams.uni-graz.at/o:konde.34}{Kommentar}, \href{https://gams.uni-graz.at/o:konde.175}{Tagebuchedition}, \href{https://gams.uni-graz.at/o:konde.39}{Briefedition}, \href{https://gams.uni-graz.at/o:konde.59}{Digitale Edition}\subsection*{Projekte:}\href{http://musilonline.at}{Musil Online}, \href{www.stefanzweig.digital}{Stefan Zweig
                           digital}, \href{http://www.nietzschesource.org/#eKGWB}{Nietzsche Source - Digitale kritische Gesamtausgabe}, \href{http://bruemmer.staatsbibliothek-berlin.de/nlbruemmer/}{Nachlass
                           Franz Brümmer}, \href{http://www.wittgensteinsource.org}{Wittgenstein
                           Source}, \href{http://burckhardtsource.org}{Burckhardt
                           Source}, \href{https://gams.uni-graz.at/context:bag}{Franz
                           Brentano Archiv}, \href{http://www.hhp.uni-trier.de/Projekte/HHP/}{Heinrich Heine Portal}\subsection*{Themen:}Einführung\subsection*{Lexika}\begin{itemize}\item \href{https://edlex.de/index.php?title=Nachlassedition}{Edlex: Editionslexikon}\end{itemize}\subsection*{Zitiervorschlag:}Zangerl, Lina Maria. 2021. Nachlassedition. In: KONDE Weißbuch. Hrsg. v. Helmut W. Klug unter Mitarbeit von Selina Galka und Elisabeth Steiner im HRSM Projekt "Kompetenznetzwerk Digitale Edition". URL: https://gams.uni-graz.at/o:konde.140\newpage\section*{Nachschlagewerke zur Editionswissenschaft} \emph{Klug, Helmut W.; helmut.klug@uni-graz.at }\\
        
    Aktuell findet man im Internet neben dem \emph{KONDE-Weißbuch}
                  mehrere Nachschlagewerke mit editionswissenschaftlichem Fokus:\\
            
        \begin{itemize}\item {\textbf{Dictionnaire de critique génétique}  (Monica Zanardo und
                     Franz Johansson, Paris)Die französische Website präsentiert vorab
                     veröffentlichte Einträge eines geplanten gedruckten Nachschlagewerks zur
                     editorischen Forschungsmethode der \emph{\href{http://gams.uni-graz.at/o:konde.46}{critique génétique}}, das am \emph{Institut des textes et manuscrits modernes}
                     entstehen soll(te). Es sind nur wenige, teils unvollständige Artikel vorhanden;
                     Fachbegriffe werden in mehreren Sprachen angeboten.}\item {\textbf{Edlex: Editionslexikon } (Roland S. Kamzelak, Marbach)Das
                     Lexikon entsteht als \href{http://gams.uni-graz.at/o:konde.47}{Crowdsourcing-Projekt} im engen Umfeld der
                     Arbeitsgemeinschaft für germanistische Edition und stellt Erläuterungen von
                     Fachbegriffen bzw. Kurzbiografien von Persönlichkeiten der Editionswissenschaft
                     in deutschsprachigen Artikeln zur Verfügung. Die Artikel sind unter der Lizenz
                     CC-BY-SA verfügbar.}\item {\textbf{Lexicon of Scholarly Editing}  (\emph{European
                        Society for Textual Scholarship}, Antwerpen)Die einzelnen, oft
                     mehrsprachigen Artikel, die explizit nicht als Begriffserklärungen gedacht
                     sind, stellen Sammlungen von direkten, erklärenden Textzitaten zu den
                     jeweiligen Begriffen dar, mit denen die Fachdiskussion dazu dargestellt bzw.
                     angeregt werden soll. Die Inhalte sind nicht eigens für die Nachnutzung
                     lizenziert.}\item {\textbf{Parvum Lexicon Stemmatologicum}  (Caroline Macé und Philipp
                     Roelli, Helsinki)In englischsprachigen Artikeln werden Fachbegriffe und
                     Persönlichkeiten aus dem editionswissenschaftlichen Teilbereich der \href{http://gams.uni-graz.at/o:konde.172}{Stemmatologie} erklärt bzw.
                     dargestellt. Die Fachbegriffe werden in mehrere Sprachen übersetzt. Die Artikel
                     sind unter der Lizenz CC-BY-NC verfügbar.}\end{itemize}\subsection*{Literatur:}\begin{itemize}\item Dictionnaire en ligne de critique génétique. URL: \url{http://www.item.ens.fr/dictionnaire/}\item Edlex: Editionslexikon. URL: \url{https://edlex.de/}\item Lexicon of Scholarly Editing. URL: \url{https://lexiconse.uantwerpen.be/}\item Parvum Lexicon Stemmatologicum. URL: \url{https://wiki.helsinki.fi/display/stemmatology/Parvum+lexicon+stemmatologicum}\end{itemize}\subsection*{Verweise:}\href{https://gams.uni-graz.at/o:konde.46}{critique génétique}, \href{https://gams.uni-graz.at/o:konde.47}{Crowdsourcing}, \href{https://gams.uni-graz.at/o:konde.172}{Stemmatologie}\subsection*{Projekte:}\href{https://www.ag-edition.org}{Arbeitsgemeinschaft
                           für germanistische Edition}, \href{https://edlex.de/}{Edlex}, \href{https://wiki.helsinki.fi/display/stemmatology}{Parvum Lexicon Stemmatologicum}, \href{https://textualscholarship.eu}{European Society
                           for Textual Scholarship}, \href{http://www.item.ens.fr/dictionnaire/}{Dictionnaire de critique génétique}\subsection*{Themen:}Einführung, Digitale Editionswissenschaft\subsection*{Zitiervorschlag:}Klug, Helmut W. 2021. Nachschlagewerke zur Editionswissenschaft. In: KONDE Weißbuch. Hrsg. v. Helmut W. Klug unter Mitarbeit von Selina Galka und Elisabeth Steiner im HRSM Projekt "Kompetenznetzwerk Digitale Edition". URL: https://gams.uni-graz.at/o:konde.228\newpage\section*{Named Entity Recognition (NER)} \emph{Eder, Elisabeth; elisabeth.eder@aau.at }\\
        
    \emph{Named Entity Recognition} (NER) bezeichnet die Erkennung von Eigennamen (\emph{named entities}) in Texten sowie auch deren Klassifizierung in verschiedene Entitätstypen. In den meisten Fällen wird \emph{Named Entity Recognition} als \emph{Sequence Tagging-} oder \emph{Sequence Labeling}-Aufgabe aufgefasst, bei der jedem Token einer Sequenz eine bestimmte Kategorie bzw. ein bestimmter Entitätstyp zugewiesen wird (siehe auch \emph{\href{http://gams.uni-graz.at/o:konde.156}{Part-of-Speech-Tagging}}). Standardmäßig zählen Personen, Orte und Organisationen zu diesen Entitätstypen. Je nach Anwendungsfall werden aber auch andere Kategorien, wie zum Beispiel kommerzielle Produktnamen, Werktitel oder Fahrzeuge, berücksichtigt. (Jurafsky/Martin 2009, S. 761–768) Im Deutschen können auch Derivative (z  B. ‘österreichisch’) und partielle Entitäten, die nur einen Teil eines Token betreffen (z. B. KONDE-Beitrag), eine Rolle spielen und dementsprechende Unterkategorien bilden. (Benikova et al. 2014)\emph{Named Entity Recognition} ist in vielen \emph{Libraries} für NLP implementiert, z. B. \emph{\href{http://gams.uni-graz.at/o:konde.170}{spaCy}} oder \emph{Natural Language Toolkit} (nltk), und auch \emph{\href{http://gams.uni-graz.at/o:konde.212}{WebLicht}} bietet Tools dafür an.\\
            
        \subsection*{Literatur:}\begin{itemize}\item Akbik, Alan; Blythe, Duncan; Vollgraf, Roland: Contextual String Embeddings for Sequence Labeling. In: Proceedings of the 27th International Conference on Computational Linguistics COLING. Santa Fe, New Mexico, USA: 2018, S. 1638–1649.\item Benikova, Darina; Biemann, Chris; Reznicek, Marc: NoSta-D Named Entity Annotation for German: Guidelines and Dataset. In: Proceedings of 9th International Conference on Language Resources and Evaluation (LREC '14) LREC. Reykjavik, Iceland: 2014, S. 2524–2531.\item Benikova, Darina; Yimam, Seid Muhie; Santhanam, Prabhakaran; Biemann, Chris: GermaNER: Free Open German Named Entity Recognition Tool. In: Proceedings of the International Conference of the German Society for Computational Linguistics and Language Technology GSCL: 2015.\item Dernoncourt, Franck; Lee, Ji Young; Szolovits, Peter: NeuroNER: an easy-to-use program for named-entity recognition based on neural networks. In: Proceedings of the 2017 Conference on Empirical Methods in Natural Language Processing: System Demonstrations EMNLP. Copenhagen, Denmark: 2017, S. 97–102.\item Jurafsky, Daniel; Martin, James H.: Speech and Language Processing: An Introduction to Natural Language Processing, Computational Linguistics, and Speech Recognition. Upper Saddle River, New Jersey, United States of America: 2009.\item Riedl, Martin; Padó, Sebastian: A Named Entity Recognition Shootout for German. In: Proceedings of the 56th Annual Meeting of the Association for Computational Linguistics (Volume 2: Short Papers) ACL. Melbourne, Australia: 2018, S. 120–125.\end{itemize}\subsection*{Software:}\href{https://weblicht.sfs.uni-tuebingen.de/weblicht/}{weblicht}, \href{https://www.nltk.org/}{Natural Language Toolkit (nltk)}, \href{https://github.com/zalandoresearch/flair}{flair}, \href{https://www.ims.uni-stuttgart.de/forschung/ressourcen/werkzeuge/german-ner/}{German NER}, \href{https://github.com/tudarmstadt-lt/GermaNER}{GermaNER}, \href{http://neuroner.com/}{NeuroNER}\subsection*{Verweise:}\href{https://gams.uni-graz.at/o:konde.170}{spaCy}, \href{https://gams.uni-graz.at/o:konde.212}{WebLicht}, \href{https://gams.uni-graz.at/o:konde.156}{Part-of-Speech-Tagging}, \href{https://gams.uni-graz.at/o:konde.145}{NLP}, \href{https://gams.uni-graz.at/o:konde.176}{Tagger}\subsection*{Projekte:}\href{https://www.nltk.org}{Natural Language Toolkit}\subsection*{Themen:}Annotation und Modellierung, Natural Language Processing\subsection*{Zitiervorschlag:}Eder, Elisabeth. 2021. Named Entity Recognition (NER). In: KONDE Weißbuch. Hrsg. v. Helmut W. Klug unter Mitarbeit von Selina Galka und Elisabeth Steiner im HRSM Projekt "Kompetenznetzwerk Digitale Edition". URL: https://gams.uni-graz.at/o:konde.141\newpage\section*{Netzwerk} \emph{Geiger, Bernhard C.; geiger@ieee.org }\\
        
    Ein Netzwerk bzw. Graph ist ein mathematisches Objekt bestehend aus zumindest zwei Mengen: einer Menge an Knoten und einer Menge an Kanten, die Relationen zwischen den Knoten darstellen. In einem einfachen Netzwerk sind die Relationen zweistellig und Kanten verlaufen zwischen zwei Knoten (eine Kante zwischen Knoten A und Knoten B bedeutet, dass A und B in einer gewissen Relation zueinander stehen). Netzwerke können gerichtet (d. h., dass eine Relation von A nach B von einer Relation von B nach A unterschieden wird; die Kanten werden dann als Pfeile dargestellt) und/oder gewichtet (d. h., dass eine Relation von A nach B eine stärkere/schwächere Ausprägung haben kann, als eine Relation von B nach C) sein. In signierten Netzwerken können die Relationen negative Gewichte annehmen. \\
            
        Ein bipartites Netzwerk besteht aus zwei Typen von Knoten, wobei innerhalb jedes Typs von Knoten keine Kanten bestehen. Man nehme zum Beispiel die Knotenmengen ‘Schriftsteller’ und ‘Roman’ sowie die Relation ‘ist Autor von’: In diesem Fall bestehen gerichtete Kanten nur von der Menge der Schriftsteller zur Menge der Romane. In einem multipartiten Netzwerk gibt es mehrere solcher Typen von Knoten. \\
            
        Relevant ist auch der Multigraph, in dem mehrere Typen von Relationen dargestellt werden. Fügen wir zu obigem Beispiel noch die Relation ‘hat gelesen’ hinzu, dann erhalten wir einen bipartiten, gerichteten Multigraph, der darstellt, welche Schriftsteller welche Romane gelesen bzw. welche sie verfasst haben.\\
            
        \subsection*{Literatur:}\begin{itemize}\item Thurner, Stefan; Hanel, Rudolf; Klimek, Peter: Introduction to the Theory of Complex Systems. Oxford, New York: 2018.\item Trilcke, Peer: Social Network Analysis (SNA) als  Methode einer  extempirischen Literaturwissenschaft. In: Empirie in der Literaturwissenschaft. Münster: 2013, S. 201–247.\end{itemize}\subsection*{Software:}\href{https://d3js.org}{D3js}, \href{https://gephi.org/}{Gephi}, \href{https://nodegoat.net/}{Node Goat}, \href{https://public.tableau.com/s/}{Tableau}\subsection*{Verweise:}\href{https://gams.uni-graz.at/o:konde.73}{Dramennetzwerk}, \href{https://gams.uni-graz.at/o:konde.74}{Dramennetzwerkanalyse}, \href{https://gams.uni-graz.at/o:konde.54}{Datenvisualisierung}, \href{https://gams.uni-graz.at/o:konde.210}{Visualisierungstools}\subsection*{Themen:}Einführung, Datenanalyse\subsection*{Zitiervorschlag:}Geiger, Bernhard C. 2021. Netzwerk. In: KONDE Weißbuch. Hrsg. v. Helmut W. Klug unter Mitarbeit von Selina Galka und Elisabeth Steiner im HRSM Projekt "Kompetenznetzwerk Digitale Edition". URL: https://gams.uni-graz.at/o:konde.144\newpage\section*{Normalisierung} \emph{Galka, Selina; selina.galka@uni-graz.at }\\
        
    Unter Normalisierung versteht man die Anpassung von historischen oder persönlichen
                  Schreibweisen an eine (standardisierte) Norm. Je nach Zweck und Zielgruppe der
                  Edition können mehr oder weniger tiefgreifende Normalisierungsmaßnahmen getroffen
                  werden. \\
            
        In der Editionsgeschichte haben sich unterschiedliche Ansätze herausgebildet:
                  Während Karl Lachmann, Wegbereiter der wissenschaftlichen Textkritik und
                  Editionswissenschaft, von einer übergeordneten, normierten Literatursprache
                  ausging, an die der Text angepasst werden muss, gibt es daneben den diplomatischen
                  Abdruck (\href{http://gams.uni-graz.at/o:konde.65}{diplomatische Edition}),
                  der versucht, den historischen Text ohne jegliche Normalisierung so genau wie
                  möglich abzubilden. \\
            
        Die Normalisierungsmaßnahmen können beispielsweise historische Klein- und
                  Großschreibung, Interpunktion, Konsonantenverdoppelung oder Getrennt- und
                  Zusammenschreibung betreffen. Normalisierungen können explizit gekennzeichnet oder
                  stillschweigend vorgenommen werden; der Umgang mit Normalisierungen sollte jedoch
                  genau in den \href{http://gams.uni-graz.at/o:konde.198}{Transkriptionsrichtlinien} dokumentiert und erklärt werden. \\
            
        Die Normalisierung bei \href{http://gams.uni-graz.at/o:konde.59}{Digitalen
                     Editionen} spielt sowohl bei der \href{http://gams.uni-graz.at/o:konde.137}{Modellierung}, Auszeichnung und Anreicherung der
                  Texte mit \href{http://gams.uni-graz.at/o:konde.215}{XML}/\href{http://gams.uni-graz.at/o:konde.178}{TEI} eine Rolle als auch bei der
                  Webrepräsentation. Normalisierungen können mit dementsprechenden TEI-Elementen
                  kodiert werden, beispielsweise mit <choice>:\\
            
        \begin{verbatim}<choice>
    <orig>parceque</orig>
    <reg>parce que</reg>
</choice>\end{verbatim}Somit können auch die unterschiedlichen Varianten, wenn gewollt, in der
                  Webrepräsentation der \href{http://gams.uni-graz.at/o:konde.59}{Digitalen
                     Edition} angezeigt werden.\\
            
        \subsection*{Literatur:}\begin{itemize}\item Oellers, Norbert: Angleichung, Normalisierung, Restitution. Die Edition
                              hybrida als Schicksal der deutschen Klassiker? In: Oellers, Norbert; Steinecke, Hartmut: Probleme
                              neugermanistischer Edition. Zeitschrift für deutsche Philologie
                              101: 1982.\item Bein, Thomas (Hrsg.): Roundtable. Normalisierung und Modernisierung der
                              historischen Graphie. Mit Beiträgen von: Thomas Bein (Aachen), Kurt
                              Gärtner (Trier), Andrea Hofmeister-Winter (Graz), Ulrike Leuschner
                              (Darmstadt), Wolfgang Lukas (Wuppertal), Hans-Gert Roloff (Berlin),
                              Claudia Schumacher (Aachen), Winfried Woesler (Osnabrück,
                              Koordinator). In: Vom Nutzen der Editionen. Zur Bedeutung moderner
                              Editorik für die Erforschung von Literatur- und
                              Kulturgeschichte. Berlin, Boston: 2015, S. 419–460.\item Woesler, Winfried: Die Normalisierung historischer Orthographie als
                              wissenschaftliche Aufgabe. In: Zeitschrift für deutsche Philologie 105,
                              Sonderheft: 1986, S. 69–83.\end{itemize}\subsection*{Verweise:}\href{https://gams.uni-graz.at/o:konde.65}{diplomatische Edition}, \href{https://gams.uni-graz.at/o:konde.126}{Markup}, \href{https://gams.uni-graz.at/o:konde.137}{Modellierung}, \href{https://gams.uni-graz.at/o:konde.195}{Textmodellierung}, \href{https://gams.uni-graz.at/o:konde.178}{TEI}, \href{https://gams.uni-graz.at/o:konde.215}{XML}, \href{https://gams.uni-graz.at/o:konde.59}{Digitale Edition}, \href{https://gams.uni-graz.at/o:konde.198}{Transkriptionsrichtlinien}, \href{https://gams.uni-graz.at/o:konde.50}{Datenmodell "hyperdiplomatische
                           Transkription"}\subsection*{Themen:}Einführung, Digitale Editionswissenschaft\subsection*{Lexika}\begin{itemize}\item \href{https://wiki.helsinki.fi/display/stemmatology/Normalisation}{Parvum Lexicon Stemmatologicum}\end{itemize}\subsection*{Zitiervorschlag:}Galka, Selina. 2021. Normalisierung. In: KONDE Weißbuch. Hrsg. v. Helmut W. Klug unter Mitarbeit von Selina Galka und Elisabeth Steiner im HRSM Projekt "Kompetenznetzwerk Digitale Edition". URL: https://gams.uni-graz.at/o:konde.146\newpage\section*{Normdaten} \emph{Steiner, Christian; christian.steiner@uni-graz.at / Fritze, Christiane; christiane.fritze@onb.ac.at }\\
        
    Normdaten (im Englischen \emph{authority control} oder \emph{authority files}) sind kontrollierte Vokabulare für bestimmte Domänen. In einer Normdatei (Datenbank oder Katalog) hat eine Entität einen eindeutigen Eintrag, der mit weiteren erläuternden oder spezifizierenden Informationen versehen sein kann. \\
            
        Ursprünglich waren Normdaten hauptsächlich für den Bibliotheksbereich relevant. Spätestens mit dem Aufkommen des \emph{\href{http://gams.uni-graz.at/o:konde.167}{Semantic Web}} gewann die Möglichkeit der Kontrolle von Datenbeständen durch Normdaten jedoch an Bedeutung für alle Bereiche, die Informationen austauschen wollen und müssen. Große Fortschritte wurden mit der Einrichtung des \emph{Virtual Integrated Authority File} (\href{http://gams.uni-graz.at/o:konde.111}{VIAF}) gemacht, welches eine Zusammenführung von verschiedenen Normdateien für Personen- und Organisationsnamen erlaubt. Das Projekt wurde von der Deutschen Nationalbibliothek und der \emph{Library of Congress} initiiert, mit dem Ziel der Vernetzung mehrerer Normdateien nationaler Behörden wie etwa der \href{http://gams.uni-graz.at/o:konde.109}{GND} (Gemeinsame Normdatei) der Deutschen Nationalbibliothek. Für den Bereich der geographischen Verortung wird häufig \emph{\href{http://gams.uni-graz.at/o:konde.107}{GeoNames}} und der \emph{\href{http://gams.uni-graz.at/o:konde.108}{Getty}}\emph{-Thesaurus of Geographic Names} verwendet. Auch Wissensdatenbanken wie \emph{\href{http://gams.uni-graz.at/o:konde.112}{Wikidata}} können als Normdatei fungieren, auch wenn ihre Funktionalität darüber hinaus geht. \\
            
        \subsection*{Literatur:}\begin{itemize}\item Bates, Marcia J: Understanding Information Retrieval Systems: Management, Types, and Standards Understanding Information Retrieval Systems: 2011.\item Getty Vocabularies (Getty Research Institute). URL: \url{https://www.getty.edu/research/tools/vocabularies/}\item Siegfried, Susan: An Analysis of Search Terminology Used by Humanities Scholars: The Getty Online Searching Project Report Number 1 An Analysis of Search Terminology Used by Humanities Scholars. In: The Library Quarterly 63: 1993, S. 1-39.\item Stadler, Peter: Normdateien in Editionen. In: editio 26: 2012, S. 174–183.\end{itemize}\subsection*{Software:}\href{https://viaf.org}{VIAF}, \href{geonames.org}{Geonames}, \href{http://www.dnb.de/DE/Standardisierung/GND/gnd_node.html}{GND}, \href{http://commons.pelagios.org/}{Pelagios commons}, \href{https://www.wikidata.org/wiki/Wikidata:Main_Page}{Wikidata}\subsection*{Verweise:}\href{https://gams.uni-graz.at/o:konde.167}{Semantic Web}, \href{https://gams.uni-graz.at/o:konde.109}{GND}, \href{https://gams.uni-graz.at/o:konde.107}{GeoNames}, \href{https://gams.uni-graz.at/o:konde.111}{VIAF}, \href{https://gams.uni-graz.at/o:konde.112}{Wikidata}, \href{https://gams.uni-graz.at/o:konde.108}{Getty}, \href{https://gams.uni-graz.at/o:konde.8}{Linked Open Data}\subsection*{Projekte:}\href{https://weber-gesamtausgabe.de/de/Index}{Carl Maria von Weber: Gesamtausgabe}, \href{https://medea.hypotheses.org}{MEDEA. Modelling semantically Enriched Digital Edition of Accounts}\subsection*{Themen:}Einführung, Annotation und Modellierung\subsection*{Lexika}\begin{itemize}\item \href{https://edlex.de/index.php?title=Normdaten}{Edlex: Editionslexikon}\end{itemize}\subsection*{Zitiervorschlag:}Steiner, Christian; Fritze, Christiane. 2021. Normdaten. In: KONDE Weißbuch. Hrsg. v. Helmut W. Klug unter Mitarbeit von Selina Galka und Elisabeth Steiner im HRSM Projekt "Kompetenznetzwerk Digitale Edition". URL: https://gams.uni-graz.at/o:konde.147\newpage\section*{OAIS RM} \emph{Steiner, Elisabeth; elisabeth.steiner@uni-graz.at }\\
        
    Das OAIS\emph{(Open Archival Information System)}-Referenzmodell wurde vom \emph{Consultative Committee for Space Data Systems }(CCSDS) entwickelt und weiterführend auch in einen ISO-Standard überführt. Es besteht aus Empfehlungen, welche die verlässliche und langfristige Archivierung von digitaler Information zum Ziel haben. Das OAIS RM ist ein abstraktes Datenmodell, das keinerlei Aussagen über die konkrete technische Umsetzung der Prinzipien enthält. Verschiedene Systemarchitekturen können diese Herausforderungen unterschiedlich lösen, aber trotzdem mit den Prinzipien des Referenzmodells im Einklang stehen ( \emph{OAIS Compliance}). \\
            
        Zentrale Bestandteile des OAIS RM sind die \emph{Information Package}s. In diesen konzeptuellen Containern werden Inhalte, \href{http://gams.uni-graz.at/o:konde.35}{Metadaten} und Identifizierungsinformationen zusammengefasst. Je nach Funktion unterscheidet man SIP\emph{ (Submission Information Package)}, AIP \emph{(Archival Information Package)} und DIP\emph{ (Dissemination Information Package)}. Das SIP wird vom Produzenten der Information an das Archiv übergeben. Durch den Vorgang des Ingest wird das SIP in ein AIP umgewandelt, das innerhalb des Archivs gespeichert wird. Dabei werden beispielsweise notwendige Informationen hinzugefügt oder Dateiformate geändert und eine Qualitätskontrolle durchgeführt, sodass das Material zur Langzeitarchivierung geeignet ist. Dem AIP wird ein \href{http://gams.uni-graz.at/o:konde.12}{PID} (\emph{Persistent Identifier}) zugewiesen. Aus dem AIP wird nach Anfrage durch die/den Benutzer/in wiederum ein DIP generiert, das zur Ausgabe der im AIP gespeicherten Information dient. Innerhalb des Archivs muss eine ständige Planung der \href{http://gams.uni-graz.at/o:konde.6}{Langzeitarchivierung} und der damit verbundenen Aktivitäten gegeben sein. \\
            
        \emph{OAIS Compliance} bildet die Grundlage für die Arbeit jedes Langzeitrepositoriums und auch die Basis für viele Zertifizierungen als \emph{\href{http://gams.uni-graz.at/o:konde.13}{trusted digital repository}} (bspw. in der Weiterentwicklung ISO 16363:2012).\\
            
        \subsection*{Literatur:}\begin{itemize}\item ISO 14721:2012. Space data and information transfer systems — Open archival information system (OAIS) — Reference model. URL: \url{https://www.iso.org/standard/57284.html}\item ISO 16363:2012. Space data and information transfer systems — Audit and certification of trustworthy digital repositories. URL: \url{https://www.iso.org/standard/56510.html}\item The Consultative Committee for Space Data Systems: Reference Model for an Open Archival Information System (OAIS): 2012, URL: \url{https://public.ccsds.org/pubs/650x0m2.pdf}.\end{itemize}\subsection*{Verweise:}\href{https://gams.uni-graz.at/o:konde.13}{Trusted Repository & Zertifizierungen}, \href{https://gams.uni-graz.at/o:konde.7}{FAIR-Prinzipien}, \href{https://gams.uni-graz.at/o:konde.6}{Digitale Nachhaltigkeit}, \href{https://gams.uni-graz.at/o:konde.4}{nestor}\subsection*{Themen:}Archivierung\subsection*{Zitiervorschlag:}Steiner, Elisabeth. 2021. OAIS RM. In: KONDE Weißbuch. Hrsg. v. Helmut W. Klug unter Mitarbeit von Selina Galka und Elisabeth Steiner im HRSM Projekt "Kompetenznetzwerk Digitale Edition". URL: https://gams.uni-graz.at/o:konde.11\newpage\section*{OCR} \emph{Fritze, Christiane; christiane.fritze@onb.ac.at / Mühlberger, Günter;
                  guenter.muehlberger@uibk.ac.at }\\
        
    OCR steht für \emph{Optical Character Recognition}, also für
                  ‘optische Zeichenerkennung’. Eine OCR-Software liest ein \href{http://gams.uni-graz.at/o:konde.37}{Bilddigitalisat} ein und erkennt darauf vorhandenen
                  Text, sodass dieser nach der OCR-Erkennung durchsuchbar ist (STRG F). Bevor die
                  OCR-Software mit dem Erkennungsdurchlauf beginnt, wird das zu analysierende
                  Bilddigitalisat binarisiert, d. h. aus einem Farbscan wird ein bitonales Bild
                  erstellt. Anschließend wird das Layout des Bilddigitalisats erfasst und in
                  Erkennungszonen eingeteilt. Für die Erkennung werden die einzelnen Pixelcluster
                  mit in der Software verankerten Pixelmustern abgeglichen und der wahrscheinlichste
                  Wert ausgewählt. \\
            
        Der Leistungsumfang von Softwareangeboten ist unterschiedlich: Je nach Software
                  und Ausgabeformat kann die Erkennungswahrscheinlichkeit in den \href{http://gams.uni-graz.at/o:konde.25}{Metadaten} vermerkt werden. Je nach
                  Software kann für die Erkennung ein Wörterbuch für bestimmte Sprachen oder
                  diachrone Sprachstufen hinterlegt sein. Je nach Bedarf können die zu
                  analysierenden Zonen vorab markiert werden bzw. ist eine händische Fehlerkorrektur
                  des erkannten Textes möglich.\\
            
        Für die Weiterverarbeitung stehen verschiedene Ausgabeformate bereit: einfacher
                  Text, \href{http://gams.uni-graz.at/o:konde.154}{pageXML}, das \href{http://gams.uni-graz.at/o:konde.215}{XML}-basierte Format ALTO oder hOCR
                  mit Layoutinformationen.\\
            
        \emph{Optical Character Recognition} gehört zu den bekanntesten
                  Anwendungen der Mustererkennung und wurde bereits in den 70er-Jahren von Raymond
                  Kurzweil als Standardverfahren entwickelt. Zuerst nur für spezielle Schriften
                  geeignet, erweiterten sich die Einsatzmöglichkeiten Schritt für Schritt. In den
                  späten 90er-Jahren erreichte die Erkennungsgenauigkeit einen ersten Höhepunkt:
                  Geschäftsdokumente wie Briefe, Rechnungen, aber auch Bücher und moderne Zeitungen
                  konnten nun zuverlässig erkannt werden. Doch Schriften aus dem 15. bis zum 19.
                  Jahrhundert konnten trotz intensiver Bemühungen auch weiterhin nicht
                  zufriedenstellend gelesen werden. Ein bekanntes Beispiel hierfür sind die anfangs
                  von \emph{Google} gelieferten Volltexte im Zusammenhang mit dem
                     \emph{Google Books}-Projekt.\\
            
        Die Revolution in der Erkennung von herausfordernden, nicht-standardisierten
                  Schriften fand erst vor wenigen Jahren statt und beruht auf neuen Methoden des
                  Maschinenlernens, die zuvor schon bei der automatisierten Sprach- und
                  Handschriftenerkennung entwickelt wurden. Statt eine Zeile in Wörter oder gar
                  einzelne Buchstaben zu zerlegen, wie dies bei traditioneller OCR der Fall ist,
                  wird nunmehr das Bild einer vollständigen Zeile einem neuronalen Netz zusammen mit
                  dem dazugehörigen Text als Trainingsmaterial vorgesetzt. Das Netz lernt dann
                  selbständig den Zusammenhang zwischen Bild und Text. Wie die nachfolgenden
                  Beispiele zeigen werden, können mit der nunmehr dominierenden Methode der
                  Texterkennung sowohl historische Bücher, als auch beliebige Handschriften (\href{http://gams.uni-graz.at/o:konde.224}{HTR}) mit einer erstaunlichen
                  Genauigkeit erkannt werden. \\
            
        Für unsere Beispiele verwenden wir Zahlen, die wir im Rahmen der Arbeit mit der
                  Transkribus-Plattform gemacht haben. Dort sind zwei OCR-\emph{Engines} im Einsatz, einmal die vom CITlab-Team der Universität Rostock
                  entwickelte HTR+ und zum andern die von der PRHLT-Gruppe der Universität Valencia
                  entwickelte \emph{PyLaia Engine}. Letztere ist auch als
                  Open Source-Software verfügbar. Für unsere Experimente haben wir die HTR+-\emph{Engine} verwendet. Daneben gibt es noch eine Reihe anderer
                  OCR-Anwendungen, die bekanntesten darunter sind wahrscheinlich \emph{Tesseract} von \emph{Google} oder \emph{Ocropy} von Thomas Breuel, zusätzlich haben auch \emph{Google,
                     Amazon, Microsoft} und \emph{Facebook} jeweils eigene
                     OCR-\emph{Engines} entwickelt und bieten diese über ihre
                  Plattformen an.\\
            
        Das erste Modell, das wir kurz vorstellen wollen, wurde im Rahmen des \emph{NewsEye}-Projekts erstellt und 2019 auf der \emph{Transkribus}-Plattform veröffentlicht. Ein Update ist in Vorbereitung. Das
                  Modell beruht auf Trainingsdaten österreichischer Zeitungen des späten 19. und
                  frühen 20. Jahrhunderts im Umfang von 442.141 Wörtern. Das Ergebnis am
                  Validierungsset beträgt 1,66 Prozent \emph{Character Error Rate}.
                  Diese Messung beinhaltet auch Satzzeichen, Zahlen in Tabellen, Werbeeinschaltungen
                  und ähnliches, liefert also ein eher pessimistisches Ergebnis. Für laufenden Text
                  werden hingegen deutlich bessere Fehlerraten von unter 1% erreicht. Wie
                  leistungsfähig das Netz ist, kann im folgenden Beispiel gezeigt werden. Es handelt
                  sich um einen Ausschnitt aus der \emph{Rheinischen Volksstimme}
                  aus dem frühen 20. Jahrhundert. Die Bildqualität ist aufgrund des schlechten
                  Drucks (durchscheinende Buchstaben) sowie der Tatsache, dass hier ein Mikrofilm
                  der Digitalisierung zugrunde lag, stark eingeschränkt. \\
            
        Abbildung: Beispiel Texterkennung - Vergleich Abbyy
                           FineReader - Google OCR - Transrkibus\\
            
        Noch bessere Ergebnisse lassen sich mit dem Modell \emph{Noscemus}
                  erzielen, das im Rahmen eines ERC-Grants am Institut für Neulatein der Universität
                  Innsbruck erstellt und in \emph{Transkribus} veröffentlicht wurde.
                  Es zeigt eine durchschnittliche Fehlerquote am Validierungsset von 0,86 Prozent
                  CER auf. Hier liegen dem Modell neulateinische Schriften des 16. bis 18.
                  Jahrhunderts im Umfang von 307.329 Wörtern zugrunde. Zudem wurde diesem Netz auch
                  beigebracht, die typischen Abkürzungen wie sie im Neulateinischen gebräuchlich
                  sind, gleich bei der Erkennung aufzulösen.\\
            
        \\
            
        Abbildung: Beispiel Texterkennung - Neulateinisches Dokument\\
            
        \subsection*{Literatur:}\begin{itemize}\item READ-COOP SCE: Public Models in Transkribus: 2020. URL: \url{https://readcoop.eu/transkribus/public-models/}.\end{itemize}\subsection*{Software:}\href{https://www.abbyy.com/de-de/}{Abby
                           Finereader}, \href{https://launchpad.net/cuneiform-linux}{Cuneiform}, \href{https://github.com/tesseract-ocr/}{Tesseract}, \href{https://github.com/tmbarchive/ocropy}{The
                           OCRopus OCR System and Related Software}, \href{https://transkribus.eu/Transkribus/}{Transkribus}, \href{https://readcoop.eu/wp-content/uploads/2018/12/D7.9_HTR_NN_final.pdf}{HTR+}, \href{https://github.com/jpuigcerver/PyLaia}{PyLaia}, \href{https://github.com/OCR4all}{OCR4all}, \href{https://github.com/OCR-D}{OCR-d}, \href{http://wlt.synat.pcss.pl/}{Virtual
                           Transcription Laboratory}\subsection*{Projekte:}\href{https://books.google.at}{Google Books}, \href{https://readcoop.eu/transkribus/}{Transkribus}\subsection*{Verweise:}\href{https://gams.uni-graz.at/o:konde.60}{Digitalisierung}, \href{https://gams.uni-graz.at/o:konde.224}{HTR}, \href{https://gams.uni-graz.at/o:konde.197}{Transkription}, \href{https://gams.uni-graz.at/o:konde.199}{Transkriptionswerkzeuge}\subsection*{Themen:}Digitalisierung\subsection*{Lexika}\begin{itemize}\item \href{https://edlex.de/index.php?title=Optical_Character_Recognition_(OCR)}{Edlex: Editionslexikon}\item \href{https://lexiconse.uantwerpen.be/index.php/lexicon/ocr/}{Lexicon of Scholarly Editing}\end{itemize}\subsection*{Zitiervorschlag:}Fritze, Christiane; Mühlberger, Günter. 2021. OCR. In: KONDE Weißbuch. Hrsg. v. Helmut W. Klug unter Mitarbeit von Selina Galka und Elisabeth Steiner im HRSM Projekt "Kompetenznetzwerk Digitale Edition". URL: https://gams.uni-graz.at/o:konde.149\newpage\section*{ODD} \emph{Bleier, Roman; roman.bleier@uni-graz.at }\\
        
    Der Standard der \emph{Text Encoding Initiative} (\href{http://gams.uni-graz.at/o:konde.178}{TEI}) ist in der \href{http://gams.uni-graz.at/o:konde.166}{Schema}-Metasprache ODD (\emph{One Document Does it All}) definiert. Ein ODD ist ein
                  TEI-Dokument, das Schema-Fragmente, Dokumentationstexte, Codebeispiele und
                  Verweise auf die TEI \emph{Guidelines} in einem Dokument enthält.
                  Für Dokumentationszwecke gibt es in der TEI ein eigenes Modul \emph{tagdocs — Documentation Elements}, in dem Elemente und Attribute definiert
                  sind, die für das Schreiben von ODDs verwendet werden. (TEI: 22
                     Documentation Elements)\\
            
        ODD ist von zentraler Bedeutung für die Entwicklung lokaler, projektspezifischer
                  Schemata. Da die TEI über 500 Elemente und Attribute definiert hat und nicht alle
                  für ein Editionsprojekt relevant sind, ist es nötig, das verwendete Subset und
                  mögliche Änderungen zum Gesamt-Schema durch ein ODD zu dokumentieren. Mit \emph{Roma} stellt die TEI-Community ein Werkzeug für die Anpassung
                  von ODDs und zur Generierung von Schemata (\href{http://gams.uni-graz.at/o:konde.163}{RELAX NG}, XML-Schema, DTD) und Dokumentation (als
                  HTML oder PDF) zur Verfügung. (Mittelbach et al. 2019)\\
            
        \subsection*{Literatur:}\begin{itemize}\item Burnard, Lou: Customizing the TEI - OpenEdition Press. In: What is the Text Encoding Initiative?: 2014.\item Roma: generating customizations for the TEI. URL: \url{https://roma2.tei-c.org/}\item TEI. 22 Documentation Elements. URL: \url{https://www.tei-c.org/release/doc/tei-p5-doc/en/html/TD.html}\item TEI. 23 Using the TEI. URL: \url{https://tei-c.org/release/doc/tei-p5-doc/en/html/USE.html}\item Getting Started with P5 ODDs. URL: \url{https://tei-c.org/guidelines/customization/getting-started-with-p5-odds/}\end{itemize}\subsection*{Verweise:}\href{https://gams.uni-graz.at/o:konde.166}{Schema}, \href{https://gams.uni-graz.at/o:konde.163}{RelaxNG}, \href{https://gams.uni-graz.at/o:konde.178}{TEI}\subsection*{Software:}\href{https://roma.tei-c.org}{Roma: generating
                           customizations for the TEI}\subsection*{Themen:}Annotation und Modellierung, Digitale Editionswissenschaft\subsection*{Zitiervorschlag:}Bleier, Roman. 2021. ODD. In: KONDE Weißbuch. Hrsg. v. Helmut W. Klug unter Mitarbeit von Selina Galka und Elisabeth Steiner im HRSM Projekt "Kompetenznetzwerk Digitale Edition". URL: https://gams.uni-graz.at/o:konde.150\newpage\section*{Office-Formate} \emph{Stigler, Johannes; johannes.stigler@uni-graz.at}\\
        
    \emph{Microsoft Office} und \emph{Open} bzw. \emph{LibreOffice}-Applikationen verwenden schon seit geraumer Zeit \href{http://gams.uni-graz.at/o:konde.215}{XML}-basierte Datenformate zur Speicherung von Text- und Arbeitsblattdaten. Diese Produkte können daher sehr einfach in Transkriptions- und Editionsworkflows eingesetzt werden. Auch dann, wenn am anderen Ende für eine \href{http://gams.uni-graz.at/o:konde.59}{Digitale Edition} ein Dokument gemäß den Konventionen der \emph{Text Encoding Initiative} (\href{http://gams.uni-graz.at/o:konde.178}{TEI}) stehen soll. So ist es z. B. möglich, über eine intelligente Verwendung von Formatvorlagen durch Markieren mit der Maus semantische \href{http://gams.uni-graz.at/o:konde.17}{Annotationen} in den Text einzubringen.\\
            
        Sowohl DOCX als auch ODT sind ISO-zertifizierte Container-Formate auf Basis von XML und können daher von einschlägigen Tools direkt weiterverarbeitet werden. Dateien dieses Formates sind eigentlich ZIP-Archive, die mehrere Dateien mit Text und Formatierungsinformationen in menschenlesbarer Notation enthalten. \emph{Oxygen}, ein in der Community weit verbreiteter XML-Editor, etwa kann beide Datenformate direkt einlesen. Genauso ist es möglich, Dateien dieser Formate über ein Webservice der TEI-Community (\emph{OxGarage}) direkt ins TEI-Format zu konvertieren. Aufschlussreiche Informationen zu beiden Office-Formaten finden sich auf einer Seite der\emph{ Library of Congress}.\\
            
        \subsection*{Literatur:}\begin{itemize}\item Sustainability of Digital Formats: Planning for Library of Congress Collections. DOCX Transitional (Office Open XML), ISO 29500:2008-2016, ECMA-376, Editions 1-5 Sustainability of Digital Formats. URL: \url{https://www.loc.gov/preservation/digital/formats/fdd/fdd000397.shtml}\item OpenDocument. URL: \url{https://en.wikipedia.org/wiki/OpenDocument}\item Office Open XML. URL: \url{https://de.wikipedia.org/wiki/Office_Open_XML}\end{itemize}\subsection*{Software:}\href{https://oxgarage.tei-c.org/}{OxGarage}, \href{http://oxygenxml.com/}{Oxygen}\subsection*{Verweise:}\href{https://gams.uni-graz.at/archive/objects/context:konde/methods/sdef:Context/get?mode=workflow}{Workflow Digitalisierung}\subsection*{Themen:}Annotation und Modellierung, Digitale Editionswissenschaft\subsection*{Zitiervorschlag:}Stigler, Johannes. 2021. Office-Formate. In: KONDE Weißbuch. Hrsg. v. Helmut W. Klug unter Mitarbeit von Selina Galka und Elisabeth Steiner im HRSM Projekt "Kompetenznetzwerk Digitale Edition". URL: https://gams.uni-graz.at/o:konde.120\newpage\section*{Ontologie} \emph{Galka, Selina; selina.galka@uni-graz.at }\\
        
    In der Informationswissenschaft und den Digital Humanities handelt es sich bei Ontologien um eine Form der Wissensrepräsentation. Man versucht Wissen aus der Welt oder thematisch kleiner gefassten Bereichen zu modellieren, abzubilden und zu formalisieren. Durch die Formalisierung wird es möglich, Daten und Wissen weiterzuverwenden, auszutauschen und sogar logische Schlussfolgerungen zu ziehen. (Rehbein 2017, S. 162) Ontologien stehen auch sehr stark in Zusammenhang mit dem \emph{\href{http://gams.uni-graz.at/o:konde.167}{Semantic Web}}, da sie dabei helfen, Informationen explizit  und weiterverarbeitbar zu machen.\\
            
        Eine Ontologie bestimmt mittels grundlegender Bestandteile ein Begriffssystem; bei diesen Bestandteilen handelt es sich um Klassen (es werden Objekte zusammengefasst, die Eigenschaften miteinander teilen, wie beispielsweise Personen oder Orte), Attribute und Eigenschaften (Charakterisierung der Objekte, wie z. B. die möglichen Geschlechter einer Person) und Relationen. Außerdem können auch Bedingungen festgelegt werden. Mit all diesen Bestandteilen können Aussagen formuliert werden, die gemeinsam ein Regelwerk bilden – somit sind Ontologien im Gegensatz zu anderen Formen der Wissensrepräsentation, wie \href{http://gams.uni-graz.at/o:konde.126}{Markup}, äußerst expressiv. (Rehbein 2017, S. 164ff.)\\
            
        Es kann zwischen \emph{Top-Level}-Ontologien, die sehr allgemeine Begriffe wie Zeit und Raum beschreiben, \emph{Domain}-Ontologien, die einen großen Gegenstandsbereich beschreiben, \emph{Task}-Ontologien und \emph{Application}-Ontologien unterschieden werden. Ontologien können aber auch nach ihrem Formalisierungsgrad charakterisiert werden, z. B. nach \emph{Lightweight Ontologies} oder \emph{Heavyweight Ontologies}. (Rehbein 2017, S. 165ff.)\\
            
        Für die Formalisierung von Ontologien gibt es mehrere Techniken, wie z. B. \href{http://gams.uni-graz.at/o:konde.131}{RDF, OWL und RDFS}. Die Erstellung von Ontologien kann sehr komplex werden, im Idealfall kann und sollte aber auf bereits bestehende Ressourcen zurückgegriffen werden, wie auf \href{http://gams.uni-graz.at/o:konde.133}{CIDOC CRM}, \href{http://gams.uni-graz.at/o:konde.132}{SKOS} oder andere bestehende kontrollierte Vokabularien.\\
            
        Semantische Technologien bekommen im Bezug auf \href{http://gams.uni-graz.at/o:konde.59}{Digitale Editionen} immer mehr Bedeutung – Jörg Wettlaufer argumentierte 2018, dass textbasierte Digitale Editionen beispielsweise auf den etablierten \href{http://gams.uni-graz.at/o:konde.178}{TEI}-Standard zurückgreifen würden, dies aber nicht zu einer Interoperabilität zwischen den unterschiedlichen Editionen führe, sondern eher zu einer immer stärkeren Auffächerung des \emph{Markups}. (Wettlaufer 2018) Dadurch komme es zu einem stärkeren Bedarf an Lösungen zur Vernetzung und Nachnutzung von Digitalen Editionen, “sowie auch zur Erschließung über eine maschinenlesbare Semantik, die über \emph{Linked Open Data }(\href{http://gams.uni-graz.at/o:konde.8}{LOD}), \href{http://gams.uni-graz.at/o:konde.147}{Normdaten} und andere \href{http://gams.uni-graz.at/o:konde.25}{Metadaten} jedweder Form Verknüpfungen herzustellen in der Lage ist.” (Wettlaufer 2018)\\
            
        \subsection*{Literatur:}\begin{itemize}\item Münnich, Stefan: Ontologien als semantische Zündstufe für die digitale Musikwissenschaft? In: Bibliothek Forschung und Praxis 42: 2018, S. 184–193.\item Jannidis, Fotis; Kohle, Hubertus: Digital Humanities. Eine Einführung. Mit Abbildungen und Grafiken Digital Humanities. Hrsg. von  und Malte Rehbein. Stuttgart: 2017.\item Romanello - Berti - Boschetti - Babeu - Crane, Matteo - Monica - Federico - Alison - Gregory: Rethinking critical editions of fragmentary texts by ontologies. In: Proceedings of 13th International Conference on Electronic Publishing: Rethinking Electronic Publishing: Innovation in Communication Paradigms and Technologies: 2009, S. 155–174.\item Wettlaufer, Jörg: Der nächste Schritt? Semantic Web und digitale Editionen. In: Digitale Metamorphose: Digital Humanities und Editionswissenschaft: 2018.\end{itemize}\subsection*{Verweise:}\href{https://gams.uni-graz.at/o:konde.5}{DHA-Ontologie}, \href{https://gams.uni-graz.at/o:konde.133}{Metadaten Schemata für LZA: CIDOC CRM}, \href{https://gams.uni-graz.at/o:konde.131}{Metadaten Schemata für LZA: RDF, RDFS, OWL u.a.}, \href{https://gams.uni-graz.at/o:konde.132}{Metadaten Schemata für LZA: SKOS}, \href{https://gams.uni-graz.at/o:konde.167}{Semantic Web}, \href{https://gams.uni-graz.at/o:konde.168}{Semantic Web TEchnologien}, \href{https://gams.uni-graz.at/o:konde.147}{Normdaten}, \href{https://gams.uni-graz.at/o:konde.109}{Kontrollierte Vokabularien: GND}, \href{https://gams.uni-graz.at/o:konde.107}{Kontrollierte Vokabularien: GeoNames}, \href{https://gams.uni-graz.at/o:konde.111}{Kontrollierte Vokabularien: VIAF}, \href{https://gams.uni-graz.at/o:konde.108}{Kontrollierte Vokabularien: Getty}, \href{https://gams.uni-graz.at/o:konde.112}{Kontrollierte Vokabularien: Wikidata}\subsection*{Projekte:}\href{https://medea.hypotheses.org}{MEDEA. Modelling semantically Enriched Digital Edition of Accounts}, \href{http://www.blumenbach-online.de}{Johann Friedrich Blumenbach - Online}\subsection*{Themen:}Einführung, Annotation und Modellierung, Digitale Editionswissenschaft\subsection*{Zitiervorschlag:}Galka, Selina. 2021. Ontologie. In: KONDE Weißbuch. Hrsg. v. Helmut W. Klug unter Mitarbeit von Selina Galka und Elisabeth Steiner im HRSM Projekt "Kompetenznetzwerk Digitale Edition". URL: https://gams.uni-graz.at/o:konde.151\newpage\section*{Open Access} \emph{Klug, Helmut W.; helmut.klug@uni-graz.at }\\
        
    \emph{Open Access} bezeichnet den freien Zugang zu Forschungsdaten und wissenschaftlichen Publikationen im Internet. ‘Frei’ bedeutet in diesem Zusammenhang, “[…] dass diese Literatur kostenfrei und öffentlich im Internet zugänglich sein sollte, sodass Interessierte die Volltexte lesen, herunterladen, kopieren, verteilen, drucken, in ihnen suchen, auf sie verweisen und sie auch sonst auf jede denkbare legale Weise benutzen können, ohne finanzielle, gesetzliche oder technische Barrieren jenseits von denen, die mit dem Internetzugang selbst verbunden sind.” (Budapest Open Access Initiative)\\
            
        Im Rahmen der Budapester \emph{Open Access Initiative} wird in Bezug auf die \href{http://gams.uni-graz.at/o:konde.119}{Lizenzierung} derartiger Materialien die Nutzung möglichst offener Lizenzen vorgeschlagen: “In allen Fragen des Wiederabdrucks und der Verteilung und in allen Fragen des Copyright überhaupt sollte die einzige Einschränkung darin bestehen, den jeweiligen Autorinnen und Autoren Kontrolle über ihre Arbeit zu belassen und deren Recht zu sichern, dass ihre Arbeit angemessen anerkannt und zitiert wird.” (Budapest Open Access Initiative) Im Rahmen des österreichischen Urheberrechts wäre das die entsprechende \href{http://gams.uni-graz.at/o:konde.45}{Creative Commons}-\href{http://gams.uni-graz.at/o:konde.9}{Lizenz} ‘CC-BY’, da diese nur die Nennung von Urheberin oder Urheber und Miturheberin oder Miturheber verlangt und alle anderen Arten der Werknutzung (teilen, bearbeiten, ...) ermöglicht.\\
            
        \subsection*{Literatur:}\begin{itemize}\item Brintzinger, Klaus-Rainer: Piraterie oder Allmende der Wissenschaften? Zum Streit um Open Access und der Rolle von Wissenschaft,  Bibliotheken und Markt bei der Verbreitung von  Forschungsergebnissen. In: Leviathan: 2010, S. 331–346.\item . URL: \url{https://www.budapestopenaccessinitiative.org/translations/german-translation}\item Graf, Klaus; Merta, Brigitte; Sommerlechner, Andrea; Weigl, Herwig: Edition und Open Access. In: Vom Nutzen des Edierens. Akten des internationalen Kongresses zum 150-jährigen Bestehen des Instituts für Österreichische Geschichtsforschung, Wien, 3.-5. Juni 2004. Wien: 2005, S. 197–203.\item Kennedy, Dennis M.: A primer on open source licensing legal issues: copyright, copyleft and copyfuture Primer on open source licensing legal issues. In: Louis Univ. Public Law Rev. 20: 2001, S. 345.\item Spoo, Robert: “For God’s sake, publish; only be sure of your rights”: Virginia Woolf, Copyright, and Scholarship. In: Woolf Editing/Editing Woolf. Clemson: 2009, S. 227–231.\end{itemize}\subsection*{Software:}\href{Vectr}{CC Lizenzgenerator}\subsection*{Verweise:}\href{https://gams.uni-graz.at/o:konde.119}{Lizenzierung}, \href{https://gams.uni-graz.at/o:konde.9}{Lizenzmodelle}, \href{https://gams.uni-graz.at/o:konde.45}{Creative Commons}, \href{https://gams.uni-graz.at/o:konde.44}{Urheberrecht}\subsection*{Projekte:}\href{https://open-access.net/}{Informationsplattform Open Access}, \href{http://www.berlin9.org/about/declaration/}{Berlin Declaration on Open Access}, \href{https://www.fwf.ac.at/de/forschungsfoerderung/open-access-policy/}{FWF Open Access Policy}, \href{https://www.ris.bka.gv.at/GeltendeFassung.wxe?Abfrage=Bundesnormen&Gesetzesnummer=10001848}{Bundesgesetz über das Urheberrecht an Werken der Literatur und der Kunst und über verwandte Schutzrechte (Urheberrechtsgesetz}, \href{https://creativecommons.org}{Creative Commons}\subsection*{Themen:}Einführung, Rechtliche Aspekte\subsection*{Lexika}\begin{itemize}\item \href{https://edlex.de/index.php?title=Open_Access}{Edlex: Editionslexikon}\end{itemize}\subsection*{Zitiervorschlag:}Klug, Helmut W. 2021. Open Access. In: KONDE Weißbuch. Hrsg. v. Helmut W. Klug unter Mitarbeit von Selina Galka und Elisabeth Steiner im HRSM Projekt "Kompetenznetzwerk Digitale Edition". URL: https://gams.uni-graz.at/o:konde.152\newpage\section*{PAGE-XML} \emph{Klug, Helmut W.; helmut.klug@uni-graz.at }\\
        
    Die PAGE-XML-Formate werden verwendet, um den Seiteninhalt von Quellendokumenten, die als Bilddigitalisate vorliegen, zu beschreiben und über Bildkoordinaten in den Digitalisaten zu verorten. Die Daten umfassen Layout, Textinhalt und mögliche Informationen zu Bildbearbeitung (\emph{dewarping}, \emph{deskewing}) sowie Ground-Truth-Daten, die eine Beurteilung der Ergebnisse von automatisierten Erkennungsroutinen zulassen, bei denen PAGE-XML gerne als Output verwendet wird. \\
            
        \subsection*{Literatur:}\begin{itemize}\item Pletschacher, S.; Antonacopoulos, A.: The PAGE (Page Analysis and Ground-truth Elements) Format Framework. In: Proceedings of the 20th International Conference on Pattern Recognition (ICPR2010), Istanbul, Turkey, August 23‐26, 2010: 2010, S. 257‐260.\end{itemize}\subsection*{Software:}\href{http://t-pen.org/TPEN/}{T-Pen}, \href{https://transkribus.eu/Transkribus/}{Transkribus}\subsection*{Verweise:}\href{https://gams.uni-graz.at/o:konde.196}{Topografische Edition}, \href{https://gams.uni-graz.at/o:konde.215}{XML}, \href{https://gams.uni-graz.at/o:konde.60}{Digitalisierung}, \href{https://gams.uni-graz.at/o:konde.224}{HTR}, \href{https://gams.uni-graz.at/o:konde.149}{OCR}\subsection*{Projekte:}\href{https://github.com/PRImA-Research-Lab/PAGE-XML}{PAGE-XML auf Git-Hub}, \href{https://www.primaresearch.org}{Pattern Recognition & Image Analysis Research Lab der Universität Sanfort, Manchester}\subsection*{Themen:}Digitalisierung\subsection*{Zitiervorschlag:}Klug, Helmut W. 2021. PAGE-XML. In: KONDE Weißbuch. Hrsg. v. Helmut W. Klug unter Mitarbeit von Selina Galka und Elisabeth Steiner im HRSM Projekt "Kompetenznetzwerk Digitale Edition". URL: https://gams.uni-graz.at/o:konde.154\newpage\section*{PREMIS – Preservation Metadata: Implementation Strategies} \emph{Stigler, Johannes; johannes.stigler@uni-graz.at }\\
        
    Der PREMIS-Standard zielt darauf ab, Aspekte des Lebenszyklus eines Objektes in einem digitalen Archiv (Repositorium) zu beschreiben. Basierend auf Überlegungen des \emph{Open Archival Information System}(\href{http://gams.uni-graz.at/o:konde.11}{OAIS})-Referenzmodells ist er damit unmittelbar von zentraler \href{http://gams.uni-graz.at/o:konde.6}{Nachhaltigkeit} in der Archivierung von digitalen Inhalten. Die internationale PREMIS-Arbeitsgruppe wurde vom \emph{Online Computer Library Center} (OCLD) 2003 gegründet, die das \emph{PREMIS Data Dictionary} pflegt und weiterentwickelt. Aktuell ist es in Version 3.0 vom Juni 2015 verfügbar.\\
            
        Im \emph{PREMIS Data Dictionary} wird ein Set an sogenannten semantischen Einheiten definiert, die einem Langzeitarchiv bekannt sein sollten, damit es seine Archivierungsfunktion erfüllen kann. Das PREMIS-Datenmodell unterscheidet dabei Objekte (\emph{Objects}), Ereignisse (\emph{Event}s), Agenten (\emph{Agents}), Rechte (\emph{Rights}) sowie Umgebungen (\emph{Environments}). Auf Basis dieser semantischen Einheiten können Aspekte der (a) Provenienz, (b) der Zugänglichkeit, (c) der durch Langzeitarchivierungsmaßnahmen erhaltenswerten signifikanten Eigenschaften und (d) der für die Langzeitarchivierung relevanten rechtlichen Informationen digitaler Objekte beschrieben werden.\\
            
        Die digitale Provenienz ist die Dokumentation der Verarbeitungskette und der Veränderungshistorie eines digitalen Objekts. Regeln für die Zugänglichkeit bestimmen im Falle einer Benutzung oder einer notwendigen Migration, welche Operationen erlaubt sind. Signifikante Eigenschaften sind Charakteristika eines Objekts, die mittels Langzeitarchivierungsmaßnahmen erhalten werden sollen. Hier gilt es z. B. zu definieren, ob in einem Textdokument bloß der Text und die Bilder entscheidend sind oder ob auch Schriften, der Hintergrund, die Formatierung und weitere Eigenschaften des \emph{Look and Feel} eines Objekts genauso wichtig sind. Rechtliche Informationen sind natürlich nicht ausschließlich für die Langzeitarchivierung relevant, aber zu wissen, was man mit einem Objekt machen darf, ist für den Erhaltungsprozess wichtig. Alle bekannten rechtlichen Informationen, inklusive des \href{http://gams.uni-graz.at/o:konde.44}{urheberrechtlichen} Status, der \href{http://gams.uni-graz.at/o:konde.9}{Lizenzbedingungen} sowie spezieller Befugnisse, sollten dabei aufgezeichnet werden.\\
            
        Ganz explizit außerhalb der Betrachtung bleiben unter PREMIS Metadatenelemente zu formatspezifischen Aspekten, deskriptive Metadaten, Metadaten, die anwendungs- oder geschäftsprozessspezifisch sind, z. B. Metadaten, die die Sammlungs-Policy eines Archivs beschreiben, detaillierte Informationen über Speichermedien oder Hardware, detaillierte Informationen über Agenten (Personen, Organisationen oder Software) sowie umfangreiche Informationen über Rechte und Genehmigungen. Insgesamt entspricht es der Philosophie von PREMIS, hier auf etablierte, eingeführte Metadatenstandards zurückzugreifen.\\
            
        \subsection*{Verweise:}\href{https://gams.uni-graz.at/o:konde.6}{Digitale Nachhaltigkeit}, \href{https://gams.uni-graz.at/o:konde.11}{OAIS RM}, \href{https://gams.uni-graz.at/o:konde.25}{Metadaten}, \href{https://gams.uni-graz.at/o:konde.128}{Metadaten Schemata für LZA: DCMI}, \href{https://gams.uni-graz.at/o:konde.129}{Metadaten Schemata für LZA: METS}, \href{https://gams.uni-graz.at/o:konde.131}{Metadaten Schemata für LZA: RDF, RDFS, OWL u.a}, \href{https://gams.uni-graz.at/o:konde.133}{Metadaten Schemata für LZA: CIDOC CRM}, \href{https://gams.uni-graz.at/o:konde.9}{Lizenzmodelle}, \href{https://gams.uni-graz.at/o:konde.44}{Urheberrecht}\subsection*{Literatur:}\begin{itemize}\item Digital preservation: OAIS. URL: \url{http://oais.info/}\item OCLC. URL: \url{https://oclc.org/}\item PREMIS. URL: \url{http://loc.gov/standards/premis/}\end{itemize}\subsection*{Themen:}Metadaten, Archivierung\subsection*{Zitiervorschlag:}Stigler, Johannes. 2021. PREMIS – Preservation Metadata: Implementation Strategies. In: KONDE Weißbuch. Hrsg. v. Helmut W. Klug unter Mitarbeit von Selina Galka und Elisabeth Steiner im HRSM Projekt "Kompetenznetzwerk Digitale Edition". URL: https://gams.uni-graz.at/o:konde.130\newpage\section*{Paläographie} \emph{Rieger, Lisa; lrieger@edu.aau.at }\\
        
    Unter Paläographie versteht man die „Wissenschaft bzw. Lehre von den Formen und Mitteln der Schrift“ (Best 1991, S. 359), was neben den Schriftformen auch die Beschreibstoffe, Schreibmittel, Schreibgewohnheiten und Buchformen als Untersuchungsgegenstände miteinschließt. Sie nahm ihren Ursprung bei den Benediktinern von St. Maur, die anhand von Schriftmerkmalen versuchten, Herkunftsort und Entstehungszeit von Handschriften zu bestimmen. (Debes 1986, S. 384) Als Begründer der modernen wissenschaftlichen Paläographie gelten Wilhelm Meyer, Paul von Winterfeld und Ludwig Traube. (Bruckner 1967, S. 218)\\
            
        Um die Entwicklung der Schrift in den Schriftdenkmälern im Detail nachvollziehen zu können, dienen datierbare und lokalisierbare Schriftstücke als Gerüst. Die dazu angewandte vergleichende Methode zielt auf das Erkennen von Zusammenhängen und regionalen Stilarten ab, u. a. helfen Veränderungen bestimmter Formen in der Schrift bei der zeitlichen Einordnung. Während früher v. a. das frühe und das hohe Mittelalter im Interesse der Forschung standen, findet in der heutigen Forschung auch zunehmend das Spätmittelalter Beachtung, aus dem sich auch wesentlich mehr Handschriften bis heute erhalten haben. Eine Darstellung der deutschen Paläographie kann nicht unabhängig vom lateinischen Schriftwesen erfolgen, dennoch bedingen die Spezifika deutschsprachiger Handschriften eine Reihe von eigenen Hilfsmitteln, der sich Germanisten bedienen. Bedeutendes Studienmaterial zu deutschen Handschriften inklusive fundierter Schriftanalysen bietet dabei z. B. die Sammlung von Petzet und Glauning „Deutsche schrifttafeln des IX. bis XVI. jahrhunderts aus handschriften der Bayerischen staatsbibliothek in München”, die von 1910 bis 1930 in fünf Bänden erschienen ist. (Schneider 2014, S. 13–16)\\
            
        Neben den Buchschriften zählen zum Forschungsgegenstand auch die Urkunden-, Kanzlei- und Geschäftsschriften. Während in Früh- und Hochmittelalter für Erstere ausschließlich die kalligraphische, geformte Schrift verwendet wurde, wurden Letztere in verschiedenen stilistischen Formen der zusammenhängend geschriebenen Kursivschrift verfasst. Im Spätmittelalter kam es immer stärker zur gegenseitigen Beeinflussung der beiden Grundtypen, durch die auch zahlreiche Mischformen entstanden. Abgesehen von der Komplexität des Schriftbildes muss beachtet werden, dass aufgrund von individuellen, unabwägbaren Faktoren eine exakte Datierung und/oder räumliche Einordnung eines handschriftlichen Textes in vielen Fällen trotz paläographischer Analysen nicht immer möglich ist. (Schneider 2014, S. 16–19)\\
            
        Während früher vor allem mit Fotografien und optischen Reproduktionen in Bildarchiven gearbeitet wurde, wird heute dank Digitalkameras und Scanner mit digitalen Bildern gearbeitet. (Jannidis/Kohle/Rehbein 2017, S. 187 f.) Zu Beginn des 21. Jahrhunderts entstanden im Bereich der Paläographie neben digitalen Archiven auch einige digitale Projekte, welche die Arbeit mit Bleistift und Fotokopien allmählich überflüssig machen sollten. (Hodel/Nadig 2019, S. 144 f.) Wie bei vielen verwandten Disziplinen konnte man sich auch in der Paläographie bis auf wenige Bereiche noch auf keine einheitliche Terminologie einigen. (Schneider 2014, S. 16)\\
            
        \subsection*{Literatur:}\begin{itemize}\item Best, Otto: Handbuch literarischer Fachbegriffe. Definitionen und Beispiele. Überarbeitete und erweiterte Ausgabe Handbuch literarischer Fachbegriffe. Frankfurt am Main: 1991.\item Bruckner, Albert: Zur Paläographie und Handschriftenkunde Zur Paläographie und Handschriftenkunde. In: Schweizerische Zeitschrift für Geschichte/Revue suisse d'historie/Rivista storica svizzera 17: 1967, S. 218–221.\item Debes, D.: Paläographie. In: Wörterbuch der Literaturwissenschaft. Leipzig: 1986, S. 384.\item Jannidis, Fotis; Kohle, Hubertus: Digital Humanities. Eine Einführung. Mit Abbildungen und Grafiken Digital Humanities. Hrsg. von  und Malte Rehbein. Stuttgart: 2017.\item Hodel, Tobias; Nadig, Michael: Grundlagen der Mediävistik digital vermitteln: 'Ad fontes', aber wie? Grundlagen der Mediävistik digital vermitteln In: Das Mittelalter 24: 2019.\item Schneider, Karin: Paläographie und Handschriftenkunde für Germanisten. Eine Einführung Paläographie und Handschriftenkunde für Germanisten. Berlin, Boston: 2014, URL: \url{https://www.degruyter.com/view/title/304681?tab_body=toc}.\end{itemize}\subsection*{Verweise:}\href{https://gams.uni-graz.at/o:konde.55}{Denkmäleredition}, \href{https://gams.uni-graz.at/o:konde.103}{Kodikologie}, \href{https://gams.uni-graz.at/o:konde.37}{Bilddigitalisierungstechniken}, \href{https://gams.uni-graz.at/o:konde.64}{Digitalisierungsstandards}, \href{https://gams.uni-graz.at/o:konde.60}{Digitalisierung}, \href{https://gams.uni-graz.at/o:konde.127}{Materialität}, \href{https://gams.uni-graz.at/o:konde.221}{Paläotypie}\subsection*{Themen:}Einführung, Digitale Editionswissenschaft\subsection*{Zitiervorschlag:}Rieger, Lisa. 2021. Paläographie. In: KONDE Weißbuch. Hrsg. v. Helmut W. Klug unter Mitarbeit von Selina Galka und Elisabeth Steiner im HRSM Projekt "Kompetenznetzwerk Digitale Edition". URL: https://gams.uni-graz.at/o:konde.155\newpage\section*{Paläotypie} \emph{Neuber, Frederike; frederike.neuber@bbaw.de }\\
        
    Anliegen der Paläotypie ist die Erforschung historischer Druckschriften. Primärer Forschungsgegenstand der buchwissenschaftlichen Hilfswissenschaft sind dabei Inkunabeln, d. h. Frühdrucke aus der Zeit von 1438, als Johannes Gutenberg seine frühesten Druckversuche unternahm, bis etwa 1500. Die Angabe zur Zeitspanne der Wiegendruckproduktion dient dabei lediglich zur Orientierung, denn die Inkunabelzeit auf einen klaren zeitlichen Rahmen festzulegen gestaltet sich schwierig, u. a. da die Entwicklung über Län­dergrenzen hinweg einen unterschiedlichen Verlauf nahm. (Haebler 1979, S. 2f.)\\
            
        Ziel der Paläotypie ist die Datierung und Identifikation unfirmierter Wie­gendrucke, d. h. von Druckerzeugnissen ohne oder mit nur unzureichenden Angaben über Drucker, Druckort und Druckdatum. Die Zuordnung zu einer Offizine erfolgt mittels der Identifikation von Typen, da Drucker des 15. Jahrhunderts Typen ent­weder selbst herstellten oder Spezialisten in ihren Druckwerkstätten damit beauftragten, wodurch Stempel, Matrizen und Letternmaterial im Besitz einer be­stimmten Druckerei blieben. Durch den Abgleich mit bereits identifizierten Schriftquellen können Schriften so chronolo­gisch eingeordnet bzw. einer Offizine zugeordnet werden. (Duntze 2007, S. 23)\\
            
        Anfang des 19. Jahrhundert leistete Ludwig Hain mit dem vierbändigen \emph{Repertorium bibliogra­phicum}(Hain 1826–1838) paläotypische Pionierarbeit, indem er die Drucke nicht nur nach inhaltlichen Aspekten, sondern auch nach formalen Kriterien und dem Schrifttyp klassifizierte. (Haebler 1979, S. 11–19) 1898 führte der britische Buchwissenschaftler Robert Proctor den buchtech­nisch motivierten Ansatz fort und führte den Durchschnittswert der Messung von zwanzig Zeilen als weiteres Identifikationskriterium ein. (Proctor 1898–1903)\\
            
        Anfang des 20. Jahrhunderts erweiterte Konrad Haebler die Proctor’sche Zeilen­messung mit einen zweiten exakten Faktor zur Identifizierung von Inkunabeln: der Form der Majuskel ‘M’, da diese nach Haeblers Auf­fassung in den gotischen Schriften der Frühdruckzeit die zahlreichsten Formva­rianten aufweist. Analog zur ‘M-Form’ als Leitbuchstabe der ge­brochenen Schriften, wurde für Antiquaschriften die ‘Qu’-Form als Identifizierungsmerkmal fest­gelegt. (Haebler 1979, S. 88–91) In Haeblers fünfbändigem \emph{Typenrepertorium der Wiegendrucke}, das zwischen 1905 und 1924 er­schien, sind fast 4000 exemplarische Druckschriften verzeichnet. (Haebler 1905–1924)\\
            
        Seit Ende der 1990er Jahre werden die Daten aus Haeblers Typenrepertorium im Zusammenhang mit der Erstellung des Gesamtkatalogs der Wiegendrucke am In­kunabelreferat der Staatsbibliothek zu Berlin in eine Datenbank übertragen. Das digitale Typenrepertorium der Wiegendrucke baut auf der Proctor-Haebler-Methode auf und verzeichnet mehr als 6000 Drucktypen.\\
            
        \subsection*{Literatur:}\begin{itemize}\item Duntze, Oliver: Ein Verleger sucht sein Publikum: die Strassburger Offizin des Matthias Hupfuff (1497/98-1520). München: 2007.\item Haebler, Konrad: Handbuch der Inkunabelkunde. Stuttgart: 1979.\item Haebler, Konrad: Typenrepertorium der Wiegendrucke. Wiesbaden: 1968.\item Hain, Ludwig: Repertorium bibliographicum, in quo libri omnes ab arte typographica inventa usque ad annum MD. typis expressi, ordine alphabetico vel simpliciter enumerantur vel adcuratius recensentur. Stuttgart, Tübingen: 1826.\item Proctor, Robert: An index to the early printed books in the British Museum: from the invention of printing to the year 1500, Bd. 1–4. London: 1898.\end{itemize}\subsection*{Verweise:}\href{https://gams.uni-graz.at/o:konde.200}{Typografie}, \href{https://gams.uni-graz.at/o:konde.77}{Editionstypografie}, \href{https://gams.uni-graz.at/o:konde.43}{copy-text}, \href{https://gams.uni-graz.at/o:konde.155}{Paläographie}\subsection*{Projekte:}\href{http://tw.staatsbibliothek-berlin.de/}{Typenrepertorium der Wiegendrucke}\subsection*{Themen:}Digitale Editionswissenschaft\subsection*{Zitiervorschlag:}Neuber, Frederike. 2021. Paläotypie. In: KONDE Weißbuch. Hrsg. v. Helmut W. Klug unter Mitarbeit von Selina Galka und Elisabeth Steiner im HRSM Projekt "Kompetenznetzwerk Digitale Edition". URL: https://gams.uni-graz.at/o:konde.221\newpage\section*{Part-of-Speech-Tagging} \emph{Resch, Claudia; claudia.resch@oeaw.ac.at }\\
        
    Eine der häufigsten fachspezifischen \href{http://gams.uni-graz.at/o:konde.17}{Annotationen} von Texten besteht in der Zuweisung linguistischer Information nach morphosyntaktischen Merkmalen. Das sogenannte \emph{Part-of-Speech-Tagging} (PoS-Tagging) ist das Klassifizieren eines Textes nach Wortarten und stellt neben der Tokenisierung und \href{http://gams.uni-graz.at/o:konde.115}{Lemmatisierung} einen wesentlichen Bestandteil der linguistischen Basisannotation dar: Dabei werden die in einem Text vorkommenden Wörter und Satzzeichen mit einem vordefinierten Inventar von verfügbaren Wortarten (\href{http://gams.uni-graz.at/o:konde.177}{TagSet}) einer grammatikalischen Klasse zugewiesen, wodurch eine Suche nach abstrakten sprachlichen Phänomenen möglich wird. Auf diese Weise kann die Abfrage generalisiert werden, etwa indem nach bestimmten Wortarten oder Sequenzen von Wortarten gesucht wird. Die Abfrage kann damit jedoch auch weiter spezifiziert werden, zum Beispiel, wenn gezielt nach allen Belegen der Wortform ‘sein’ in der Funktion des Possessivpronomens gesucht wird, hingegen die Vorkommen von ‘sein’ als Auxiliarverb ausgeschlossen werden sollen.\\
            
        Die Zuordnung der Wortformen zu einer Wortart kann manuell, halb-automatisch oder – bei sehr großen Textsammlungen – automatisch durch sogenannte \emph{Part-of-Speech}-\href{http://gams.uni-graz.at/o:konde.176}{Tagger} (kurz: PoS-Tagger) erfolgen. Deren Zuweisungen und Disambiguierungen basieren entweder auf Regeln (symbolische Tagger) oder auf statischen Verfahren bzw. maschinellen Lernverfahren (stochastische Tagger). Sogenannte hybride oder transformationsbasierte Tagger kombinieren beide Verfahren, indem sie bei der Disambiguierung mehrdeutiger Einheiten zunächst von der wahrscheinlichsten Wortart ausgehen, um diese dann durch kontextspezifische Regeln zu korrigieren (Perkuhn/Keibel/Kupietz 2012, S. 59). Automatische Tagger können an verschiedene Sprachen und Sprachstufen angepasst werden – eine Liste von frei verfügbaren PoS-Taggern bietet die Universität Stanford an.
               \\
            
        Als Referenzsysteme und Grundlage für das Trainieren des PoS-Taggings werden Texte von bester Qualität (Goldstandard) herangezogen, deren automatische Annotation manuell überprüft und nachkorrigiert wurde. Generell ist die Automatisierung des PoS-Taggings für große, moderne Standardsprachen bereits sehr weit fortgeschritten – allerdings „bleibt hier für historische oder variante Spracherzeugnisse oder bestimmte literarische Genres noch viel zu tun“  (Rapp 2017, S. 259).
               \\
            
        Die Qualität von manuell durchgeführten Annotationen beruht letztlich auf \href{http://gams.uni-graz.at/o:konde.100}{interpretativen} Entscheidungen, deren Zuverlässigkeit durch die Anwendung des \emph{Inter-Annotator-Agreements} oder des \emph{Intra-Annotatator-Agreements} gesichert werden soll. Im Sinne der Nachnutzbarkeit von Annotationen ist es wichtig, deren Qualität einzuschätzen, mögliche Fehlerquellen zu diskutieren und die Verwendung der Labels sowie getroffene Entscheidungen in den Tagging-Guidelines nachvollziehbar zu dokumentieren.\\
            
        \subsection*{Literatur:}\begin{itemize}\item Bański, Piotr; Haaf, Susanne; Mueller, Martin: Lightweight Grammatical Annotation in the TEI: New Perspectives. In: LREC 2018 – 11th International Conference on Language Resources and Evaluation. Japan, S. 1795–1802.\item Lemnitzer, Lothar; Zinsmeister, Heike: Korpuslinguistik. Eine Einführung. Tübingen: 2010.\item Perkuhn, Rainer; Keibel, Holger; Kupietz, Marc: Korpuslinguistik. Paderborn: 2012.\item Rapp, Andrea: Manuelle und automatische Annotation. In: Digital Humanities. Eine Einführung. Stuttgart: 2017, S. 253–267.\item Schiller, Anne; Teufel, Simone; Stöckert, Christine; Thielen, Christine: Guidelines für das Tagging deutscher Textcorpora mit STTS: 1999. URL: \url{http://www.sfs.uni-tuebingen.de/resources/stts-1999.pdf}.\end{itemize}\subsection*{Software:}\href{https://www.clarin.eu/content/services}{CLARIN-mediated NLP-services}, \href{https://weblicht.sfs.uni-tuebingen.de/weblicht/}{weblicht}, \href{http://opennlp.apache.org/}{Apache OPENNLP}, \href{http://ucrel.lancs.ac.uk/claws/}{CLAWS POS-Tagger for English}, \href{https://www.cis.uni-muenchen.de/~schmid/tools/TreeTagger/}{TreeTagger}, \href{https://www.cis.uni-muenchen.de/~schmid/tools/RNNTagger/}{RNNTagger}, \href{https://github.com/tsproisl/SoMeWeTa}{SoMeWeTa}, \href{https://spacy.io/}{spacy }, \href{https://nlp.stanford.edu/software/tagger.shtml}{Stanford Log-linear Part-Of-Speech-Tagger}, \href{https://universaldependencies.org/}{Universal Dependencies}\subsection*{Verweise:}\href{https://gams.uni-graz.at/o:konde.115}{Lemmatisierung}, \href{https://gams.uni-graz.at/o:konde.145}{NLP}, \href{https://gams.uni-graz.at/o:konde.170}{SpaCy}, \href{https://gams.uni-graz.at/o:konde.176}{Tagger}, \href{https://gams.uni-graz.at/o:konde.177}{Tagsets}, \href{https://gams.uni-graz.at/o:konde.17}{Textannotation}, \href{https://gams.uni-graz.at/o:konde.212}{Weblicht}, \href{https://gams.uni-graz.at/o:konde.80}{Elemente digitaler Editionen}\subsection*{Projekte:}\href{http://www.deutschestextarchiv.de/}{Deutsches Textarchiv}, \href{https://acdh.oeaw.ac.at/abacus/}{Austrian Baroque Corpus (ABaC:us)}, \href{https://traveldigital.acdh.oeaw.ac.at/}{travel!digital}, \href{https://nlp.stanford.edu/links/statnlp.html#Taggers}{Liste von Part-of-Speech-Taggern}\subsection*{Themen:}Einführung, Natural Language Processing\subsection*{Zitiervorschlag:}Resch, Claudia. 2021. Part-of-Speech-Tagging. In: KONDE Weißbuch. Hrsg. v. Helmut W. Klug unter Mitarbeit von Selina Galka und Elisabeth Steiner im HRSM Projekt "Kompetenznetzwerk Digitale Edition". URL: https://gams.uni-graz.at/o:konde.156\newpage\section*{Persistent Identifier} \emph{Bleier, Roman; roman.bleier@uni-graz.at / Klug, Helmut W.; helmut.klug@uni-graz.at }\\
        
    Als \emph{Persistent Identifier} (PID) werden langzeitverfügbare Referenzen auf digitale Objekte bezeichnet. \emph{Persistent Identifier} können systemspezifisch oder global vergeben werden. In der \href{http://gams.uni-graz.at/o:konde.6}{Langzeitarchivierung} wird eine Reihe von PID-Systemen verwendet, die digitalen Objekten einen global eindeutigen Namen zuweisen. Die bekanntesten Systeme sind \emph{Digital Object Identifier} (DOI), \emph{handle.net}, \emph{Archival Resource Key} (ARK) und \emph{Uniform Resource Names} (URNs). \\
            
        Manchmal werden in der Fachliteratur auch \emph{Persistent Uniform Resource Locators} (PURLs) zu den PID-Systemen gezählt. Ein PURL ist jedoch eigentlich ein Permalink, der durch ein Redirect auf die gewünschte Ressource im Internet aufgelöst wird. PURLs waren ursprünglich als Übergangslösung gedacht, bis sich URNs etabliert haben, und können in URLs überführt werden. Ein zentraler Unterschied zwischen PIDs und Permalinks (inkl. PURLs) ist, dass letztere http-URIs sind, die sofort über den Webbrowser aufgelöst werden können, während PIDs einen Resolver brauchen, der den PID mit der URL der entsprechenden Ressource verbindet.\\
            
        Im Kontext von \href{http://gams.uni-graz.at/o:konde.59}{Digitalen Editionen} gibt es auch Versuche, traditionelle, kanonische Zitiersysteme in \emph{Persistenten Identifiers} auszudrücken und dadurch eine übliche Zitierpraxis in das digitale Medium zu überführen. Das bekannteste Protokoll für diesen Zweck ist das \emph{Canonical Text Service} (CTS). \\
            
        \subsection*{Literatur:}\begin{itemize}\item Arnold, Eckhart; Müller, Stefan: Wie permanent sind Permalinks? In: Informationspraxis 3: 2017.\item Bleier, Roman: Canonical structure and the referencing of digital resources for the study of ancient and medieval Christianity. In: Digital Humanities and Christianity: An Introduction. Berlin: 2021.\item Persistent Identifier: eindeutige Bezeichner für digitale Inhalte. URL: \url{http://www.persistent-identifier.de/}\item DOI® Handbook: 2019. URL: \url{https://www.doi.org/hb.html}.\item Klump, Jens; Huber, Robert: 20 Years of Persistent Identifiers – Which Systems are Here to Stay? In: Data Science Journal 16: 2017, S. 1-7.\item Schroeder, Kathrin: 9.4. Persistent Identifier (PI) - ein Überblick. In: nestor Handbuch. Eine keine Enzyklopädie der digitalen Langzeitarchivierung. Version 2.3. Glückstadt: 2009.\item Sompel, Herbert Van de; Sanderson, Robert; Shankar, Harihar; Klein, Martin: Persistent Identifiers for Scholarly Assets and the Web: The Need for an Unambiguous Mapping Persistent Identifiers for Scholarly Assets and the Web. In: International Journal of Digital Curation 9: 2014, S. 331–342.\end{itemize}\subsection*{Verweise:}\href{https://gams.uni-graz.at/o:konde.6}{Digitale Nachhaltigkeit}, \href{https://gams.uni-graz.at/o:konde.8}{Linked Open Data}, \href{https://gams.uni-graz.at/o:konde.219}{Zitierbarkeit digitaler Ressourcen}, \href{https://gams.uni-graz.at/o:konde.220}{Zitiervorschlag}\subsection*{Projekte:}\href{http://handle.net}{handle.net}, \href{https://www.doi.org}{Digital Object Identifier}, \href{http://cite-architecture.github.io/cts_spec/specification.html#cts}{Canonical Text Services}\subsection*{Themen:}Archivierung, Digitale Editionswissenschaft\subsection*{Lexika}\begin{itemize}\item \href{https://edlex.de/index.php?title=Persistant_Identifier_(PI)}{Edlex: Editionslexikon}\end{itemize}\subsection*{Zitiervorschlag:}Bleier, Roman; Klug, Helmut W. 2021. Persistent Identifier. In: KONDE Weißbuch. Hrsg. v. Helmut W. Klug unter Mitarbeit von Selina Galka und Elisabeth Steiner im HRSM Projekt "Kompetenznetzwerk Digitale Edition". URL: https://gams.uni-graz.at/o:konde.12\newpage\section*{Philosophische Edition} \emph{Lobis, Ulrich; ulrich.lobis@uibk.ac.at / Wang-Kathrein, Joseph; joseph.wang@uibk.ac.at}\\
        
    Unter einer philosophischen Edition ist zunächst eine Edition von philosophischen Primärtexten zu verstehen. Dabei kann es sich um ein philosophisches Werk, um das philosophische Schaffen einer Person oder um eine Sammlung von Texten, die zum gleichen philosophischen Thema passen, handeln. Das bedeutet insbesondere, dass eine philosophische Edition zugleich eine \href{http://gams.uni-graz.at/o:konde.93}{historisch-kritische} oder eine \href{http://gams.uni-graz.at/o:konde.90}{genetische} Edition sein kann.\\
            
        Bei einigen philosophischen Editionen liegt ein besonderes Augenmerk auf den verschiedenen Überlieferungen und Varianten, die in der Edition verzeichnet werden. Diese Editionen versuchen den ‘Originaltext’ zu rekonstruieren, der in den unterschiedlichen Überlieferungen (z. B. durch Abschreibfehler) verändert worden ist. Andere zielen auf die Unterschiede zwischen publizierten und unpublizierten Fassungen. Diese Editionen ermöglichen ihren Leserinnen und Lesern das genaue Studium von Texten, deren Autorinnen und Autoren besondere Acht auf die Wortwahl geben. Bei beiden bietet das \href{http://gams.uni-graz.at/o:konde.59}{digitale Edieren} v. a. den Vorteil, dass die verschiedenen Überlieferungen und Fassungen quasi ‘auf Knopfdruck’ repliziert werden können.\\
            
        Bei einigen philosophischen Editionen stellt das Erstellen des \href{http://gams.uni-graz.at/o:konde.34}{Kommentars} die Hauptaufgabe dar. Bei den Kommentaren kann es sich zwar sowohl um Einzelstellen- wie um Überblickskommentare handeln, aber im Gegensatz zu anderen Editionen scheint der Kommentar hier sehr viel Platz einzunehmen. Es kommt häufig vor, dass der Kommentar sogar quantitativ länger ist als der edierte Primärtext. Gerade daran kann man erkennen, dass die Hauptziele einer philosophischen Edition nicht nur die Darbietung eines Textes beinhalten. Vielmehr will sie den Leserinnen und Lesern eine \href{http://gams.uni-graz.at/o:konde.100}{Interpretation} des Textes nahe bringen und ihn in der Ideengeschichte einordnen.\\
            
        \subsection*{Verweise:}\href{https://gams.uni-graz.at/o:konde.59}{Digitale Edition}, \href{https://gams.uni-graz.at/o:konde.34}{Kommentar}, \href{https://gams.uni-graz.at/o:konde.93}{historisch-kritische Edition}, \href{https://gams.uni-graz.at/o:konde.90}{genetische Edition}, \href{https://gams.uni-graz.at/o:konde.100}{Interpretation}, \href{https://gams.uni-graz.at/o:konde.32}{Apparat}, \href{https://gams.uni-graz.at/o:konde.174}{Synopse}\subsection*{Themen:}Einführung, Digitale Editionswissenschaft\subsection*{Lexika}\begin{itemize}\item \href{https://edlex.de/index.php?title=Philosophische_Edition}{Edlex: Editionslexikon}\end{itemize}\subsection*{Zitiervorschlag:}Lobis, Ulrich; Wang-Kathrein, Joseph. 2021. Philosophische Edition. In: KONDE Weißbuch. Hrsg. v. Helmut W. Klug unter Mitarbeit von Selina Galka und Elisabeth Steiner im HRSM Projekt "Kompetenznetzwerk Digitale Edition". URL: https://gams.uni-graz.at/o:konde.157\newpage\section*{Pseudonymisierung} \emph{Eder, Elisabeth; elisabeth.eder@aau.at }\\
        
    Pseudonymisierung stellt ein mögliches Verfahren zum Schutz persönlicher Daten in
                  Korpora und Datensammlungen dar. Dabei wird Information, die eine Identifikation
                  natürlicher Personen ermöglicht, durch realistische Bezeichnungsalternativen
                  ersetzt, um die Integrität der Daten zu wahren. ‘Irene Adler’ könnte etwa durch
                  ‘Hermine Granger’ ersetzt werden. Dazu sind zwei Schritte notwendig: die Erkennung
                  der individuenidentifizierenden Information und die Substitution der
                  entsprechenden Textteile. Erstere Aufgabe weist Ähnlichkeiten zur \emph{\href{http://gams.uni-graz.at/o:konde.141}{Named Entity Recognition}} auf. Die möglichen Entitätstypen können je nach Texttyp jedoch abweichen und
                  differenzieren beispielsweise Personennamen in Nachnamen und weibliche und
                  männliche Vornamen aus, erfassen verschiedene Unterkategorien für Orte oder
                  enthalten IDs sowie Kontaktdaten (Telefonnummer, URIs, E-Mail-Adressen) etc.
                  (siehe z. B. Stubbs/Uzuner 2015b oder Eder et al. 2019) Aufgrund dessen müssen
                  vorhandene Tools für \href{http://gams.uni-graz.at/o:konde.141}{NER} und \emph{Sequence}-Tagging potentiell auf dementsprechend annotierten
                  Daten trainiert werden. Explizit für die automatische Erkennung von
                  personenidentifizierender Information entwickelte Programme existieren vor allem
                  für den medizinischen Bereich. (Überblick z. B. in Stubbs et al. 2015c, 2017) In einem zweiten
                  Schritt werden die gefundenen Entitäten durch realistische Alternativen
                  substituiert. Mit dem \emph{Surrogate Generation-Tool} lassen sich
                  Ersetzungen fürs Deutsche automatisch generieren.\\
            
        \subsection*{Literatur:}\begin{itemize}\item Eder, Elisabeth; Krieg-Holz, Ulrike; Hahn, Udo: De-identification of emails: pseudonymizing
                              privacy-sensitive data in a German email corpus. In: Proceedings of the International Conference Recent
                              Advances in Natural Language Processing RANLP. Varna, Bulgaria: 2019, S. 259–269.\item Medlock, Ben: An introduction to NLP-based textual
                              anonymization. In: Proceedings of the Fifth International Conference on
                              Language Resources and Evaluation LREC. Genoa, Italy: 2006.\item Stubbs, Amber; Uzuner, Özlem; Kotfila, Christopher; Goldstein, Ira; Szolovits, Peter: Challenges in synthesizing surrogate PHI in narrative
                              EMRs. In: Medical Data Privacy Handbook. Cham, Heidelberg, New York, Dordrecht, London: 2015, S. 717–735.\item Stubbs, Amber; Uzuner, Özlem: Annotating longitudinal clinical narratives for
                              de-identification: the 2014 i2b2/UTHealth corpus. In: Journal of Biomedical Informatics 58: 2015, S. 20–29.\item Stubbs, Amber; Kotfila, Christopher; Uzuner, Özlem: Automated systems for the de-identification of
                              longitudinal clinical narratives: overview of 2014 i2b2/UTHealth
                              Shared Task Track 1. In: Journal of Biomedical Informatics 58: 2015, S. 11–19.\item Stubbs, Amber; Filannino, Michele; Uzuner, Özlem: De-identification of psychiatric intake records.
                              Overview of 2016 CEGS NGRID Shared Task Track 1. In: Journal of Biomedical Informatics 75: 2017, S. 4-18.\item Yeniterzi, Reyyan; Aberdeen, John S.; Bayer, Samuel; Wellner, Benjamin; Hirschman, Lynette; Malin, Bradley A.: Effects of personal identifier resynthesis on clinical
                              text de-identification. In: Journal of the American Medical Informatics
                              Association 17: 2010, S. 159–168.\end{itemize}\subsection*{Software:}\href{https://spacy.io/}{spacy }, \href{https://github.com/zalandoresearch/flair}{flair}, \href{https://www.ims.uni-stuttgart.de/forschung/ressourcen/werkzeuge/german-ner/}{German NER}, \href{https://github.com/tudarmstadt-lt/GermaNER}{GermaNER}, \href{https://github.com/ee-2/SurrogateGeneration}{Surrogate Generation}, \href{http://neuroner.com/}{NeuroNER}\subsection*{Verweise:}\href{https://gams.uni-graz.at/o:konde.141}{Named Entity Recognition}, \href{https://gams.uni-graz.at/o:konde.223}{Digitalisierung:
                           Rechtliches}, \href{https://gams.uni-graz.at/o:konde.145}{NLP}\subsection*{Themen:}Natural Language Processing\subsection*{Zitiervorschlag:}Eder, Elisabeth. 2021. Pseudonymisierung. In: KONDE Weißbuch. Hrsg. v. Helmut W. Klug unter Mitarbeit von Selina Galka und Elisabeth Steiner im HRSM Projekt "Kompetenznetzwerk Digitale Edition". URL: https://gams.uni-graz.at/o:konde.159\newpage\section*{Quellenedition} \emph{Vogeler, Georg; georg.vogeler@uni-graz.at / Wallnig, Thomas;
                  thomas.wallnig@univie.ac.at }\\
        
    Als Quellenedition bezeichnet man alle Editionen, deren Ziel es ist, Texte zu
                  veröffentlichen, aus denen historische Informationen gezogen werden können. Als
                  Quellen können alle aus der Alltagspraxis stammenden Dokumente (‘Überreste’
                     (Droysen 1937)) und alle erzählend über die Vergangenheit
                  informierenden Texte verstanden werden. Das Editionsinteresse liegt also nicht
                  ausschließlich in der Erzeugung eines philologisch zuverlässigen Textes, sondern
                  ebensosehr darin, Zeugnisse von historischen Prozessen der politisch und
                  wissenschaftlich interessierten Öffentlichkeit zugänglich zu machen. Dafür ist der
                  wissenschaftliche \href{http://gams.uni-graz.at/o:konde.34}{Kommentar} (vgl.
                  z. B. die Publikation von Hitlers \emph{Mein Kampf} durch das
                  Institut für Zeitgeschichte, München (Hitler et al. 2016)) und die
                  Erschließung der Texte durch Register und Regesten von besonderem Interesse. \\
            
        Diese Editionstradition ist in der Mediävistik eng mit den Editionen der \emph{Monumenta Germaniae Historica}, für die Neuere Geschichte und
                  Zeitgeschichte mit der systematischen Edition der Akten der höchsten politischen
                  Entscheidungsträger (Akten der Reichskanzlei, Ministerratsprotokolle,
                  Reichtstagsakten) und der Publikation von außenpolitischen Dokumenten als Reaktion
                  auf die politischen Katastrophen der Weltkriege des 20. Jahrhunderts verbunden.
                  Nationalstaatlich konzipierte Quelleneditionen greifen frühneuzeitliche
                  Traditionen auf, in denen Streit um Staatsrecht und Kirchengeschichte die
                  Publikation von historischen Quellen motivierte. Eng verwandt mit den
                  Quelleneditionen sind \href{http://gams.uni-graz.at/o:konde.140}{Nachlasspublikationen} von Personen des öffentlichen Lebens, wie
                  Politikerinnen und Politikern, aber auch Künstlerinnen und Künstlern sowie
                  Wissenschafterinnen und Wissenschaftern.\\
            
        Aufgrund unterschiedlicher Traditionen disziplinärer und institutioneller Kontexte
                  haben Quelleneditionen keine gemeinsamen Verfahren in der Textbehandlung
                  entwickelt. So gibt es z. B. eine Reihe nationaler oder sprachbezogener
                  Empfehlungen für Quelleneditionen, wie z. B. Ihnatowicz (1962), Joly
                     (2003), AHF (2007), Heinemeyer (1978),
                  Normas de transcripción y edición (1944), Norme per la stampa delle
                  Fonti (1906), Smith (2001), obwohl die \emph{Commission Internationale de Diplomatique} noch 1984 den Versuch
                  internationaler Empfehlungen gemacht hat (Diplomatica et sigillographica
                     1984). Die Logik der Quelleneditionen reicht von besonders dokumentnaher
                  Edition (\emph{\href{http://gams.uni-graz.at/o:konde.72}{documentary editing}}) bis zu datenbankartiger Dokumentation der Quellen in Regesten (\href{http://gams.uni-graz.at/o:konde.162}{Regestenedition}(Steinecke 1982, Sprengel et al. 2013)), auch wenn diese Form aus
                  Sicht von Historikerinnen und Historikern klar von der Volltextwiedergabe in einer
                  Edition zu unterscheiden ist.\\
            
        Quelleneditionen grenzen sich von anderen Editionsformen meist dadurch ab, dass
                  sie es den Benutzerinnen und Benutzern erleichtern, die historischen Informationen
                  in den Dokumenten ausfindig zu machen, indem sie Kopfregesten, umfangreiche
                  Register und insbesondere einen historischen Kommentar liefern. Mit digitalen
                  Mitteln können diese Editionen insbesondere mit Hilfe von Methoden des \emph{\href{http://gams.uni-graz.at/o:konde.167}{Semantic Web}} als “assertive editions” (Vogeler 2019) realisiert werden.\\
            
        Mehrere Institutionen und editorische Unternehmen weisen in Österreich eine
                  längere Tradition auf. Das heute an der Universität Wien angesiedelte Institut für
                  Österreichische Geschichtsforschung ist seit der Mitte des 19. Jahrhunderts mit
                  der Ausbildung von Editorinnen und Editoren historischer Quellen befasst. Dort
                  werden auch bis heute editorische Langzeitunternehmen (mit-)betreut und in
                  mehreren Reihen publiziert, etwa in den \emph{QIÖG}. Ähnliches
                  gilt für die Österreichische Akademie der Wissenschaften, deren ‘Historische
                  Kommission’ ebenfalls seit rund 170 Jahren mit der Herausgabe von historischen
                  Quellen befasst ist, beispielsweise in der Reihe \emph{Fontes Rerum
                     Austriacarum}. Eine ähnliche Rolle erfüllen die Veröffentlichungen der
                  Kommission für Neuere Geschichte Österreichs.\\
            
        \subsection*{Literatur:}\begin{itemize}\item AHF: Arbeitsgemeinschaft außeruniversitärer historischer
                              Forschungseinrichtungen: Empfehlungen zur Edition frühneuzeitlicher
                              Texte: 2007. URL: \url{https://www.heimatforschung-regensburg.de/280/1/E-Forum_AHF-Empfehlungen.pdf}.\item , : Diplomatica et sigillographica. Travaux préliminaires de
                              la CID et de la commission internationale de sigillographie pour une
                              normalisation internationale des éditions de document et un
                              Vocabulaire international de la Diplomatique et de la
                              Sigillographie. Zaragoza: 1984.\item Droysen, Johann Gustav: Historik. Vorlesungen über Enzyklopädie und Methodologie
                              der Geschichte. Hrsg. von  und  . München: 1937.\item Richtlinien für die Edition landesgeschichtlicher
                              Quellen. Hrsg. von  und Walter Heinemeyer. Marburg, Köln: 1978.\item Hitler, Adolf; Hartmann, Christian; Raim, Edith: Hitler, Mein Kampf: eine kritische Edition Hitler, Mein Kampf. München: 2016.\item Ihnatowicz, Ireneusz: Projekt instrukcji wydawniczej dla źródeł historycznych
                              XIX i początku XX wieku. In: Studia Źródłoznawcze 7: 1962.\item Joly, Bertrand: L’édition des documents des XIXe et XXe siècles. In: Bibliothèque de l'École des chartes 161: 2003, S. 537–552.\item Normas de transcripción y edición de textos y
                              documentos. Madrid: 1944.\item Norme per la stampa delle Fonti per la storia
                              d'Italia. In: Bulletino dell'Istituto storico italiano 28: 1906, S. XI-XXI.\item Smith, Marc: Conseils pour l'éditions des document en langue
                              italienne (XIVe-XVIIe siècle). In: BECh 159: 2001, S. 541–578.\item Sprengel, Peter; Wack, Edith; Lörke, Tim: Gerhart Hauptmann digital. Probleme und
                              Herausforderungen einer Briefregestenedition in 'Kalliope'. In: Im Dickicht der Texte. Editionswissenschaft als
                              interdisziplinäre Grundlagenforschung: 2013, S. 183–208.\item Steinecke, Hartmut: Brief-Regesten. Theorie und Praxis einer neuen
                              Editionsform. In: Zeitschrift für Deutsche Philologie 101: 1982, S. 199–210.\item Vogeler, Georg: The ‘assertive edition’: On the consequences of digital
                              methods in scholarly editing for historians The ‘assertive edition’. In: International Journal of Digital Humanities 1: 2019, S. 309–322.\end{itemize}\subsection*{Verweise:}\href{https://gams.uni-graz.at/o:konde.72}{Documentary Editing}, \href{https://gams.uni-graz.at/o:konde.162}{Regestausgabe}, \href{https://gams.uni-graz.at/o:konde.34}{Kommentar}, \href{https://gams.uni-graz.at/o:konde.140}{Nachlassedition}, \href{https://gams.uni-graz.at/o:konde.167}{Semantic Web}\subsection*{Themen:}Einführung, Digitale Editionswissenschaft\subsection*{Lexika}\begin{itemize}\item \href{https://edlex.de/index.php?title=Quellenedition}{Edlex: Editionslexikon}\end{itemize}\subsection*{Zitiervorschlag:}Vogeler, Georg; Wallnig, Thomas. 2021. Quellenedition. In: KONDE Weißbuch. Hrsg. v. Helmut W. Klug unter Mitarbeit von Selina Galka und Elisabeth Steiner im HRSM Projekt "Kompetenznetzwerk Digitale Edition". URL: https://gams.uni-graz.at/o:konde.160\newpage\section*{RDF, RDFS, OWL} \emph{Pollin, Christopher; christopher.pollin@uni-graz.at }\\
        
    Das \emph{Resource Description Framework} (RDF) ist ein Datenmodell zur Darstellung und für den Austausch von Daten im Web. Daten werden als Resource definiert, wobei eine Resource alles sein kann: ein Dokument, ein physisches Objekt oder ein abstraktes Konzept. Über jede \emph{resource} werden Statements der Form Subjekt-Prädikat-Objekt formuliert und repräsentieren Beziehungen zwischen zwei \emph{Resources}, den sogenannte \emph{Triples}. Betrachtet man den Satz “Bob ist befreundet mit Alice.'', dann lässt sich folgendes \emph{Triple} extrahieren: <Bob> als Subjekt, <ist befreundet mit> als Prädikat und <Alice> als Objekt. Jede Relation in RDF ist nur in eine Richtung definiert und insgesamt entspricht RDF einem gerichteten Graphen. Jeder Bestandteil des \emph{Triples} ist entweder ein URI oder ein \emph{Literal}.\\
            
        Um die Art der Beziehungen und die Ressourcen zu klassifizieren, liefert das \emph{Resource Description Framework Schema }(RDFs) eine semantische Erweiterung für RDF. Dies umfasst die Möglichkeit, Klassen (\emph{rdfs:Class}) und Relationen (\emph{rdf:Property}) zu definieren, und folgt dem Paradigma der Objektorientierung. Es lassen sich auf diese Weise Instanzen von Klassen erzeugen, die alle Eigenschaften der Klasse und ihrer übergeordneten Klassen erben. Im RDFS-\emph{Namespace} sind neben den Konzepten zur Definitionen von Klassen und ihren Relationen weitere Standardisierungen zur Referenzierung (\emph{rdfs:seeAlso}), Typisierung (\emph{rdfs:Datatype}) und Beschreibung (\emph{rdfs:label,rdfs:comment}) gegeben.\\
            
        Die \emph{Web Ontology Language} (OWL) erweitert RDFs um die Möglichkeit der formalen Beschreibung durch deskriptive Logiken. Es gibt mehrere Untermengen der Sprache, die sich in ihrer Ausdrucksstärke und Entscheidbarkeit unterscheiden. OWL Lite ermöglicht es Taxonomien zu definieren, OWL DL führt deskriptive Logik der Prädikatenlogik erster Stufe ein und OWL Full ist in vollem Umfang ausdrucksstark zu ungunsten der Entscheidbarkeit. Entscheidbarkeit ist notwendig, um das \emph{Reasoning}, das automatisierte logische Schlussfolgern, zu ermöglichen.\\
            
        \subsection*{Literatur:}\begin{itemize}\item Powers, Shelley: Practical RDF: solving problems with the resource description framework. Köln: 2003.\item RDF 1.1 Primer. URL: \url{https://www.w3.org/TR/rdf11-primer/}\item RDF Schema 1.1 RDFS. URL: \url{https://www.w3.org/TR/rdf-schema/}\item Hitzler, Pascal; Krötzsch, Markus; Rudolph, Sebastian; Sure, York: Semantic Web. Grundlagen: 2008, S. 33–88.\item OWL Web Ontology Language Overview. URL: \url{http://www.ksl.stanford.edu/people/dlm/webont/OWLOverview.htm}\item Rehbein, Malte: Ontologien. Stuttgart: 2017, S. 162–176.\end{itemize}\subsection*{Verweise:}\href{https://gams.uni-graz.at/o:konde.137}{Modellierung}, \href{https://gams.uni-graz.at/o:konde.151}{Ontologie}, \href{https://gams.uni-graz.at/o:konde.167}{Semantic Web}, \href{https://gams.uni-graz.at/o:konde.168}{Semantic Web Technologien}\subsection*{Themen:}Annotation und Modellierung\subsection*{Lexika}\begin{itemize}\item \href{https://edlex.de/index.php?title=Resource_Description_Framework_(RDF)}{Edlex: Editionslexikon}\end{itemize}\subsection*{Zitiervorschlag:}Pollin, Christopher. 2021. RDF, RDFS, OWL. In: KONDE Weißbuch. Hrsg. v. Helmut W. Klug unter Mitarbeit von Selina Galka und Elisabeth Steiner im HRSM Projekt "Kompetenznetzwerk Digitale Edition". URL: https://gams.uni-graz.at/o:konde.131\newpage\section*{Regestausgabe} \emph{Rieger, Lisa; lrieger@edu.aau.at }\\
        
    Der Begriff ‘Regesten’ stammt ursprünglich aus der Mediävistik und Geschichtswissenschaft;  Regesten stellten dort „Auszüge von Urkunden (mit Inhaltsangabe, Datierung etc.) bzw. deren gedruckte Sammlungen“ (Best 1991, S. 416) dar. Von der Literaturwissenschaft wurden sie für die \href{http://gams.uni-graz.at/o:konde.39}{Edition von Briefen} adaptiert, die aufgrund ihres Umfangs nicht vollständig abgedruckt werden können. Durch die Zusammenfassung der wesentlichen Informationen in Regestform können diese Briefe systematisch erschlossen werden, ohne in ihrer vollen Form abgedruckt zu werden. (Plachta 1997, S. 25 f.) Innerhalb einer Edition folgen Regesten stets demselben Muster: Der Regestkopf enthält Informationen über Briefschreiber, Ort, Entstehungsdatum sowie, wenn vorhanden, die Überlieferungslage und bisherigen Abdruck, während im eigentlichen Regest der Inhalt des Briefes zusammengefasst wird. Zusätzlich können Register der Briefschreiber, der erwähnten Personen und/oder Werke zur besseren Orientierung hinzugefügt werden. (Hagen 1988, S. 53 f.)\\
            
        Zu den bekanntesten Regestausgaben zählen die der Briefe Thomas Manns sowie die der Briefe an Goethe (Hagen 1988, S. 53) – wobei bei letzterer geschätzt wird, dass ein Gesamtabdruck der ca. 20.000 überlieferten Briefe bis zu 80 Bände umfassen würde. (Werber 1986, S. 426) Als Nachteile der komprimierten Darstellung wird oft genannt, dass Sprache und Stil der Briefe, aber auch die besonderen Charakteristika eines Briefes, wie z. B. die Formulierung der Anrede, nicht erfasst werden. (Hagen 1988, S. 54; Plachta 1997, S. 26). Dabei sollte jedoch beachtet werden, dass eine Regestausgabe in erster Linie dem Nachweis von Existenz und Standort eines Dokuments dient und dem Benutzer durch die inhaltliche Erschließung einen zusätzlichen Filter für seine weitere Recherche bietet. Die zielgerichtete Suche nach bestimmten Briefen wird dabei durch die \href{http://gams.uni-graz.at/o:konde.60}{Digitalisierung} der Editionen weiter erleichtert. (Koltes 2013, S. 75–77)\\
            
        \subsection*{Literatur:}\begin{itemize}\item Best, Otto: Handbuch literarischer Fachbegriffe. Definitionen und Beispiele. Überarbeitete und erweiterte Ausgabe Handbuch literarischer Fachbegriffe. Frankfurt am Main: 1991.\item Hagen, Waltraud: Von den Ausgabetypen. In: Vom Umgang mit Editionen. Eine Einführung in Verfahrensweisen und Methoden der Textologie. Berlin: 1998, S. 31–54.\item Manfred Koltes: Probleme bei der Retro-Konversion. Die Regestausgabe der Briefe an Goethe Probleme bei der Retro-Konversion. In: Brief-Edition im digitalen Zeitalter 34: 2013, S. 75–86.\item Plachta, Bodo: Editionswissenschaft. Eine Einführung in Methode und Praxis der Edition neuerer Texte Editionswissenschaft: 1997.\item Werner, J: Regesten Regesten. In: Wörterbuch der Literaturwissenschaft 1. Leipzig: 1986, S. 426.\end{itemize}\subsection*{Verweise:}\href{https://gams.uni-graz.at/o:konde.39}{Briefedition}, \href{https://gams.uni-graz.at/o:konde.28}{Textgenese}, \href{https://gams.uni-graz.at/o:konde.60}{Digitalisierung}, \href{https://gams.uni-graz.at/o:konde.59}{Digitale Edition}\subsection*{Projekte:}\href{http://www.regesta-imperii.de/unternehmen/ri-online.html}{Regesta Imperii}, \href{https://correspsearch.net}{correspSearch}, \href{http://www.blumenbach-online.de}{Johann Friedrich Blumenbach - Online}\subsection*{Themen:}Einführung, Digitale Editionswissenschaft\subsection*{Lexika}\begin{itemize}\item \href{https://edlex.de/index.php?title=Regestausgabe}{Edlex: Editionslexikon}\end{itemize}\subsection*{Zitiervorschlag:}Rieger, Lisa. 2021. Regestausgabe. In: KONDE Weißbuch. Hrsg. v. Helmut W. Klug unter Mitarbeit von Selina Galka und Elisabeth Steiner im HRSM Projekt "Kompetenznetzwerk Digitale Edition". URL: https://gams.uni-graz.at/o:konde.162\newpage\section*{Relax NG} \emph{Galka, Selina; selina.galka@uni-graz.at }\\
        
    Relax NG oder auch RNG (\emph{Regular Language Description for XML New
                     Generation}) ist eine \href{http://gams.uni-graz.at/o:konde.166}{Schemasprache} für XML-Dokumente. Es handelt sich dabei um eine sehr
                  mächtige Schemasprache, deren Syntax sich an \href{http://gams.uni-graz.at/o:konde.215}{XML} orientiert. Relax NG existiert in zwei
                  verschiedenen Versionen, einerseits in XML (.xml), andererseits in einer etwas
                  kompakteren Version (.rnc). Durch die Anwendung unterschiedlicher
                  Annotationssprachen (\href{http://gams.uni-graz.at/o:konde.178}{TEI}, \emph{OpenDocument}) ist die Schemasprache weit verbreitet.\\
            
        In der Relax NG Compact-Syntax werden sogenannte Regelstatements definiert, die in
                  Form von \emph{named patterns} geschrieben werden. Ein Beispiel
                  für ein XML-\emph{Snippet}:\\
            
        \begin{verbatim}<persons>
    <title></title>
    <person></person>
</persons>\end{verbatim}und ein mögliches dazugehöriges RNC:\\
            
        \begin{verbatim}start = persons
persons = element persons {title, persons}\end{verbatim}Es wird also zunächst das Wurzelelement definiert, danach die Elemente mit deren
                  Inhaltsmodellen, die in geschwungenen Klammern notiert werden. Im Inhaltsmodell
                  wird festgelegt, was innerhalb von Elementen und Attributen erlaubt ist, wobei
                  hier auch die Reihenfolge und die Anzahl berücksichtigt werden. Außerdem ist die
                  Angabe von Kardinalitäten und Gruppierungen möglich, wodurch sehr präzise Regeln
                  definiert werden können.\\
            
        XML-Editoren wie \emph{Oxygen} erlauben das Einbinden von
                  Schemata, wodurch XML-Dokumente gegen das jeweilige Schema validiert werden
                  können.\\
            
        \subsection*{Literatur:}\begin{itemize}\item Vogeler, Georg; Sahle, Patrick: XML. In: Digital Humanities. Eine Einführung. Stuttgart: 2017, S. 128–148.\item Relax NG. URL: \url{https://relaxng.org}\item Relax NG Compact Tutorial. URL: \url{https://relaxng.org/compact-tutorial-20030326.html}\item Guide to Schema Writing with Relax NG. URL: \url{https://dh.newtfire.org/explainRNG.html}\end{itemize}\subsection*{Software:}\href{http://oxygenxml.com/}{Oxygen}\subsection*{Projekte:}\href{https://relaxng.org}{RELAX NG}\subsection*{Verweise:}\href{https://gams.uni-graz.at/o:konde.166}{Schema}, \href{https://gams.uni-graz.at/o:konde.215}{XML}\subsection*{Themen:}Annotation und Modellierung\subsection*{Zitiervorschlag:}Galka, Selina. 2021. Relax NG. In: KONDE Weißbuch. Hrsg. v. Helmut W. Klug unter Mitarbeit von Selina Galka und Elisabeth Steiner im HRSM Projekt "Kompetenznetzwerk Digitale Edition". URL: https://gams.uni-graz.at/o:konde.163\newpage\section*{Responsive Web Design} \emph{Galka, Selina; selina.galka@uni-graz.at }\\
        
    Der Begriff \emph{Responsive Web Design} wurde 2010 vom Webdesigner Ethan Marcotte geprägt. Lange war es üblich, Webseiten statisch zu entwerfen und dieses Ergebnis genauso statisch in HTML und CSS umzusetzen; mit der steigenden Anzahl an unterschiedlichen Displaygrößen (z. B. Laptop, Smartphone, Tablet) wurde es jedoch nötig, diese Webseiten reaktionsfähig und flexibler zu gestalten. (Zillgens 2013, S. 2ff.) Es ist nicht mehr klar zu definieren, mit welcher Display-Größe die Nutzerinnen und Nutzer einer Website deren Inhalte konsumieren. (Zillgens 2013, S. 12)\\
            
        Die grundlegenden Bausteine für ein flexibles Webdesign sind ein flexibles Gestaltungsraster (\emph{Grid}), flexible Bilder und Medien sowie \emph{Mediaqueries} (ein Modul aus der CSS3-Spezifikation). \\
            
        Bei der Umsetzung eines Projektes ist auch der Ausgangspunkt des Designs je nach Zielgruppe der Website zu bedenken – \emph{\href{http://gams.uni-graz.at/o:konde.134}{mobile first}} oder \emph{desktop first}.\\
            
        Responsive Webseiten werden häufig mit CSS-Frameworks gestaltet, wie z. B. mit \emph{Bootstrap}. Derartige Frameworks stellen Komponenten und Funktionen bereit, die sich leicht an unterschiedliche Displaygrößen und Auflösungen anpassen lassen.\\
            
        \subsection*{Literatur:}\begin{itemize}\item Marcotte, Ethan: Responsive Webdesign. New York: 2010.\item Responsive Webdesign. URL: \url{https://de.wikipedia.org/wiki/Responsive_Webdesign}\item Zillgens, Christoph: Responsive Webdesign - Reaktionsfähige Websites gestalten und umsetzen. München: 2013.\end{itemize}\subsection*{Software:}\href{https://getbootstrap.com/}{Bootstrap}\subsection*{Verweise:}\href{https://gams.uni-graz.at/o:konde.56}{Design}, \href{https://gams.uni-graz.at/o:konde.134}{mobile first}\subsection*{Themen:}Interfaces\subsection*{Zitiervorschlag:}Galka, Selina. 2021. Responsive Web Design. In: KONDE Weißbuch. Hrsg. v. Helmut W. Klug unter Mitarbeit von Selina Galka und Elisabeth Steiner im HRSM Projekt "Kompetenznetzwerk Digitale Edition". URL: https://gams.uni-graz.at/o:konde.164\newpage\section*{Ressourcenerschließung mit Normdaten in Archiven und Bibliotheken (RNAB)} \emph{Lenhart, Elmar; elmar.lenhart@aau.at   }\\
        
    Die RNAB ist ein Regelwerk zur Erschließung von Personen-, Familien- und Körperschaftsarchiven sowie Sammlungen, das im Jänner 2019 verabschiedet wurde. Es ersetzte die bis dahin gültigen \emph{Regeln zur Erschließung von Nachlässen und Autographen} (RNA).\\
            
        Im Jahr 2010 wurden für publizierte Werke die Regeln \emph{Resource Description and Access} (RDA) verabschiedet und stellten damit einen internationalen bibliothekarischen Standard der Erschließung dar. Literaturarchive waren ab 2014 als erster Sondermaterialien-Bereich in der Organisationsstruktur des RDA-Standardisierungsausschusses vertreten. In etwa zur gleichen Zeit etablierten sich das \emph{Library Reference Model} (LRM) und die \emph{Records in Context} (RiC-CM) als moderne Metadatenmodelle. Die Entwicklung der RNAB orientierte sich an diesen bibliothekarischen Standards und ihren Methoden, aufbauend auf den Beschreibungsstandards für Archive, den ISAD(G).\\
            
        Grundgedanke und Ziele: Im Zentrum der RNAB stehen Ressourcen, also Informationsobjekte, die bestimmt, geordnet und verzeichnet werden. Die erzeugten Datensätze sollen die Inhalte von Archiven zugänglich, recherchier- und referenzierbar machen. Im Unterschied zu früheren Regelwerken werden die Ressourcen nicht nur in ihren Merkmalen, sondern auch in ihren mehrdimensionalen Relationen beschrieben. Entscheidend ist hier der Einsatz von \href{http://gams.uni-graz.at/o:konde.147}{Normdaten} und von kontrollierten Vokabularien zur standardisierten Beschreibung von Objekten. Die RNAB soll auch sicherstellen, dass bisher geleistete Erschließungen innerhalb der neuen Regeln gültig bleiben und hat eine weitgehende Interoperabilität mit anderen verwendeten Standards zum Ziel.\\
            
        Aufbau: Die RNAB ist in drei Abschnitte eingeteilt. Der erste Abschnitt behandelt den Umgang mit archivalischen Beständen, ihre Ordnung und das Vokabular zur Beschreibung der Ressourcenarten sowie Standards der Verzeichnung und die Anbindung an Normen und Normdatensätze wie ISO-Normen und \href{http://gams.uni-graz.at/o:konde.109}{GND}. Im zweiten Abschnitt sind 22 Regeln gelistet, die sich allen Verzeichnungskategorien widmen. Kernelemente liefern obligatorische Informationen nach RDA, Zusatzelemente enthalten weitere Informationen zur jeweiligen Ressource. Jeder Regel sind Beispiele angefügt. Der dritte Abschnitt enthält Listen kontrollierter Vokabulare, insbesondere Beziehungskennzeichnungen und ausführliche Glossare.\\
            
        \subsection*{Literatur:}\begin{itemize}\item Ressourcenerschließung mit Normdaten in Archiven und Bibliotheken (RNAB). Richtlinie und Regeln, RalfBreslau VolkerKaukoreit RudolfProbst JuttaWeber MartinWedl: 2019. URL: \url{https://d-nb.info/1186104252/34}.\item RDA und Sondermaterialien. URL: \url{https://wiki.dnb.de/display/RDAINFO/AG+RNAB+%7C+RDA+und+Sondermaterialien}\end{itemize}\subsection*{Verweise:}\href{https://gams.uni-graz.at/o:konde.147}{Normdaten}, \href{https://gams.uni-graz.at/o:konde.109}{GND}\subsection*{Themen:}Metadaten, Archivierung\subsection*{Zitiervorschlag:}Lenhart, Elmar. 2021. Ressourcenerschließung mit Normdaten in Archiven und Bibliotheken (RNAB). In: KONDE Weißbuch. Hrsg. v. Helmut W. Klug unter Mitarbeit von Selina Galka und Elisabeth Steiner im HRSM Projekt "Kompetenznetzwerk Digitale Edition". URL: https://gams.uni-graz.at/o:konde.165\newpage\section*{Schema} \emph{Galka, Selina; selina.galka@uni-graz.at }\\
        
    Mittels \href{http://gams.uni-graz.at/o:konde.215}{XML} können Dokumente und Daten strukturiert dargestellt werden; es ist zunächst aber nicht festgelegt, welche Elemente und Elementnamen bei der Auszeichnung verwendet werden. Schemata definieren mögliche Elemente, ihre Attribute und Werte eines XML-Dokuments. Außerdem kann auch definiert werden, wie diese Elemente und Attribute zu verwenden sind (z. B. obligatorisch, optional). Mit einem Schema kann also eine bestimmte Logik und Struktur der Texte und Daten festgelegt und ein Modell kodifiziert werden. (Vogeler/Sahle 2017, S. 135)\\
            
        XML-Dokumente brauchen grundsätzlich kein Schema, aber ohne eines können die Elemente, Attribute und ihre Werte nicht auf ihre Richtigkeit überprüft werden, die Daten sind nicht interpretierbar, Eingabefehler können nicht überprüft werden und der Datenaustausch gestaltet sich schwierig. Neben der Wohlgeformtheit von XML-Dokumenten wird in der Regel auch die Validität überprüft – ein XML-Dokument ist valide, wenn es seinem Schema entspricht. Schemata sind somit für Editionsprojekte relevant, um die Qualität der Daten zu sichern. \\
            
        Schemata können in unterschiedlichen Sprachen ausgedrückt sein. Zur Verfügung stehen beispielsweise DTD, \href{http://gams.uni-graz.at/o:konde.163}{Relax NG} (auch: RNG) oder XML Schema (auch: XSD), wobei sich die Schemasprachen in ihrem Funktionsumfang unterscheiden. Relax NG ist eine sehr mächtige Schemasprache, deren Syntax sich an XML orientiert. XML Schema ist seit 2001 Empfehlung des W3C und wird ebenfalls in XML notiert; es ist möglich, die Struktur eines XML-Dokuments differenziert zu beschreiben, zwischen unterschiedlichen Datentypen zu unterscheiden und Kardinalitätsangaben zu machen. XML Schema ist umfangreicher als DTD – DTD ist zwar leicht zu lesen, wird aber kaum mehr verwendet, weil es nicht in XML notiert wird und und nicht sonderlich ausdrucksmächtig ist. (Vogeler/Sahle 2017, S. 135f.)\\
            
        XML-Editoren, wie z. B. \emph{Oxygen}, können das Zuweisen von Schemata ermöglichen und stellen ein Werkzeug zum Validieren bereit.\\
            
        Der Standard der \href{http://gams.uni-graz.at/o:konde.178}{TEI} wird in der Schema-Metasprache \href{http://gams.uni-graz.at/o:konde.150}{ODD} definiert.\\
            
        \subsection*{Literatur:}\begin{itemize}\item Relax NG. URL: \url{https://relaxng.org}\item DTD Tutorial. URL: \url{https://www.w3schools.com/xml/xml_dtd_intro.asp}\item XML Schema Tutorial. URL: \url{https://www.w3schools.com/xml/schema_intro.asp}\item Vogeler, Georg; Sahle, Patrick: XML. In: Digital Humanities. Eine Einführung. Stuttgart: 2017, S. 128–148.\item Burnard, Lou: Customizing the TEI - OpenEdition Press. In: What is the Text Encoding Initiative?: 2014.\end{itemize}\subsection*{Software:}\href{http://oxygenxml.com/}{Oxygen}\subsection*{Verweise:}\href{https://gams.uni-graz.at/o:konde.215}{XML}, \href{https://gams.uni-graz.at/o:konde.163}{Relax NG}, \href{https://gams.uni-graz.at/o:konde.150}{ODD}\subsection*{Themen:}Annotation und Modellierung\subsection*{Zitiervorschlag:}Galka, Selina. 2021. Schema. In: KONDE Weißbuch. Hrsg. v. Helmut W. Klug unter Mitarbeit von Selina Galka und Elisabeth Steiner im HRSM Projekt "Kompetenznetzwerk Digitale Edition". URL: https://gams.uni-graz.at/o:konde.166\newpage\section*{Semantic Web} \emph{Hinkelmanns, Peter; peter.hinkelmanns@sbg.ac.at }\\
        
    Mit dem \emph{Semantic Web} wird das reguläre World Wide Web um
                  semantische Informationen ergänzt, die einen globalen Informationsgraphen ergeben.
                  Geprägt wurde die Idee des \emph{Semantic Webs} durch Tim
                  Berners-Lee (Berners-Lee/Fischetti 2000), eine ausführliche
                  Einführung bieten etwa Dean Allemang und Jim Hendler (Allemang/Hendler
                     2011).\\
            
        Der Informationsgraph besteht aus Aussagen zu den unterschiedlichsten Entitäten.
                     \href{http://gams.uni-graz.at/o:konde.147}{Normdaten}, etwa zu Personen,
                  Orten oder Werken, stellen Projekte wie die \href{http://gams.uni-graz.at/o:konde.109}{Gemeinsame Normdatei} (GND) (Gemeinsame
                     Normdatei 2019) oder \href{http://gams.uni-graz.at/o:konde.112}{Wikidata}(Wikidata 2019) bereit.\\
            
        Eine Information besteht dabei immer aus einem Tripel, also einer
                  Subjekt-Prädikat-Objekt-Beziehung. Beispielsweise:\\
            
        \begin{itemize}\item {[Stefan Zweig] [ist] [Person]}\item {[Stefan Zweig] [ist Autor von] [Schachnovelle]}\item {[Schachnovelle] [ist] [Literarisches Werk]}\end{itemize}Aus diesen Triples kann ein gerichteter Graph aufgebaut werden, sodass die
                  Subjekte und Objekte die Knoten und die Prädikate die Kanten bilden. Die
                  Ausrichtung der Kanten erfolgt dabei immer vom Subjekt auf das Objekt:\\
            
        Abbildung 1: Graph aus drei Triples\\
            
        Übergreifende Abfragen im \emph{Semantic Web} funktionieren dann,
                  wenn im gesamten Netz einheitliche Identifikatoren, genannt URI\emph{
                     – Uniform Resource Identifier}, genutzt werden (siehe auch \emph{\href{http://gams.uni-graz.at/o:konde.12}{Persistent Identifier}}). Um diese notwendige Einheit zu erzielen, ermöglichen \emph{\href{http://gams.uni-graz.at/o:konde.151}{Ontologien}} die Erstellung von Regeln für \emph{Triples}. Im obigen
                  Beispiel könnten etwa die Prädikate ‘ist’ und ‘ist Autor von’ und die beiden
                  Objekte ‘Person’ und ‘Literarisches Werk’ in einer solchen Ontologie beschrieben
                  werden. Eine Regel könnte sein, dass ‘ist Autor von’ nur von einer ‘Person’ zu
                  einem ‘literarischen Werk’ zeigen darf. Mit diesen wenigen, definierten
                  Eigenschaften können nun weitere Datensätze zu Autorinnen und Autoren und deren
                  Werken erstellt werden. Nun könnte etwa abgefragt werden, welche Autorinnen und
                  Autoren existieren und wie viele Werke sie jeweils verfasst haben.\\
            
        Ein anderes Projekt könnte diese Daten ergänzen, indem es etwa die Lebensdaten
                  Stefan Zweigs aufnimmt und sich dabei auf denselben Knoten ‘Stefan Zweig’ bezieht.
                  Die Gesamtheit dieser Informationen wird als \emph{\href{http://gams.uni-graz.at/o:konde.8}{Linked Data}}\emph{ (LD)} oder \emph{\href{http://gams.uni-graz.at/o:konde.8}{Linked open Data}} (LoD) bezeichnet. Wichtigstes Grundprinzip ist, dass im \emph{Semantic Web}\textbf{a} lle Teilnehmerinnen und Teilnehmer \textbf{a} lles über \textbf{a} lle Themen sagen können (AAA-Modell).\\
            
        \subsection*{Literatur:}\begin{itemize}\item Allemang, Dean; Hendler, Jim: What is the Semantic Web? In: Semantic Web for the Working Ontologist: 2011, S. 1-12.\item Berners-Lee, Tim; Fischetti, Mark: Weaving the Web: the original design and ultimate
                              destiny of the World Wide Web by its inventor Weaving the Web. San Francisco: 1999.\item Gemeinsame Normdatei. URL: \url{https://www.dnb.de/gnd}\end{itemize}\subsection*{Verweise:}\href{https://gams.uni-graz.at/o:konde.168}{Semantic Web-Technologien}, \href{https://gams.uni-graz.at/o:konde.8}{Linked Open Data}, \href{https://gams.uni-graz.at/o:konde.147}{Normdaten}, \href{https://gams.uni-graz.at/o:konde.12}{Persistent Identifier}, \href{https://gams.uni-graz.at/o:konde.109}{Kontrollierte Vokabularien:
                           GND}, \href{https://gams.uni-graz.at/o:konde.112}{Kontrollierte Vokabularien:
                           Wikidata}, \href{https://gams.uni-graz.at/o:konde.133}{CIDOC}, \href{https://gams.uni-graz.at/o:konde.131}{RDF}\subsection*{Themen:}Einführung, Annotation und Modellierung\subsection*{Projekte:}\href{https://www.wikidata.org/wiki/Wikidata:Main_Page}{Wikidata}\subsection*{Lexika}\begin{itemize}\item \href{https://edlex.de/index.php?title=Semantic_Web}{Edlex: Editionslexikon}\end{itemize}\subsection*{Zitiervorschlag:}Hinkelmanns, Peter. 2021. Semantic Web. In: KONDE Weißbuch. Hrsg. v. Helmut W. Klug unter Mitarbeit von Selina Galka und Elisabeth Steiner im HRSM Projekt "Kompetenznetzwerk Digitale Edition". URL: https://gams.uni-graz.at/o:konde.167\newpage\section*{Semantic Web-Technologien} \emph{Hinkelmanns, Peter; peter.hinkelmanns@sbg.ac.at / Zangerl, Lina Maria; linamaria.zangerl@sbg.ac.at }\\
        
    Das \emph{\href{http://gams.uni-graz.at/o:konde.167}{Semantic Web}} baut auf denselben Technologien wie das reguläre Internet auf, ergänzt diese aber. Basismodell ist das \emph{Resource Description Framework }(\href{http://gams.uni-graz.at/o:konde.131}{RDF}) (Cyganiak/Wood/Lanthaler 2014), das es ermöglicht, Ressourcen mittels einfacher \emph{Triples} (Subjekt – Prädikat – Objekt) zu beschreiben. Auf RDF basieren auch die Sprachen \emph{RDF Schema} (RDFS) (Brickley/Guha 2014) und die \emph{Web Ontology Language} (OWL) (Motik/Patel-Schneider/Parsia 2012), die zur Beschreibung von \href{http://gams.uni-graz.at/o:konde.151}{Ontologien} verwendet werden können. In einer Ontologie werden die Beziehungen zwischen Knoten und deren Typisierung beschrieben. Beispielsweise bietet \emph{Friend of a Friend} (FOAF) (Brickley/Miller 2014), die Möglichkeit, Personen und deren Beziehungen zueinander zu beschreiben. \\
            
        Ontologien gehen von einer \emph{Open World Assumption} aus, was bedeutet, dass es grundsätzlich keine vollständige Kenntnis über die im \emph{Semantic Web} enthaltenen Daten geben kann. Eine Validierung bestehender Datensätze gegen eine Ontologie ist also nicht möglich. Ob ein Datensatz konform zu einer Vorlage ist, kann daher mittels der \emph{Shapes Constraint Language} (SHACL) (Knublauch/Kontokostas 2017) ermittelt werden.\\
            
        Eine weitere technische Grundlage des \emph{Semantic Web} sind kontrollierte Vokabulare bzw. Thesauri. Mit ihnen können strukturierte Klassifikationen, also etwa eine Systematik der Pflanzenarten, angelegt werden. Die maßgebliche Ontologie, nach der strukturierte Vokabulare angelegt werden können, ist das \emph{Simple Knowledge Organisation System} (\href{http://gams.uni-graz.at/o:konde.132}{SKOS}). (Miles/Bechhofer 2009)\\
            
        Verwaltet werden RDF-Daten in \emph{Triple Stores}, also Graph-Datenbanken. Einige verbreitete Produkte sind \emph{Apache Jena}, \emph{GraphDB} oder \emph{AllegroGraph}. Mit der \emph{SPARQL Protocol And RDF Query Language} (SPARQL) (Harris/Seaborne 2013) existiert eine umfangreiche Abfragesprache für diese Datenbanken, die auch die Bearbeitung und Erstellung von Graphen ermöglicht.\\
            
        \subsection*{Literatur:}\begin{itemize}\item RDF Schema 1.1 RDFS. URL: \url{https://www.w3.org/TR/rdf-schema/}\item FOAF Vocabulary Specification 0.99 FOAF. URL: \url{http://xmlns.com/foaf/spec/}\item RDF 1.1 Concepts and Abstract Syntax. URL: \url{https://www.w3.org/TR/rdf11-concepts/}\item SPARQL 1.1 Query Language SPARQL. URL: \url{http://www.w3.org/TR/sparql11-query/}\item Shapes Constraint Language (SHACL) SHACL. URL: \url{https://www.w3.org/TR/shacl/}\item SKOS Simple Knowledge Organization System Reference SKOS. URL: \url{http://www.w3.org/TR/skos-reference}\item OWL 2 Web Ontology Language Structural Specification and Functional-Style Syntax (Second Edition) OWL. URL: \url{https://www.w3.org/TR/owl2-syntax/}\end{itemize}\subsection*{Software:}\href{https://jena.apache.org/}{Apache Jena}, \href{https://www.blazegraph.com/}{BlazeGraph}, \href{https://www.ontotext.com/graphdb‎}{GraphDB}, \href{https://franz.com/agraph/allegrograph/}{AllegroGraph}\subsection*{Verweise:}\href{https://gams.uni-graz.at/o:konde.167}{Semantic Web}, \href{https://gams.uni-graz.at/o:konde.8}{Linked Open Data}, \href{https://gams.uni-graz.at/o:konde.147}{Normdaten}, \href{https://gams.uni-graz.at/o:konde.131}{RDF}, \href{https://gams.uni-graz.at/o:konde.132}{SKOS}\subsection*{Themen:}Einführung, Annotation und Modellierung\subsection*{Zitiervorschlag:}Hinkelmanns, Peter; Zangerl, Lina Maria. 2021. Semantic Web-Technologien. In: KONDE Weißbuch. Hrsg. v. Helmut W. Klug unter Mitarbeit von Selina Galka und Elisabeth Steiner im HRSM Projekt "Kompetenznetzwerk Digitale Edition". URL: https://gams.uni-graz.at/o:konde.168\newpage\section*{Social Edition} \emph{Klug, Helmut W.; helmut.klug@uni-graz.at }\\
        
    Charakteristisch für diesen Editionstyp ist die Einbeziehung irgendeiner Art von ‘Öffentlichkeit’, welche die Editorin oder den Editor oder das Editionsteam bei der Erstellung einer Edition unterstützt. Dieser Personenkreis kann von einem erweiterten Kreis von Fachleuten bis hin zu \emph{\href{http://gams.uni-graz.at/o:konde.47}{Crowdsourcing}} reichen. \href{http://gams.uni-graz.at/o:konde.59}{Digitale Editionen} eignen sich für dieses Modell besonders, da entweder spezielle Onlinesoftware oder im Rahmen des Webauftritts unterschiedlichste Social Media und Kollaborationstools eingesetzt werden können. Die Herausforderung bei einem derartigen Editionsunternehmen wird weniger die technische Umsetzung als die kontinuierliche Motivation der Beteiligten sowie das Erreichen bzw. die Aufrechterhaltung eines entsprechenden Qualitätsstandards sein. \\
            
        \subsection*{Literatur:}\begin{itemize}\item Crompton, Constance; Arbuckle, Alyssa; Siemens, Raymond: Understanding the Social Edition Through Iterative Implementation: The Case of the Devonshire MS (BL Add MS 17492) Understanding the Social Edition Through Iterative Implementation. In: Scholarly and Research Communication 4: 2013.\item Driscoll, Matthew James; Pierazzo, Elena: Introduction: Old Wine in New Bottles? In: Digital Scholarly Editing. Theories and Practices. Cambrige: 2016, S. 1-15.\item Price, Kenneth M.: Social Scholarly Editing. In: A New Companion to Digital Humanities. Chichester: 2016, S. 137–149.\item Sahle, Patrick: What is a Scholarly Digital Edition. In: Digital Scholarly Editing: Theories and Practices. Cambridge: 2016.\item Siemens, R.; Timney, M.; Leitch, C.; Koolen, C.; Garnett, A.; with the ETCL, INKE, and PKP Research Groups: Toward modeling the social edition: An approach to understanding the electronic scholarly edition in the context of new and emerging social media Toward modeling the social edition. In: Literary and Linguistic Computing 27: 2012, S. 445–461.\item Shillingsburg, Peter: Is Reliable-Social-Scholarly-Editing an Oxymoron. In: Center for Textual Studies and Digital Humanities Publications: 2013.\end{itemize}\subsection*{Software:}\href{http://transcribe-bentham.ucl.ac.uk/td/Transcribe_Bentham}{Bentham Transcription Desk}, \href{https://diyhistory.lib.uiowa.edu}{Civil War Diaries & Letters Transcription Project}, \href{https://github.com/gsbodine/crowd-ed}{Crowd-Ed}, \href{http://www.ala.org.au/get-involved/citizen-science/fielddata-software/}{FieldData}, \href{https://fromthepage.com/}{FromThePage}, \href{http://edgerton-digital-collections.org/notebooks}{Harold "Doc" Edgerton Project}, \href{http://www.digiverso.com/de/products/viewer}{Gobi viewer}, \href{https://islandora.ca/}{Citizen Science, Collaboration}, \href{http://www.mom-wiki.uni-koeln.de/}{Itineranova-Editor}, \href{http://pybossa.com/}{PyBOSSA}, \href{http://github.com/zooniverse/Scribe}{Scribe}, \href{http://scripto.org/}{scripto}, \href{https://textgrid.de/}{TextGrid}, \href{https://textualcommunities.org/app/}{Textual Communities}, \href{https://transkribus.eu/Transkribus/}{Transkribus}, \href{http://bencrowder.net/coding/unbindery/}{Unbindery}, \href{http://menus.nypl.org/}{What's On the Menu?}, \href{http://en.wikisource.org/wiki/Main_Page}{Wikisource}\subsection*{Verweise:}\href{https://gams.uni-graz.at/o:konde.47}{Crowdsourcing}, \href{https://gams.uni-graz.at/o:konde.104}{Kollaboration}, \href{https://gams.uni-graz.at/o:konde.41}{Citizen Science}, \href{https://gams.uni-graz.at/o:konde.89}{Gamification}\subsection*{Projekte:}\href{https://en.wikibooks.org/wiki/The_Devonshire_Manuscript}{A Social Edition of the Devonshire MS (BL Add. MS 17492)}, \href{https://www.ucl.ac.uk/bentham-project/}{Transcribe Bentham Project}, \href{http://cds.library.brown.edu/projects/pico/}{Conclusiones CM}\subsection*{Themen:}Einführung, Digitale Editionswissenschaft\subsection*{Lexika}\begin{itemize}\item \href{https://edlex.de/index.php?title=Social_Editing}{Edlex: Editionslexikon}\end{itemize}\subsection*{Zitiervorschlag:}Klug, Helmut W. 2021. Social Edition. In: KONDE Weißbuch. Hrsg. v. Helmut W. Klug unter Mitarbeit von Selina Galka und Elisabeth Steiner im HRSM Projekt "Kompetenznetzwerk Digitale Edition". URL: https://gams.uni-graz.at/o:konde.169\newpage\section*{Stand-off-Markup} \emph{Klug, Helmut W.; helmut.klug@uni-graz.at}\\
        
    Als \emph{Stand-off-} oder externes \href{http://gams.uni-graz.at/o:konde.126}{Markup} wird die \href{http://gams.uni-graz.at/o:konde.17}{Annotation} von Primärdaten (Text- und Binärdateien) bezeichnet, wenn die Annotationen getrennt von den Primärdaten gespeichert werden. Die Idee dahinter ist es, zum einen binäre Dateien annotieren zu können, in die kein Markup eingefügt werden kann, zum anderen können parallel zur Textebene mehrere Annotationsebenen vorhanden sein, die sich als \emph{Inline-}Markup in \href{http://gams.uni-graz.at/o:konde.215}{XML} überlappen würden. Der Primärtext wird auf diese Art nicht oder nur geringfügig durch das eingefügte Markup verändert. Prinzipiell sollten Primärdaten und Annotation auch physisch (in unterschiedlichen Dateien) voneinander getrennt werden. Wenn alle Daten in einer Datei gespeichert werden, sind die Annotationen von den Primärdaten z. B. in der \href{http://gams.uni-graz.at/o:konde.178}{TEI} durch die XML-Hierarchie getrennt: <standOff> als Geschwisterelement von <text>. \\
            
        Von zentraler Bedeutung für diese Art der Annotation ist die Verknüpfung von Primärdaten mit den Annotationen. Das kann z. B. über \emph{XInclude} und \emph{XPointer}(vgl. TEI P5, Ch. 16.9f.) oder über entsprechende IDs, die vordefinierten Primärdateneinheiten (Indexierung auf Zeichen-, Wortebene, …) zugewiesen sind, erfolgen. Das Referenzsystem muss jedenfalls stabil und idealerweise auf nachträgliche Änderungen in den Primärdaten ausgerichtet sein. \emph{Stand-off-}Markup kann für unterschiedliche Anwendungsfälle geeignet sein, aktiv eingesetzt wird es z. B. im Rahmen der \emph{Music Encoding Initiative} (MEI) für Codierung von Dynamik, Artikulation, Ornamentation u. dgl. oder in den Bibelwissenschaften als \emph{Open Scripture Information Standard} (OSIS) und \emph{Theological Markup Language} (ThML).\\
            
        \emph{Stand-off-}Markup wird seit den 1990ern diskutiert. Burghardt/Wolff fassen die Entwicklungen und die Überlegungen zu einer Standardisierung für literatur- und sprachwissenschaftliche Anwendungen zusammen und formulieren dazu grundlegende Empfehlungen  (Burghardt/Wolff 2009, S. 57):\\
            
        \begin{itemize}\item {Der Originaltext sollte in seinem ursprünglichen Zustand im Dateisystem der Annotationsdatei gesichert werden, um die Lesbarkeit und Wiederverwendbarkeit zu gewährleisten.}\item {Die Indexierung des Originaltextes sollte in einer gesonderten Datei gespeichert werden.}\item {Die Annotationssoftware, die das \emph{stand-off}-Format generiert, sollte Synchronisierungsmecha­nismen enthalten, die es erlauben[,] den Originaltext auch während des laufenden Annotationspro­zesses zu ändern.}\item {Die Software sollte Versionskontrolle und Änderungshistorie der Primärdaten unterstützen.}\item {Bei der Indexierung sollte der Text am besten zeichenweise erfasst werden, da so später beliebig feinkörnige Annotationen hinzugefügt werden können.}\item {Die Speicherung von Originaltext und Annotation in unterschiedlichen Dateien erhöht die Lesbarkeit und ermöglicht die Konservierung der Primärdaten.}\end{itemize}\subsection*{Literatur:}\begin{itemize}\item Stand-off Markup. URL: \url{https://beethovens-werkstatt.de/glossary/stand-off-markup/}\item Burghardt, Manuel; Wolff, Christian: Stand off-Annotation für Textdokumente: Vom Konzept zur Implementierung (zur Standardisierung?). In: Von der Form zur Bedeutung: Texte automatisch verarbeiten / From Form to Meaning: Processing Texts Automatically. Narr: 2009.\item Spadini, Elena; Turska, Magdalena: XML-TEI Stand-off Markup: One Step Beyond. In: Digital Philology: A Journal of Medieval Cultures 8: 2019, S. 225–239.\item 16 Linking, Segmentation, and Alignment. URL: \url{https://tei-c.org/release/doc/tei-p5-doc/en/html/SA.html}\item Vigilante, Raffaele: Why TEI Stand-off Markup Authoring Needs Simplification. In: JTEI 2016–2019.\end{itemize}\subsection*{Verweise:}\href{https://gams.uni-graz.at/o:konde.52}{Datenmodell "MHDBDB"}, \href{https://gams.uni-graz.at/o:konde.185}{Textformate: GrAF}, \href{https://gams.uni-graz.at/o:konde.17}{Annotation (grundsätzlich)}, \href{https://gams.uni-graz.at/o:konde.139}{Digitale Musikedition}, \href{https://gams.uni-graz.at/o:konde.126}{Markup}\subsection*{Software:}\href{http://www.cs.unibo.it/projects/xslt%2B%2B/XPointerTest.html}{XPointer - Test implementation}, \href{http://www.anc.org/tools/using-gate.html}{GATE for ANC}\subsection*{Projekte:}\href{https://music-encoding.org}{Music Encoding Initiative (MEI) }, \href{https://www.w3.org/TR/xinclude/}{XML Inclusions (XInclude)}, \href{https://www.w3.org/TR/xptr-framework/}{XPointer Framework}, \href{https://crosswire.org/osis/}{Open Scripture Information Standard (OSIS)}, \href{https://www.ccel.org/ThML/index.html}{Theological Markup Language (ThML)}\subsection*{Themen:}Annotation und Modellierung\subsection*{Lexika}\begin{itemize}\item \href{https://lexiconse.uantwerpen.be/index.php/lexicon/markup-standoff/}{Lexicon of Scholarly Editing}\end{itemize}\subsection*{Zitiervorschlag:}Klug, Helmut W. 2021. Stand-off-Markup. In: KONDE Weißbuch. Hrsg. v. Helmut W. Klug unter Mitarbeit von Selina Galka und Elisabeth Steiner im HRSM Projekt "Kompetenznetzwerk Digitale Edition". URL: https://gams.uni-graz.at/o:konde.171\newpage\section*{Stemmatologie} \emph{Andrews, Tara; tara.andrews@univie.ac.at }\\
        
    Stemmatologie ist eine Subdisziplin der Textwissenschaften, die sich mit der
                  Ermittlung der wahrscheinlichsten Überlieferungsabhängigkeiten eines Textes
                  befasst. Im Falle von vormodernen Texten bedeutet dies meist die Erstellung eines
                  Stammbaums (Stemma), der anzeigt, von welchem Exemplar (oder welchen Exemplaren)
                  Abschriften angefertigt wurden. Der Begriff wird auch mitunter im Kontext \href{http://gams.uni-graz.at/o:konde.90}{genetischer Textkritik} verwendet und
                  bezeichnet dort die Reihenfolge der von einer einzigen Autorin oder einem einzigen
                  Autor erzeugten Redaktionen. \\
            
        Die traditionelle Methode zur Erstellung von Stemmata wird (nach Karl Lachmann,
                  einem deutschen Philologen des 19. Jahrhunderts) ‘Lachmannsche Methode’ genannt.
                  Sie beruht auf der Vorstellung, sogenannte ‘wahre Lesarten’ des ursprünglichen
                  Textes von ‘Fehlern’ oder Änderungen späterer Abschreiberinnen und Abschreiber
                  unterscheiden zu können, seien sie absichtlich vorgenommen worden oder
                  unabsichtlich zustande gekommen. Wenn eine diesbezügliche Einschätzung vor der
                  Texterstellung nicht möglich oder angebracht erscheint, oder wenn die Zahl der
                  Überlieferungen besonders hoch ist, greift man heute vermehrt auf statistische
                  Methoden zurück, um das Verhältnis zwischen den Handschriften zu bestimmen.
                  Während es üblich ist, breit zugängliche Software zu verwenden, die ursprünglich
                  für Evolutionsbiologie geschrieben wurde (etwa \emph{Phylip
                        PARS} oder \emph{SplitsTree}), existieren auch
                  Algorithmen wie RHM (Roos et al. 2006) sowie die Leitfehler-basierte
                  Methode (Roelli 2014), die mit Blick auf die speziellen Bedürfnisse
                  der Textwissenschaften entwickelt worden sind.\\
            
        \subsection*{Literatur:}\begin{itemize}\item Howe, Christopher J; Connolly, Ruth; Windram, Heather F: Responding to Criticisms of Phylogenetic Methods in
                              Stemmatology. In: Studies in English Literature 1500-1900 52: 2012, S. 51–67.\item Parvum Lexicon Stemmatologicum. URL: \url{https://wiki.helsinki.fi/display/stemmatology/Parvum+lexicon+stemmatologicum}\item Mooney, Linne Ruth; Robinson, Peter; Howe, Christopher J.; Barbrook, Adrian C.: Parallels Between Stemmatology and
                              phylogenetics. In: Studies in Stemmatology. Amsterdam: 2004, S. 3-11.\item Roelli, Philipp: Petrus Alfonsi or On the mutual benefit of traditional
                              and computerised stemmatology. In: Analysis of Ancient and Medieval Texts and Manuscripts:
                              Digital Approaches. Turnhaut: 2014, S. 45–70.\item Roos, Teemu; Heikkilä, Tuomas; Myllymäki, Petri; Brewka, G.; Coradeschi, S.; Perini, A.; Traverso, P.: A Compression-Based Method for Stemmatic
                              Analysis. In: Proceeding of the 2006 conference on ECAI 2006 ECAI. Amsterdam: 2006, S. 805–806.\item Trovato, Paolo: Everything You Always Wanted to Know about Lachmann's
                              Method. Padova: 2014.\end{itemize}\subsection*{Software:}\href{http://www.traviz.vizcovery.org}{TRAViz}, \href{https://stemmaweb.net/}{The Stemmaweb
                           Project}\subsection*{Projekte:}\href{https://evolution.genetics.washington.edu/phylip.html}{PHYLIP}, \href{http://www.splitstree.org}{SplitsTree}\subsection*{Verweise:}\href{https://gams.uni-graz.at/o:konde.90}{Genetische Edition}, \href{https://gams.uni-graz.at/o:konde.93}{historisch-kritische Edition}, \href{https://gams.uni-graz.at/o:konde.46}{critique genetique}\subsection*{Themen:}Digitale Editionswissenschaft\subsection*{Lexika}\begin{itemize}\item \href{https://edlex.de/index.php?title=Stemma}{Edlex: Editionslexikon}\item \href{https://wiki.helsinki.fi/display/stemmatology/Stemmatology}{Parvum Lexicon Stemmatologicum}\item \href{https://lexiconse.uantwerpen.be/index.php/lexicon/stemmatology/}{Lexicon of Scholarly Editing}\end{itemize}\subsection*{Zitiervorschlag:}Andrews, Tara. 2021. Stemmatologie. In: KONDE Weißbuch. Hrsg. v. Helmut W. Klug unter Mitarbeit von Selina Galka und Elisabeth Steiner im HRSM Projekt "Kompetenznetzwerk Digitale Edition". URL: https://gams.uni-graz.at/o:konde.172\newpage\section*{Studienausgabe} \emph{Galka, Selina; selina.galka@uni-graz.at }\\
        
    Bei einer Studienausgabe handelt es sich um einen \href{http://gams.uni-graz.at/o:konde.76}{Editionstyp}. Während \href{http://gams.uni-graz.at/o:konde.93}{historisch-kritische Ausgaben} darauf ausgelegt sind, alle Varianten eines Textes zu verzeichnen und einen Vollständigkeitsanspruch erheben, stellt die Studienausgabe einen in Umfang und Inhalt reduzierten Editionstyp dar.\\
            
        Von Studienausgaben darf keine vollständige Wiedergabe aller Textfassungen erwartet werden; die Zielsetzung ist primär Texterschließung und Textinterpretation. Es wird zunächst ein zuverlässiger, textkritisch erarbeiteter und gesicherter Text zu Verfügung gestellt. Als Kernstück ist ein \href{http://gams.uni-graz.at/o:konde.34}{Kommentar} zum Text enthalten, der nicht nur Sachverhalte im Hinblick auf das historische Textumfeld erläutert, sondern auch Hinweise auf das aktuelle Textverständnis geben kann. (Plachta 2013, S. 17f.; Erler 1980, S. 289)\\
            
        Umstritten ist, inwiefern in den historischen Text zugunsten der Lesefreundlichkeit eingegriffen werden kann und soll (\href{http://gams.uni-graz.at/o:konde.146}{Normalisierung}). (Plachta 2013, S. 18) Studienausgaben richten sich nicht an Laien, sondern an ein vorwiegend akademisches Publikum bzw. fachspezifische Interessenten. (Bein 2011, S. 73; Erler 1980, S. 288)\\
            
        Im Rahmen von \href{http://gams.uni-graz.at/o:konde.59}{Digitale Editionen} können Studienausgaben als (gedruckte) Derivate (\href{http://gams.uni-graz.at/o:konde.96}{Hybridedition}) der Datenbasis erstellt werden.\\
            
        \subsection*{Literatur:}\begin{itemize}\item Bein, Thomas: Textkritik. Eine Einführung in Grundlagen germanistisch-mediävistischer Editionswissenschaft. Lehrbuch mit Übungsteil. Frankfurt am Main: 2011.\item Erler, Gotthart: Plädoyer für einen Editionstyp. Zu einigen konzeptionellen und editionstechnischen Aspekten von Lese- und Studienausgaben. In: Zeitschrift für Germanistik 1: 1980, S. 287–298.\item Göttsche, Dirk: Ausgabentypen und Ausgabenbenutzer. In: Text und Edition - Positionen und Perspektiven. Berlin: 2000, S. 37–64.\item Plachta, Bodo: Editionswissenschaft. Eine Einführung in Methode und Praxis der Edition neuerer Texte. Stuttgart: 2013.\end{itemize}\subsection*{Verweise:}\href{https://gams.uni-graz.at/o:konde.76}{Editionstypen}, \href{https://gams.uni-graz.at/o:konde.93}{Historisch-kritische Ausgabe / Edition}, \href{https://gams.uni-graz.at/o:konde.146}{Normalisierung}, \href{https://gams.uni-graz.at/o:konde.34}{Kommentar}, \href{https://gams.uni-graz.at/o:konde.75}{Editionstext}\subsection*{Themen:}Einführung, Digitale Editionswissenschaft\subsection*{Lexika}\begin{itemize}\item \href{https://edlex.de/index.php?title=Studienausgabe}{Edlex: Editionslexikon}\end{itemize}\subsection*{Zitiervorschlag:}Galka, Selina. 2021. Studienausgabe. In: KONDE Weißbuch. Hrsg. v. Helmut W. Klug unter Mitarbeit von Selina Galka und Elisabeth Steiner im HRSM Projekt "Kompetenznetzwerk Digitale Edition". URL: https://gams.uni-graz.at/o:konde.173\newpage\section*{Synopse} \emph{Rieger, Lisa; lrieger@edu.aau.at }\\
        
    Unter Synopse versteht man im Allgemeinen eine knappe Zusammenfassung oder Übersicht, in der Literaturwissenschaft bezeichnet sie jedoch genauer die „parallele Wiedergabe versch. Fassungen eines literar. Werkes“ (Best 1991, S. 513). Dazu folgt der Grundform des Textes über alle Bearbeitungsstufen hinweg die parallele Wiedergabe sämtlicher Varianten, Korrekturen und Ergänzungen. (Plachta 1997, S. 106 f.) Da in den jeweiligen Zeilen nur die abweichenden Stellen wiedergegeben werden, kennzeichnen die Lücken in den untereinanderstehenden Zeilen jene Stellen, an denen der Text der darüberliegenden Zeile weiterhin gültig ist. (Nutt-Kofoth 2007, S. 16)\\
            
        Die synoptische Darstellung bietet sich v. a. bei Autoren an, die sämtliche Ansätze ihres Textes sofort zu Papier bringen – bei sogenannten ‘Papierarbeitern’ – und dadurch immer wieder eine große Anzahl von Veränderungen, oft sogar auf unterschiedlichen Textträgern, vornehmen. Sie wird meist für Texte in Versform herangezogen, da die vorgegebene Form des Verses von Korrekturen normalerweise unberührt bleibt, während Prosatexte über kein formales Schema verfügen und Änderungen somit auch in beliebiger Länge und Anzahl vorgenommen werden können. (Scheibe 1988, S. 143–147) Der Vorteil der synoptischen Darstellung liegt in der Darstellung vollständiger Texte, wodurch bei nur geringem Platzbedarf stets der gesamte Kontext erhalten bleibt. Durch flexible Einschubmöglichkeiten kann die Synopse zudem die \href{http://gams.uni-graz.at/o:konde.28}{Textentwicklung} chronologisch genauer darstellen als Einzeldarstellungen. Für die Darstellung stark voneinander abweichender Textträger ist die Synopse allerdings ungeeignet. (Zeller 1996, S. 99 f.)\\
            
        In \href{http://gams.uni-graz.at/o:konde.59}{Digitalen Editionen} findet man heute oft die synoptische, d. h. parallele Darstellung von \href{http://gams.uni-graz.at/o:konde.83}{Faksimile} und \href{http://gams.uni-graz.at/o:konde.75}{Text}, fallweise aber auch Textsynopsen (z. B. in den Editionen \emph{Lyrik des deutschen Mittelalters} oder \emph{Der Welsche Gast}). (Jannidis/Kohle/Rehbein 2017, S. 235) Wie bei der Buchpublikation von synoptischen Darstellungen ist auch bei der Onlinepublikation die Größe der Darstellungsfläche (Bildschirm) oft ein Problem beim \href{http://gams.uni-graz.at/o:konde.56}{Design} des \href{http://gams.uni-graz.at/o:konde.98}{Interface}.\\
            
        \subsection*{Literatur:}\begin{itemize}\item Best, Otto: Handbuch literarischer Fachbegriffe. Definitionen und Beispiele. Überarbeitete und erweiterte Ausgabe Handbuch literarischer Fachbegriffe. Frankfurt am Main: 1991.\item Jannidis, Fotis; Kohle, Hubertus: Digital Humanities. Eine Einführung. Mit Abbildungen und Grafiken Digital Humanities. Hrsg. von  und Malte Rehbein. Stuttgart: 2017.\item Nutt-Kofoth, Rüdiger: Editionsphilologie Editionsphilologie. In: Handbuch Literaturwissenschaft. Gegenstände - Konzepte - Institutionen. Stuttgart, Weimar: 2007, S. 1-27.\item Plachta, Bodo: Editionswissenschaft. Eine Einführung in Methode und Praxis der Edition neuerer Texte Editionswissenschaft: 1997.\item Scheibe, Siegfried: Zur Anwendung der synoptischen Variantendarstellung bei komplizierter Prosaüberlieferung. Mit einem Beispiel aus Franz Fühmanns "Das Judenauto" Zur Anwendung der synoptischen Variantendarstellung bei komplizierter Prosaüberlieferung. In: editio 2: 1988, S. 142–191.\item Zeller, Hans: Die synoptisch-textgenetische Darstellung. Dafür und dawider Die synoptisch-textgenetische Darstellung. In: editio 10: 1996, S. 99–115.\end{itemize}\subsection*{Verweise:}\href{https://gams.uni-graz.at/o:konde.32}{Apparat}, \href{https://gams.uni-graz.at/o:konde.28}{Textgenese}, \href{https://gams.uni-graz.at/o:konde.59}{Digitale Edition}, \href{https://gams.uni-graz.at/o:konde.75}{Editionstext}, \href{https://gams.uni-graz.at/o:konde.83}{Faksimileausgabe/edition}\subsection*{Projekte:}\href{http://www.ldm-digital.de}{Lyrik des deutschen Mittelalters}, \href{http://digi.ub.uni-heidelberg.de/wgd/}{Welscher Gast Digital}\subsection*{Themen:}Digitale Editionswissenschaft\subsection*{Lexika}\begin{itemize}\item \href{https://edlex.de/index.php?title=Synopse}{Edlex: Editionslexikon}\item \href{https://wiki.helsinki.fi/display/stemmatology/Edition%2C+synoptic}{Parvum Lexicon Stemmatologicum}\end{itemize}\subsection*{Zitiervorschlag:}Rieger, Lisa. 2021. Synopse. In: KONDE Weißbuch. Hrsg. v. Helmut W. Klug unter Mitarbeit von Selina Galka und Elisabeth Steiner im HRSM Projekt "Kompetenznetzwerk Digitale Edition". URL: https://gams.uni-graz.at/o:konde.174\newpage\section*{TEI (Text Encoding Initiative)} \emph{Eibinger, Julia; julia.eibinger@uni-graz.at }\\
        
    Die \emph{Text Encoding Initiative} hat sich die laufende Weiterentwicklung von Standards für digitale Textkodierung und deren Aufrechterhaltung zur Aufgabe gemacht. Ihren Ursprung hat die TEI in einer Konferenz von 1987, aus der im Folgejahr ein Projekt und schließlich das heute bestehende Konsortium entstand. \\
            
        Seit ihrer Entstehung hat die TEI insgesamt fünf Versionen ihrer Richtlinien zum Umgang mit digitalen Textressourcen veröffentlicht. Ende der 1990er-Jahre sprach die TEI eine explizite Empfehlungung für \href{http://gams.uni-graz.at/o:konde.215}{XML} aus und widmete die vierte Fassung ihrer Guidelines auch ganz den nötigen Anpassungen für diese Markup-Sprache. Heute wird die TEI als Quasi-Standard für digitale Texte verstanden, die in \href{http://gams.uni-graz.at/o:konde.215}{XML} kodiert sind. Diese enge Verbundenheit resultiert nicht zuletzt aus den personellen Überlappungen, da Mitglieder des \emph{Technical Council} der TEI stark an der Entwicklung von \href{http://gams.uni-graz.at/o:konde.215}{XML} beteiligt waren. (TEI P5 2020, S. xxiii) Allerdings ist die TEI als eigenständige Markup-Sprache nicht abhängig von \href{http://gams.uni-graz.at/o:konde.215}{XML} und könnte in Zukunft für andere Markup-Sprachen ebenso adaptiert werden, wie das Ende der 1990er-Jahre in Bezug auf \href{http://gams.uni-graz.at/o:konde.215}{XML} der Fall war. (TEI P5 2020, S. xvi)\\
            
        Der große Vorteil einer solchen standardisierten Diskussionsgrundlage für Markup-Sprachen ist das Potenzial für Austausch zwischen einzelnen Wissenschaftlerinnen und Wissenschaftlern, aber auch ganzen Forschungsprojekten und -teams. Dabei kann unabhängig von System oder Applikation gearbeitet werden und Daten können problemlos transferiert werden. Zudem bietet die TEI ein Framework für unterschiedliche Textsorten und Anwendungsfälle. Anwenderinnen und Anwender müssen also nicht selbst ein vollständiges Inventar an Tags für ihre Projekte aufstellen. Es besteht allerdings die Möglichkeit, grundlegend den TEI-XML-Dialekt zu verwenden und in der Praxis um eigene notwendige Elemente zu ergänzen. Dabei muss allerdings geregelt vorgegangen werden, um die intendierte Austauschbarkeit der Daten zu gewährleisten.  (TEI P5 2020, S. xvi)\\
            
        Die TEI bietet eine Reihe von Modulen, die \href{http://gams.uni-graz.at/o:konde.215}{XML}-Elemente bestimmten Klassen zuweisen und ihre Attribute definieren. Manche fungieren als Basismodule bzw. werden für jede Art der Textkodierung empfohlen, während Elemente in anderen Modulen für die Kodierung ganz bestimmter Textsorten angelegt sind (z. B. poetische Verstexte). Für ein TEI-konformes \href{http://gams.uni-graz.at/o:konde.215}{XML}-Dokument braucht es ein hinterlegtes TEI-Schema, gegen das validiert werden kann. Für dieses Schema können einzelne Module ausgewählt und auch Elemente entsprechend den eigenen Anforderungen adaptiert werden. (TEI P5 2020, Kap. 1)\\
            
        Die Verwendung von TEI-XML bringt auch Vorteile für die Transformation von Dokumenten, da die TEI hierfür ein Set an \href{http://gams.uni-graz.at/o:konde.86}{XSLT}-Stylesheets zur Verfügung stellt. Damit können einfache TEI-Dokumente in z. B. \emph{Word, LaTeX, ePub} und andere Formate konvertiert werden. Das frei zugängliche Konvertierungstool \emph{OxGarage} ermöglicht zusätzlich die Transformation nach TEI für verschiedene Dokumenttypen wie Word, Spreadsheets oder Präsentationen. Ebenso gibt es online frei zugängliche Werkzeuge für die Publikation und Präsentation von TEI-Dokumenten im Web. Eine verhältnismäßig unkomplizierte Variante ist TEI-\emph{Boilerplate}. Dabei wird ein TEI-Dokument mittels \href{http://gams.uni-graz.at/o:konde.86}{XSLT} in einer HTML5-Shell eingebettet, die Gestaltung wird mittels CSS realisiert.\\
            
        Es erscheint wenig überraschend, dass sich die TEI aufgrund der vorgenannten Vorteile als go-to-Lösung für \href{http://gams.uni-graz.at/o:konde.59}{Digitale Editionen} etabliert hat. Die gemeinsame Basis gibt der DH-Community die Möglichkeit, sich über generelle Problemstellungen auszutauschen, Herangehensweisen weiterzuentwickeln und gegebenenfalls auch Daten nachzunutzen. Auf der Suche nach Vorbildern und Beispielen gibt es verschiedene Möglichkeiten zur Orientierung; die TEI führt z. B. selbst Projekte an, die TEI-XML verwenden.\\
            
        \subsection*{Literatur:}\begin{itemize}\item TEI: P5 Guidelines TEI Guidelines. URL: \url{http://www.tei-c.org/Guidelines/P5/}\end{itemize}\subsection*{Software:}\href{https://oxgarage.tei-c.org/}{OxGarage}, \href{http://dcl.ils.indiana.edu/teibp/index.html}{TEI Boilerplate}\subsection*{Verweise:}\href{https://gams.uni-graz.at/o:konde.215}{XML}, \href{https://gams.uni-graz.at/o:konde.86}{XSLT}, \href{https://gams.uni-graz.at/o:konde.79}{Was ist XML-TEI?}, \href{https://gams.uni-graz.at/o:konde.126}{Markup}, \href{https://gams.uni-graz.at/o:konde.17}{Annotation}, \href{https://gams.uni-graz.at/o:konde.137}{Modellierung}\subsection*{Projekte:}\href{https://tei-c.org}{Text Encoding Initiative}, \href{https://tei-c.org/activities/projects/}{Projects using the TEI}\subsection*{Themen:}Einführung, Annotation und Modellierung\subsection*{Lexika}\begin{itemize}\item \href{https://edlex.de/index.php?title=Text_Encoding_Initiative_(TEI)}{Edlex: Editionslexikon}\item \href{https://lexiconse.uantwerpen.be/index.php/lexicon/tei/}{Lexicon of Scholarly Editing}\end{itemize}\subsection*{Zitiervorschlag:}Eibinger, Julia. 2021. TEI (Text Encoding Initiative). In: KONDE Weißbuch. Hrsg. v. Helmut W. Klug unter Mitarbeit von Selina Galka und Elisabeth Steiner im HRSM Projekt "Kompetenznetzwerk Digitale Edition". URL: https://gams.uni-graz.at/o:konde.178\newpage\section*{TEI Customization} \emph{Galka, Selina; selina.galka@uni-graz.at }\\
        
    Da das \href{http://gams.uni-graz.at/o:konde.178}{TEI}-Tagset für sehr unterschiedliche thematische Bereiche und Nutzungergruppen eingesetzt werden kann, stellt die Anpassbarkeit (\emph{i. e. customization}) für projektspezifische Anforderungen einen wichtigen Aspekt dieses XML-Dialekts dar. (TEI Customization) In der Regel benötigt man für die Auszeichnung eines Textes nicht alle Module, die die TEI zur Verfügung stellt, denn diese hat über 500 Elemente und Attribute für unterschiedlichste Anwendungsfälle definiert. Deshalb wird mit \href{http://gams.uni-graz.at/o:konde.150}{ODD} eine Schema-Metasprache bereitgestellt, mit welcher ein \href{http://gams.uni-graz.at/o:konde.166}{Schema} zur projektspezifischen Dokumentation und Validierung erstellt werden kann. (Burnard 2014) Die TEI stellt mit \emph{Roma} ein Tool zur verfügung, das Benutzer bei der Erstellung eines Schemas unterstützt.\\
            
        Die einfachste TEI \emph{Customization} erlaubt schlicht alle Elemente, die von der TEI definiert werden; das daraus resultierende Schema nennt sich \emph{tei\_all}. Am häufigsten genutzt wird jedoch \emph{tei\_lite}, welches rund 50 Elemente und ihre Attribute enthält und den Ansprüchen der meisten TEI-Benutzerinnen und -Benutzer genügt. (Burnard 2014)\\
            
        Es gibt TEI \emph{Customizations}, die vom TEI-Konsortium entwickelt werden, aber auch welche, die von der Forschercommunity entwickelt und zur Verfügung gestellt werden. Einen Überblick bietet hier die Webseite der TEI (TEI Customization) oder auch das TEI Wiki (TEI Wiki: Customization). Erwähnenswert ist z. B. das DTA-Basisformat, ein TEI-Format, welches sich vor allem für historische Drucktexte der Frühen Neuzeit eignet. \\
            
        \subsection*{Literatur:}\begin{itemize}\item Burnard, Lou: Customizing the TEI - OpenEdition Press. In: What is the Text Encoding Initiative?: 2014.\item DTABf. Deutsches Textarchiv – Basisformat (2011–2020). URL: \url{http://www.deutschestextarchiv.de/doku/basisformat/}\item TEI: Customization. URL: \url{https://tei-c.org/guidelines/customization/}\item TEI Wiki: Customization. URL: \url{https://wiki.tei-c.org/index.php/Category:Customization}\end{itemize}\subsection*{Verweise:}\href{https://gams.uni-graz.at/o:konde.166}{Schema}, \href{https://gams.uni-graz.at/o:konde.150}{ODD}, \href{https://gams.uni-graz.at/o:konde.178}{TEI}\subsection*{Software:}\href{https://roma.tei-c.org}{Roma: generating customizations for the TEI}\subsection*{Themen:}Annotation und Modellierung, Digitale Editionswissenschaft\subsection*{Zitiervorschlag:}Galka, Selina. 2021. TEI Customization. In: KONDE Weißbuch. Hrsg. v. Helmut W. Klug unter Mitarbeit von Selina Galka und Elisabeth Steiner im HRSM Projekt "Kompetenznetzwerk Digitale Edition". URL: https://gams.uni-graz.at/o:konde.180\newpage\section*{TEI msDesc} \emph{Raunig, Elisabeth; elisabeth.raunig@uni-graz.at }\\
        
    Das TEI-Element <msDesc> wird verwendet, um Manuskripte, andere
                  Text beinhaltende Objekte und alles handgeschriebene nicht Publizierte zu
                  beschreiben. Die <msDesc> kann entweder Prosatext beinhalten
                  oder aber mit bestimmten Headings und Subelementen strukturiert werden. Für
                     <msDesc> gilt generell: Überall, wo ein
                     <p> gesetzt werden kann, kann auch
                     <msDesc> platziert werden. Das Element kann demnach Teil des
                     <body>-Elements sein oder im <teiHeader>
                  untergebracht werden. Als Teil des <teiHeader> ist es ein
                  Kindelement von <sourceDesc>.\\
            
        \begin{verbatim}<teiHeader>
    <fileDesc>
        <titleStmt>
            <title>Title</title>
        </titleStmt>
        <publicationStmt>
            <p>Publication Information</p>
        </publicationStmt>
        <sourceDesc>
            <msDesc>
                <msIdentifier></msIdentifier>
                <p>Prosa Beschreibung</p>
            </msDesc>
        </sourceDesc>
    </fileDesc>
</teiHeader>\end{verbatim}Die wichtigsten Elemente für eine strukturierte <msDesc> sind
                     <msIdentifier>, <head>,
                     <msContents>, <physDesc>,
                     <history> und <additional>. Für
                  Sammelhandschriften, wo jeder Teil seine eigenen Spezifika aufweist, gibt es das
                  Element <msPart> und für lose Handschriftenfragmente oder
                  Fragmente, die Teil einer Handschrift sind, wird <msFrag>
                  verwendet. \\
            
        Grundstruktur mit wichtigsten Subelementen:\\
            
        \begin{verbatim}<msDesc>
    <msIdentifier> siehe unten</msIdentifier>
    <head>
        <title></title>
        <origDate></origDate>
        <origPlace></origPlace>
    </head>
    <msContents>
        <textLang></textLang>
        <msItem></msItem>
    </msContents>
    <physDesc>siehe unten</physDesc>
    <history>
        <origin></origin>
        <provenance></provenance>
        <acquisition></acquisition>
    </history>
    <additional>Bibliografische Informationen</additional>
    <msPart>gleichen Elemente wie msDesc</msPart>
    <msFrag>gleichen Elemente wie msDesc</msFrag>
</msDesc>\end{verbatim}Jedes dieser Elemente hat seine eigenen Subelemente, je nach Detailanspruch der
                  Beschreibung können diese verwendet oder weggelassen werden. Jedoch muss man
                  beachten, dass manche Elemente andere Elemente und/oder eine bestimmte Reihenfolge
                  erzwingen – mit Ausnahme des <msIdentifier> erfordert jedes
                  Element zumindest ein <p>. <msIdentifier>
                  ist das einzige Element, das obligatorisch verlangt wird; es sollte zumindest je
                  ein Element, das zu den Gruppen ‘Ort’, ‘Repositorium’ oder ‘Identifikation’
                  gehört, beinhalten.\\
            
        Vollständiger <msIdentifier> am Beispiel der Handschrift Graz,
                  Ms. 1609:\\
            
        \begin{verbatim}<msIdentifier>
    <country>Austria</country>
    <region>Styria</region>
    <settlement>Graz</settlement>
    <institution>Karl-Franzens-Universität Graz</institution>
    <repository>Universitätsbibliothek</repository>
    <collection>Abteilung für Sondersammlung</collection>
    <idno>1609</idno>
    <altIdentifier>
        <idno>34/8</idno>
    </altIdentifier>
    <msName>Graz, UB, Ms. 1609</msName>
</msIdentifier>\end{verbatim}Nur wenige Elemente weisen verpflichtende Attribute auf. Die Verwendung von
                  Attributen ist jedoch generell zu empfehlen, da sie den Elementen mehr Semantik
                  verleihen.\\
            
        Beispiel aus der Ms. 1609 mit Attributen, die am Zentrum für
                  Informationsmodellierung ausgewählt wurden (für Details vgl. Klug et al.
                     2019):\\
            
        \begin{verbatim}<physDesc>
    <objectDesc>
        <supportDesc material="paper">
            <support><p>paper</p></support>
            <extent>
                <measure type="leavesCount" unit="leaf">469</measure>
                <dimensions type="block" unit="mm">
                    <height>140</height>
                    <width>110</width>
                 </dimensions>
            </extent>
            <foliation>
                <p>Prosabeschreibung Foliierung</p>
            </foliation>
            <collation>
                <formula style="chroust">Chroust'sche Formel</formula>
                <catchwords>Kustoden</catchwords>
            </collation>
            <condition>
                <desc>Prosabeschreibung oder:</desc>
                <material ana="folio">Paper</material>
                <material ana="binding">Prosatext</material>
                <watermark xml:id="WaageKreis">Prosatext</watermark>
            </condition>
        </supportDesc>
        <layoutDesc>
            <p> Prosabeschreibung zu Spalten und Zeilen </p>
        </layoutDesc>
    </objectDesc>
    <handDesc><p>Text</p></handDesc>
    <scriptDesc><p>Text</p></scriptDesc>
    <musicNotation><p>Text</p></musicNotation>
    <decoDesc><p>Text</p></decoDesc>
    <bindingDesc><p>Text</p></bindingDesc>
    <sealDesc><p>Text</p></sealDesc>
    <accMat><p>Text</p></accMat>
</physDesc>\end{verbatim}\subsection*{Literatur:}\begin{itemize}\item TEI: 10 Manuscript Description. URL: \url{https://www.tei-c.org/release/doc/tei-p5-doc/en/html/MS.html}\item Kodikologie. URL: \url{hdl.handle.net/11471/521.60}\end{itemize}\subsection*{Projekte:}\href{https://gams.uni-graz.at/corema}{CoReMA -
                           Cooking Recipes of the Middle Ages}, \href{http://gams.uni-graz.at/context:malab}{Mittelalterlabor}\subsection*{Themen:}Annotation und Modellierung, Digitale Editionswissenschaft\subsection*{Zitiervorschlag:}Raunig, Elisabeth. 2021. TEI msDesc. In: KONDE Weißbuch. Hrsg. v. Helmut W. Klug unter Mitarbeit von Selina Galka und Elisabeth Steiner im HRSM Projekt "Kompetenznetzwerk Digitale Edition". URL: https://gams.uni-graz.at/o:konde.179\newpage\section*{Tagebuchedition} \emph{Galka, Selina; selina.galka@uni-graz.at }\\
        
    Grundsätzlich werden Tagebuchtexte wie Werktexte nach gängiger Editionspraxis erschlossen. Eine Besonderheit der Tagebuchedition liegt darin, dass die Texte von der Autorin bzw. vom Autor verfasst wurden, ohne dass dahinter ursprünglich eine Veröffentlichungsintention stand. So werden in der Editionswissenschaft Tagebücher und Briefe häufig ähnlich behandelt, da die Edition dieser Texte genau genommen gegen die Intention ihrer Urheberinnen und Urheber erfolgt (\href{http://gams.uni-graz.at/o:konde.39}{Briefedition}). (Hurlebusch 1995, S. 26) Da es sich um private Aufzeichnungen handelt, sollten sich Editorinnen und Editoren hauptsächlich vermittelnd zwischen Autorin bzw. Autor und Leserin bzw. Leser platzieren. (Hurlebusch 1995, S. 30) Es kann nötig sein, gewisse Textstellen zu anonymisieren oder zu streichen. Weiters kann einschränkend auf die editorische Erkenntnis wirken, wenn unbekannte Personen aus dem nahen Lebensbereich der Autorin oder des Autors genannt werden, über die in Nachschlagewerken oder Normdateien keine Informationen gefunden werden können. (Hurlebusch 1995, S. 31)\\
            
        Durch die Adressierung der Texte – im Fall der Tagebücher an niemanden bzw. im Fall von Briefen an einen sehr eingeschränkten Personenkreis – erschließt sich der Inhalt für die gemeine Leserschaft oft nur geringfügig. (Dumont 2020, S. 175) Die Aufgabe einer Tagebuchedition ist es nun, den Text möglichst rezipierbar und verständlich zur Verfügung zu stellen; aber auch, den ursprünglichen Sinnzusammenhang für jede und jeden zu erschließen und zu erläutern. (Hurlebusch 1995, S. 29) Dies betrifft beispielsweise Andeutungen, die für die Autorin oder den Autor zu seiner Zeit völlig klar waren, heute aber nur mehr schwer nachzuvollziehen sind, oder genannte Personen, Werke oder Ereignisse aus der persönlichen Welt des Schreibenden. (Dumont 2020, S. 175f.) So sind \href{http://gams.uni-graz.at/o:konde.34}{Kommentare}, Register und einführende Texte oft wesentliche Bestandteile einer Tagebuchedition. \\
            
        Bei \href{http://gams.uni-graz.at/o:konde.59}{Digitalen Editionen} sind Register nun nicht mehr im Platz begrenzt, sondern können umfangreichere Informationen enthalten. (Dumont 2020, S. 184) Außerdem werden die einzelnen Einträge mit \href{http://gams.uni-graz.at/o:konde.147}{Normdaten} angereichert, wie z. B. in Bezug auf Personen Normdaten aus der \href{http://gams.uni-graz.at/o:konde.109}{GND}, wo die eindeutige Identifikationsnummer eine “projektübergreifende Identifizierung und Vernetzung” (Dumont 2020, S. 185) ermöglicht. Für Normdaten zur geographischen Verortung werden meist \emph{\href{http://gams.uni-graz.at/o:konde.107}{GeoNames}} oder der \emph{\href{http://gams.uni-graz.at/o:konde.108}{Getty Thesaurus of Geographic Names}} verwendet. Gegenstand der Forschung ist momentan auch, wie sich der Kommentar in digitalen Tagebucheditionen manifestieren kann (Dumont 2020). Zur \href{http://gams.uni-graz.at/o:konde.17}{Annotation} der Texte wird in der Regel \href{http://gams.uni-graz.at/o:konde.178}{TEI} verwendet.\\
            
        Analog zum Korrespondenzdatennetzwerk \emph{correspSearch} wurde bei der DHd 2020 die Idee präsentiert, ein Ereignisdatennetzwerk \emph{\href{http://gams.uni-graz.at/o:konde.53}{eventSearch}}, aufzubauen, welches standardisiert modellierte Ereignisdaten zusammenführt und sie datumsbezogen zur Verfügung stellt. (DHd 2020)\\
            
        \subsection*{Literatur:}\begin{itemize}\item Berbig - Hettche, Roland - Walter: Die Tagebücher Paul Heyses und Julius Rodenbergs. Möglichkeiten ihrer Erschließung und Dokumentation. In: Edition von autobiographischen Schriften und Zeugnissen zur Biographie. Beihefte zu Editio Bd. 7: 1995, S. 105–118.\item Dumont, Stefan: Kommentieren in digitalen Brief- und Tagebuch-Editionen. In: Annotieren, Kommentieren, Erläutern: Aspekte des Medienwandels. Berlin, Boston: 2020, S. 175–193.\item Fritze, Christiane; Klug, Helmut W.; Kurz, Stephan; Steindl, Christoph: Events: Modellierungen und Schnittstellen. In: Abstracts DHd 2020: 2020, S. 62–65.\item Hurlebusch, Klaus: Divergenzen des Schreibens vom Lesen. Besonderheiten der Tagebuch- und Briefedition. In: editio. Internationales Jahrbuch für Editionswissenschaft 9: 1995, S. 18–36.\item Meise, Helga: Höfische Tagebücher in der frühen Neuzeit. Überlegungen zu ihrer Edition und Kommentierung. In: Edition von autobiographischen Schriften und Zeugnissen zur Biographie. Beihefte zu Editio Bd. 7: 1995, S. 27–37.\item Plachta, Bodo: "Je trouve que je réfléchis assez …". Erwartungen an eine Edition der Tagebuchaufzeichnungen Franz von Fürstenbergs. In: Edition von autobiographischen Schriften und Zeugnissen zur Biographie. Beihefte zu Editio Bd. 7: 1995, S. 48–61.\item Somavilla, Ilse: Ludwig Wittgenstein: Tagebücher Ludwig Wittgenstein. In: Von der ersten zur letzten Hand. Theorie und Praxis der literarischen Edition. Hg. von Bernhard Fetz: 2000, S. 98–99.\end{itemize}\subsection*{Verweise:}\href{https://gams.uni-graz.at/o:konde.39}{Briefedition}, \href{https://gams.uni-graz.at/o:konde.178}{TEI}, \href{https://gams.uni-graz.at/o:konde.53}{eventSearch}, \href{https://gams.uni-graz.at/o:konde.147}{Normdaten}, \href{https://gams.uni-graz.at/o:konde.34}{Kommentar}, \href{https://gams.uni-graz.at/o:konde.109}{GND}, \href{https://gams.uni-graz.at/o:konde.107}{Geonames}, \href{https://gams.uni-graz.at/o:konde.108}{Getty}\subsection*{Projekte:}\href{https://edition.onb.ac.at/okopenko/context:okopenko/methods/sdef:Context/get}{Tagebücher Andreas Okopenko}, \href{https://schnitzler-tagebuch.acdh.oeaw.ac.at/pages/index.html}{Arthur Schnitzler Tagebuch}, \href{https://edition-humboldt.de}{edition humboldt digital}, \href{http://gams.uni-graz.at/context:ome}{Oskar Morgenstern Tagebuchedition}, \href{http://www.geonames.org}{geonames.org}, \href{https://www.dnb.de/DE/Professionell/Standardisierung/GND/gnd.html}{GND}, \href{https://www.getty.edu/research/tools/vocabularies/}{Getty Vocabularies}\subsection*{Themen:}Einführung, Digitale Editionswissenschaft\subsection*{Zitiervorschlag:}Galka, Selina. 2021. Tagebuchedition. In: KONDE Weißbuch. Hrsg. v. Helmut W. Klug unter Mitarbeit von Selina Galka und Elisabeth Steiner im HRSM Projekt "Kompetenznetzwerk Digitale Edition". URL: https://gams.uni-graz.at/o:konde.175\newpage\section*{Tagger} \emph{Eder, Elisabeth; elisabeth.eder@aau.at }\\
        
    Tagger sind Programme, die Text, meistens in tokenisierter Form (\emph{\href{http://gams.uni-graz.at/o:konde.216}{Tokenizer}}), automatisch mit entsprechenden Tags aus festgelegten \href{http://gams.uni-graz.at/o:konde.177}{Tagsets} annotieren. Sie basieren großteils auf \emph{Machine Learning} und wurden auf ausgewählten Korpora trainiert, die bereits \href{http://gams.uni-graz.at/o:konde.17}{Annotationen} nach bestimmten Tagsets enthalten. In vielen Fällen lassen sich die Tagger auch auf eigenen annotierten Daten trainieren, zum Beispiel auf einer neuen Sprache oder mit einem alternativen Tagset. In Bezug auf \emph{\href{http://gams.uni-graz.at/o:konde.156}{Part-of-Speech-Tagging}} sind hier der TreeTagger (Schmid 1994; Schmid 1995) sowie der neuere RNNTagger (Schmid 2019), die beide zudem die jeweiligen Lemmata der einzelnen Token ausgeben (\href{http://gams.uni-graz.at/o:konde.115}{Lemmatisierung}), zu erwähnen. Neben einer Auswahl von PoS-Taggern ist auch der TreeTagger in \emph{\href{http://gams.uni-graz.at/o:konde.212}{WebLicht}} inkludiert. Der SoMeWeTa (\emph{Social Media and Web Tagger}) (Proisl 2018) eignet sich speziell für deutsche Texte aus dem Social Media- und Web-Bereich. Die \emph{Python-Libraries}\emph{\href{http://gams.uni-graz.at/o:konde.170}{spaCy}}, \emph{Natural Language Toolkit} (nltk) und \emph{flair} bieten ebenfalls PoS-Tagging an.\\
            
        \subsection*{Literatur:}\begin{itemize}\item Akbik, Alan; Blythe, Duncan; Vollgraf, Roland: Contextual String Embeddings for Sequence Labeling. In: Proceedings of the 27th International Conference on Computational Linguistics COLING. Santa Fe, New Mexico, USA: 2018, S. 1638–1649.\item Proisl, Thomas: SoMeWeTa: A Part-of-Speech Tagger for German Social Media and Web Texts. In: Proceedings of the Eleventh International Conference on Language Resources and Evaluation (LREC 2018) LREC. Miyazaki, Japan: 2018.\item Schmid, Helmut: Probabilistic Part-of-Speech Tagging Using Decision Trees. In: Proceedings of International Conference on New Methods in Language Processing. Manchester, United Kingdom: 1994.\item Schmid, Helmut: Improvements in Part-of-Speech Tagging with an Application to German. In: Proceedings of the ACL SIGDAT-Workshop: 1995.\item Schmid, Helmut: Deep Learning-Based Morphological Taggers and Lemmatizers for Annotating Historical Texts. In: Proceedings of the 3rd International Conference on Digital Access to Textual Cultural Heritage DaTeCH. Brussels, Belgium: 2019.\end{itemize}\subsection*{Software:}\href{https://spacy.io/}{spacy }, \href{https://github.com/zalandoresearch/flair}{flair}, \href{https://www.cis.uni-muenchen.de/~schmid/tools/TreeTagger/}{TreeTagger}, \href{https://www.cis.uni-muenchen.de/~schmid/tools/RNNTagger/}{RNNTagger}, \href{https://github.com/tsproisl/SoMeWeTa}{SoMeWeTa}\subsection*{Projekte:}\href{https://github.com/tsproisl/SoMeWeTa}{SoMeWeTa}, \href{https://nlp.stanford.edu/links/statnlp.html#Taggers}{Liste von Part of Speech Taggern}\subsection*{Verweise:}\href{https://gams.uni-graz.at/o:konde.156}{Part-of-Speech-Tagging}, \href{https://gams.uni-graz.at/o:konde.177}{Tagsets}, \href{https://gams.uni-graz.at/o:konde.170}{spaCy}, \href{https://gams.uni-graz.at/o:konde.212}{WebLicht}, \href{https://gams.uni-graz.at/o:konde.115}{Lemmatisierung}, \href{https://gams.uni-graz.at/o:konde.216}{Tokenizer}\subsection*{Themen:}Natural Language Processing\subsection*{Zitiervorschlag:}Eder, Elisabeth. 2021. Tagger. In: KONDE Weißbuch. Hrsg. v. Helmut W. Klug unter Mitarbeit von Selina Galka und Elisabeth Steiner im HRSM Projekt "Kompetenznetzwerk Digitale Edition". URL: https://gams.uni-graz.at/o:konde.176\newpage\section*{Tagsets (allgemein und linguistisch)} \emph{Fröstl, Michael; frostlmichael@gmail.com / Eder, Elisabeth; elisabeth.eder@aau.at}\\
        
    Tagsets sind Annotationsinventare bzw. einfache Annotationsrichtlinien. Sie bilden
                  die Grundlage digitaler Annotationsprozesse (linguistischer Natur) und machen
                  diese transparent und nachvollziehbar. Tagsets stellen sich zumeist in Gestalt von
                  Listen standardisierter Abkürzungen dar. Diese Abkürzungen bilden jene Tags (\emph{Labels}), die im (linguistischen) Annotationsprozess
                  verwendet werden. Sie beruhen auf Konvention. Jeweils ein Tag steht dabei
                  eindeutig für exakt \textbf{ein}  Phänomen eines Textes oder einer
                  Quelle, respektive eines Gegenstandes, dessen inhärente Eigenschaften (digital)
                  beschrieben, explizit und computerlesbar gemacht werden. Bei digitaler
                  Kennzeichnung und Beschreibung von linguistischen Phänomenen beschreiben einzelne
                  Tags jeweils \textbf{ein}  linguistisches Phänomen eines Wortes bzw.
                  Satzzeichens (eines Tokens; z. B. seine Wortart = \emph{part of
                     speech} = PoS oder eine morphologische Erscheinung wie etwa den Kasus
                  etc.). Dabei können an das einzelne Wort mehrere Tags angelagert werden – in
                  Abhängigkeit davon, welche linguistischen Phänomene beschrieben werden sollen.
                  Linguistische Tags entsprechen in \href{http://gams.uni-graz.at/o:konde.215}{XML} einem Attribut. Die grammatische Kategorie bildet dabei den
                  Attributnamen links des Gleichheitszeichens. Als Attributwert rechts des
                  Gleichheitszeichens (zwischen "...") fungiert der eigentliche linguistische Tag
                  als Teil von Tagsets und als Repräsentant des sprachwissenschaftlichen
                  Einzelphänomens, z. B.: \\
            
        \begin{verbatim}<w pos="noun" numerus="plural">characters</w>\end{verbatim}Ihrer Funktion nach können linguistische Tagsets grob in zwei Gruppen eingeteilt
                  werden: solche, die allein der morphologischen und/oder der
                  Wortarten-(PoS)-Annotation (\emph{\href{http://gams.uni-graz.at/o:konde.156}{Part-of-Speech-Tagging}}) dienen, andererseits solche, die zur Erstellung von \emph{Treebanks} (Baumbanken) vorgesehen sind, also zur syntaktischen \href{http://gams.uni-graz.at/o:konde.17}{Annotation} geparster Texte. Je nach
                  Sprache haben sich in der Corpuslinguistik verschiedene Tagsets de facto als
                  Standard durchgesetzt, so etwa das Stuttgart-Tübingen-Tagset (STTS)
                     (Schiller et al. 1999) im Falle der PoS-Annotation deutscher
                  Texte, das beispielsweise beim TreeTagger (\href{http://gams.uni-graz.at/o:konde.176}{Tagger}) eingesetzt wird, sowie das darauf
                  aufbauende, aber leicht abgeänderte TIGER-Annotationsschema (Albert et al.
                     2003) (Verwendung bei \emph{\href{http://gams.uni-graz.at/o:konde.170}{spaCy}}). Daneben existieren Ansätze zu universell verwendbaren PoS-Tagsets, wie das
                     \emph{Universal Dependencies PoS-T}, ebenfalls bei \emph{spaCy} im Gebrauch.\\
            
        Je reduzierter und kürzer Tagsets aus linguistischer Sicht gestaltet sind, desto
                  bessere und schnellere Tagging-Ergebnisse können bei automatischer Annotation
                  mittels Tagger für gewöhnlich erzielt werden, allerdings auf Kosten
                  sprachwissenschaftlicher Differenzierung und Genauigkeit.\\
            
        \subsection*{Literatur:}\begin{itemize}\item Albert, Stefanie; Anderssen, Jan; Bader, Regine; Becker, Stephanie; Bracht, Tobias; Brants, Sabine; Brants, Thorsten; Demberg, Vera; Dipper, Stefanie; Eisenberg, Peter; Hansen, Silvia; Hirschmann, Hagen; Janitzek, Juliane; Kirstein, Carolin; Langner, Robert; Michelbacher, Lukas; Plaehn, Oliver; Preis, Cordula; Pußel, Marcus; Schrader, Bettina; Schwartz, Anne; Smith, George; Uszkoreit, Hans: TIGER Annotationsschema: 2003. URL: \url{https://www.linguistics.ruhr-uni-bochum.de/~dipper/pub/tiger_annot.pdf}.\item Schiller, Anne; Stöckert, Christine; Teufel, Simone; Thielen, Christine: Guidelines für das Tagging deutscher Textcorpora mit
                              STTS (Kleines und großes Tagset): 1999. URL: \url{http://www.sfs.uni-tuebingen.de/resources/stts-1999.pdf}.\end{itemize}\subsection*{Software:}\href{https://spacy.io/}{spacy }, \href{https://www.cis.uni-muenchen.de/~schmid/tools/TreeTagger/}{TreeTagger}\subsection*{Projekte:}\href{https://universaldependencies.org/u/pos/all.html}{Universal POS
                           tags}, \href{https://homepage.ruhr-uni-bochum.de/Stephen.Berman/Korpuslinguistik/Tagsets-STTS.html}{Stuttgart Tübingen Tagset}, \href{https://www.ims.uni-stuttgart.de/forschung/ressourcen/lexika/germantagsets/}{Tagsets für das Deutsche}\subsection*{Verweise:}\href{https://gams.uni-graz.at/o:konde.17}{Annotation}, \href{https://gams.uni-graz.at/o:konde.29}{Annotationsstandards}, \href{https://gams.uni-graz.at/o:konde.115}{Lemmatisierung}, \href{https://gams.uni-graz.at/o:konde.126}{Markup}, \href{https://gams.uni-graz.at/o:konde.156}{Part-of-Speech-Tagging}, \href{https://gams.uni-graz.at/o:konde.176}{Tagger}, \href{https://gams.uni-graz.at/o:konde.178}{TEI}, \href{https://gams.uni-graz.at/o:konde.170}{spaCy}, \href{https://gams.uni-graz.at/o:konde.145}{NLP}\subsection*{Themen:}Natural Language Processing\subsection*{Zitiervorschlag:}Tagsets (allgemein und linguistisch). In: KONDE Weißbuch. Hrsg. v. Helmut W. Klug unter Mitarbeit von Selina Galka und Elisabeth Steiner im HRSM Projekt "Kompetenznetzwerk Digitale Edition". URL: https://gams.uni-graz.at/o:konde.177\newpage\section*{Testautomatisierung} \emph{Stoff, Sebastian; sebastian.stoff@uni-graz.at }\\
        
    Automatisierte Tests zeichnen sich durch die Verwendung von Test-Frameworks aus. Oftmalig werden sie in Verbindung mit Ablaufsteuerungssystemen (wie \emph{Continuous Integration} oder \emph{Continuous Delivery}) eingesetzt und sind mithilfe dieser technischen Hilfen beliebig oft und einfach wiederholbar. Innerhalb von Sekunden können tausende Tests durchlaufen werden. Gängigerweise werden automatisierte Tests in Unittests, Integrationstests und Systemtests eingeteilt. (Demant 2018) Dabei werden im Allgemeinen einzelne Softwarekomponenten aber auch ganze Modulverbünde automatisiert überprüft. In der Regel sind solche Überprüfungen anhand der Anforderungen an Softwaresysteme gestaltet und werden in einer isolierten Umgebung ausgeführt. (Broy/Kuhrmann 2013)\\
            
        Der Automatisierungsanteil der Tests bezieht sich bei automatischen Testverfahren weniger auf die Erfassung und Implementierung der eigentlichen Testfälle, sondern mehr auf die Art und Weise der Durchführung und Berichterstattung. Weit verbreitet sind Roboter bzw. Werkzeuge, die in regelmäßigen Abständen einen vorab definierten Testkatalog durchlaufen. Dabei sorgen diese Automatisierungstools für Reproduzierbarkeit der Testabläufe, für eine Einstellbarkeit des Testzeitpunktes und für eine verständliche Berichterstattung. So kann zum Beispiel bei Performance-intensiven Überprüfungsvorgängen der Ablaufzeitpunkt um Mitternacht definiert werden, um die Stabilität des Gesamtsystems während der Arbeitszeit zu gewährleisten und gleichzeitig trotzdem gravierende Fehler im System aufdecken zu können. (Brandes/Heller 2017)\\
            
        \subsection*{Literatur:}\begin{itemize}\item Brandes, Christian; Heller, Michael: Qualitätsmanagement in agilen IT-Projekten – quo vadis?: 2017, URL: \url{doi.org/10.1007/978-3-658-18085-0}.\item Broy, Manfred; Kuhrmann, Marco: Projektorganisation und Management im Software Engineering: 2013, URL: \url{doi.org/10.1007/978-3-642-29290-3}.\item Demant, Christian: Software Due Diligence: 2018, URL: \url{doi.org/10.1007/978-3-662-53062-7}.\end{itemize}\subsection*{Verweise:}\href{https://gams.uni-graz.at/o:konde.182}{Testen als Qualitätssicherung}, \href{https://gams.uni-graz.at/o:konde.183}{Testsystematisierung}, \href{https://gams.uni-graz.at/o:konde.56}{Design Digitaler Editionen}\subsection*{Themen:}Software und Softwareentwicklung\subsection*{Zitiervorschlag:}Stoff, Sebastian. 2021. Testautomatisierung. In: KONDE Weißbuch. Hrsg. v. Helmut W. Klug unter Mitarbeit von Selina Galka und Elisabeth Steiner im HRSM Projekt "Kompetenznetzwerk Digitale Edition". URL: https://gams.uni-graz.at/o:konde.181\newpage\section*{Testen als Qualitätssicherung} \emph{Stoff, Sebastian; sebastian.stoff@uni-graz.at }\\
        
    Softwaretesten ist ein zentraler Teil der Qualitätssicherung eines Softwareprojektes. Ein Test ist ein ablauffähiges, eigenständiges Programm mit eigenen Eingabedaten und Durchführungsszenarien. Der Testfall dient somit dazu, zu überprüfen, ob ein Programm mit gegebenen Eingabedaten unter einem vorab bestimmten Ablaufszenario die erwartete Ausgabe erzeugt. Bei ‘ordentlich’ abgewickelten Softwareprojekten ist es nicht selten der Fall, dass Anforderungsanalyse und Qualitätssicherung den Hauptteil des Arbeitsaufwandes ausmachen. Da gerade die Qualitätssicherung wesentlich vom Testen getragen wird, muss ein großer Teil des Projektaufwandes für die Gestaltung, Planung und Anwendung geeigneter Testverfahren eingerechnet werden. (Broy/Kuhrmann 2013) Christian Brandes und Michael Heller verweisen dabei grundsätzlich darauf, dass nicht nur die Qualitätssicherung \textbf{durch}  Tests, sondern insbesondere auch die Qualitätssicherung \textbf{der}  Tests notwendig ist. (Brandes/Heller 2017)\\
            
        Dazu kommt außerdem, dass nicht nur das direkte Schreiben, Planen bzw. Implementieren von Tests einen größeren Mehraufwand bedeuten kann, sondern, dass auch die regelmäßige wiederholte Ausführung der Tests und die anschließende Aufarbeitung der Testresultate größere Ressourcen beanspruchen. Johannes Brauer schätzt den Zeitbedarf für sachgerechtes Testen während der Implementierung eines Systems auf  25% bis 50% der gesamten Arbeitszeit einer Entwicklerin oder eines Entwicklers im jeweiligen Projekt. (Brauer 2014)\\
            
        Im Gesamten wird das Software-Testing der konstruktiven und analytischen Qualitätssicherung zugerechnet. (Broy/Kuhrmann 2013)\\
            
        \subsection*{Literatur:}\begin{itemize}\item Brandes, Christian; Heller, Michael: Qualitätsmanagement in agilen IT-Projekten – quo vadis?: 2017, URL: \url{doi.org/10.1007/978-3-658-18085-0}.\item Brauer, Johannes: Grundkurs Smalltalk - Objektorientierung von Anfang an: 2014, URL: \url{doi.org/10.1007/978-3-658-00631-0}.\item Broy, Manfred; Kuhrmann, Marco: Projektorganisation und Management im Software Engineering: 2013, URL: \url{doi.org/10.1007/978-3-642-29290-3}.\end{itemize}\subsection*{Verweise:}\href{https://gams.uni-graz.at/o:konde.181}{Testautomatisierung}, \href{https://gams.uni-graz.at/o:konde.183}{Testsystematisierung}, \href{https://gams.uni-graz.at/o:konde.56}{Design Digitaler Editionen}, \href{https://gams.uni-graz.at/o:konde.57}{Design to Test}, \href{https://gams.uni-graz.at/o:konde.99}{Interface Design Cycle}\subsection*{Themen:}Software und Softwareentwicklung\subsection*{Zitiervorschlag:}Stoff, Sebastian. 2021. Testen als Qualitätssicherung. In: KONDE Weißbuch. Hrsg. v. Helmut W. Klug unter Mitarbeit von Selina Galka und Elisabeth Steiner im HRSM Projekt "Kompetenznetzwerk Digitale Edition". URL: https://gams.uni-graz.at/o:konde.182\newpage\section*{Testsystematisierung} \emph{Stoff, Sebastian; sebastian.stoff@uni-graz.at }\\
        
    Ein Black-Box-Test oder auch Schwarz-Kasten-Test zeichnet sich dadurch aus, dass die Testerin oder der Tester keinerlei oder nur eingeschränkte Ahnung von der konkreten Implementierung des zu überprüfenden Programms hat. Bei dieser Testgattung wird keine Kenntnis vom Test-Ziel verlangt. Es ist nicht möglich, sämtliche Eingabevarianten eines Programmes oder eines Unterprogrammes auf korrekte Umsetzung zu überprüfen. (Halang/Konakovsky 2013) Black-Box-Testing wird auch funktionales Testen genannt. (Kolhaupt 2017)\\
            
        Im Gegensatz zum Black-Box-Test kann ein White-Box-Test nur dann funktionieren, wenn die Testerin oder der Tester die innere Implementierung des zu untersuchenden Programms beherrscht. Ziel ist es, durch spezifische Anpassung die innere Implementierung eines anderen Programms zu überprüfen. (Halang/Konakovsky 2013)\\
            
        Als ein Beispiel für eine typische und weit verbreitete White-Box-Testgattung kann die statische Codeanalyse genannt werden. Durch falsifizierende Verfahren werden hierbei Fehler in der erarbeiteten Software bzw. Datenstruktur aufgefunden und diese der Entwicklerin oder dem Entwickler kommuniziert. Viele größere, integrierte Entwicklungsumgebungen besitzen umfangreiche Werkzeuge zur statischen Codeanalyse, welche oftmals bereits in der Standardkonfiguration mit ausgeliefert werden. Zum Beispiel könnte eine solche statische Codeanalyse der Entwicklerin oder dem Entwickler melden, dass die Initialisierung einer gerade verwendeten Variablen ein paar Zeilen zuvor vergessen wurde.  (Broy/Kuhrmann 2013)\\
            
        \subsection*{Literatur:}\begin{itemize}\item Broy, Manfred; Kuhrmann, Marco: Projektorganisation und Management im Software Engineering: 2013, URL: \url{doi.org/10.1007/978-3-642-29290-3}.\item Halang, Wolfgang; Konakovsky, Rudolf: Sicherheitsgerichtete Echtzeitsysteme. Berlin, Heidelberg: 2013, URL: \url{doi.org/10.1007/978-3-642-37298-8}.\item Kolhaupt, Nikolaus: Automated Software Testing. Wien: 2017. URL: \url{https://permalink.obvsg.at/AC14476572}.\end{itemize}\subsection*{Verweise:}\href{https://gams.uni-graz.at/o:konde.181}{Testautomatisierung}, \href{https://gams.uni-graz.at/o:konde.182}{Testen als Qualitätssicherung}, \href{https://gams.uni-graz.at/o:konde.56}{Design Digitaler Editionen}\subsection*{Themen:}Software und Softwareentwicklung\subsection*{Zitiervorschlag:}Stoff, Sebastian. 2021. Testsystematisierung. In: KONDE Weißbuch. Hrsg. v. Helmut W. Klug unter Mitarbeit von Selina Galka und Elisabeth Steiner im HRSM Projekt "Kompetenznetzwerk Digitale Edition". URL: https://gams.uni-graz.at/o:konde.183\newpage\section*{Text / Dokument (Fokus: Literaturwissenschaft – Bsp. Musil)} \emph{Fanta, Walter; walter.fanta@aau.at / Boelderl, Artur R.;
                  artur.boelderl@aau.at}\\
        
    Eine wichtige textologische Prämisse für \href{http://gams.uni-graz.at/o:konde.90}{textgenetische Editionen} im digitalen Medium betrifft die Unterscheidung
                  und Trennung von Text und Dokument bzw. von Repräsentation und Präsentation. Text
                  und Dokument bezeichnen mediale Aspekte der Quelle, Repräsentation und
                  Präsentation den Modus der Vermittlung. ‘Text’ ist der in Sprache und Schrift
                  gefasste, gedanklich erfassbare Inhalt – also eine immaterielle Kategorie, das
                  Werk in allen seinen Bestandteilen. ‘Dokument’ ist der Textträger, „materielles
                  Substrat textlicher Überlieferung“ (Gabler 2007, Abs. 2) – also die
                  Bücher, die Zeitschriften und Zeitungen, in denen das Werk publiziert ist und −
                  für die genetische Betrachtung vor allem wichtig − die Manuskripte des Nachlasses.
                  Edition kann mit Hans Walter Gabler begriffen werden als Akt der Vermittlung
                  zwischen Text und Dokument: „Edieren heißt, Texte von und aus Dokumenten
                  abzuleiten.“ (Gabler 2007, Abs. 15)\\
            
        Mit den Medien, die zum Einsatz gelangen – Druck oder digitale Medien −, können
                  einerseits Texte, andererseits Dokumente entweder repräsentiert oder präsentiert
                  werden. Der Modus ‘Repräsentation’ bedeutet die Übertragung des Textes oder des
                  Dokuments in ein anderes Medium, d.h. Text oder Dokument sind in dem anderen
                  Medium in allen Einzelheiten vertreten, Texte durch akkurate Wiedergabe aller
                  ihrer Zeichen, Dokumente durch stellvertretende Wiedergabe aller die Materialität
                  ausdrückenden optischen Elemente. Der Modus ‘Präsentation’ bedeutet, Texte oder
                  Dokumente in einem anderen Medium so verändert an ein Publikum zu vermitteln, dass
                  sie von diesem rezipiert werden können (z. B. durch literarische Lektüre oder
                  wissenschaftliche Nachnutzung). Unter dieser Voraussetzung bedeutet \href{http://gams.uni-graz.at/o:konde.17}{Annotation} in einer digitalen
                  textgenetischen Edition die Anreicherung der Repräsentation des Textes mit
                  Information über das Dokument; auf der Ebene des Dokuments betrachtet ist die
                  digitale Repräsentation als digitale Abbildung mit Information über den Text
                  angereichert. Bei beiden Arten der Anreicherung handelt es sich um Annotation von
                     \href{http://gams.uni-graz.at/o:konde.25}{Metadaten}. \\
            
        \subsection*{Literatur:}\begin{itemize}\item Clausen, Hans; Klug, Helmut: Die textgenetische Darstellung des Romans Der Mann ohne
                              Eigenschaften von Robert Musil auf MUSIL ONLINE. In: Textgenese in der digitalen Edition. Berlin, Boston: 2019, S. 229–249.\item Das wissenschaftliche Edieren als Funktion der
                              Dokumente. URL: \url{http://computerphilologie.digital-humanities.de/jg06/gabler.html}\end{itemize}Dieser Beitrag wurden im Kontext des FWF-Projekts "MUSIL ONLINE – interdiskursiver Kommentar" 
                  (P 30028-G24) verfasst.\subsection*{Verweise:}\href{https://gams.uni-graz.at/o:konde.17}{Annotation (Literaturwissenschaft:
                           grundsätzlich)}, \href{https://gams.uni-graz.at/o:konde.25}{Metadaten (Fokus:
                           Literaturwissenschaft - Bsp. Musil)}, \href{https://gams.uni-graz.at/o:konde.18}{Benutzerschnittstelle (Fokus:
                           Literaturwissenschaft - Bsp. Musil)}, \href{https://gams.uni-graz.at/o:konde.96}{Hybridedition}, \href{https://gams.uni-graz.at/o:konde.22}{Lesetext (Hybridedition)}\subsection*{Projekte:}\href{http://musilonline.at}{Musil Online}\subsection*{Themen:}Digitale Editionswissenschaft\subsection*{Lexika}\begin{itemize}\item \href{https://edlex.de/index.php?title=Text}{Edlex: Editionslexikon}\item \href{https://wiki.helsinki.fi/display/stemmatology/Text}{Parvum Lexicon Stemmatologicum}\item \href{https://lexiconse.uantwerpen.be/index.php/lexicon/text/}{Lexicon of Scholarly Editing}\end{itemize}\subsection*{Zitiervorschlag:}Fanta, Walter; Boelderl, Artur R. 2021. Text / Dokument (Fokus: Literaturwissenschaft – Bsp. Musil). In: KONDE Weißbuch. Hrsg. v. Helmut W. Klug unter Mitarbeit von Selina Galka und Elisabeth Steiner im HRSM Projekt "Kompetenznetzwerk Digitale Edition". URL: https://gams.uni-graz.at/o:konde.27\newpage\section*{Text Mining} \emph{Lang, Sarah; sarah.lang@uni-graz.at }\\
        
    Der Begriff des \emph{Text Mining} (TM) wurde 1995 durch Ronen Feldman und Ido Dagan unter dem Titel \emph{Knowledge Discovery from Text} (KDT) eingeführt; er bleibt jedoch bis heute wenig klar abgegrenzt. Allgemein wird TM aus dem Blickwinkel der Informatik als “a subfield devoted to the extraction of knowledge from unstructured text” angesehen (Jockers/Underwood 2016, S. 292). Zugehörig zur Domäne von \emph{\href{http://gams.uni-graz.at/o:konde.48}{Data Mining}} und \emph{Data Science}, wird es als ‘\emph{Data Mining} unter Benutzung von Textdaten’ definiert und mitunter auch \emph{Text Data Mining} genannt. Ziele dabei sind \emph{Information Extraction} (IE), \emph{Information Retrieval} (IR) und \emph{Knowledge Discovery}. \\
            
        Zur Datenverarbeitung wird \emph{\href{http://gams.uni-graz.at/o:konde.145}{Natural Language Processing (NLP)}} verwendet, wodurch eine Nähe zum Feld der Computerlinguistik entsteht. Im Gegensatz zum \emph{\href{http://gams.uni-graz.at/o:konde.71}{Distant Reading}} wird der Begriff \emph{Text Mining} eher im Kontext der Informatik verwendet. Distant Reading findet sich in den Digital Humanities zumeist in Form von Computerphilologie (\emph{Computational Literary Studies}), wobei digitale Analysemethoden für die \href{http://gams.uni-graz.at/o:konde.100}{Interpretation} von Text fruchtbar gemacht werden sollen. \emph{Text Mining} dagegen verfolgt Ziele aus dem Bereich der \emph{Information Extraction}. \emph{Text Mining} versteht Text als reines Datenbündel (vgl. \emph{bag-of-words}) oder als Datenlieferanten, der selbst in der Analyse keine weitere Bedeutung mehr haben muss. Die Resultate werden dem Text nur ‘entnommen’. Also ist \emph{Text Mining} nicht primär ein hermeneutisches Tool zur Textinterpretation, sondern eher ein Werkzeug zur Textauswertung. \\
            
        Die Aufgabe des \emph{Text Mining} besteht in statistischer Pattern-Erkennung, die in Anwendungen wie \emph{Text Clustering}, \emph{Text Categorization}, \emph{Entity Extraction}, \emph{Document Summarization} oder auch \emph{Sentiment Analysis} vorkommt. Aber auch TF-IDF (\emph{term frequency-inverse document frequency}), Intertextualitäts- oder Plagiatserkennung (\emph{Intertextuality / Text Reuse / Plagiarism Detection}) gehören dazu sowie das Pre-Processing von Inputtext durch Parsen und \emph{\href{http://gams.uni-graz.at/o:konde.145}{Natural Language Processing (NLP)}}, um eine gewisse Strukturierung der ansonsten als  unstrukturiert bezeichneten Datengattung ‘Text’ zu erzielen. Die Bezeichnung des ‘Mining’ verweist auch besonders auf die Analyse der Big Data des Internet (\emph{Web Mining}). Mitunter wird \emph{Text Mining} auch mithilfe von \emph{Machine Learning}-Algorithmen betrieben. Ressourcen, die speziell zum \emph{Text Mining} erarbeitet wurden, sind außerdem zumeist nicht primär für die Anwendung auf historische Texte und Sprachen beziehungsweise  überhaupt auf geisteswissenschaftliche Anwendungsszenarien ausgelegt. Im Fall von \emph{Text Mining} wird außerdem tendenziell eher von Big Data-Anwendungen ausgegangen, wohingegen \emph{Distant Reading}-Methoden zur quantitativen Textanalyse auch schon mit verhältnismäßig kleineren Textkorpora durchgeführt werden. \\
            
        \subsection*{Literatur:}\begin{itemize}\item Jockers, Matthew L.; Underwood, Ted: Text‐Mining the Humanities. In: A New Companion to Digital Humanities. Chichester: 2016, S. 291–306.\end{itemize}\subsection*{Software:}\href{https://www.nltk.org/}{Natural Language Toolkit (nltk)}, \href{https://www.r-project.org}{R}\subsection*{Verweise:}\href{https://gams.uni-graz.at/o:konde.48}{Data Mining}, \href{https://gams.uni-graz.at/o:konde.71}{Distant Reading}, \href{https://gams.uni-graz.at/o:konde.145}{NLP}, \href{https://gams.uni-graz.at/o:konde.16}{Analysemethoden}, \href{https://gams.uni-graz.at/o:konde.141}{NER}, \href{https://gams.uni-graz.at/o:konde.54}{Datenvisualisierung}, \href{https://gams.uni-graz.at/o:konde.100}{Interpretation}\subsection*{Themen:}Datenanalyse\subsection*{Zitiervorschlag:}Lang, Sarah. 2021. Text Mining. In: KONDE Weißbuch. Hrsg. v. Helmut W. Klug unter Mitarbeit von Selina Galka und Elisabeth Steiner im HRSM Projekt "Kompetenznetzwerk Digitale Edition". URL: https://gams.uni-graz.at/o:konde.194\newpage\section*{Textformate: FtanML} \emph{Hinkelmanns, Peter; peter.hinkelmanns@sbg.ac.at }\\
        
    FtanML (Kay 2013), benannt nach dem Schweizer Dorf Ftan, ist der Entwurf für ein Datenmodell, das die Defizite von \href{http://gams.uni-graz.at/o:konde.215}{XML}, etwa die Komplexität, die Beschränkung auf ein hierarchisches Modell, die Signifikanz oder Nicht-Signifikanz von Weißraum oder technisch nicht genau definierte Anforderungen beheben soll. Die zugehörige Schemasprache \emph{FtanGram} ermöglicht es, Datenstrukturen zu definieren und zu validieren, mittels \emph{FtanSkrit} können Dokumente abgefragt und transformiert werden.\\
            
        Ein FtanML-Element kann aus einem Namen, einer Liste von Attributen und Werten und einem Wert für das FtanML-Element selbst bestehen:\\
            
        \begin{verbatim}"<" name? (name "=" value)* content? ">"\end{verbatim}(Kay 2013, S. 6)\\
            
        Anders als in XML können FtanML-Elemente auch Listen und weitere Datentypen als Attributwerte enthalten. Ebenfalls möglich sind Listen, die wiederum FtanML-Elemente enthalten.\\
            
        Die Entwicklung von FtanML wurde eingestellt, das zugehörige Repository 2013 zuletzt aktualisiert. (Github: FtanML)\\
            
        \subsection*{Literatur:}\begin{itemize}\item Kay, Michael: The FtanML Markup Language. In: Proceedings of the Balisage: The Markup Conference 2013 Balisage: The Markup Conference 2013. Montréal, Canada.\end{itemize}\subsection*{Verweise:}\href{https://gams.uni-graz.at/o:konde.15}{Alternativen zur Textkodierung mit TEI}, \href{https://gams.uni-graz.at/o:konde.178}{TEI}, \href{https://gams.uni-graz.at/o:konde.166}{Schema}, \href{https://gams.uni-graz.at/o:konde.137}{Modellierung}\subsection*{Projekte:}\href{https://github.com/FtanML-WG}{FtanML}\subsection*{Themen:}Annotation und Modellierung\subsection*{Zitiervorschlag:}Hinkelmanns, Peter. 2021. Textformate: FtanML. In: KONDE Weißbuch. Hrsg. v. Helmut W. Klug unter Mitarbeit von Selina Galka und Elisabeth Steiner im HRSM Projekt "Kompetenznetzwerk Digitale Edition". URL: https://gams.uni-graz.at/o:konde.184\newpage\section*{Textformate: GrAF} \emph{Hinkelmanns, Peter; peter.hinkelmanns@sbg.ac.at }\\
        
    GrAF (= \emph{Graph Annotation Format}) (Ide/Suderman 2007;
                     Ide/Suderman 2014) ist eine \href{http://gams.uni-graz.at/o:konde.215}{XML}-Implementation des \emph{Linguistic Annotation
                     Formats} (\href{http://gams.uni-graz.at/o:konde.187}{Textformat LAF})
                  und fokussiert auf die Repräsentation von Texten und dazugehörigen linguistischen
                  Annotationen. Texte werden mittels \emph{\href{http://gams.uni-graz.at/o:konde.178}{Stand-off-Markup}} annotiert, wobei die Annotationen selbst Teil eines Baumes sein können, um
                  etwa Dependenzen zu annotieren. Annotationen bestehen aus einem Label und einer
                  sogenannten \emph{Feature Structure}, die die Einzelelemente – \emph{Features –} einer Annotation beinhaltet.\\
            
        Dass nach GrAF eine Vielzahl linguistischer Daten konvertiert werden können, haben
                  Nancy Ide und Harry Bunt gezeigt. (Ide/Bunt 2010)\\
            
        \subsection*{Literatur:}\begin{itemize}\item Ide, Nancy; Bunt, Harry: Anatomy of annotation schemes. Mapping to GrAF. In: Proceeding LAW IV '10 Proceedings of the Fourth
                              Linguistic Annotation Workshop Fourth Linguistic Annotation
                              Workshop. Uppsala: 2010, S. 247–255.\item Ide, Nancy; Suderman, Keith: GrAF: A Graph-based Format for Linguistic
                              Annotations GrAF. In: Proceedings of the Linguistic Annotation
                              Workshop Linguistic Annotation Workshop. Prague: 2007, S. 1–8.\item Ide, Nancy; Suderman, Keith: The Linguistic Annotation Framework: a standard for
                              annotation interchange and merging The Linguistic Annotation Framework. In: Language Resources and Evaluation 48: 2014, S. 395–418.\end{itemize}\subsection*{Verweise:}\href{https://gams.uni-graz.at/o:konde.15}{Alternativen zur Textkodierung mit
                           TEI}, \href{https://gams.uni-graz.at/o:konde.178}{TEI}, \href{https://gams.uni-graz.at/o:konde.187}{Textformate: LAF}, \href{https://gams.uni-graz.at/o:konde.215}{XML}, \href{https://gams.uni-graz.at/o:konde.171}{Stand-Off-Markup}\subsection*{Themen:}Annotation und Modellierung\subsection*{Zitiervorschlag:}Hinkelmanns, Peter. 2021. Textformate: GrAF. In: KONDE Weißbuch. Hrsg. v. Helmut W. Klug unter Mitarbeit von Selina Galka und Elisabeth Steiner im HRSM Projekt "Kompetenznetzwerk Digitale Edition". URL: https://gams.uni-graz.at/o:konde.185\newpage\section*{Textformate: Kadmos} \emph{Hinkelmanns, Peter; peter.hinkelmanns@sbg.ac.at }\\
        
    \emph{Kadmos}(Efer 2016) ist eine von Thomas Efer vorgestellte Rechercheanwendung für graphenbasierte Texte. Zugrunde liegt ein Property-Graph, also ein Graph, dessen Kanten auch mit Eigenschaften versehen werden können. Ein Text wird tokenisiert eingespeist, zudem werden auch die Types der Tokens in \emph{Kadmos} verwaltet. Der Textverlauf wird als Pfad durch die Menge der Tokens-Knoten umgesetzt. Durch weitere Kanten können beliebige Annotationsebenen ergänzt werden.\\
            
        \subsection*{Literatur:}\begin{itemize}\item Efer, Thomas: Graphdatenbanken für die textorientierten e-Humanities. Leipzig: 2016. URL: \url{https://nbn-resolving.org/urn:nbn:de:bsz:15-qucosa-219122}.\end{itemize}\subsection*{Verweise:}\href{https://gams.uni-graz.at/o:konde.15}{Alternativen zur Textkodierung mit TEI}, \href{https://gams.uni-graz.at/o:konde.178}{TEI}\subsection*{Themen:}Annotation und Modellierung\subsection*{Zitiervorschlag:}Hinkelmanns, Peter. 2021. Textformate: Kadmos. In: KONDE Weißbuch. Hrsg. v. Helmut W. Klug unter Mitarbeit von Selina Galka und Elisabeth Steiner im HRSM Projekt "Kompetenznetzwerk Digitale Edition". URL: https://gams.uni-graz.at/o:konde.186\newpage\section*{Textformate: LAF} \emph{Hinkelmanns, Peter; peter.hinkelmanns@sbg.ac.at }\\
        
    LAF (= \emph{Linguistic Annotation Format}) (Ide/Romary/de la Clergerie 2003) ist ein Framework für linguistische Annotationen, das dem ISO-Standard TC37 SC4 \emph{Language resource management} folgt. LAF unterscheidet zwischen Segmenten, die auf Texte referenzieren, und linguistischen Annotationen, die sich wiederum auf die Segmente beziehen.\\
            
        \subsection*{Literatur:}\begin{itemize}\item Ide, Nancy; Romary, Laurent; de la Clergerie, Eric: International Standard for a Linguistic Annotation Framework. In: Proceedings of the HLT-NAACL 2003 Workshop on Software Engineering and Architecture of Language Technology Systems (SEALTS) HLT-NAACL 2003 Workshop on Software Engineering and Architecture of Language Technology Systems (SEALTS): 2003, S. 25–30.\end{itemize}\subsection*{Verweise:}\href{https://gams.uni-graz.at/o:konde.15}{Alternativen zur Textkodierung mit TEI}, \href{https://gams.uni-graz.at/o:konde.178}{TEI}, \href{https://gams.uni-graz.at/o:konde.185}{GrAF}\subsection*{Themen:}Annotation und Modellierung\subsection*{Zitiervorschlag:}Hinkelmanns, Peter. 2021. Textformate: LAF. In: KONDE Weißbuch. Hrsg. v. Helmut W. Klug unter Mitarbeit von Selina Galka und Elisabeth Steiner im HRSM Projekt "Kompetenznetzwerk Digitale Edition". URL: https://gams.uni-graz.at/o:konde.187\newpage\section*{Textformate: LMNL} \emph{Hinkelmanns, Peter; peter.hinkelmanns@sbg.ac.at }\\
        
    LMNL (= \emph{Layered Markup and Annotation Language}) (Tennison/Piez 2002) ist eine \href{http://gams.uni-graz.at/o:konde.126}{Markup}-Sprache, die geschaffen wurde, um die Probleme mit überlappenden Annotationen in \href{http://gams.uni-graz.at/o:konde.178}{TEI-XML} zu lösen. Vorgeschlagen wurde auch eine Umsetzung von TEI in LMNL (Piez 2014), wobei TEI-XML automatisch nach TEI-LMNL transformiert werden kann. \\
            
        \subsection*{Literatur:}\begin{itemize}\item Piez, Wendell: TEI in LMNL: Implications for Modeling TEI in LMNL. In: Journal of the Text Encoding Initiative: 2014.\item Tennison, Jeni; Piez, Wendell: The Layered Markup and Annotation Language (LMNL) LMNL. In: Proceedings of Extreme Markup 2002 Extreme Markup 2002: 2002.\end{itemize}\subsection*{Verweise:}\href{https://gams.uni-graz.at/o:konde.15}{Alternativen zur Textkodierung mit TEI}, \href{https://gams.uni-graz.at/o:konde.178}{TEI}, \href{https://gams.uni-graz.at/o:konde.126}{Markup}\subsection*{Themen:}Annotation und Modellierung\subsection*{Zitiervorschlag:}Hinkelmanns, Peter. 2021. Textformate: LMNL. In: KONDE Weißbuch. Hrsg. v. Helmut W. Klug unter Mitarbeit von Selina Galka und Elisabeth Steiner im HRSM Projekt "Kompetenznetzwerk Digitale Edition". URL: https://gams.uni-graz.at/o:konde.188\newpage\section*{Textformate: TAGML} \emph{Hinkelmanns, Peter; peter.hinkelmanns@sbg.ac.at }\\
        
    TAGML (= \emph{Text-As-Graph Markup Language}) (Haentjens Dekker et al. 2018) modelliert Texte mit mehreren Ebenen, die voneinander unabhängig sein können. Anders als in \href{http://gams.uni-graz.at/o:konde.215}{XML} kann also ein Element einem bestimmten Layer zugeordnet sein. Textknoten können demnach Bestandteil mehrerer strukturierter Layer sein und nicht nur Teil eines Gesamtbaumes. TAGML ermöglicht auch eine \href{http://gams.uni-graz.at/o:konde.137}{Modellierung} nichtlinearer Texte, etwa durch Hinzufügungen und Ersetzungen im Text. Ein weiterer Unterschied zu XML ist die Möglichkeit von überlappenden und sich unterbrechenden Tags.\\
            
        \subsection*{Literatur:}\begin{itemize}\item Haentjens Dekker, Ronald; Bleeker, Elli; Buitendijk, Bram; Kulsdom, Astrid; Birnbaum, David J: TAGML: A markup language of many dimensions TAGML. In: Balisage: The Markup Conference 2018. Washington, DC: 2018.\end{itemize}\subsection*{Verweise:}\href{https://gams.uni-graz.at/o:konde.15}{Alternativen zur Textkodierung mit TEI}, \href{https://gams.uni-graz.at/o:konde.178}{TEI}, \href{https://gams.uni-graz.at/o:konde.137}{Modellierung}, \href{https://gams.uni-graz.at/o:konde.215}{XML}\subsection*{Themen:}Annotation und Modellierung\subsection*{Zitiervorschlag:}Hinkelmanns, Peter. 2021. Textformate: TAGML. In: KONDE Weißbuch. Hrsg. v. Helmut W. Klug unter Mitarbeit von Selina Galka und Elisabeth Steiner im HRSM Projekt "Kompetenznetzwerk Digitale Edition". URL: https://gams.uni-graz.at/o:konde.189\newpage\section*{Textformate: TexMECS (GODDAG)} \emph{Hinkelmanns, Peter; peter.hinkelmanns@sbg.ac.at}\\
        
    Der Ansatz von TexMECS (Huitfeldt/Sperberg-McQueen 2001) und dessen Vorgänger GODDAG (= \emph{General Ordered-Descendant Directed Acyclic Graph}) (Sperberg-McQueen/Huitfeldt 2004) ist es, eine Alternative zu \href{http://gams.uni-graz.at/o:konde.215}{XML} für die Textkodierung zur Verfügung zu stellen, die einen Bruch mit der strikt hierarchischen Struktur erlaubt. TexMECS erlaubt diskontinuierliche Elemente, Überlappungen und virtuelle Elemente, die Verweise auf andere Elemente sind.\\
            
        \subsection*{Literatur:}\begin{itemize}\item TexMECS. An experimental markup meta-language for complex documents. URL: \url{http://mlcd.blackmesatech.com/mlcd/2003/Papers/texmecs.html}\item Sperberg-McQueen, C. M.; Huitfeldt, Claus: GODDAG: A Data Structure for Overlapping Hierarchies GODDAG. In: Digital Documents: Systems and Principles. Springer: 2004, S. 139–160.\end{itemize}\subsection*{Verweise:}\href{https://gams.uni-graz.at/o:konde.15}{Alternativen zur Textkodierung mit TEI}, \href{https://gams.uni-graz.at/o:konde.178}{TEI}, \href{https://gams.uni-graz.at/o:konde.126}{Markup}, \href{https://gams.uni-graz.at/o:konde.137}{Modellierung}, \href{https://gams.uni-graz.at/o:konde.215}{XML}\subsection*{Themen:}Annotation und Modellierung\subsection*{Zitiervorschlag:}Hinkelmanns, Peter. 2021. Textformate: TexMECS (GODDAG). In: KONDE Weißbuch. Hrsg. v. Helmut W. Klug unter Mitarbeit von Selina Galka und Elisabeth Steiner im HRSM Projekt "Kompetenznetzwerk Digitale Edition". URL: https://gams.uni-graz.at/o:konde.190\newpage\section*{Textformate: XStandoff} \emph{Hinkelmanns, Peter; peter.hinkelmanns@sbg.ac.at }\\
        
    XStandoff (zuvor auch SGF \emph{(Sekimo Generic Format}) genannt) (Stührenberg/Jettka 2009) ist ein \href{http://gams.uni-graz.at/o:konde.215}{XML}-basiertes Format, das mittels Zeichenadressierung beliebig viele Annotationsschichten zu einem Text ermöglicht.\\
            
        \subsection*{Literatur:}\begin{itemize}\item Stührenberg, Maik; Jettka, Daniel: A toolkit for multi-dimensional markup: The development of SGF to XStandoff A toolkit for multi-dimensional markup. In: Balisage: The Markup Conference 2009. Montréal, Canada.\end{itemize}\subsection*{Verweise:}\href{https://gams.uni-graz.at/o:konde.15}{Alternativen zur Textkodierung mit TEI}, \href{https://gams.uni-graz.at/o:konde.178}{TEI}, \href{https://gams.uni-graz.at/o:konde.215}{XML}, \href{https://gams.uni-graz.at/o:konde.126}{Markup}\subsection*{Themen:}Annotation und Modellierung\subsection*{Zitiervorschlag:}Hinkelmanns, Peter. 2021. Textformate: XStandoff. In: KONDE Weißbuch. Hrsg. v. Helmut W. Klug unter Mitarbeit von Selina Galka und Elisabeth Steiner im HRSM Projekt "Kompetenznetzwerk Digitale Edition". URL: https://gams.uni-graz.at/o:konde.191\newpage\section*{Textgenese} \emph{Fanta, Walter; walter.fanta@aau.at / Boelderl, Artur R.;
                  artur.boelderl@aau.at}\\
        
    Im digitalen Kontext erfährt das aus der \emph{\href{http://gams.uni-graz.at/o:konde.46}{Critique génétique}}(Grésillon 1999) bekannte textgenetische Dossier (TGD) eine um
                  Erkenntnisse der Schreibprozessforschung erweiterte Umsetzung, die dem Umstand
                  Rechnung trägt, dass aus der Perspektive des Prozesses jedes Schriftzeugnis immer
                  zu einem Projekt gehört, ohne Verortung in einem Werk als Bezugspunkt.
                     (Sahle 2013, S. 38) Von der Schreibszene als historischem Akt des
                  Schreibens in Raum und Zeit mit allen beteiligten Körperteilen, Gesten, Geräten,
                  Materialien stellt die Edition nur jene Spuren, die sich im Schrift-Dokument
                  befinden (Campe 2012, S. 270f), in einem textgenetischen
                  Annotationsmodell dar, dessen drei Stufen den drei Arten genetischer Varianz
                  entsprechen: (1) Mikro-, (2) Meso- und (3) Makrogenese. (Nutt-Kofoth
                     2019) Im Textdokument ersichtliche, einzelne Revisionsschritte (wie
                  Streichung, Einfügung, Umstellung) fallen in den Bereich der \href{http://gams.uni-graz.at/o:konde.25}{Mikrogenese}; die \href{http://gams.uni-graz.at/o:konde.24}{Mesogenese} berücksichtigt zusätzlich jene
                  Textdokumente, die für die Abfassung oder das Umschreiben des ersten Dokuments
                  eine Rolle spielen (wie Entwurf, Notiz, Reinschrift); die \href{http://gams.uni-graz.at/o:konde.23}{Makrogenese} schließlich betrifft das gesamte
                  jeweilige Schreibprojekt (Textgeschichte, Fassungen, Versionen). Bei der Datierung
                  der Dokumente werden die relative Chronologie, die sich beispielsweise auf
                  intratextuelle Verweise gründet, und die absolute Chronologie, die sich aus
                  Datumsangaben des Autors ergibt, miteinander in Beziehung gesetzt. \\
            
        \subsection*{Literatur:}\begin{itemize}\item Campe, Rüdiger: Die Schreibszene, Schreiben Schreiben als Kulturtechnik. In: Schreiben als Kulturtechnik: Grundlagentexte. Berlin: 2012, S. 269–282.\item Grésillon, Almuth: Literarische Handschriften: Einführung in die "critique
                              génétique" Literarische Handschriften. Bern: 1999.\item Nutt-Kofoth, Rüdiger: Textgenese analog und digital: Ziele, Standards,
                              Probleme. In: Textgenese in der digitalen Edition. Berlin, Boston: 2019, S. 3-22.\item Sahle, Patrick: Digitale Editionsformen. Zum Umgang mit der
                              Überlieferung unter den Bedingungen des Medienwandels. Teil 3:
                              Textbegriffe und Recodierung. Norderstedt: 2013.\end{itemize}Dieser Beitrag wurden im Kontext des FWF-Projekts "MUSIL ONLINE – interdiskursiver Kommentar" 
                  (P 30028-G24) verfasst.\subsection*{Software:}\href{http://oxygenxml.com/}{Oxygen}\subsection*{Verweise:}\href{https://gams.uni-graz.at/o:konde.17}{Annotation (Literaturwissenschaft:
                           grundsätzlich)}, \href{https://gams.uni-graz.at/o:konde.19}{Interdiskursivität (Fokus:
                           Literaturwissenschaft, Bsp. Musil)}, \href{https://gams.uni-graz.at/o:konde.20}{Intertextualität (Fokus:
                           Literaturwissenschaft)}, \href{https://gams.uni-graz.at/o:konde.21}{Intratextualität (Fokus:
                           Literaturwissenschaft, Bsp. Musil)}, \href{https://gams.uni-graz.at/o:konde.23}{Makrogenese (Fokus:
                           Literaturwissenschaft, Bsp. Musil)}, \href{https://gams.uni-graz.at/o:konde.24}{Mesogenese (Fokus:
                           Literaturwissenschaft, Bsp. Musil)#24}, \href{https://gams.uni-graz.at/o:konde.26}{Mikrogenese (Fokus:
                           Literaturwissenschaft, Bsp. Musil)}, \href{https://gams.uni-graz.at/o:konde.22}{Hybridedition}, \href{https://gams.uni-graz.at/o:konde.90}{Genetische Edition}\subsection*{Projekte:}\href{http://musilonline.at}{Musil Online}\subsection*{Themen:}Annotation und Modellierung, Digitale Editionswissenschaft\subsection*{Lexika}\begin{itemize}\item \href{https://edlex.de/index.php?title=Textgenese}{Edlex: Editionslexikon}\item \href{https://lexiconse.uantwerpen.be/index.php/lexicon/genetic-text/}{Lexicon of Scholarly Editing}\end{itemize}\subsection*{Zitiervorschlag:}Fanta, Walter; Boelderl, Artur R. 2021. Textgenese. In: KONDE Weißbuch. Hrsg. v. Helmut W. Klug unter Mitarbeit von Selina Galka und Elisabeth Steiner im HRSM Projekt "Kompetenznetzwerk Digitale Edition". URL: https://gams.uni-graz.at/o:konde.28\newpage\section*{Textkritik in Digitalen Editionen} \emph{Schwentner, Isabella; isabella.schwentner@univie.ac.at   }\\
        
    Zentrale Aufgaben der Textkritik sind das Suchen und Sichten von Überlieferungsträgern, die editorische Überprüfung von Texten auf ihre Authentizität sowie deren Bereitstellung in zuverlässiger Gestalt. Ziel ist die Erarbeitung einer wissenschaftlichen, meist einer \href{http://gams.uni-graz.at/o:konde.93}{historisch-kritischen Edition}. (Bohnenkamp 1996, S. 179–203; Plachta 2006, S. 139)\\
            
        Je nach Entstehungszeit und Überlieferungsbedingungen des Textes werden dafür unterschiedliche Methoden (Heuristik, \href{http://gams.uni-graz.at/o:konde.105}{Kollation}, Recensio, Examinatio, Emendatio) eingesetzt. Während die Textkritik bei antiken und mittelalterlichen Texten einen Archetyp zu rekonstruieren versuchte, konzentriert sich die moderne Textkritik auf die Analyse und kritische Sichtung der vorliegenden originalen Überlieferungsträger (wie etwa Manuskripte, Typoskripte, autorisierte Drucke, aber auch Vorarbeiten, Vorstufen u. Entwürfe) und späterer Drucke in Hinblick auf Textfehler und -verderbnisse (Schreibversehen oder Irrtümer der Autorin oder des Autors, fehlerhafte Abschriften, Druckfehler, Herausgebereingriffe, Zensur). (Bohnenkamp 1996, S. 185f.)\\
            
        Gedruckte wie auch \href{http://gams.uni-graz.at/o:konde.59}{Digitale Editionen} sollen zunächst den durch textkritische Methoden geprüften und edierten Text präsentieren, wobei z. B. für die Kollation auf bestimmte Software-Tools wie \emph{CollateX}, \emph{Juxta-Commons}, \emph{TUSTEP} oder \emph{Versioning Machine} etc. zurückgegriffen werden kann. In der Digitalen Edition kann der edierte Text mit unterschiedlichen Anzeigemodi und Visualisierungen präsentiert werden.\\
            
        Als eine der wichtigsten editorischen Aufgaben gilt es, die \href{http://gams.uni-graz.at/o:konde.28}{Genese eines Textes} über mehrere Bearbeitungsstufen hinweg in einem \href{http://gams.uni-graz.at/o:konde.32}{kritischen Apparat} nachvollziehbar zu machen. (Plachta 2006, S. 99) Die französische \emph{\href{http://gams.uni-graz.at/o:konde.46}{critique génétique}} propagierte früh, dass sich \href{http://gams.uni-graz.at/o:konde.28}{Textgenese} in Digitalen Editionen einfacher und besser veranschaulichen lasse. (Grésillon 1999, S. 244ff.) Das auf diese Theorie zurückgehende Konzeptpapier der TEI \emph{An Encoding Model for }\emph{\href{http://gams.uni-graz.at/o:konde.90}{Genetic Editions}} enthält Elemente zur Beschreibung von textgenetischen – vornehmlich \href{http://gams.uni-graz.at/o:konde.26}{mikrogenetischen} – Phänomenen wie Textmodifikationen (Hinzufügungen, Streichungen, Sofortkorrekturen etc.), Umstellungen, Verweise etc. Weitere Spezifizierungen finden sich in drei Kapiteln der \href{http://gams.uni-graz.at/o:konde.178}{TEI}\emph{-Guidelines}:\\
            
        \begin{itemize}\item {Kapitel 10: \emph{Manuscript Description}  ist relevant für die Beschreibung von Textzeugen, vornehmlich Handschriften [\href{http://gams.uni-graz.at/o:konde.92}{Handschriftenbeschreibung}].}\item {Kapitel 11: \emph{Representation of Primary Sources} bietet Kodierungsvorschläge für \href{http://gams.uni-graz.at/o:konde.197}{Transkriptionen} von Handschriften, um Textmodifikationen auszeichnen zu können, wie etwa: Streichungen <del>, Hinzufügungen <add>, Umstellungen <transpose>, Einfügezeichen <metamark> etc.; ebenso ist es möglich, fehlerhafte Stellen mit <sic> und Korrekturvorschläge mit <corr> zu markieren.}\item {Kapitel 12: \emph{Critical Apparatus} zeigt Strategien und Mittel zur Auszeichnung von Textvarianz – z. B. in verschiedenen Drucken – im textkritischen Apparat (<app>).}\end{itemize}\href{http://gams.uni-graz.at/o:konde.178}{XML-TEI} hat sich als \href{http://gams.uni-graz.at/o:konde.126}{Markup}-Instrument in der digitalen Editionspraxis etabliert, wobei es nicht für alle Phänomene befriedigende Lösungen gibt – so werden etwa die unzureichenden Auszeichnungsmöglichkeiten von \href{http://gams.uni-graz.at/o:konde.23}{makrogenetischen} Beziehungen problematisiert. (Bosse/Fanta 2019, IX)\\
            
        \subsection*{Literatur:}\begin{itemize}\item An Encoding Model for Genetic Editions. URL: \url{http://www.tei-c.org/Activities/Council/Working/tcw19.html}\item Bohnenkamp, Anne: Textkritik und Textedition. In: Grundzüge der Literaturwissenschaft. Hg. von Heinz Ludwig Arnold und Heinrich Detering: 1996, S. 179–203.\item Textgenese in der digitalen Edition. Hrsg. von Anke Bosse und Walter Fanta. Berlin: 2019.\item Grésillon, Almuth: Literarische Handschriften: Einführung in die "critique génétique" Literarische Handschriften. Bern: 1999.\item Textgenese und digitales Edieren. Wolfgang Koeppens "Jugend" im Kontext der Editionsphilologie. Hrsg. von Katharina Krüger, Elisabetta Mengaldo und Eckhard Schumacher. Berlin, Boston: 2016.\item Martens, Gunter; Zeller, Hans: Textgenetische Edition. Tübingen: 1998.\item Plachta, Bodo: Editionswissenschaft. Eine Einführung in Methode und Praxis der Edition neuerer Texte Editionswissenschaft: 1997.\item TEI: P5 Guidelines TEI Guidelines. URL: \url{http://www.tei-c.org/Guidelines/P5/}\end{itemize}\subsection*{Software:}\href{https://collatex.net}{CollateX}, \href{http://juxtacommons.org}{Juxta-Commons}, \href{http://oxygenxml.com/}{Oxygen}, \href{http://www.tustep.uni-tuebingen.de/}{TUSTEP}, \href{http://v-machine.org/}{Versioning Machine}, \href{http://lera.uzi.uni-halle.de}{Lera}\subsection*{Verweise:}\href{https://gams.uni-graz.at/o:konde.46}{critique génétique}, \href{https://gams.uni-graz.at/o:konde.59}{Digitale Edition}, \href{https://gams.uni-graz.at/o:konde.90}{genetische Edition}, \href{https://gams.uni-graz.at/o:konde.93}{historisch-kritische Edition}, \href{https://gams.uni-graz.at/o:konde.105}{Kollation}, \href{https://gams.uni-graz.at/o:konde.32}{kritischer Apparat}, \href{https://gams.uni-graz.at/o:konde.126}{Markup}, \href{https://gams.uni-graz.at/o:konde.23}{Makrogenese}, \href{https://gams.uni-graz.at/o:konde.26}{Mikrogenese}, \href{https://gams.uni-graz.at/o:konde.178}{TEI}, \href{https://gams.uni-graz.at/o:konde.28}{Textgenese}, \href{https://gams.uni-graz.at/o:konde.34}{Kommentar}, \href{https://gams.uni-graz.at/o:konde.197}{Transkription}, \href{https://gams.uni-graz.at/o:konde.210}{Visualisierungstools}, \href{https://gams.uni-graz.at/o:konde.54}{Datenvisualisierung}\subsection*{Themen:}Digitale Editionswissenschaft\subsection*{Lexika}\begin{itemize}\item \href{https://edlex.de/index.php?title=Textkritik}{Edlex: Editionslexikon}\item \href{https://wiki.helsinki.fi/display/stemmatology/Textual+criticism}{Parvum Lexicon Stemmatologicum}\item \href{https://lexiconse.uantwerpen.be/index.php/lexicon/textual-criticism/}{Lexicon of Scholarly Editing}\end{itemize}\subsection*{Zitiervorschlag:}Schwentner, Isabella. 2021. Textkritik in Digitalen Editionen. In: KONDE Weißbuch. Hrsg. v. Helmut W. Klug unter Mitarbeit von Selina Galka und Elisabeth Steiner im HRSM Projekt "Kompetenznetzwerk Digitale Edition". URL: https://gams.uni-graz.at/o:konde.192\newpage\section*{Textmodellierung} \emph{Galka, Selina; selina.galka@uni-graz.at }\\
        
    Texte können unterschiedliche Textstrukturen, Inhalte, Erzählebenen, Lesearten, historische Informationen, Datierungen oder topografische Informationen enthalten. Diese impliziten Strukturen werden bei der \href{http://gams.uni-graz.at/o:konde.137}{Modellierung} je nach Zweck und Forschungsfrage identifiziert und mit Hilfe von Auszeichnungssprachen für den Computer explizit gemacht, damit dieses Wissen maschinell weiterverarbeitet werden kann. Bei der Textmodellierung versucht man also, Strukturen und Informationen von Texten herauszuarbeiten.\\
            
        Bei Brieftexten könnte man beispielsweise als modellierbare Elemente das Absendedatum und den Absendeort sehen, Sender, Empfänger und Grußformeln, aber auch z. B. im Brieftext selbst erwähnte Personen und Orte (vgl. Beispiel im Artikel zu \href{http://gams.uni-graz.at/o:konde.126}{Markup}). Bei Gedichten könnte man einzelne Strophen oder Verse modellieren; bei Tagebuchtexten z. B. als größte Struktur das Tagebuch und darin eventuell die einzelnen Tagebucheinträge mit Datumsangaben und erwähnten Entitäten wie Personen, Orten oder Institutionen. Bei linguistischen Forschungsfragen werden einzelne \href{http://gams.uni-graz.at/o:konde.156}{Wörter und Wortteile getaggt}.\\
            
        Die Modellierung betrifft sowohl inhaltliche als auch strukturelle Ebenen – so können bei mehrere Seiten umfassenden Texten auch die einzelnen Seiten als Entitäten erfasst werden. Auch der Modellierungstiefe sind grundsätzlich keine Grenzen gesetzt; Texte könnten sogar bis hin zu einzelnen Wörtern oder sogar Zeichen modelliert werden. Dabei sollte jedoch immer der Zweck hinterfragt und das Forschungsziel nicht aus den Augen verloren werden. \\
            
        Die modellierten Entitäten, Strukturen, Informationen usw. werden bei \href{http://gams.uni-graz.at/o:konde.59}{Digitalen Editionen} mit Markup explizit gemacht und können somit mithilfe des Computers \href{http://gams.uni-graz.at/o:konde.16}{analysiert}, \href{http://gams.uni-graz.at/o:konde.54}{visualisiert} oder ausgewertet – allgemein gesagt also weiterverarbeitet – werden. Als Quasi-Standard hat sich in den Digitalen Geisteswissenschaften die \href{http://gams.uni-graz.at/o:konde.178}{TEI} zur Codierung von Texten etabliert. Für das Einfügen des Markups existieren spezielle \href{http://gams.uni-graz.at/o:konde.30}{Annotationsumgebungen}.\\
            
        \subsection*{Literatur:}\begin{itemize}\item The shape of data in digital humanities: modeling texts and text-based resources. Hrsg. von Julia Flanders und Fotis Jannidis. London: 2019.\end{itemize}\subsection*{Verweise:}\href{https://gams.uni-graz.at/o:konde.137}{Modellierung}, \href{https://gams.uni-graz.at/o:konde.29}{Annotationsstandards}, \href{https://gams.uni-graz.at/o:konde.30}{Annotationsumgebungen}, \href{https://gams.uni-graz.at/o:konde.17}{Annotation}, \href{https://gams.uni-graz.at/o:konde.126}{Markup}, \href{https://gams.uni-graz.at/o:konde.178}{TEI}, \href{https://gams.uni-graz.at/o:konde.100}{Interpretation}, \href{https://gams.uni-graz.at/o:konde.215}{XML}\subsection*{Themen:}Annotation und Modellierung\subsection*{Zitiervorschlag:}Galka, Selina. 2021. Textmodellierung. In: KONDE Weißbuch. Hrsg. v. Helmut W. Klug unter Mitarbeit von Selina Galka und Elisabeth Steiner im HRSM Projekt "Kompetenznetzwerk Digitale Edition". URL: https://gams.uni-graz.at/o:konde.195\newpage\section*{Topografische Edition} \emph{Klug, Helmut W.; helmut.klug@uni-graz.at }\\
        
    Im Rahmen einer topografischen Edition wird die räumliche Anordnung des Textes in der historischen Quelle berücksichtigt und im Datenmodell sowie in der editorischen Umsetzung abgebildet, um z. B. \href{http://gams.uni-graz.at/o:konde.28}{textgenetische} Fragestellungen beantworten zu können. Die Reproduktion der Quelle geht damit über eine rein \href{http://gams.uni-graz.at/o:konde.66}{diplomatische Abbildung} des Quellentextes hinaus. In der Regel sind \href{http://gams.uni-graz.at/o:konde.75}{Editionstext} und Faksimile der Quelle über Bildkoordinaten verknüpft, wie das z. B. mit \href{http://gams.uni-graz.at/o:konde.154}{PAGE-XML} oder den \href{http://gams.uni-graz.at/o:konde.178}{TEI}-Elementen <surface> und <zone> möglich ist.\\
            
        \subsection*{Literatur:}\begin{itemize}\item Digital Humanities. Eine Einführung. Hrsg. von Fotis Jannidis, Hubertus Kohle und Malte Rehbein. Stuttgart: 2017, URL: \url{https://doi.org/10.1007%2f978-3-476-05446-3}.\item Rafiyenko, Dariya: Tracing: A Graphical-Digital Method for Restoring Damaged Manuscripts. In: Kodikologie und Paläographie im digitalen Zeitalter 4. Norderstedt: 2017.\item Sahle, Patrick: Digitale Editionsformen. Zum Umgang mit der Überlieferung unter den Bedingungen des Medienwandels. Teil 3: Textbegriffe und Recodierung. Norderstedt: 2013.\end{itemize}\subsection*{Software:}\href{https://github.com/oxygenxml/TEI-Facsimile-Plugin}{Oxygen-TEI-Facsimile-Plugin}, \href{http://t-pen.org/TPEN/}{T-Pen}, \href{https://transkribus.eu/Transkribus/}{Transkribus}\subsection*{Verweise:}\href{https://gams.uni-graz.at/o:konde.99}{Transkriptionswerkzeuge}, \href{https://gams.uni-graz.at/o:konde.50}{Datenmodell "hyperdiplomatische Transkription"}, \href{https://gams.uni-graz.at/o:konde.59}{Digitale Edition}, \href{https://gams.uni-graz.at/o:konde.28}{Textgenese}, \href{https://gams.uni-graz.at/o:konde.195}{Textmodellierung}\subsection*{Projekte:}\href{http://research.cch.kcl.ac.uk/proust_prototype/}{Proust Prototype}\subsection*{Themen:}Einführung, Digitale Editionswissenschaft\subsection*{Zitiervorschlag:}Klug, Helmut W. 2021. Topografische Edition. In: KONDE Weißbuch. Hrsg. v. Helmut W. Klug unter Mitarbeit von Selina Galka und Elisabeth Steiner im HRSM Projekt "Kompetenznetzwerk Digitale Edition". URL: https://gams.uni-graz.at/o:konde.196\newpage\section*{Transkription} \emph{Helmut W. Klug; helmut.klug@uni-graz.at}\\
        
    Als Transkription wird in den Editionswissenschaften die Übertragung eines
                  historischen Quellentexts in ein modernes Medium, heutzutage in der Regel
                  maschinenlesbarer Text, verstanden. Im Zuge der \href{http://gams.uni-graz.at/o:konde.60}{Quellendigitalisierung} ist das Transkribieren einer
                  der ersten Schritte, die durchgeführt werden. Das Ergebnis einer Transkription
                  orientiert sich an den spezifischen Fragestellungen und historisch gewachsenen
                  Richtlinien der einzelnen Disziplinen und diese sollten in den \href{http://gams.uni-graz.at/o:konde.198}{Transkriptions- /
                     Editionsrichtlinien} erläutert werden. Eine Transkription kann händisch
                  (keying, double-keying) oder automatisiert (\href{http://gams.uni-graz.at/o:konde.149}{OCR}, \href{http://gams.uni-graz.at/o:konde.224}{HTR}) erfolgen; diese Arbeit wird auch gerne als \href{http://gams.uni-graz.at/o:konde.47}{Crowd-sourcing}-Aufgabe verteilt. (Jannidis, Kohle, Rehbein, Kap. 12.4)
                  Als Hilfsmittel zum Transkribieren werden unterschiedliche Softwarelösungen (\href{http://gams.uni-graz.at/o:konde.199}{Transkriptionswerkzeuge}) angeboten,
                  die meist an die individuellen Bedürfnisse von Editionsprojekten angepasst werden
                  müssen. \\
            
        \subsection*{Literatur:}\begin{itemize}\item Brokfeld, Jens: Die digitale Edition der „preußischen Zeitungsberichte“:
                              Evaluation von Editionswerkzeugen zur nutzergenerierten Transkription
                              handschriftlicher Quellen. Fachhochschule Potsdam: 2012. URL: \url{https://nbn-resolving.org/urn:nbn:de:kobv:525-3319}.\item DFG-Praxisregeln "Digitalisierung", Deutsche Forschungsgemeinschaft: 2016. URL: \url{https://www.dfg.de/formulare/12_151/}.\item Digital Humanities. Eine Einführung. Hrsg. von Fotis Jannidis, Hubertus Kohle und Malte Rehbein. Stuttgart: 2017, URL: \url{https://doi.org/10.1007%2f978-3-476-05446-3}.\item Pierazzo, Elena; Stokes, Peter A.: Putting the Text back into Context: A Codicological
                              Approach to Manuscript Transcription. In: Codicology and Palaeography in the Digital Age 2 3: 2011.\item Sahle, Patrick: Digitale Editionsformen. Zum Umgang mit der
                              Überlieferung unter den Bedingungen des Medienwandels. Teil 1: Das
                              typografische Erbe. Norderstedt: 2013.\item Sahle, Patrick: Digitale Editionsformen. Zum Umgang mit der
                              Überlieferung unter den Bedingungen des Medienwandels. Teil 3:
                              Textbegriffe und Recodierung. Norderstedt: 2013.\end{itemize}\subsection*{Software:}\href{https://www.annotationstudio.org/}{Annotation
                           Studio}, \href{http://transcribe-bentham.ucl.ac.uk/td/Transcribe_Bentham}{Bentham Transcription Desk}, \href{https://diyhistory.lib.uiowa.edu}{Civil War
                           Diaries & Letters Transcription Project}, \href{http://cte.oeaw.ac.at/}{Classical Text
                           Editor}, \href{https://github.com/gsbodine/crowd-ed}{Crowd-Ed}, \href{https://wiki.tei-c.org/index.php/CWRC-Writer}{CWRC-Writer}, \href{https://ecdosis.rocks/Home/}{Ecdosis}, \href{http://www.bbaw.de/telota/software/ediarum}{Ediarum}, \href{https://www.e-laborate.nl/en/}{eLaborate}, \href{http://linhd.es/en/}{EVI-Lindh}, \href{https://www.nch.com.au/scribe/index.html}{Express Scribe}, \href{https://fromthepage.com/}{FromThePage}, \href{http://edgerton-digital-collections.org/notebooks}{Harold "Doc"
                           Edgerton Project}, \href{https://islandora.ca/}{Citizen Science,
                           Collaboration}, \href{http://www.mom-wiki.uni-koeln.de/}{Itineranova-Editor}, \href{https://kcl-ddh.github.io/kiln/}{KILN}, \href{http://lombardpress.org/}{LombardPress}, \href{https://manuscriptdesk.uantwerpen.be/md/Main_Page}{Manuscript
                           Desk}, \href{https://www.archives.gov/citizen-archivist/missions}{Citizen
                           Archivist Dashboard}, \href{http://ntvmr.uni-muenster.de/de/manuscript-workspace}{NTVMR
                           (manuscript workspace)}, \href{http://code.google.com/p/openscribe/}{Open
                           Scribe}, \href{http://oxygenxml.com/}{Oxygen}, \href{https://github.com/oxygenxml/TEI-Facsimile-Plugin}{Oxygen-TEI-Facsimile-Plugin}, \href{http://www.fabiovitali.it/filologia/}{PhiloEditor}, \href{http://pybossa.com/}{PyBOSSA}, \href{http://github.com/zooniverse/Scribe}{Scribe}, \href{http://scripto.org/}{scripto}, \href{http://t-pen.org/TPEN/}{T-Pen}, \href{https://textgrid.de/}{TextGrid}, \href{https://www.textlab.org/about/}{TextLab}, \href{https://textualcommunities.org/app/}{Textual
                           Communities}, \href{http://transcribo.org/en/}{Transcribo}, \href{https://transkribus.eu/Transkribus/}{Transkribus}, \href{http://www.tustep.uni-tuebingen.de/}{TUSTEP}, \href{http://bencrowder.net/coding/unbindery/}{Unbindery}, \href{https://docs.google.com/document/d/1QsFodbmuOld4ZAmnURR2tKewE1tgRo1cGxpaIUy92Mw/edit}{EditMOM3}, \href{http://wlt.synat.pcss.pl/}{Virtual
                           Transcription Laboratory}, \href{http://menus.nypl.org/}{What's On the
                           Menu?}, \href{http://en.wikisource.org/wiki/Main_Page}{Wikisource}, \href{http://community.ancestry.co.uk/awap}{World
                           Archives Project}, \href{https://www.zooniverse.org/}{zooniverse}, \href{http://www.teitok.org/index.php?action=about}{TEITOK}, \href{https://www.digitisation.eu}{IMPACT Tools and
                           Data}, \href{https://3pc.de/}{Refine!Editor}\subsection*{Verweise:}\href{https://gams.uni-graz.at/o:konde.198}{Transkriptionsrichtlinien}, \href{https://gams.uni-graz.at/o:konde.65}{Diplomatische Transkription}, \href{https://gams.uni-graz.at/o:konde.50}{Datenmodell "hyperdiplomatische
                           Transkription"}, \href{https://gams.uni-graz.at/o:konde.199}{Transkriptionswerkzeuge}\subsection*{Themen:}Einführung, Digitalisierung, Digitale Editionswissenschaft\subsection*{Lexika}\begin{itemize}\item \href{https://lexiconse.uantwerpen.be/index.php/lexicon/transcription/}{Lexicon of Scholarly Editing}\end{itemize}\subsection*{Zitiervorschlag:}Klug, Helmut W. 2021. Transkription. In: KONDE Weißbuch. Hrsg. v. Helmut W. Klug unter Mitarbeit von Selina Galka und Elisabeth Steiner im HRSM Projekt "Kompetenznetzwerk Digitale Edition". URL: https://gams.uni-graz.at/o:konde.197\newpage\section*{Transkriptionsrichtlinien} \emph{Andrews, Tara; tara.andrews@univie.ac.at }\\
        
    Nahezu alle Editionsprojekte beginnen mit der \href{http://gams.uni-graz.at/o:konde.197}{Transkription} eines oder mehrerer Texte in einem
                  digitalen Format. Zu diesem Zweck müssen vorab editorische Entscheidungen
                  getroffen werden, die von der Art des Textes ebenso abhängen wie vom adressierten
                  Zielpublikum der Edition. Angeraten ist das Verfassen von Transkriptions- und
                  Editionsrichtlinien, in welchen diese Entscheidungen beschrieben und begründet
                  sind. Solch ein Dokument ist in der Regel dynamisch, es wird während des
                  Transkriptionsprozesses häufig dann aktualisiert, wenn Stellen im Original neue
                  Entscheidungen (oder eine Verfeinerung der bereits getroffenen Entscheidungen)
                  nötig machen. Im Fall der Transkriptionen, die nach den \href{http://gams.uni-graz.at/o:konde.178}{TEI}\emph{-Guidelines} durchgeführt werden, sollen solche
                  Transkriptionsrichtlinien laut aktueller Best Practice als \emph{\href{http://gams.uni-graz.at/o:konde.180}{Customization}} im \href{http://gams.uni-graz.at/o:konde.150}{ODD} (\emph{One Document Does it All})-Format gewartet werden.\\
            
        Obwohl große Editionsunternehmen meist über diesbezügliche Vorgaben verfügen,
                  variieren Transkriptionsrichtlinien von Projekt zu Projekt. Sie hängen ab von der
                  Gesamtlänge des Textes, der Zahl an Textzeugen und der für die Edition verfügbaren
                  Zeit. Die Richtlinien müssen etwa die folgenden Fragen adressieren: \\
            
        \begin{itemize}\item {Werden alle Texte transkribiert oder lediglich der Text der Leithandschrift,
                     mit dem dann andere Varianten verglichen werden? }\item {Werden Fehler des Schreibers / der Schreiberin dokumentiert, und wenn ja,
                     gibt es eine Klassifikation dieser Fehler? }\item {Werden Abkürzungen buchstabengetreu wiedergegeben oder (stillschweigend)
                     aufgelöst? }\item {Wie werden, sofern vorhanden, Überschriften oder Rubriken im Text
                     dargestellt?}\end{itemize}Diese und andere Fragen stellen sich zwangsläufig während des
                  Transkriptionsprozesses, weshalb das Pflegen eines Richtliniendokuments empfohlen
                  wird. Dieses garantiert die editionstechnische Kohärenz während eines oft Monate
                  oder Jahre währenden Bearbeitungsprozesses. \\
            
        \subsection*{Literatur:}\begin{itemize}\item Transcription. URL: \url{https://lexiconse.uantwerpen.be/index.php/lexicon/transcription/}\item Robinson, Peter; Solopova, Elizabeth: Guidelines for Transcription of the Manuscripts of the
                              Wife of Bath’s Prologue. In: The Canterbury Tales Project Occasional Papers 1: 1993, S. 19–52.\item TEI. 23 Using the TEI. URL: \url{https://tei-c.org/release/doc/tei-p5-doc/en/html/USE.html}\end{itemize}\subsection*{Software:}\href{http://www.txstep.de}{TXSTEP}\subsection*{Verweise:}\href{https://gams.uni-graz.at/o:konde.197}{Transkription}, \href{https://gams.uni-graz.at/o:konde.180}{TEI Customization}, \href{https://gams.uni-graz.at/o:konde.66}{Diplomatische Transkription}, \href{https://gams.uni-graz.at/o:konde.50}{Datenmodell "hyperdiplomatische
                           Transkription"}, \href{https://gams.uni-graz.at/o:konde.199}{Transkriptionswerkzeuge}, \href{https://gams.uni-graz.at/o:konde.146}{Normalisierung}\subsection*{Themen:}Einführung, Digitale Editionswissenschaft\subsection*{Zitiervorschlag:}Andrews, Tara. 2021. Transkriptionsrichtlinien. In: KONDE Weißbuch. Hrsg. v. Helmut W. Klug unter Mitarbeit von Selina Galka und Elisabeth Steiner im HRSM Projekt "Kompetenznetzwerk Digitale Edition". URL: https://gams.uni-graz.at/o:konde.198\newpage\section*{Transkriptionswerkzeuge} \emph{Klug, Helmut W.; helmut.klug@uni-graz.at }\\
        
    Transkriptionswerkzeuge sind Softwareanwendungen, die on- oder offline den Prozess der \href{http://gams.uni-graz.at/o:konde.197}{Transkription} einer historischen Quelle unterstützen, indem sie z. B. ein auch für Laien bedienbares GUI (\emph{Graphical User Interface}) zur Verfügung stellen, Layout- und/oder Texterkennung anbieten oder den transkribierten Text mithilfe von Koordinatenangaben in den digitalen Faksimiles verankern. Teilweise bieten sie die Möglichkeit, diese Arbeit auch in Form von \emph{\href{http://gams.uni-graz.at/o:konde.47}{Crowdsourcing}} durchzuführen. Abhängig von den jeweiligen Anwendungen, können die Daten in unterschiedlichen Formaten gespeichert werden; ein gängiges Format für die Abbildung von Seitenstrukturen ist \href{http://gams.uni-graz.at/o:konde.154}{PAGE-XML}. \\
            
        Diese Programme wurden meist für individuelle Anwendungszwecke geschaffen, sodass eine Nachnutzung entweder nur im vorgegebenen Anwendungskontext oder erst durch eine projektspezifische Anpassung an neue Arbeitsabläufe möglich ist.\\
            
        \subsection*{Literatur:}\begin{itemize}\item Digital Humanities. Eine Einführung. Hrsg. von Fotis Jannidis, Hubertus Kohle und Malte Rehbein. Stuttgart: 2017, URL: \url{https://doi.org/10.1007%2f978-3-476-05446-3}.\item Brokfeld, Jens: Die digitale Edition der „preußischen Zeitungsberichte“: Evaluation von Editionswerkzeugen zur nutzergenerierten Transkription handschriftlicher Quellen. Fachhochschule Potsdam: 2012. URL: \url{https://nbn-resolving.org/urn:nbn:de:kobv:525-3319}.\item DFG-Praxisregeln "Digitalisierung", Deutsche Forschungsgemeinschaft: 2016. URL: \url{https://www.dfg.de/formulare/12_151/}.\item Pierazzo, Elena; Stokes, Peter A.: Putting the Text back into Context: A Codicological Approach to Manuscript Transcription. In: Codicology and Palaeography in the Digital Age 2 3: 2011.\item Sahle, Patrick: Digitale Editionsformen. Zum Umgang mit der Überlieferung unter den Bedingungen des Medienwandels. Teil 1: Das typografische Erbe. Norderstedt: 2013.\item Sahle, Patrick: Digitale Editionsformen. Zum Umgang mit der Überlieferung unter den Bedingungen des Medienwandels. Teil 3: Textbegriffe und Recodierung. Norderstedt: 2013.\end{itemize}\subsection*{Software:}\href{https://www.annotationstudio.org/}{Annotation Studio}, \href{http://transcribe-bentham.ucl.ac.uk/td/Transcribe_Bentham}{Bentham Transcription Desk}, \href{https://diyhistory.lib.uiowa.edu}{Civil War Diaries & Letters Transcription Project}, \href{http://cte.oeaw.ac.at/}{Classical Text Editor}, \href{https://github.com/gsbodine/crowd-ed}{Crowd-Ed}, \href{https://wiki.tei-c.org/index.php/CWRC-Writer}{CWRC-Writer}, \href{https://ecdosis.rocks/Home/}{Ecdosis}, \href{http://www.bbaw.de/telota/software/ediarum}{Ediarum}, \href{https://www.e-laborate.nl/en/}{eLaborate}, \href{http://linhd.es/en/}{EVI-Lindh}, \href{https://www.nch.com.au/scribe/index.html}{Express Scribe}, \href{https://fromthepage.com/}{FromThePage}, \href{http://edgerton-digital-collections.org/notebooks}{Harold "Doc" Edgerton Project}, \href{https://islandora.ca/}{Citizen Science, Collaboration}, \href{http://www.mom-wiki.uni-koeln.de/}{Itineranova-Editor}, \href{https://kcl-ddh.github.io/kiln/}{KILN}, \href{http://lombardpress.org/}{LombardPress}, \href{https://manuscriptdesk.uantwerpen.be/md/Main_Page}{Manuscript Desk}, \href{https://www.archives.gov/citizen-archivist/missions}{Citizen Archivist Dashboard}, \href{http://ntvmr.uni-muenster.de/de/manuscript-workspace}{NTVMR (manuscript workspace)}, \href{http://code.google.com/p/openscribe/}{Open Scribe}, \href{http://oxygenxml.com/}{Oxygen}, \href{https://github.com/oxygenxml/TEI-Facsimile-Plugin}{Oxygen-TEI-Facsimile-Plugin}, \href{http://www.fabiovitali.it/filologia/}{PhiloEditor}, \href{http://pybossa.com/}{PyBOSSA}, \href{http://github.com/zooniverse/Scribe}{Scribe}, \href{http://scripto.org/}{scripto}, \href{http://t-pen.org/TPEN/}{T-Pen}, \href{https://textgrid.de/}{TextGrid}, \href{https://www.textlab.org/about/}{TextLab}, \href{https://textualcommunities.org/app/}{Textual Communities}, \href{http://transcribo.org/en/}{Transcribo}, \href{https://transkribus.eu/Transkribus/}{Transkribus}, \href{http://www.tustep.uni-tuebingen.de/}{TUSTEP}, \href{http://bencrowder.net/coding/unbindery/}{Unbindery}, \href{https://docs.google.com/document/d/1QsFodbmuOld4ZAmnURR2tKewE1tgRo1cGxpaIUy92Mw/edit}{EditMOM3}, \href{http://wlt.synat.pcss.pl/}{Virtual Transcription Laboratory}, \href{http://menus.nypl.org/}{What's On the Menu?}, \href{http://en.wikisource.org/wiki/Main_Page}{Wikisource}, \href{http://community.ancestry.co.uk/awap}{World Archives Project}, \href{https://www.zooniverse.org/}{zooniverse}, \href{http://www.teitok.org/index.php?action=about}{TEITOK}, \href{https://www.digitisation.eu}{IMPACT Tools and Data}, \href{https://3pc.de/}{Refine!Editor}\subsection*{Verweise:}\href{https://gams.uni-graz.at/o:konde.198}{Transkriptionsrichtlinien}, \href{https://gams.uni-graz.at/o:konde.66}{Diplomatische Transkription}, \href{https://gams.uni-graz.at/o:konde.50}{Datenmodell "hyperdiplomatische Transkription"}, \href{https://gams.uni-graz.at/o:konde.47}{Crowdsourcing}, \href{https://gams.uni-graz.at/o:konde.196}{Topografische Edition}, \href{https://gams.uni-graz.at/o:konde.197}{Transkription}\subsection*{Themen:}Digitalisierung, Software und Softwareentwicklung\subsection*{Zitiervorschlag:}Klug, Helmut W. 2021. Transkriptionswerkzeuge. In: KONDE Weißbuch. Hrsg. v. Helmut W. Klug unter Mitarbeit von Selina Galka und Elisabeth Steiner im HRSM Projekt "Kompetenznetzwerk Digitale Edition". URL: https://gams.uni-graz.at/o:konde.199\newpage\section*{Trusted Repository, Zertifizierungen} \emph{Steiner, Elisabeth; elisabeth.steiner@uni-graz.at}\\
        
    Digitale Forschungsdaten sollen zum Zwecke der Zitier- und Nachverfolgbarkeit langzeitarchiviert werden, was auch den dauerhaften (freien) Zugang zu den Ressourcen einschließt. Die Aufgabe der \href{http://gams.uni-graz.at/o:konde.6}{Langzeitarchivierung} wird von Repositorien übernommen. Ein \emph{trusted} oder \emph{trustworthy digital repository} (TDR) erfüllt bestimmte Kriterien, deren Einhaltung zur bestmöglichen Durchführung dieser Aufgabe befähigen sollen. Das Konzept des \emph{trust}b eschreibt hier sowohl das Vertrauen des Produzenten wie auch des Konsumenten, dass die Ressourcen im Repositorium ordnungsgemäß verwaltet, archiviert und zur Verfügung gestellt werden. Diese Kriterien umfassen nicht nur technische, sondern in erster Linie auch administrative und organisatorische Gesichtspunkte (Stichwort: Dokumentation). Einer der umfangreichsten Kriterienkataloge wurde von der \emph{Research Libraries Group} (RLG) herausgegeben, andere Organisationen und Projekte (z. B. \emph{\href{http://gams.uni-graz.at/o:konde.4}{nestor}}) berufen sich im Wesentlichen auf die gleichen Punkte und ergänzen diese. Grundlage ist oft die Konformität mit dem \href{http://gams.uni-graz.at/o:konde.11}{OAIS-Referenzmodell}. \\
            
        Um die Erfüllung dieser Anforderungen objektiv und extern zu überprüfen, kann sich ein Repositorium verschiedenen Zertifizierungsprozessen unterziehen. Besonders häufig werden Repositorien im Fachbereich Digital Humanities mit dem \emph{CoreTrustSeal }(früher: \emph{Data Seal of Approval}) zertifiziert. Dies entspricht der Stufe \emph{Basic Certification} des \emph{European Framework for Audit and Certification of Digital Repositories}. Weitere Möglichkeiten sind das \emph{nestor}-Siegel (basierend auf DIN 31644) oder eine formelle ISO 16363-Zertifizierung. Die Zertifizierung als TDR wird immer mehr nicht nur zur reinen Qualitätskontrolle, sondern auch als Vergabekriterium in der Einwerbung von Drittmitteln relevant.\\
            
        \subsection*{Literatur:}\begin{itemize}\item European Framework for Audit and Certification of Digital Repositories.
                              Memorandum of Understanding. URL: \url{http://www.trusteddigitalrepository.eu/Memorandum%20of%20Understanding.html}\item ISO 16363:2012. Space data and information transfer systems — Audit and certification of trustworthy digital repositories. URL: \url{https://www.iso.org/standard/56510.html}\item Research Libraries Group: Trusted Digital Repositories: Attributes and Responsibilities. RLG: 2012. URL: \url{http://www.oclc.org/research/activities/past/rlg/trustedrep/repositories.pdf}.\item The Consultative Committee for Space Data Systems: Reference Model for an Open Archival Information System (OAIS): 2012, URL: \url{https://public.ccsds.org/pubs/650x0m2.pdf}.\end{itemize}\subsection*{Verweise:}\href{https://gams.uni-graz.at/o:konde.11}{OAIS RM}, \href{https://gams.uni-graz.at/o:konde.7}{FAIR-Prinzipien}, \href{https://gams.uni-graz.at/o:konde.6}{Digitale Nachhaltigkeit}, \href{https://gams.uni-graz.at/o:konde.4}{nestor}\subsection*{Projekte:}\href{https://crosswire.org/osis/}{Open Scripture Information Standard (OSIS)}, \href{https://www.langzeitarchivierung.de/Webs/nestor/DE/Zertifizierung/nestor_Siegel/siegel.html}{nestor-Siegel für vertrauenswürdige digitale Langzeitarchive}\subsection*{Themen:}Archivierung\subsection*{Zitiervorschlag:}Steiner, Elisabeth. 2021. Trusted Repository, Zertifizierungen. In: KONDE Weißbuch. Hrsg. v. Helmut W. Klug unter Mitarbeit von Selina Galka und Elisabeth Steiner im HRSM Projekt "Kompetenznetzwerk Digitale Edition". URL: https://gams.uni-graz.at/o:konde.13\newpage\section*{Typografie} \emph{Neuber, Frederike; frederike.neuber@bbaw.de   }\\
        
    Typografie bezeichnet die Gestaltung eines Druckerzeugnisses (z. B. Seitengestaltung, Layout, Schrift) sowie die technisch-materiellen Herstellungsverfahren des Druckes. (Wehde 2000, S. 3f.) In der angewandten Typografie und im Sprachgebrauch differenziert man typografische Gestaltungsressourcen auf zwei Ebenen: (Forssman/Wilberg 2001, S. 9f.)\\
            
        \begin{itemize}\item {Mikrotypografie, auch Detailtypografie: Schriftgestaltung und Schriftsatz}\item {Makrotypografie: Buch- und Seitengestaltung}\end{itemize}Ein daran anschließendes und weiter differenzierendes, kommunikationsanalytisch geprägtes Schema baut auf vier Gestaltungsdimensionen auf: (Stöckl 2004, S. 22f.)\\
            
        \begin{itemize}\item {Mikrotypografie: Design und Wahl der Schriften}\item {Mesotypografie: Gebrauch der Schriftzeichen in der Zeile}\item {Makrotypografie: Anordnen von Textkörpern und graphischen Elementen}\item {Paratypografie: Material und Praktik des Druckes}\end{itemize}Der editorische Umgang mit der Typografie gedruckter Quellen ist stark kulturell geprägt. Im deutschsprachigen Raum spricht die Literaturwissenschaft Typografie zwar semantisches Potential zu (Nehrlich 2016; Veitenheimer 2016), gleichzeitig gibt es in der Editionsphilologie aber keine konventionalisierten Verfahren zur Erschließung druckspezifischer Phänomene wie Layout, Satz und Schrift. Gerade in jüngerer Zeit kann man hier jedoch einen Wandel beobachten, beispielsweise durch Annika Rockenbergers und Per Röckens Vorschlag einer systematischen Beschreibung barocker Drucke oder die von Frederike Neuber vorgelegte digitale Schriftontologie zur Erschließung der Mikrotypografie in Stefan Georges Werk. (Rockenberger/Röcken 2009; Neuber 2017)\\
            
        Im Gegensatz zur deutschsprachigen blickt die angelsächsische Editionswissenschaft auf eine lange Methodentradition bei der Erschließung von Druckerzeugnissen zurück. So ist das editorische Wissen und das Erschließen der typografischen Eigenschaften einer Quelle die Basis für das \emph{Copy Text}-Verfahren (Greg 1950), in dem bibliographische Analysen beispielsweise zur Stemmata-Erstellung durchgeführt werden. Entstehungskontext des Verfahrens ist die \emph{Critical Bibliography}, ein methodischer Zusammenschluss von Buchkunde und Literaturwissenschaft.\\
            
        \subsection*{Literatur:}\begin{itemize}\item Forssman, Friedrich; Willberg, Hans-Peter: Erste Hilfe in Typografie. Ratgeber für den Umgang mit Schrift. Mainz: 2001.\item Greg, W. W: The Rationale of Copy-Text. In: Studies in Bibliography 3: 1950, S. 19–36.\item Nehrlich, Thomas: Typographie als Bedeutungsträger bei Kleist. In: Typographie & Literatur: 2016, S. 75–103.\item Neuber, Frederike: Typografie und Varianz in Stefan Georges Werk. Konzeptionelle Überlegungen zu einer ,typografiekritischen‘ Edition. In: editio 31: 2017.\item Rockenberger, Annika; Röcken, Per: Vom Offensichtlichen. Über Typographie und Edition am Beispiel barocker Drucküberlieferung (Grimmelshausens »Simplicissimus«). In: editio 23: 2009, S. 21–45.\item Stöckl, Hartmut: Typographie: Gewand und Körper des Textes. In: Zeitschrift für Angewandte Linguistik 41: 2004, S. 5–48.\item Veitenheimer, Bernhard: Synästhetik des Bedeutens. Zur Semantik der Typographie bei Johann Georg Hamann. In: Typographie & Literatur: 2016, S. 105–129.\item Wehde, Susanne: Typographische Kultur: eine zeichentheoretische und kulturgeschichtliche Studie zur Typographie und ihrer Entwicklung. Tübingen: 2000.\end{itemize}\subsection*{Verweise:}\href{https://gams.uni-graz.at/o:konde.43}{Copy Text Edition}, \href{https://gams.uni-graz.at/o:konde.77}{Editionstypografie}, \href{https://gams.uni-graz.at/o:konde.221}{Paläotypie}\subsection*{Projekte:}\href{http://gams.uni-graz.at/context:stgd¡}{Stefan George Digital}\subsection*{Themen:}Digitale Editionswissenschaft\subsection*{Lexika}\begin{itemize}\item \href{https://edlex.de/index.php?title=Typografie}{Edlex: Editionslexikon}\end{itemize}\subsection*{Zitiervorschlag:}Neuber, Frederike. 2021. Typografie. In: KONDE Weißbuch. Hrsg. v. Helmut W. Klug unter Mitarbeit von Selina Galka und Elisabeth Steiner im HRSM Projekt "Kompetenznetzwerk Digitale Edition". URL: https://gams.uni-graz.at/o:konde.200\newpage\section*{Universität Graz} \emph{Klug, Helmut W.; helmut.klug@uni-graz.at}\\
        
    siehe \href{http://gams.uni-graz.at/o:konde.217}{Zentrum für Informationsmodellierung}\\
            
        \subsection*{Verweise:}\href{https://gams.uni-graz.at/o:konde.217}{Zentrum für Informationsmodellierung}\subsection*{Zitiervorschlag:}Klug, Helmut W. 2021. Universität Graz. In: KONDE Weißbuch. Hrsg. v. Helmut W. Klug unter Mitarbeit von Selina Galka und Elisabeth Steiner im HRSM Projekt "Kompetenznetzwerk Digitale Edition". URL: https://gams.uni-graz.at/o:konde.227\newpage\section*{Universität Innsbruck} \emph{Lobis, Ulrich; ulrich.lobis@uibk.ac.at / Wang-Kathrein, Joseph;
                  joseph.wang@uibk.ac.at}\\
        
    An der Universität Innsbruck gibt es zahlreiche Projekte und Arbeitsgruppen, die
                  sich mit der Digitalisierung beschäftigen. Zu den ersten Abteilungen, die sich mit
                  der Digitalisierung von Texten beschäftigten, zählte die \emph{Digitalisierung \& Elektronische Archivierung (DEA)} am Institut für
                  Germanistik. Im Laufe der Zeit beschäftigten sich zahlreiche andere Institute mit
                  Digital Humanities, was 2013 zur Gründung des \emph{Forschungszentrums
                     Digital Humanities} führte, welches das Ziel hatte, eine Anlaufstelle für
                  Fragen zur Digitalisierung zu werden. 2019 wurde weiters das \emph{Digital Science Center (DiSC)} gegründet, das einen umfassenderen Ansatz
                  verfolgt und sich der Digitalisierung aus der Perspektive verschiedenster
                  Disziplinen nähert. Zwei weitere Einrichtungen sind zu nennen, die aus Projekten
                  an der Universität Innsbruck hervorgingen und noch eng mit ihr verbunden sind: Das
                  erste ist READ-COOP SCE, in dem die Plattform \emph{Transkribus}
                  entwickelt wurde (\emph{Transkribus} wurde im Rahmen von HF7 und
                  Horizon 2020 entwickelt), und die \emph{innsbruck university
                     innovations GmbH} (iui), die ein Spin-Off-Unternehmen der Universität
                  Innsbruck ist und die Dienstleistungen im Bereich der Digitalisierung
                  anbietet.\\
            
        Einen Überblick über die Digital Humanities-Projekte an der Universität Innsbruck
                  findet man auf https://www.uibk.ac.at/digital-humanities/projekte.html. \\
            
        \subsection*{Projekte:}\href{https://www.uibk.ac.at/germanistik/einrichtungen/dea}{Digitalisierung & Elektronische Archivierung (DEA)}, \href{https://www.uibk.ac.at/disc/}{Digital Science
                           Center (DiSC)}, \href{https://readcoop.eu}{READ-COOP SCE}, \href{https://digital-innsbruck.at}{innsbruck
                           university innovations GmbH (iui)}, \href{https://www.uibk.ac.at/digital-humanities/projekte.html}{DH
                           Projekte der Universität Innsbruck}\subsection*{Themen:}Institutionen\subsection*{Zitiervorschlag:}Lobis, Ulrich; Wang-Kathrein, Joseph. 2021. Universität Innsbruck. In: KONDE Weißbuch. Hrsg. v. Helmut W. Klug unter Mitarbeit von Selina Galka und Elisabeth Steiner im HRSM Projekt "Kompetenznetzwerk Digitale Edition". URL: https://gams.uni-graz.at/o:konde.201\newpage\section*{Universität Klagenfurt (AAU)} \emph{Bosse, Anke; anke.bosse@aau.at / Krieg-Holz, Ulrike; ulrike.krieg-holz@aau.at }\\
        
    Digitales Edieren steht an der Universität Klagenfurt in einem breiten, Disziplinen übergreifenden Kontext. Dass die Digitalisierung weite Teile menschlicher Lebensbereiche durchdringt und umfassende Veränderungsprozesse mit sich bringt, hat die Universität Klagenfurt veranlasst, 2017 den Forschungsbereich \emph{Humans in the Digital Age }(HDA) als langfristigen interdisziplinären Initiativschwerpunkt zu implementieren. HDA nimmt dabei eine auf den Menschen und seine Verhaltensweisen gerichtete Perspektive ein, in der technische und nicht-technische Aspekte miteinander verschränkt sind. In die zentrale sozial- und kulturwissenschaftliche Sicht auf die Digitalisierung und ihre Folgen sind technik- bzw. ingenieurwissenschaftliche Perspektiven eingebettet. Als Kernbereich wird seit 2019 das \emph{Digital Age Research Center} (D!ARC) aufgebaut, an dem fakultätsübergreifende Forschungsvorhaben zu Aspekten der Digitalisierung angesiedelt werden. Einen besonderen Bezug zum Digitalen Edieren hat dabei etwa das Projekt \emph{Digitale Biographien in Österreich: Synthese und Vernetzung individueller Lebensläufe aus dem WWW}, bei dem es um die Frage geht, welche ‘digitalen Spuren’ österreichische Bürger (gewollt oder ungewollt) im WWW hinterlassen und wie sich diese Spuren zu einer individuellen digitalen Biographie zusammensetzen lassen. Methodisch werden hierfür informatische und geisteswissenschaftliche Problemlösungsverfahren eng verzahnt.\\
            
        Gleichzeitig zu \emph{Humans in the Digital Age} und damit eng verbunden bündelt das HSM-Projekt \emph{Kompetenznetzwerk Digitale Edition} (2017-2020) laufende und neu entstehende Projekte in der Digitalen Edition; es hat vor allem die Kooperationen mit Partnern außerhalb der Universität Klagenfurt verstärkt.\\
            
        Digitales Edieren hat an der Universität Klagenfurt bereits eine längere Tradition, denn es startete 2004 mit der Einrichtung der Arbeitsstelle für Digitale Edition am Robert-Musil-Institut für Literaturforschung / Kärntner Literaturarchiv.Die hier entstandene Klagenfurter Musil-Ausgabe auf DVD wird seit 2015 abgelöst durch die Musil-\href{http://gams.uni-graz.at/o:konde.96}{Hybridausgabe}, die aus einer Leseausgabe in Buchform und der Open-Access-Internetplattform \emph{MUSIL ONLINE} besteht und sämtliche Werke und Vorabdrucke sowie den überaus umfänglichen Nachlass Musils anbietet, \href{http://gams.uni-graz.at/o:konde.178}{TEI}-codiert. \emph{MUSIL ONLINE} wird in enger Kooperation mit der Österreichischen Nationalbibliothek kontinuierlich zu einer Musil-Edition ausgebaut, die auf der Editionsplattform der ÖNB implementiert wird. In sie fließen die Ergebnisse des FWF-Projekts \emph{MUSIL ONLINE - interdiskursiver Kommenar} (2018-2022) ein; es entwickelt innovative digitale Lösungen für Online-Kommentare, die die \href{http://gams.uni-graz.at/o:konde.19}{Interdiskursivität} sowohl der Schriften Musils als auch der entsprechenden Sekundärliteratur berücksichtigt.\\
            
        Am Musil-Institut/Kärntner Literaturarchiv wird die Werner-Kofler-Hybridausgabe mit dem FWF-Projekt \emph{Werner Kofler intermedial} (2018-2022), in enger Kooperation mit dem \href{http://gams.uni-graz.at/o:konde.217}{Zentrum für Informationsmodellierung} der Universität Graz, weitergeführt.\\
            
        In Kooperation mit der Europäischen Genossenschaft \emph{Transkribus} läuft am Musil-Institut/Kärntner Literaturarchiv die Digitalisierung und halbautomatische \href{http://gams.uni-graz.at/o:konde.197}{Transkription} der 100 Notizbücher Josef Winklers aus dessen umfänglichem Bestand am Kärntner Literaturarchiv (1977-2013).\\
            
        Am Institut für Germanistik werden folgende Projekte durchgeführt: Das größte deutschsprachige E-Mail-Korpus \emph{CodE Alltag} wurde grundlegend überarbeitet und annotiert; die \href{http://gams.uni-graz.at/o:konde.159}{Pseudonymisierung} individuenidentifizierender Merkmale und deren Evaluation erlauben die unrestringierte Weitergabe der Korpusdaten. Das Korpus wird derzeit u. a. für stilometrische Untersuchungen genutzt. Mit Hilfe von bestehenden Wörterbuchangaben und \emph{Word Embeddings} entsteht zunächst ein deutschsprachiges Lexikon zur Vulgärsprache (VulGer), das zur digitalen Aufbereitung und Analyse von Textkorpora genutzt werden kann, etwa als Ressource für Stilberechnungen. In der \emph{Virtuellen Benediktiner Bibliothek Millstatt} werden die an elf Orten in Europa verstreuten Handschriften virtuell wieder zusammengeführt und der weiteren Erforschung zugänglich gemacht.Die Internetplattform \emph{litkult1920er}, die auf Basis zweier FWF-Projekte 2008-2018 erstellt wurde, bietet ein Epochenprofil zu den transdisziplinären Konstellationen in der österreichischen Literatur, Kunst und Kultur der Zwischenkriegszeit.\\
            
        Am Institut für Slawistik läuft die Longitudinalstudie \emph{Zweisprachiger Spracherwerb anhand schriftlicher Texte von Schülerinnen und Schülern der Hermagoras Volksschule}; es erfolgen Korpuserstellung, \href{http://gams.uni-graz.at/o:konde.60}{Digitalisierung}, \href{http://gams.uni-graz.at/o:konde.17}{Auszeichnung}, Auswertung zur Erforschung der Sprach- und Textkompetenz im dualen deutsch-slowenischen Immersionsunterricht sowie die Digitalisierung von schriftlichen Sprachbeherrschungsprüfungen des Slowenischstudiums. Das zweijährige Projekt \emph{Unknown and Little-Known Manuscripts and Printed Texts of Older Slovenian Literature in the Wider Slovenian, Regional Austrian Carinthian, and Austrian Context} ist 2020 abgeschlossen.\\
            
        Die \href{http://gams.uni-graz.at/o:konde.59}{Digitale Edition} ist eine Teildisziplin der Digital Humanities. Als Vertragspartnerin im Forschungsinfrastrukturkonsortium CLARIAH-AT verfolgt die Universität Klagenfurt das Ziel, digitale Infrastrukturen auf- und auszubauen und insbesondere die Digital Humanities zu fördern.\\
            
        Im Rahmen des HRSM-Projekts\emph{ E-infrastructures Austria} schließlich erstellt die Universitätsbibliothek das Repositorium \emph{Netlibrary}, in das laufend Datensätze und Volltexte implementiert, langzeitarchiviert und mit URN versehen werden.\\
            
        \subsection*{Verweise:}\href{https://gams.uni-graz.at/o:konde.16}{Analysemethoden}, \href{https://gams.uni-graz.at/o:konde.218}{Zielgruppen digitaler Editionen}, \href{https://gams.uni-graz.at/o:konde.59}{Digitale Edition}, \href{https://gams.uni-graz.at/o:konde.60}{Digitalisierung}, \href{https://gams.uni-graz.at/o:konde.65}{diplomatische Edition}, \href{https://gams.uni-graz.at/o:konde.75}{Editionstext}, \href{https://gams.uni-graz.at/o:konde.83}{Faksimileausgabe/edition}, \href{https://gams.uni-graz.at/o:konde.85}{Filmedition}, \href{https://gams.uni-graz.at/o:konde.90}{Genetische Edition}, \href{https://gams.uni-graz.at/o:konde.96}{Hybridedition}, \href{https://gams.uni-graz.at/o:konde.98}{Interface}, \href{https://gams.uni-graz.at/o:konde.103}{Kodikologie}, \href{https://gams.uni-graz.at/o:konde.139}{Musikedition}, \href{https://gams.uni-graz.at/o:konde.140}{Nachlassedition}, \href{https://gams.uni-graz.at/o:konde.152}{Open Access}, \href{https://gams.uni-graz.at/o:konde.155}{Paläografie}, \href{https://gams.uni-graz.at/o:konde.162}{Regestausgabe/Regesten}, \href{https://gams.uni-graz.at/o:konde.174}{Synopse}, \href{https://gams.uni-graz.at/o:konde.192}{Textkritik}, \href{https://gams.uni-graz.at/o:konde.34}{Kommentar}, \href{https://gams.uni-graz.at/o:konde.36}{Bereitstellung von Digitalisaten}, \href{https://gams.uni-graz.at/o:konde.126}{Markup}, \href{https://gams.uni-graz.at/o:konde.178}{TEI}, \href{https://gams.uni-graz.at/o:konde.141}{Named Entity Recognition / NER}, \href{https://gams.uni-graz.at/o:konde.159}{Pseudonymisierung}, \href{https://gams.uni-graz.at/o:konde.176}{Tagger}, \href{https://gams.uni-graz.at/o:konde.177}{Tagsets}, \href{https://gams.uni-graz.at/o:konde.135}{Mockup}, \href{https://gams.uni-graz.at/o:konde.205}{Usability}, \href{https://gams.uni-graz.at/o:konde.210}{Visualisierungstools}, \href{https://gams.uni-graz.at/o:konde.17}{Annotation (grundsätzlich)}, \href{https://gams.uni-graz.at/o:konde.19}{Annotation: Interdiskursivität}, \href{https://gams.uni-graz.at/o:konde.20}{Annotation: Intertextualität}, \href{https://gams.uni-graz.at/o:konde.21}{Annotation: Intratextualität}, \href{https://gams.uni-graz.at/o:konde.26}{Annotation: Mikrogenese}, \href{https://gams.uni-graz.at/o:konde.24}{Annotation: Mesogenese}, \href{https://gams.uni-graz.at/o:konde.23}{Annotation: Makrogenese}, \href{https://gams.uni-graz.at/o:konde.28}{Annotation: Textgenese}, \href{https://gams.uni-graz.at/o:konde.41}{Citizen Science}, \href{https://gams.uni-graz.at/o:konde.47}{Crowdsourcing}, \href{https://gams.uni-graz.at/o:konde.153}{Österreichische Nationalbibliothek}, \href{https://gams.uni-graz.at/o:konde.217}{Zentrum für Informationsmodellierung / Uni Graz}, \href{https://gams.uni-graz.at/o:konde.117}{Liste der Hybrideditionen}, \href{https://gams.uni-graz.at/o:konde.79}{Einführung: Was ist XML/TEI?}\subsection*{Software:}\href{https://transkribus.eu/Transkribus/}{Transkribus}\subsection*{Projekte:}\href{http://www.digitale-edition.at}{KONDE - Kompetenznetzwerk Digitale Edition}, \href{https://www.aau.at/musil/literaturforschung/musilforschung/#musil-hybrid}{Musil-Hybrid}, \href{http://musilonline.at}{Musil Online}, \href{https://www.aau.at/musil/literaturforschung/kofler/}{Kofler intermedial. Kommentierte Werkausgabe Werner Kofler (Radio, Film, Theater)}, \href{https://github.com/codealltag}{CodE Alltag - A German-Language Email Corpus}, \href{https://www.aau.at/universitaetsbibliothek-klagenfurt/sondersammlungen/kostbarkeiten-aus-der-bibliothek/handschriften_millstatt/}{Die Handschriften des Benediktinerstiftes Millstatt}, \href{https://www.e-infrastructures.at/de/}{E-infrastructures Austria}, \href{http://netlibrary.aau.at}{netlibrary}, \href{https://www.aau.at/hda/}{Digital Age Research Center (D!ARC)}, \href{ https://litkult1920er.aau.at/}{litkult1920er}\subsection*{Themen:}Institutionen\subsection*{Zitiervorschlag:}Bosse, Anke; Krieg-Holz, Ulrike. 2021. Universität Klagenfurt (AAU). In: KONDE Weißbuch. Hrsg. v. Helmut W. Klug unter Mitarbeit von Selina Galka und Elisabeth Steiner im HRSM Projekt "Kompetenznetzwerk Digitale Edition". URL: https://gams.uni-graz.at/o:konde.202\newpage\section*{Universität Salzburg} \emph{Zeppezauer-Wachauer, Katharina; katharina.wachauer@sbg.ac.at / Zangerl, Lina Maria; linamaria.zangerl@sbg.ac.at }\\
        
    Die Paris Lodron Universität Salzburg (PLUS) vertritt die Auffassung, dass die \href{http://gams.uni-graz.at/o:konde.60}{Digitalisierung} gerade in den Geistes-, Sozial- und Kulturwissenschaften viele Möglichkeiten bietet, Inhalte digital aufzubereiten und Forscherinnen, Forschern und Interessierten weltweit zugänglich zu machen. Befördert werden Forschung und Lehre an der Schnittstelle von Geisteswissenschaften und zeitgemäßen technologischen Umsetzungen sowie computergestützten Methoden. Als Vertragspartner im Forschungsinfrastrukturkonsortium CLARIAH-AT strebt die PLUS den Auf- und Ausbau digitaler Infrastrukturen und die Etablierung digitaler Forschungsprozesse in den Kultur- und Geisteswissenschaften an.\\
            
        Digital Humanities-Projekte werden an der PLUS in den verschiedensten Themenbereichen durchgeführt, wobei aktuell besonderes Augenmerk auf der systematischen Erschließung und digitalen Dissemination genuin analoger Quellen (überwiegend Texte, aber auch Bilder und Artefakte) liegt. \href{http://gams.uni-graz.at/o:konde.59}{Digitale Editionen} werden somit nicht nur neu erstellt, sondern auch analoge Editionen retrodigitalisiert und dieses hybride Material für die weitere Nutzung im digitalen Raum auf der Basis wissenschaftlich etablierter Standards annotiert bzw. aufbereitet. Die Forschungsinfrastruktur-Datenbank des BMBWF bietet eine Übersicht über infrastrukturelle Angebote aus dem Bereich der digitalen Geisteswissenschaften der PLUS. \\
            
        Der \href{http://gams.uni-graz.at/o:konde.152}{Open Access}-Publikationsserver ePLUS der Universitätsbibliothek ist ein nach den gängigen Richtlinien der \href{http://gams.uni-graz.at/o:konde.6}{Langzeitarchivierung} angelegtes Repository, das ein stabiles Hosting sowie die Verbreitung digitaler Forschungsdaten und -ressourcen bietet. Es stellt einen Archivdienst für die Publikation von Forschungsarbeiten der Mitarbeiterinnen und Mitarbeiter sowie Lehrbeauftragten, Emeriti und Studierenden (Qualifikationsschriften) dar.\\
            
        Die an den IT-Services im Aufbau befindliche Serviceplattform \href{http://gams.uni-graz.at/o:konde.68}{dhPLUS} soll zukünftig den dauerhaften und sicheren Betrieb von Digital Humanities-Projekten an der Universität Salzburg gewährleisten. Zu diesem Zwecke werden mit Unterstützung des KONDE-Projektes drei Pilotprojekte auf dhPLUS veröffentlicht, die in enger methodischer Verbindung mit Digitalen Editionen stehen oder selbst solche darstellen: die von der Universitätsbibliothek präsentierte und von den IT-Services gewartete Hybridedition \emph{EbnerOnline} (UBS/ITS), die das Ziel verfolgt, sämtliche Werke Ferdinand Ebners zu veröffentlichen; die am Literaturarchiv in Zusammenarbeit mit dem \href{http://gams.uni-graz.at/o:konde.217}{ZIM} in Graz entstehende Archivpräsentation \emph{STEFAN ZWEIG DIGITAL}, die durch die Rekonstruktion von Zweigs weltweit verstreutem Nachlass Ausgangsbasis und Recherchetool für künftige Werk- und \href{http://gams.uni-graz.at/o:konde.140}{Nachlasseditionen} darstellen kann, sowie der Relaunch der \emph{Mittelhochdeutschen Begriffsdatenbank} (MHDBDB) des Interdisziplinären Zentrums für Mittelalter und Frühneuzeit (IZMF), die mehr als 650 born digital oder retrodigitalisierte Texteditionen mittelalterlicher, deutschsprachiger Literatur verarbeitet. Der Disseminationsservice wird gleichermaßen als Archiv- \textbf{und}  Arbeitsplattform angelegt: Er soll nicht nur der Sicherung der Projektressourcen dienen, sondern genauso eine ständige Bearbeitung ermöglichen.\\
            
        \subsection*{Verweise:}\href{https://gams.uni-graz.at/o:konde.152}{Open Access}, \href{https://gams.uni-graz.at/o:konde.6}{Langzeitarchivierung}, \href{https://gams.uni-graz.at/o:konde.68}{Dissemination-Services: DHPLUS}, \href{https://gams.uni-graz.at/o:konde.217}{Zentrum für Informationsmodellierung}, \href{https://gams.uni-graz.at/o:konde.140}{Nachlassedition}, \href{https://gams.uni-graz.at/o:konde.52}{Datenmodell "MHDBDB"}\subsection*{Projekte:}\href{https://www.uni-salzburg.at/index.php?id=209704}{DH Projekte der Universität Salzburg}, \href{https://forschungsinfrastruktur.bmbwf.gv.at/}{Forschungsinfrastruktur-Datenbank des BMBWF}, \href{http://wfe.sbg.ac.at/exist/apps/Frontpage/index.html}{Ebner Online}, \href{www.stefanzweig.digital}{Stefan Zweig digital}, \href{http://mhdbdb.sbg.ac.at/}{Mittelhochdeutsche Begriffsdatenbank (MHDBDB)}\subsection*{Themen:}Institutionen\subsection*{Zitiervorschlag:}Zeppezauer-Wachauer, Katharina; Zangerl, Lina Maria. 2021. Universität Salzburg. In: KONDE Weißbuch. Hrsg. v. Helmut W. Klug unter Mitarbeit von Selina Galka und Elisabeth Steiner im HRSM Projekt "Kompetenznetzwerk Digitale Edition". URL: https://gams.uni-graz.at/o:konde.203\newpage\section*{Universität Wien} \emph{Andrews, Tara; tara.andrews@univie.ac.at / Wallnig, Thomas; thomas.wallnig@univie.ac.at}\\
        
    \href{http://gams.uni-graz.at/o:konde.59}{Digitales Edieren} steht an der Universität Wien in einer langen Tradition historischer und philologischer Disziplinen, zugleich bemüht sich die Universitätsleitung um die Erarbeitung einer Strategie in digitalen Fragen. Vor diesem Hintergrund muss eine Orientierung hinsichtlich digitaler Agenden einerseits von der Dynamik laufender Prozesse, andererseits von der komplexen Struktur des institutionellen Ökosystems ausgehen. Es existieren an der Universität Wien (Stand: Januar 2020) einige Kommunikationsknoten: beispielsweise das Vizerektorat für Digitalisierung, die Forschungsplattform Data Science oder der Forschungsschwerpunkt Digital Humanities an der Historisch-Kulturwissenschaftlichen Fakultät. Der Zentrale Informatikdienst und die Universitätsbibliothek bieten Unterstützung im Bereich Forschungsdatenmanagement. \\
            
        Innerhalb Österreichs bestehen enge Kooperationen mit dem \href{http://gams.uni-graz.at/o:konde.1}{ACDH-CH der Österreichischen Akademie der Wissenschaften} sowie dem \href{http://gams.uni-graz.at/o:konde.217}{Institut Zentrum für Informationsmodellierung} an der Universität Graz, ebenso mit der Europäischen Genossenschaft \emph{Transkribus}. Die Einrichtung einer zentralen Anlaufstelle in digitalen Fragen ist Gegenstand von Diskussionen; im Hinblick auf digitales Edieren ist es gegenwärtig angeraten, sich an die Lehrstuhlinhaberin Tara Andrews bzw. an das auf historische Editionen spezialisierte Institut für Österreichische Geschichtsforschung (IOEG) zu wenden.\\
            
        Digitale oder \href{http://gams.uni-graz.at/o:konde.96}{hybride Editionen} werden an unterschiedlichen Fakultäten und Zentren durchgeführt; die oben genannten Einrichtungen und Initiativen verfolgen das Ziel von Vernetzung und Überblick, dennoch kann es von Vorteil sein, sich zum Zwecke der Orientierung bei ähnlich gelagerten Themen und Fragestellungen auch direkt an eine oder mehrere der folgenden editorisch tätigen Personen zu wenden:\\
            
        \begin{itemize}\item {\emph{Edition von Pseudo-Athanasius, Expositiones in Psalmos}: kritische Edition eines frühchristlichen Textes (5. Jh), der besondere Herausforderungen aufgrund der vielschichtigen Textgestaltung der Überlieferung mit sich bringt; geleitet von Uta Heil.}\item {\emph{Edition der Chronik des Matthäus von Edessa}: graphenbasierte Edition mit Elementen maschinellen Lernens einer armenischen Kreuzzugschronik (12. Jh.); geleitet von Tara Andrews.}\item {\emph{Edition der gelehrten Korrespondenz der Brüder Pez}: international konsortial vernetzte Volltextedition einer lateinischen Briefkorrespondenz samt Nachlass österreichischer Benediktiner (18. Jh.); geleitet von Thomas Wallnig.}\item {\emph{Edition der (Früh-) Werke von Arthur Schnitzler} (gest. 1931): progressiv angelegte kritische Werkausgabe eines führenden Vertreters der österreichischen Moderne; geleitet von Konstanze Fliedl.}\item {\emph{Onlinekommentar zur Prosawerkausgabe von Werner Kofler} (gest. 2011): hybrid konzipierte Edition eines Gegenwartsschriftstellers, deren digitaler Teil komplementär zu urheberrechtlich geschütztem Material konzipiert ist; bearbeitet von Claudia Dürr und Wolfgang Straub.}\end{itemize}Die genannten Editionen dienen als exemplarische Auswahl, die keinesfalls die Relevanz anderer, nicht genannter Initiativen schmälern soll. Ebenso ist mit zu bedenken, dass zahlreiche Initiativen und Projekte existieren, die digitale Forschung jenseits von digitalem Edieren im engeren Sinn bzw. in Überschneidung mit diesem praktizieren, darunter etwa \emph{Exhibitions of Modern European Paintings 1905-1915}.\\
            
        \subsection*{Verweise:}\href{https://gams.uni-graz.at/o:konde.59}{Digitale Edition}, \href{https://gams.uni-graz.at/o:konde.217}{ZIM}, \href{https://gams.uni-graz.at/o:konde.96}{Hybridedition}\subsection*{Projekte:}\href{https://rektorat.univie.ac.at/rektorat/vizerektorinnen/vizerektor-maier/geschaeftsbereich/}{VR Digitalisierung, Uni Wien}, \href{https://datascience.univie.ac.at/}{Data Science}, \href{https://fsp-digital-humanities.univie.ac.at/}{FSP DH, Uni Wien}, \href{https://zid.univie.ac.at/}{ZID, Uni Wien}, \href{https://bibliothek.univie.ac.at/}{UB Wien}, \href{https://phaidra.univie.ac.at/}{Phaidra}, \href{https://datamanagement.univie.ac.at/en/rdm/network-software-development/}{Forschungsdarenmanagement, Uni Wien}, \href{https://readcoop.eu/}{Transkribus}, \href{https://geschichtsforschung.univie.ac.at/ueber-uns/digitale-themen-am-ioeg/}{DH am IOEG}, \href{https://etfkg.univie.ac.at/en/about-us/team/uta-heil/}{Pseudo-Athanasius, Expositiones in Psalmos}, \href{https://editions.byzantini.st/ChronicleME/}{Chronik des Matthäus von Edessa}, \href{http://vemg.at/pez-edition-und-nachlass/}{Korrespondenz der Brüder Pez}, \href{https://www.univie.ac.at/germanistik/projekt/arthur-schnitzler-fruehwerk-3/}{Werke von Arthur Schnitzler}, \href{https://gams.uni-graz.at/archive/objects/context:kofler/methods/sdef:Context/get?mode=print}{Onlinekommentar zur Prosawerkausgabe von Werner Kofler}, \href{https://exhibitions.univie.ac.at/}{Exhibitions of Modern European Paintings 1905-1915}, \href{https://ufind.univie.ac.at/de/person.html?id=62558}{Tara Andrews}\subsection*{Themen:}Institutionen\subsection*{Zitiervorschlag:}Andrews, Tara; Wallnig, Thomas. 2021. Universität Wien. In: KONDE Weißbuch. Hrsg. v. Helmut W. Klug unter Mitarbeit von Selina Galka und Elisabeth Steiner im HRSM Projekt "Kompetenznetzwerk Digitale Edition". URL: https://gams.uni-graz.at/o:konde.204\newpage\section*{Urheberrecht} \emph{Scholger, Walter; walter.scholger@uni-graz.at }\\
        
    Forschende stehen stets in einem Spannungsverhältnis zwischen ihrem eigenen wissenschaftlichen Schaffen und jenem fremden Schaffen, das für die Bearbeitung und Erschließung im Forschungsprozess rezipiert und reflektiert werden muss. Diese Doppelrolle als Autorinnen und Autoren sowie Benutzerinnen und Benutzer und der gerechte Ausgleich zwischen den Interessen beider Parteien ist Aufgabe des Urheberrechts. \\
            
        Spricht man von ‘Urheberrecht’, so gilt es zu Beginn besonderes Augenmerk auf einen exakten Sprachgebrauch zu legen. Häufig wird sowohl in den Medien als auch in alltäglichen Gesprächen über dieses Thema fälschlich von ‘Copyright’ gesprochen: Dieser Begriff ist jedoch dem anglo-amerikanischen Rechtsraum entlehnt, dem eine gänzlich andere Rechtssystematik zugrunde liegt. Im zentraleuropäischen Raum liegt das Hauptaugenmerk auf dem Schutz der Rechte der Urheberinnen und Urheber – daher ‘Urheber-Recht’ anstelle von ‘Kopier-Recht’. \\
            
        Am Anfang des Urheberrechts steht die Definition des Werkbegriffs, da nur Werke vom Urheberrecht erfasst werden bzw. nur an Werken überhaupt ein Urheberrecht entstehen kann. Werke sind demnach “eigentümliche geistige Schöpfungen auf den Gebieten der Literatur, der Tonkunst, der bildenden Künste und der Filmkunst.” (Gesetz über Urheberrecht und verwandte Schutzrechte (Urheberrechtsgesetz), UrhG § 1 (1))\\
            
        \begin{itemize}\item {‘Eigentümlich’ bedeutet in diesem Zusammenhang, dass ein Werk eine gewisse Individualität aufweisen muss, also eine bestimmte Schöpfungshöhe gegeben ist, die eine Zuordnung zu einer bestimmten Urheberin bzw. einem bestimmten Urheber zulässt. }\item {‘Geistig’ zeigt an, dass es sich nicht um eine zufällige, unbeabsichtigte Kreation handeln darf, der Schaffensprozess muss willentlich, bewusst stattfinden. }\item {Eine ‘Schöpfung’ ist wiederum erst gegeben, wenn die individuelle Idee auch in einer greifbaren Form veröffentlicht wurde: Eine Idee allein, ungeachtet ihrer Eigentümlichkeit, ist nicht vom Urheberrecht erfasst.}\end{itemize}Die Urheberschaft ist stets an eine natürliche Person gebunden, geht nach dem Tod der Urheberin bzw. des Urhebers auf die jeweiligen Erben über und erlischt schließlich 70 Jahre nach dem Tod der Urheberin bzw. des Urhebers. Nach Ablauf dieser Schutzfrist werden Werke gemeinfrei und sind nicht mehr urheberrechtlich geschützt.\\
            
        Ein in der wissenschaftlichen Publikationspraxis sehr häufig auftretender Fall ist die kollaborative Autorschaft: Wird ein Werk von mehreren Personen kollaborativ geschaffen, so spricht man von Miturheberinnen und Miturhebern, die das Urheberrecht gemeinschaftlich wahrnehmen.\\
            
        Die Rechte der Urheberinnen und Urheber gliedern sich in zwei unterschiedliche Bereiche: Die Urheberpersönlichkeitsrechte, die dem Schutz der geistigen Interessen dienen und unübertragbar sind:\\
            
        \begin{itemize}\item {Der Schutz der Urheberschaft garantiert, dass nur die tatsächliche Urheberin bzw. der tatsächliche Urheber als Schöpferin bzw. Schöpfer eines Werkes gelten darf und kann.  }\item {Die Urheberin bzw. der Urheber allein entscheidet über die Urheberbezeichnung, die das eigene Werk trägt.}\item {Der Werkschutz besagt, dass ein Werk nicht gegen den Willen der Urheberin oder des Urhebers verfremdet, entstellt oder verändert werden darf.}\end{itemize}Die Verwertungsrechte dagegen können im Rahmen von Verträgen – z. B. mit individuellen Lizenznehmenden, mit Arbeitgebern und Ausbildungsstätten, Verwertungsgesellschaften oder Verlagen  – weitergegeben oder sogar gänzlich abgetreten werden: \\
            
        \begin{itemize}\item {Das Vervielfältigungsrecht (UrhG § 15) regelt die Anfertigung von Kopien eines Werkes (oder Teilen eines Werkes), unabhängig von der Methode und dem Träger der Vervielfältigung.}\item {Das Verbreitungsrecht (UrhG § 16) umfasst die Verbreitung analoger Werkstücke. }\item {Das Senderecht (UrhG § 17) betrifft die Sendung eines Werkes mittels Rundfunk.}\item {Das Vortrags-, Aufführungs- und Vorführungsrecht (UrhG § 18) regelt die öffentliche Wiedergabe eines Werkes. }\item {Das Zurverfügungstellungsrecht (UrhG § 18a) regelt die drahtgebundene oder drahtlose Veröffentlichung eines Werks im Internet.}\end{itemize}Diese Verwertungsrechte an Werken können in Form einer Werknutzungsbewilligung (z. B. im Rahmen einer \href{http://gams.uni-graz.at/o:konde.119}{Lizenzierung} (vgl. auch \href{http://gams.uni-graz.at/o:konde.9}{Lizenzmodelle}) oder einem exklusiven Werknutzungsrecht (z. B. im Rahmen von Verlags- oder Dienstverträgen) an Dritte abgegeben werden.\\
            
        Das Urheberrecht definiert jedoch eine Reihe von \href{http://gams.uni-graz.at/o:konde.222}{Freien Werknutzungen}, die eine Nachnutzung urheberrechtlich geschützter Werke für Wissenschaft und Lehre gestatten.\\
            
        In diesem Zusammenhang wird häufig darüber diskutiert, inwieweit eine \href{http://gams.uni-graz.at/o:konde.59}{Digitale Edition} eine eigentümliche geistige Schöpfung darstellt, wenn es sich bei ihr lediglich um die Überführung von gedruckten Quellen in das digitale Medium handelt. Grundsätzlich wird die für den Urheberrechtsschutz erforderliche Schöpfungshöhe sehr niedrig angesetzt, sodass z. B. eine kritische Edition aufgrund der erforderlichen eigenständigen wissenschaftlichen Leistung der Editorin / des Editors gemeinhin als Werk angesehen wird. Die Zurverfügungstellung einer reinen Transkription wird hingegen nur als Vervielfältigung eines bestehenden Werkes gelten können. \\
            
        \subsection*{Literatur:}\begin{itemize}\item Burgstaller, Peter: Urheberrecht für Lehrende: Ein Leitfaden für die Praxis mit 80 Fragen und Antworten. Aktuelles Urheberrecht. Wien: 2017.\item Klimpel, Paul; Weitzmann, John H: Forschen in der digitalen Welt. Juristische Handreichung für die Geisteswissenschaften. Göttingen: 2014.\item Scholger, Walter: Urheberrecht und offene Lizenzen im wissenschaftlichen Publikationsprozess. In: Publikationsberatung an Universitäten.Ein Praxisleitfaden zum Aufbau publikationsunterstützender Services. Bielefeld: 2020, S. 123–147.\item Walter, Michael M.: UrhG mit den Novellen 2009-2015, Internationales Privatrecht, Urheberrechtliche EU-Richtlinien: Mit der neueren Rechtsprechung der österreichischen Gerichte und des Gerichtshofs der Europäischen Union. Urheber- und Verwertungsgesellschaftenrecht‚ 15: Textausgabe mit Kurzkommentaren 1. Wien: 2015.\item Gesetz über Urheberrecht und verwandte Schutzrechte (Urheberrechtsgesetz). URL: \url{https://www.gesetze-im-internet.de/urhg/BJNR012730965.html}\end{itemize}\subsection*{Verweise:}\href{https://gams.uni-graz.at/o:konde.45}{Creative Commons}, \href{https://gams.uni-graz.at/o:konde.119}{Lizenzierung}, \href{https://gams.uni-graz.at/o:konde.9}{Lizenzmodelle}, \href{https://gams.uni-graz.at/o:konde.222}{Freie Werknutzungen}\subsection*{Themen:}Einführung, Rechtliche Aspekte\subsection*{Lexika}\begin{itemize}\item \href{https://edlex.de/index.php?title=Urheberrecht}{Edlex: Editionslexikon}\end{itemize}\subsection*{Zitiervorschlag:}Scholger, Walter. 2021. Urheberrecht. In: KONDE Weißbuch. Hrsg. v. Helmut W. Klug unter Mitarbeit von Selina Galka und Elisabeth Steiner im HRSM Projekt "Kompetenznetzwerk Digitale Edition". URL: https://gams.uni-graz.at/o:konde.44\newpage\section*{Usability} \emph{Galka, Selina; selina.galka@uni-graz.at }\\
        
    \emph{Usability} bezeichnet grundsätzlich die
                  Gebrauchstauglichkeit eines Produktes, eines Systems oder eines Dienstes.
                     (Thoden 2017 et al., S. 2) In den Digitalen Geisteswissenschaften
                  geht es vor allem um die \emph{Usability} der User-Interfaces, der
                  nutzerzentrierten Interaktionen mit den Tools und der Infrastruktur und um die
                  Transparenz von Workflows, die den Forschungsprozess möglichst effektiv
                  unterstützen sollen. (Bulatovic et al. 2016, S. 6)\\
            
        In den Digitalen Geisteswissenschaften sind \emph{Usability}, \emph{User Experience} (UX) und \href{http://gams.uni-graz.at/o:konde.98}{Interface} eng miteinander verbunden und spielen
                  insofern eine Rolle, als dass eine intensivere und sorgfältige Umsetzung eines
                  Tools, eines Services oder einer Infrastruktur unter Berücksichtigung der Wünsche
                  und Bedürfnisse der Benutzerinnen und Benutzer eine größere Nutzung der
                  bereitgestellten Inhalte ermöglichen kann (vgl. auch \emph{\href{http://gams.uni-graz.at/o:konde.207}{User-centered design}}). Dennoch haben Studien aber gezeigt, dass die \emph{Usability} nur selten oder erst sehr spät im Entwicklungsprozess eines
                  DH-Projekts getestet wird (Schreibman/Hanlon 2010).\\
            
        \emph{Usability} spielt in der Softwareentwicklung grundsätzlich
                  eine große Rolle und es wurden bereits einige Regeln bzw. Standards zur
                  Gebrauchstauglichkeit festgelegt. Der ISO-Standard 9241–110 definiert z. B. zehn
                  Grundsätze für \href{http://gams.uni-graz.at/o:konde.18}{Benutzerschnittstellen} von interaktiven Systemen, darunter
                  beispielsweise Aufgabenangemessenheit, Selbstbeschreibungsfähigkeit
                  (Dokumentation, Hilfen), Erwartungskonformität (Konsistenz und Anpassung an
                  Bedürfnisse der Benutzerinnen und Benutzer, vgl. \emph{\href{http://gams.uni-graz.at/o:konde.206}{User-esting}}) oder Individualisierbarkeit (vgl. Wikipedia: Usability). Steve
                  Krug definierte ebenfalls zehn Regeln für Usability von Websites (Krug
                     2005). Rosselli del Turco legte außerdem Prinzipien für das \emph{User Interface Design} im Rahmen von Digitalen Editionen
                  fest, welche auch der \emph{Usability} zugute kommen können.
                     (del Turco 2011)\\
            
        Um die Gebrauchstauglichkeit zu gewährleisten, gibt es unterschiedliche Methoden,
                  z. B. eine enge Zusammenarbeit zwischen den unterschiedlichen Beteiligten, also
                  bei \href{http://gams.uni-graz.at/o:konde.59}{Digitalen Editionen}
                  beispielsweise jenen, die die Daten zur Verfügung stellen und diese am besten
                  kennen, und jenen, die die technische Umsetzung der Edition leisten. (Thoden
                     et al. 2017, S. 2) Außerdem bietet sich das \emph{\href{http://gams.uni-graz.at/o:konde.206}{User-Testing}} an. \emph{Usability} zu gewährleisten kann in den Digitalen
                  Geisteswissenschaften aufgrund der möglichen unterschiedlichen Benutzergruppen
                  durchaus fordernd sein. (Thoden et al. 2017, S. 3)\\
            
        \subsection*{Literatur:}\begin{itemize}\item Burghardt, Manuel: Annotationsergonomie: Design-Empfehlungen für
                              Linguistische Annotationswerkzeuge. In: Information. Wissenschaft & Praxis 63: 2012, S. 200–304.\item Rosselli Del Turco, Roberto: After the Editing is Done. Designing a Graphic User
                              Interface for Digital Editions. In: Digital Medievalist 7: 2011.\item Digital Scholarly Editions as Interfaces. Hrsg. von Roman Bleier, Martina Bürgermeister, Helmut W. Klug, Frederike Neuber und Gerlinde Schneider. Norderstedt: 2018, URL: \url{https://www.i-d-e.de/publikationen/schriften/bd-12-interfaces/}.\item Gibbs, Fred; Owens, Trevor: Building Better Digital Humanities Tools: Toward broader
                              audiences and user­centered designs. In: Digital Humanities Quarterly 6: 2012.\item Kirschenbaum, Matthew G.: "So the Colors Cover the Wires": Interface, Aesthetics,
                              and Usability. In: A Companion to Digital Humanities: 2008.\item Krug, Steve: Don't Make Me Think! A Common Sense Approach to Web
                              Usability. New York: 2005.\item Schreibman, Susan; Hanlon, Ann M.: Determining Value for Digital Humanities Tools: Report
                              on a Survey of Tool Developers. In: Digital Humanities Quaterly 4: 2010.\item Thoden, Klaus; Stiller, Juliane; Bulatovic, Natasa; Meiners, Hanna-Lena; Boukhelifa, Nadia: User-Centered Design Practices in Digital Humanities –
                              Experiences from DARIAH and CENDARI. In: ABI Technik 37: 2017, S. 2-11.\item Bulatovic, Natasa; Gnadt, Timo; Romanello, Matteo; Stiller, Juliane; Thoden, Klaus: Usability in Digital Humanities - Evaluating User
                              Interfaces, Infrastructural Components and the Use of Mobile Devices
                              During Research Proces. Hannover: 2016.\item Wikipedia: ISO 9241. URL: \url{https://de.wikipedia.org/wiki/ISO_9241}\end{itemize}\subsection*{Verweise:}\href{https://gams.uni-graz.at/o:konde.206}{User Testing}, \href{https://gams.uni-graz.at/o:konde.207}{User-centered Design}, \href{https://gams.uni-graz.at/o:konde.98}{Interface}, \href{https://gams.uni-graz.at/o:konde.99}{Interface-Design-Cycle}, \href{https://gams.uni-graz.at/o:konde.164}{Responsive Design}, \href{https://gams.uni-graz.at/o:konde.18}{Benutzerschnittstellen}\subsection*{Themen:}Interfaces, Digitale Editionswissenschaft\subsection*{Zitiervorschlag:}Galka, Selina. 2021. Usability. In: KONDE Weißbuch. Hrsg. v. Helmut W. Klug unter Mitarbeit von Selina Galka und Elisabeth Steiner im HRSM Projekt "Kompetenznetzwerk Digitale Edition". URL: https://gams.uni-graz.at/o:konde.205\newpage\section*{User-centered design} \emph{Galka, Selina; selina.galka@uni-graz.at }\\
        
    Unter \emph{User-centered design} versteht man die Umsetzung einer Website, einem Tool, einem System o. Ä. unter Berücksichtigung von \emph{\href{http://gams.uni-graz.at/o:konde.205}{Usability}} und \emph{User-Experience} (UX). Das bedeutet, das Produkt soll für Nutzerinnen und Nutzer gebrauchstauglich sein und eine positive Nutzungserfahrung gewährleisten. \emph{Usability} und positive Nutzungserfahrung spielen eine große Rolle für die Etablierung und Nutzung von Tools aus den Digitalen Geisteswissenschaften; durch die Diversität der Nutzerinnen und Nutzer dieser Tools ist es jedoch oft schwierig, die geforderte Gebrauchstauglichkeit immer zu gewährleisten. (Thoden et al. 2017, S. 3) Außerdem sind auch die Forschungsgebiete und -objekte in den Digitalen Geisteswissenschaften so unterschiedlich, dass es schwierig ist, Standardprozeduren für das Design zu entwickeln. (Thoden et al. 2017, S. 3) Durchgeführte Evaluationen und Testungen haben ergeben, dass \emph{Usability}-Probleme oft durch fehlende  Dokumentation oder fehlende Benutzungshinweise entstehen (Thoden et al. 2017, S. 3; Burghardt 2012; Gibbs/Owens 2012).\\
            
        Laut Klaus Thoden wird in der aktuellen Forschung versucht, einen Ansatz zu entwickeln, der auf die Verbesserung der \emph{Usability} und der \emph{User-Experience} von DH-Tools im Generellen abzielt, anstatt für sehr spezifische DH-Tools, Infrastrukturen oder Webseiten ein \emph{User-centered design} zu entwickeln. (Thoden 2017, S. 4)\\
            
        Um die Gebrauchstauglichkeit zu gewährleisten, gibt es unterschiedliche Methoden, z. B. eine enge Zusammenarbeit zwischen den unterschiedlichen Beteiligten, also bei \href{http://gams.uni-graz.at/o:konde.59}{Digitalen Editionen} beispielsweise jenen, die die Daten zur Verfügung stellen und diese am besten kennen und jenen, die die technische Umsetzung der Edition leisten. Darüber hinaus bietet sich das \emph{\href{http://gams.uni-graz.at/o:konde.206}{User-Testing}} an. (Thoden 2017, S. 2f.)\\
            
        \subsection*{Literatur:}\begin{itemize}\item Burghardt, Manuel: Annotationsergonomie: Design-Empfehlungen für Linguistische Annotationswerkzeuge. In: Information. Wissenschaft & Praxis 63: 2012, S. 200–304.\item Rosselli Del Turco, Roberto: After the Editing is Done. Designing a Graphic User Interface for Digital Editions. In: Digital Medievalist 7: 2011.\item Di Pietro, Chiara; Del Turco, Roberto Rosselli: Between Innovation and Conservation: The Narrow Path of User Interface Design for Digital Scholarly Editions Between Innovation and Conservation. In: Digital Scholarly Editions as Interfaces. Norderstedt: 2018, S. 133–163.\item Garrett, Jesse James: The Elements of User Experience: user-centered design for the Web and beyond: 2011.\item Gibbs, Fred; Owens, Trevor: Building Better Digital Humanities Tools: Toward broader audiences and user­centered designs. In: Digital Humanities Quarterly 6: 2012.\item Green, Harriett E.: Under the Workbench: An Analysis of the Use and Preservation of MONK Text Mining Research Software. In: Literary and Linguistic Computing 29: 2014, S. 23–40.\item Heuwing, Ben; Womser-Hacker, Christa: Zwischen Beobachtung und Partizipation – nutzerzentrierte Methoden für eine Bedarfsanalyse in der digitalen Geschichtswissenschaft. In: Information. Wissenschaft & Praxis 66: 2015, S. 335–344.\item Leblanc, Elina: Design of a Digital Library Interface from User Perspective, and its Consequences for the Design of Digital Scholarly Editions: Findings of the Fonte Gaia Questionnaire. In: Digital Scholarly Editions as Interfaces 12. Norderstedt: 2018, S. 287–315.\item Thoden, Klaus; Stiller, Juliane; Bulatovic, Natasa; Meiners, Hanna-Lena; Boukhelifa, Nadia: User-Centered Design Practices in Digital Humanities – Experiences from DARIAH and CENDARI. In: ABI Technik 37: 2017, S. 2-11.\end{itemize}\subsection*{Verweise:}\href{https://gams.uni-graz.at/o:konde.205}{Usability}, \href{https://gams.uni-graz.at/o:konde.206}{User Testing}, \href{https://gams.uni-graz.at/o:konde.148}{Benutzerinnen und Benutzer Digitaler Editionen}, \href{https://gams.uni-graz.at/o:konde.98}{Interface}\subsection*{Themen:}Interfaces\subsection*{Zitiervorschlag:}Galka, Selina. 2021. User-centered design. In: KONDE Weißbuch. Hrsg. v. Helmut W. Klug unter Mitarbeit von Selina Galka und Elisabeth Steiner im HRSM Projekt "Kompetenznetzwerk Digitale Edition". URL: https://gams.uni-graz.at/o:konde.207\newpage\section*{User-testing} \emph{Galka, Selina; selina.galka@uni-graz.at}\\
        
    Um \emph{\href{http://gams.uni-graz.at/o:konde.205}{Usability}}, also die Gebrauchstauglichkeit eines Produktes, einer Software oder eines
                  Systems zu messen und ggf. zu verbessern, bietet sich das \emph{User-testing} an. Webanwendungen können mittels Evaluation durch die
                  Benutzergruppen besser auf deren Bedürfnisse angepasst werden und dadurch zu einem
                  Fortschritt in der Anwendbarkeit digitaler Tools o. Ä. beitragen. Die \emph{Usability}-Forschung liefert für die Evaluation von Tools u.
                  a. unterschiedliche Methoden, darunter beispielsweise “Benutzertests mit
                  Prototypen, Personas, Beobachtungen, Befragungen, Fokusgruppen, lautes Denken,
                  Expert-Review oder heuristische Evaluationen.” (Dewitz/Münster/Niebling
                     2019, S. 53)\\
            
        Leider sind \emph{Usability}-Tests in den Digitalen
                  Geisteswissenschaften momentan noch eher unüblich bzw. werden sie oft auch erst
                  sehr spät in der Projektumsetzung eingebunden, was in weiterer Folge Hürden für
                  die Etablierung digitaler Methoden schafft. In den meisten
                  Softwareentwicklungsprozessen sind \emph{Usability}-Tests aber
                  bereits seit langem ein Standardelement. (Dewitz/Münster/Niebling 2019, S.
                     53; Schreibmann 2010)\\
            
        \subsection*{Literatur:}\begin{itemize}\item Burghardt, Manuel: Annotationsergonomie: Design-Empfehlungen für
                              Linguistische Annotationswerkzeuge. In: Information. Wissenschaft & Praxis 63: 2012, S. 200–304.\item Caria, Federico; Mathiak, Brigitte: A Hybrid Focus Group for the Evaluation of Digital
                              Scholarly Editions of Literary Authors. In: Digital Scholarly Editions as Interfaces 12. Norderstedt: 2018, S. 267–285.\item Dewitz, Leyla; Münster, Sandra; Niebling, Florian: Usability-Testing für Softwarewerkzeuge in den Digital
                              Humanities am Beispiel von Bildrepositorien (Workshop). In: Book of Abstracts Digital Humanities im deutschsprachigen Raum
                              2019. Mainz: 2019, S. 52–55.\item Gibbs, Fred; Owens, Trevor: Building Better Digital Humanities Tools: Toward broader
                              audiences and user­centered designs. In: Digital Humanities Quarterly 6: 2012.\item Green, Harriett E.: Under the Workbench: An Analysis of the Use and
                              Preservation of MONK Text Mining Research Software. In: Literary and Linguistic Computing 29: 2014, S. 23–40.\item Kirschenbaum, Matthew G.: "So the Colors Cover the Wires": Interface, Aesthetics,
                              and Usability.. In: A Companion to Digital Humanities: 2008.\item Krug, Steve: Don't Make Me Think! A Common Sense Approach to Web
                              Usability.. New York: 2005.\item Schreibman, Susan; Hanlon, Ann M.: Determining Value for Digital Humanities Tools: Report
                              on a Survey of Tool Developers. In: Digital Humanities Quaterly 4: 2010.\item Thoden, Klaus; Stiller, Juliane; Bulatovic, Natasa; Meiners, Hanna-Lena; Boukhelifa, Nadia: User-Centered Design Practices in Digital Humanities –
                              Experiences from DARIAH and CENDARI. In: ABI Technik 37: 2017, S. 2-11.\item Bulatovic, Natasa; Gnadt, Timo; Romanello, Matteo; Stiller, Juliane; Thoden, Klaus: Usability in Digital Humanities - Evaluating User
                              Interfaces, Infrastructural Components and the Use of Mobile Devices
                              During Research Proces. Hannover: 2016.\end{itemize}\subsection*{Verweise:}\href{https://gams.uni-graz.at/o:konde.205}{Usability}, \href{https://gams.uni-graz.at/o:konde.207}{User-centered Design}\subsection*{Themen:}Interfaces\subsection*{Zitiervorschlag:}Galka, Selina. 2021. User-testing. In: KONDE Weißbuch. Hrsg. v. Helmut W. Klug unter Mitarbeit von Selina Galka und Elisabeth Steiner im HRSM Projekt "Kompetenznetzwerk Digitale Edition". URL: https://gams.uni-graz.at/o:konde.206\newpage\section*{Versionierung} \emph{Lobis, Ulrich; ulrich.lobis@uibk.ac.at / Wang-Kathrein, Joseph; joseph.wang@uibk.ac.at}\\
        
    Mit dem Begriff ‘Versionierung’ wird zunächst auf die Speicherung der Zwischenstufen von Dateien oder Verzeichnisstrukturen verwiesen. Für jede Datei oder jeden Ordner wird bei jeder Änderung eine neue Version abgelegt. Die früheren Versionen der Dateien werden aber nicht gänzlich gelöscht und stehen Benutzerinnen und Benutzern nach Bedarf zur Verfügung.\\
            
        Grundsätzlich kann man zwischen der Versionierung der eigenen oder projekt-internen Dateien und der Versionierung von bereits publizierten Daten unterscheiden. Im ersten Fall hilft der Einsatz von Versionierungsverwaltungssoftware nicht nur bei der Archivierung und Dokumentation von Projektfortschritten, sie erleichtert auch die Zusammenarbeit zwischen Projektmitgliedern. Sind Daten bzw. Editionen bereits veröffentlicht, dann kann der Einsatz von Versionierungslösungen dazu dienen, den Benutzerinnen und Benutzern den früheren Zustand des Datensatzes zur Verfügung zu stellen. Hier dient die Versionierung der Zitierbarkeit und der Nachvollziehbarkeit von Datenänderungen.\\
            
        Die bekannteste Anwendung, die mit Versionierungen arbeitet, dürfte \emph{MediaWiki} bzw. \emph{Wikipedia} sein. Diese Anwendung erlaubt ihren Nutzerinnen und Nutzern, den Zustand eines Artikels zu einem früheren Zeitpunkt wiederherzustellen.\\
            
        Für den Einsatz bei der Projektführung dürfte \emph{git} die bekannteste Hauptanwendung sein, die u. a. bei \emph{GitLab} und \emph{GitHub} zum Einsatz kommt. Mit \emph{git} ist es möglich, Verzeichnisstrukturen unter der Versionierung zu erstellen, sodass bei Bedarf der Zustand des Verzeichnisses zu einem früheren Zeitpunkt wiederhergestellt werden kann. Außerdem können die Unterschiede zwischen den einzelnen Versionen einer Datei angezeigt werden.\\
            
        \emph{Git} trennt zwischen dem Repositorium (\emph{git repository}) und dem Arbeitsverzeichnis (\emph{working tree}). Das Repositorium kann dabei zentral auf einem Server oder dezentral auf dem eigenen Computer liegen. Im Gegensatz zu anderen Versionierungsverwaltungen wie \emph{subversion} werden auch alle Änderungen im Arbeitsverzeichnis gespeichert, sodass Versionen auch dann abgerufen werden können, wenn der Server nicht zur Verfügung steht.\\
            
        Ein typischer Workflow mit \emph{git} schaut so aus: Zuerst erstellt man ein Repositorium bzw. erhält den Link zu diesem. Dieses Repositorium wird zunächst geklont (\emph{clone}), und ein Arbeitsverzeichnis wird auf dem lokalen Computer angelegt. Die Bearbeiterin bzw. der Bearbeiter verändert die Dateien im Verzeichnis nach Belieben. Nach der Bearbeitung der Dateien werden die Änderungen dem lokalen Arbeitsverzeichnis übergeben (\emph{commit}) und dann ins Repositorium geschoben (\emph{push}). Dabei vergleicht \emph{git} die Ausgangsversion der Änderungen mit der aktuellsten Version dieser Datei im Repositorium. Diese Änderungen werden angenommen, wenn beide Versionen dieselben sind. Wird eine Datei zwischendurch von einer anderen Person bearbeitet und liegt deswegen eine neue Version dieser Datei vor, warnt \emph{git} die Bearbeiterin/den Bearbeiter vor dem Versionskonflikt (\emph{conflict}). \emph{Git} lässt die Änderungen erst dann im Repositorium zu, wenn die Benutzerin/der Benutzer eigene und fremde Änderungen zusammengeführt hat (\emph{merge}).\\
            
        \subsection*{Verweise:}\href{https://gams.uni-graz.at/o:konde.104}{Kollaboration}\subsection*{Software:}\href{https://git-scm.com/}{GIT}, \href{http://github.com}{Github}, \href{http://gitlab.com/}{Gitlab}\subsection*{Projekte:}\href{https://www.mediawiki.org/wiki/MediaWiki/de}{MediaWiki}, \href{https://de.wikipedia.org/wiki/Wikipedia:Hauptseite}{Wikipedia}, \href{https://git-scm.com/}{git}, \href{https://about.gitlab.com/}{GitLab}, \href{https://github.com/}{GitHub}\subsection*{Themen:}Archivierung\subsection*{Zitiervorschlag:}Lobis, Ulrich; Wang-Kathrein, Joseph. 2021. Versionierung. In: KONDE Weißbuch. Hrsg. v. Helmut W. Klug unter Mitarbeit von Selina Galka und Elisabeth Steiner im HRSM Projekt "Kompetenznetzwerk Digitale Edition". URL: https://gams.uni-graz.at/o:konde.14\newpage\section*{Virtual International Authority File (VIAF)} \emph{Pollin, Christopher; christopher.pollin@uni-graz.at }\\
        
    VIAF ist eine virtuelle, internationale Normdatei und bietet einen Dienst an, der es Bibliotheken ermöglichen soll, abgeglichene, verlinkte und geclusterte \href{http://gams.uni-graz.at/o:konde.147}{Normdaten} zu verwenden. Als Normdaten werden standardisierte und eindeutig zuzuordnende Einträge für bestimmte Gegenstandsbereiche verstanden.\\
            
        Normdaten in VIAF werden von nationalen Partnern aggregiert, wie etwa der Gemeinsamen Normdatenbank (\href{http://gams.uni-graz.at/o:konde.109}{GND}) in Deutschland, und erlauben es, Personen, Organisationen, Werke oder Orte zu identifizieren. So entsteht ein zusammengeführter Datensatz inklusive einer Konkordanzdatei von jeder erfassten Entität in VIAF. Dieser steht für Recherchen und Datenaustausch zur Verfügung. Zusätzlich werden weiterführende Informationen zu den einzelnen Entitäten mit angeboten. \\
            
        Das \emph{Protocol for Metadata Harvesting} der \emph{Open Archives Initiative} (OAI-PMH) wird zur Aktualisierung der Datenbestände in VIAF verwendet. Monatlich findet ein Musterabgleich, ein sogenanntes ‘\emph{Pattern Matching}’, statt. Dabei werden neue Datensätze mit den bestehenden zusammengeführt, um sicherzustellen, dass Dubletten aufgelöst werden. Als Identifikator fungiert eine eigene Normdatennummer in Form eines URI. Diese können somit auch als Referenzpunkt für \emph{\href{http://gams.uni-graz.at/o:konde.8}{Linked Open Data}} genutzt werden.\\
            
        So repräsentiert die URI https://viaf.org/viaf/19685936 die Person Stefan Zweig und gleichzeitig ein Dokument im Web, das Information über Stefan Zweig anbietet.  \\
            
        \subsection*{Literatur:}\begin{itemize}\item Bennett, Rick; Hengel-Dittrich, Christina; O’Neill, Edward T.; Tillett, Barbara B.: VIAF (Virtual International Authority File): Linking Die Deutsche Bibliothek and Library of Congress Name Authority Files. In: World Library and Information Congress: 72nd IFLA General Conference and Coucil. Seoul: 2006.\item Jannidis, Fotis; Kohle, Hubertus: Digital Humanities. Eine Einführung. Mit Abbildungen und Grafiken Digital Humanities. Hrsg. von  und Malte Rehbein. Stuttgart: 2017.\item OCLC: VIAF. URL: \url{https://www.oclc.org/de/viaf.html}\end{itemize}\subsection*{Verweise:}\href{https://gams.uni-graz.at/o:konde.147}{Normdaten}, \href{https://gams.uni-graz.at/o:konde.167}{Semantic Web}, \href{https://gams.uni-graz.at/o:konde.109}{GND}, \href{https://gams.uni-graz.at/o:konde.107}{GeoNames}, \href{https://gams.uni-graz.at/o:konde.112}{Wikidata}, \href{https://gams.uni-graz.at/o:konde.108}{Getty}, \href{https://gams.uni-graz.at/o:konde.10}{Metadata Harvesting}\subsection*{Projekte:}\href{https://viaf.org}{VIAF}\subsection*{Themen:}Archivierung\subsection*{Lexika}\begin{itemize}\item \href{https://edlex.de/index.php?title=Virtual_International_Authority_File_(VIAF)}{Edlex: Editionslexikon}\end{itemize}\subsection*{Zitiervorschlag:}Pollin, Christopher. 2021. Virtual International Authority File (VIAF). In: KONDE Weißbuch. Hrsg. v. Helmut W. Klug unter Mitarbeit von Selina Galka und Elisabeth Steiner im HRSM Projekt "Kompetenznetzwerk Digitale Edition". URL: https://gams.uni-graz.at/o:konde.111\newpage\section*{Visualisierungstools} \emph{Galka, Selina; selina.galka@uni-graz.at}\\
        
    Visualisierungstools sind nützlich, um Informationen anschaulich zu machen, z. B. wenn eine komplexe Datenlage vorhanden ist, die mittels einer Visualisierung besser zugänglich wird, aber auch, um eine große Menge an Informationen visuell aufzubereiten oder überhaupt erst analysierbar zu machen (\href{http://gams.uni-graz.at/o:konde.54}{Datenvisualisierung}). (Rehbein 2017, S. 328)\\
            
        Für \href{http://gams.uni-graz.at/o:konde.59}{Digitale Editionen} gibt es eine Reihe an Tools, die für unterschiedliche Zwecke eingesetzt werden können. Raumbezogene Daten können beispielsweise mit Datenkarten visualisiert werden, wie z. B. mit \emph{Palladio} oder \emph{StoryMap.js}. Temporale Daten können mit eindimensionalen Zeitreihen oder mehreren Zeitreihen nebeneinander (vgl. \emph{Timeline.js} oder \emph{Timeline Storyteller} von \emph{Microsoft}) dargestellt werden, sodass sie vergleichbar werden. Raum- und Zeitdaten können in der Visualisierung aber auch kombiniert werden (vgl. z. B. \emph{Dariah Geobrowser} oder \emph{Neatline}). \\
            
        In Digitalen Editionen können aber auch abstrakte Daten, die keine Ordnung wie Raum oder Zeit aufweisen, visualisiert werden. Die Datenmessung erfolgt hier an Objekten, die untereinander nicht in Beziehung stehen –  solche Daten können mit unterschiedlichsten Diagrammen visualisiert werden, wie z. B. mit Punkt- oder Liniendiagrammen. (Rehbein 2017, S. 336)\emph{RAW Graphs} oder die \emph{Javascript}-Bibliothek \emph{D3.js} bieten hier eine Vielzahl an Möglichkeiten.\\
            
        Für relationale Daten, bei denen Datenelemente untereinander in Beziehung stehen (hierarchisch oder im Sinne eines Netzwerkes), bieten sich Baumstrukturen oder Graphen als Visualisierungsmöglichkeiten an. (Rehbein 2017, S. 336) Vor allem in der Stilometrie sind Visualisierungen dieser Art von großer Bedeutung; häufig genutzte Tools sind hier z. B. \emph{Voyant Tools}, \emph{Stylo} oder \emph{Gephi}, welche die Möglichkeiten bieten, Bäume, Cluster und Netzwerke zu erstellen.\\
            
        \emph{Canva} bietet unterschiedliche Designmöglichkeiten, die beispielsweise für das Webdesign nützlich sein können (z. B. Erstellung eines \emph{\href{http://gams.uni-graz.at/o:konde.135}{Mockups}}).\\
            
        \subsection*{Literatur:}\begin{itemize}\item Sinclair, Stéfan; Ruecker, Stan; Radzikowska, Milena: Information Visualization for Humanities Scholars. In: Literary Studies in the Digital Age: 2013.\item Rehbein, Malte: Informationsvisualisierung. In: Digital Humanities. Eine Einführung. Stuttgart: 2017, S. 328–342.\end{itemize}\subsection*{Software:}\href{https://www.canva.com/}{Canva}, \href{http://www.chronozoomproject.org/}{ChronoZoom}, \href{https://d3js.org}{D3js}, \href{https://geobrowser.de.dariah.eu/}{Dariah Geobrowser}, \href{https://nodegoat.net/}{Node Goat}, \href{https://www.overviewdocs.com/}{Overview}, \href{http://hdlab.stanford.edu/palladio/}{Palladio}, \href{https://rawgraphs.io/}{RAW (3D)}, \href{https://storymap.knightlab.com/}{Storymap}, \href{https://public.tableau.com/s/}{Tableau}, \href{http://timeline.knightlab.com/}{TimelineJS}, \href{https://timelinestoryteller.com/app/}{TimelineStoryteller}, \href{https://github.com/leoba/VisColl}{Viscoll}, \href{https://visone.info/}{Visione}, \href{https://voyant-tools.org/}{Voyant}, \href{http://www.teitok.org/index.php?action=about}{TEITOK}, \href{http://www.fon.hum.uva.nl/praat/}{Praat}, \href{https://stemmaweb.net/}{The Stemmaweb Project}, \href{http://corpus-tools.org/annis/}{ANNIS}\subsection*{Verweise:}\href{https://gams.uni-graz.at/o:konde.54}{Datenvisualisierung}, \href{https://gams.uni-graz.at/o:konde.113}{Lagenvisualisierung}, \href{https://gams.uni-graz.at/o:konde.135}{Mockup}, \href{https://gams.uni-graz.at/o:konde.136}{Mockup-Software}\subsection*{Themen:}Datenanalyse, Software und Softwareentwicklung\subsection*{Zitiervorschlag:}Galka, Selina. 2021. Visualisierungstools. In: KONDE Weißbuch. Hrsg. v. Helmut W. Klug unter Mitarbeit von Selina Galka und Elisabeth Steiner im HRSM Projekt "Kompetenznetzwerk Digitale Edition". URL: https://gams.uni-graz.at/o:konde.210\newpage\section*{Volltextsuche} \emph{Rastinger, Nina Claudia; ninaclaudia.rastinger@oeaw.ac.at }\\
        
    Im Rahmen einer Volltextsuche wird im gesamten Text eines Dokuments bzw. mehrerer Dokumente innerhalb einer Datenbank gesucht. Im Gegensatz zu anderen Suchstrategien, wie der  \href{http://gams.uni-graz.at/o:konde.82}{facettierten Suche}, werden eingegebene Termini also nicht mit den Metadaten der jeweiligen Dokumente, sondern mit den von ihnen beinhalteten Wörtern bzw. einem Wort(formen)-Index abgeglichen und nur bei Übereinstimmung auch gefunden (Gyorodi et al. 2010, S. 736; Beall 2008, S. 438). Dies bedeutet aber auch, dass variierende Schreibungen, wie sie etwa in historischen Texten oftmals vorliegen, ebenso immer mitgedacht werden müssen wie die Tatsache, dass Homonyme der Suchbegriffe ebenfalls gefunden werden: Die Suche nach ‘Bank’ differenziert nicht zwischen dem Geldinstitut und der Sitzgelegenheit. Ebenso wenig finden Synonyme bzw. andere Ausdrücke für ein gesuchtes Konzept automatisch Eingang in den Suchprozess; diese müssen eigens gewählt und in die Suchleiste eingegeben werden. Infolgedessen hängt der Erfolg einer Volltextsuche nicht nur von der Qualität der jeweiligen Daten ab, sondern immer auch vom individuellen Suchverhalten der Nutzerinnen und Nutzer und der Adäquatheit der von ihnen ausgewählten Keywords (Resch 2019, S. 124; Beall 2008, S. 442).
               \\
            
        Um diesen Limitationen entgegenzuwirken und Userinnen und Usern das Bestimmen über den Fein- bzw. Grobheitsgrad ihres Suchprozesses zu erleichtern, werden Volltextsuchen meist über zusätzliche Optionen – wie Wildcards, Boolesche Operatoren oder reguläre Ausdrücke (RegEx) – erweitert: Mit dem RegEx-Suchbefehl /S(a|o)l(a|o)mon*/ etwa werden mehrere Schreibweisen desselben Eigennamens gleichzeitig suchbar. Zudem birgt die Volltextsuche eine Reihe an Vorteilen in sich: So kann das Durchsuchen des wörtlichen Inhalts von Texten – im Sinne eines \emph{\href{http://gams.uni-graz.at/o:konde.71}{Distant Readings}} – unter anderem im Hinblick auf die ‘Unüberschaubarkeit’ geisteswissenschaftlichen Datenmaterials hilfreich sein (Limpinsel 2013, S. 177) und Forschende auf Text(ausschnitt)e stoßen lassen, welche ihnen über eine rein metadatenorientierte Suche eventuell entgangen wären (Kann/Hintersonnleitner 2015, S. 79). Vor diesem Hintergrund macht es Sinn, dass volltextliche Durchsuchbarkeit heutzutage nicht nur im World Wide Web den Standard repräsentiert (Beall 2008, S. 438), sondern zunehmend auch als Zielpunkt für \href{http://gams.uni-graz.at/o:konde.59}{Digitale Editionen} und Textsammlungen (vgl. etwa Resch 2019; Kann/Hintersonnleitner 2015) gesetzt wird – denn wie Müller und Hermes-Wladarsch (2017, S. 50) es treffend zusammenfassen: „Erst über durchsuchbare Volltexte potenzieren sich die Möglichkeiten wissenschaftlichen Erkenntnisgewinns.“\\
            
        \subsection*{Literatur:}\begin{itemize}\item Beall, Jeffrey: The Weaknesses of Full-Text Searching. In: The Journal of Academic Librarianship 34: 2008, S. 438–444.\item Gyorodi, Cornelia; Gyorodi, Robert; Pecherle, George; Cornea, George Mihai: Full-Text Search Engine using MySQL. In: International Journal of Computers Communications & Control 5: 2010, S. 735–743.\item Kann, Bettina; Hintersonnleitner, Michael: Volltextsuche in historischen Texten. Erfahrungen aus den Projekten der Österreichischen Nationalbibliothek. In: BIBLIOTHEK – Forschung und Praxis 39: 2015, S. 73–79.\item Limpinsel, Marco: Volltextsuche und der philologische Habitus. In: Lesen, Schreiben, Erzählen: kommunikative Kulturtechniken im digitalen Zeitalter. Frankfurt: 2013, S. 171–185.\item Müller, Maria Elisabeth; Hermes-Wladarsch, Maria: Die Digitalisierung der deutschsprachigen Zeitungen des 17. Jahrhunderts – ein Projekt mit Komplexität!. In: Die Zeitung als Medium in der neueren Sprachgeschichte. Korpora – Analyse – Wirkung. Berlin: 2017, S. 39–59.\item Resch, Claudia: Das Wien[n]erische Diarium und seine digitale Erschließung oder „Was
                              die Zeitungsleser vor Geräte haben müssen?“. In: Wiener Geschichtsblätter 74: 2019, S. 115–130.\end{itemize}\subsection*{Software:}\href{https://www.elastic.co/de/}{elasticsearch}, \href{http://lucene.apache.org/solr/}{Solr}, \href{https://textgrid.de/}{TextGrid}, \href{http://www.tustep.uni-tuebingen.de/}{TUSTEP}, \href{http://www.teitok.org/index.php?action=about}{TEITOK}, \href{http://sphinxsearch.com/about/sphinx/}{Sphinx Search}, \href{https://www.mysql.com/}{MySQL}\subsection*{Verweise:}\href{https://gams.uni-graz.at/o:konde.71}{Distant Reading}, \href{https://gams.uni-graz.at/o:konde.82}{Facettierte Suche}, \href{https://gams.uni-graz.at/o:konde.94}{Historische Korpora}, \href{https://gams.uni-graz.at/o:konde.146}{Normalisierung}, \href{https://gams.uni-graz.at/o:konde.197}{Transkription}\subsection*{Projekte:}\href{https://digitarium.acdh.oeaw.ac.at/}{Wien[n]erisches DIGITARIUM}, \href{http://anno.onb.ac.at/}{ANNO - AustriaN Newspapers Online         }, \href{http://www.deutschestextarchiv.de/}{Deutsches Textarchiv}, \href{https://acdh.oeaw.ac.at/abacus/}{Austrian Baroque Corpus (ABaC:us)}, \href{http://mhdbdb.sbg.ac.at/}{Mittelhochdeutsche Begriffsdatenbank (MHDBDB)}, \href{https://traveldigital.acdh.oeaw.ac.at/}{travel!digital}, \href{http://gams.uni-graz.at/context:ufbas}{Uhrfehdebücher der Stadt Basel - Digitale Edition}\subsection*{Themen:}Datenanalyse, Interfaces\subsection*{Zitiervorschlag:}Rastinger, Nina Claudia. 2021. Volltextsuche. In: KONDE Weißbuch. Hrsg. v. Helmut W. Klug unter Mitarbeit von Selina Galka und Elisabeth Steiner im HRSM Projekt "Kompetenznetzwerk Digitale Edition". URL: https://gams.uni-graz.at/o:konde.211\newpage\section*{WebLicht} \emph{Rieger, Lisa; lrieger@edu.aau.at }\\
        
    \emph{WebLicht} (\emph{Web-Based Linguistic Chaining Tool}) ist ein Webservice, dessen Entwicklung 2008 als Teil des Projekts D-SPIN gestartet wurde und bis heute im Rahmen des Nachfolgeprojekts CLARIN-D stets verbessert wird. Es vereint verschiedene linguistische Tools zur automatischen Textannotation, wie Tokenizer, \emph{\href{http://gams.uni-graz.at/o:konde.156}{Part-of-Speech-Tagger}} oder Parser. Dafür wurden sowohl bereits existierende als auch speziell zu diesem Zweck entwickelte Tools implementiert, die vom User über ein übersichtliches User-\href{http://gams.uni-graz.at/o:konde.98}{Interface} individuell kombiniert werden können. Die durchgeführten \href{http://gams.uni-graz.at/o:konde.17}{Annotationen} können anschließend in Form von Tabellen oder Baumdiagrammen \href{http://gams.uni-graz.at/o:konde.54}{visualisiert} werden. Für sämtliche Anwendungsmöglichkeiten stellt \emph{WebLicht} eine ausführliche Anleitung und Dokumentation zur Verfügung. (WebLicht o. J.)\\
            
        Um den Service verwenden zu können, muss vorweg ein Account angelegt werden. Dies erfolgt im Allgemeinen über die zur Auswahl stehenden Universitäten und Institutionen. Befindet sich die Heimatinstitution nicht auf der Liste, kann von Forscherinnen und Forschern und Studierenden auch persönlich eine Anfrage gestellt werden. Nach erfolgreichem Login erscheint das Input-Fenster, das einem folgende Möglichkeiten zur Verfügung stellt: die manuelle Eingabe eines Textes, die Verwendung eines Musters oder der Upload einer Datei in den unterstützten Dateiformaten. Als nächstes müssen Dokumententyp und Sprache ausgewählt werden. Im Falle der deutschen Sprache kann die Annotation im vordefinierten \emph{Easy Mode} oder im \emph{Advanced Mode}, bei dem die Tools manuell gewählt werden, durchgeführt werden. Im \emph{Easy Mode} muss nur noch ausgewählt werden, welche Form der Annotation (PoS-Tags, Morphologie, \emph{Dependency-Parsing}) vorgenommen werden soll. Dieser existiert jedoch nicht für jede Sprache. Das Ergebnis wird nach kurzer Zeit im Visualisierungsfenster angezeigt und kann in tabellarischer Form als Excel-Datei sowie auch als \href{http://gams.uni-graz.at/o:konde.215}{XML}-basierte TCF-Datei heruntergeladen und gespeichert werden. (vgl. ähnliche Anleitung bei (Hirschmann 2019, S.74))\\
            
        \subsection*{Literatur:}\begin{itemize}\item Hinrichs, Erhard; Hinrichs, Marie; Zastrow, Thomas: WebLicht: Web-based LRT services for German. In: Proceedings of the ACL 2010 System Demonstrations: 2010, S. 25–19.\item Hirschmann, Hagen: Korpuslinguistik. Eine Einführung. Mit Abbildungen und Grafiken Korpuslinguistik. Berlin: 2019, URL: \url{https://link.springer.com/book/10.1007%2F978-3-476-05493-7}.\item WebLicht. Main Page Main page. URL: \url{https://weblicht.sfs.uni-tuebingen.de/weblichtwiki/index.php/Main_Page}\end{itemize}\subsection*{Software:}\href{https://weblicht.sfs.uni-tuebingen.de/weblicht/}{weblicht}\subsection*{Verweise:}\href{https://gams.uni-graz.at/o:konde.60}{Digitalisierung}, \href{https://gams.uni-graz.at/o:konde.115}{Lemmatisierung}, \href{https://gams.uni-graz.at/o:konde.17}{Annotation}, \href{https://gams.uni-graz.at/o:konde.61}{Digitalisierungsdienste}, \href{https://gams.uni-graz.at/o:konde.99}{Transkriptionswerkzeuge}, \href{https://gams.uni-graz.at/o:konde.30}{Annotationsumgebung}, \href{https://gams.uni-graz.at/o:konde.156}{Part of Speech Tagging}, \href{https://gams.uni-graz.at/o:konde.176}{Tagger}, \href{https://gams.uni-graz.at/o:konde.216}{xTokenizer}, \href{https://gams.uni-graz.at/o:konde.31}{API}, \href{https://gams.uni-graz.at/o:konde.205}{Usability}, \href{https://gams.uni-graz.at/o:konde.79}{Einführung: Was ist XML/TEI?}\subsection*{Themen:}Natural Language Processing, Software und Softwareentwicklung\subsection*{Zitiervorschlag:}Rieger, Lisa. 2021. WebLicht. In: KONDE Weißbuch. Hrsg. v. Helmut W. Klug unter Mitarbeit von Selina Galka und Elisabeth Steiner im HRSM Projekt "Kompetenznetzwerk Digitale Edition". URL: https://gams.uni-graz.at/o:konde.212\newpage\section*{Werkausgabe} \emph{Galka, Selina; selina.galka@uni-graz.at }\\
        
    Unter einer Werkausgabe versteht man die Ausgabe/Edition von Werken einer Person. Der Begriff kann hier mit der \href{http://gams.uni-graz.at/o:konde.91}{Gesamtausgabe} überlappen, welche versucht, sämtliche Werke einer Person herauszugeben, eine Werkausgabe kann aber auch nur eine bestimmte Auswahl von Werken einer Person umfassen. \\
            
        Werkausgaben erscheinen aufgrund ihres Umfangs normalerweise (in der Buchkultur) in Reihen über einen längeren Zeitraum. Die \href{http://gams.uni-graz.at/o:konde.59}{Digitale Edition} bietet sich hier insofern besonders als Publikationsmedium an, da sie kontinuierlich und leicht erweitert werden kann. Werkausgaben können auch \href{http://gams.uni-graz.at/o:konde.96}{hybrid} publiziert werden, also digital und gedruckt, wie z. B. Werkausgabe und Kommentar zu Werner Kofler. (Kofler 2018)\\
            
        \subsection*{Literatur:}\begin{itemize}\item Lützeler, Paul Michael: Ein Plädoyer für kommentierte Werkausgaben. Zu Theorie und Praxis bei der Edition von Gesamtausgaben moderner Autoren. In: Musil-Forum 2: 1976, S. 270–286.\item Strelka, Joseph P.: Edition und Interpretation. Grundsätzliche Überlegungen zu ihrer gegenseitigen Abhängigkeit am Beispiel von Werkausgaben neuerer deutscher Literatur. In: Textkritik und Interpretation. Festschrift für Karl Konrad Polheim. Hg. von Heimo Reinitzer: 1987, S. 21–38.\item Kofler, Werner: Kommentierte Werkausgabe. Sonderzahl, 2018: 2018.\end{itemize}\subsection*{Verweise:}\href{https://gams.uni-graz.at/o:konde.91}{Gesamtausgabe}, \href{https://gams.uni-graz.at/o:konde.59}{Digitale Edition}, \href{https://gams.uni-graz.at/o:konde.96}{Hybridedition}\subsection*{Projekte:}\href{https://dme.mozarteum.at}{Digitale Mozartedition}, \href{http://www.uwe-johnson-werkausgabe.de}{Uwe Johnson Werkausgabe}, \href{https://gams.uni-graz.at/context:kofler}{Werner Kofler: Kommentar zur Werkausgabe}, \href{http://www.digital-musicology.at/de-at/edi_birck.html}{Digitale Werkausgabe Wenzel Birck}, \href{http://www.uwe-johnson-werkausgabe.de}{Uwe Johnson Werkausgabe}\subsection*{Themen:}Digitale Editionswissenschaft\subsection*{Lexika}\begin{itemize}\item \href{https://edlex.de/index.php?title=Werkausgabe}{Edlex: Editionslexikon}\end{itemize}\subsection*{Zitiervorschlag:}Galka, Selina. 2021. Werkausgabe. In: KONDE Weißbuch. Hrsg. v. Helmut W. Klug unter Mitarbeit von Selina Galka und Elisabeth Steiner im HRSM Projekt "Kompetenznetzwerk Digitale Edition". URL: https://gams.uni-graz.at/o:konde.213\newpage\section*{Wikidata} \emph{Steiner, Christian; christian.steiner@uni-graz.at }\\
        
    \emph{Wikidata} wurde 2012 von der \emph{Wikimedia
                     Foundation} ins Leben gerufen, um eine gemeinsame Wissensbasis für
                  strukturierte Daten für die verschiedenen Sprachen in \emph{Wikipedia}, \emph{Wikimedia Commons} und anderen
                  Stiftungsprojekten zu schaffen. \emph{Wikidata} teilt ihren
                  gemeinschaftsbasierten Ansatz mit \emph{Wikipedia}, wo jede und
                  jeder Daten erweitern und bearbeiten kann. Die Wissensdatenbank (\emph{knowledge base}) basiert auf den Prinzipien von \emph{\href{http://gams.uni-graz.at/o:konde.8}{Linked Open Data}} und benutzt die Software Wikibase
                  als Backend. \\
            
        \emph{Wikidata} trennt in seinem Datenmodell \emph{statements}, \emph{items}, \emph{properties} und l\emph{exemes}.
              
                  \emph{Statements} sind ganz im Sinne von \href{http://gams.uni-graz.at/o:konde.131}{RDF} in \emph{Triples} abbildbar,
                  obwohl sie formal in \emph{Wikibase key-value}-Paare sind. Sie
                  bestehen aus einem \emph{property}, dem ein oder mehrere Werte
                  zugeordnet sein können und das sich auf ein \emph{item} bezieht
                  (z. B.: \emph{University of Graz} [Q622683] [Q303] \emph{instance of} [P31] \emph{university} [Q3918],
                  übersetzt: ”Die Universität Graz ist eine Universität.”). \\
            
        \emph{Items} repräsentieren Konzepte – von konkreten Objekten oder
                  Themenfeldern bis hin zu immateriellen Ideen. Jedes \emph{item}
                  wird durch eine eindeutige Nummer identifiziert, der der Buchstabe Q vorangestellt
                  wird und die gemeinhin als ‘QID’ bezeichnet wird. Dies ermöglicht eine
                  sprachunabhängige Identifizierung der Konzepte. Beispiele für \emph{items} sind Lebensmittel [Q2095], Liebe [Q316], Apfel [Q89] oder Universum
                  [Q1]. Die \emph{labels}, also semantische Bezeichnungen der \emph{items}, müssen nicht eindeutig sein. Es gibt zum Beispiel
                  zwei Artikel mit dem Namen ‘Elvis Presley’: Elvis Presley [Q303] steht für den
                  amerikanischen Sänger und Schauspieler und Elvis Presley [Q610926] steht für sein
                  selbstbetiteltes Album. Grundsätzlich besteht ein \emph{item} aus
                  einem \emph{label}, einer \emph{description},
                  optional mehreren Aliasnamen und einer gewissen Anzahl von \emph{statements}.\\
            
        Ein \emph{property} beschreibt den Datenwert eines \emph{statements} und kann als eine Kategorie von Daten verstanden
                  werden oder als Kante im Graph. Beispiele für \emph{properties}
                  sind etwa \emph{instance of} [P31], \emph{subclass
                     of} [P279] oder \emph{color} [P462].\\
            
        \emph{Lexemes} sind in \emph{Wikidata items} mit
                  einer Datenstruktur speziell für lexikographische Daten. \\
            
        \emph{Wikidata} hat sich mittlerweile zu einer multidisziplinären,
                  maschinenlesbaren, zentralisierten und vernetzten Wissensdrehscheibe entwickelt
                  und wird für immer mehr Anwendungsfälle genutzt. Es bietet sich für \href{http://gams.uni-graz.at/o:konde.59}{Digitale Editionen} an, das
                  historische Wissen in und über das edierte Werk über \emph{Wikidata}-QIDs mit dem \emph{\href{http://gams.uni-graz.at/o:konde.167}{Semantic Web}} zu verknüpfen.\\
            
        \subsection*{Literatur:}\begin{itemize}\item Farda-Sarbas, Mariam; Müller-Birn, Claudia: Wikidata from a Research Perspective - A Systematic
                              Mapping Study of Wikidata. In: ArXiv abs/1908.11153: 2019.\item Lemus-Rojas, Mairelys; Pintscher, Lydia: Wikidata and Libraries: Facilitating Open
                              Knowledge. In: Leveraging Wikipedia: Connecting Communities of
                              Knowledge. Chicago, IL: 2018, S. 143–158.\item . In: Wikidata: A platform for data integration and
                              dissemination for the life sciences and beyond Wikidata: 2015.\item Waagmeester, Andra; Lynn, Schriml; Su, Andrew: Wikidata as a linked-data hub for Biodiversity
                              data. In: Biodiversity Information Science and Standards 3: 2019.\item Wikidata. URL: \url{http://www.wikidata.org}\end{itemize}\subsection*{Software:}\href{https://www.wikidata.org/wiki/Wikidata:Main_Page}{Wikidata}, \href{https://www.blazegraph.com/}{BlazeGraph}, \href{http://en.wikisource.org/wiki/Main_Page}{Wikisource}\subsection*{Verweise:}\href{https://gams.uni-graz.at/o:konde.8}{Linked Open Data}, \href{https://gams.uni-graz.at/o:konde.131}{RDF}, \href{https://gams.uni-graz.at/o:konde.147}{Normdaten}, \href{https://gams.uni-graz.at/o:konde.167}{Semantic Web}, \href{https://gams.uni-graz.at/o:konde.109}{GND}, \href{https://gams.uni-graz.at/o:konde.107}{GeoNames}, \href{https://gams.uni-graz.at/o:konde.111}{VIAF}, \href{https://gams.uni-graz.at/o:konde.108}{Getty}\subsection*{Projekte:}\href{https://gams.uni-graz.at/corema}{CoReMA -
                           Cooking Recipes of the Middle Ages}, \href{https://medea.hypotheses.org}{MEDEA. Modelling
                           semantically Enriched Digital Edition of Accounts}, \href{http://mhdbdb.sbg.ac.at/}{Mittelhochdeutsche
                           Begriffsdatenbank (MHDBDB)}, \href{https://blog.factgrid.de/}{FactGrid - a
                           database for historians}\subsection*{Themen:}Annotation und Modellierung, Archivierung, Digitale Editionswissenschaft\subsection*{Zitiervorschlag:}Steiner, Christian. 2021. Wikidata. In: KONDE Weißbuch. Hrsg. v. Helmut W. Klug unter Mitarbeit von Selina Galka und Elisabeth Steiner im HRSM Projekt "Kompetenznetzwerk Digitale Edition". URL: https://gams.uni-graz.at/o:konde.112\newpage\section*{XML} \emph{Raunig, Elisabeth; elisabeth.raunig@uni-graz.at }\\
        
    Die \emph{eXtensible Markup Language}, kurz XML, ist ein
                  einfaches, hierarchisch aufgebautes Textformat, dass 1998 vom World Wide
                  Web-Konsortium entwickelt wurde. Es wurde vor allem zum Aufzeigen von Strukturen
                  in Texten oder Datenbeständen, zum Austausch und zur Speicherung von Daten im Web
                  entwickelt. Es ist sowohl menschenlesbar als auch maschinenlesbar, da für die
                  Beschreibung die gleichen Zeichen verwendet werden wie die Zeichen des
                  Beschriebenen.\\
            
        XML fügt in einen Text oder Datenbestand Auszeichnungen ein, indem beispielsweise
                  ein Wort oder eine Wortkombination getagged wird. Beim \href{http://gams.uni-graz.at/o:konde.17}{Tagging} wird das Wort von einem Auszeichnungselement
                  oder ‘Tag’ eingeschlossen und damit beschrieben. Dieses \href{http://gams.uni-graz.at/o:konde.126}{Markup} steht immer zwischen den Spitzklammern <
                  und >, der Fließtext oder die Daten stehen außerhalb der Klammern. In folgendem
                  Beispiel ist deutlich erkennbar, dass das Markup den Text innerhalb der Klammern
                  genauer beschreibt und ihm eine Bedeutung gibt:
                     <name>Franz</name>. \\
            
        Die Kombination aus Anfangstag, Endtag und Text bildet ein Element. Jedes Element
                  muss mit dem Anfangstag (<name>) geöffnet und mit dem Endtag
                     (</name>) geschlossen werden, letzteres wird immer durch
                  den, dem Endtag vorangestellten, Schrägstrich gekennzeichnet. Jedes Element kann
                  weiters über Attribute und Attributwerte verfügen. Bei obigem Beispiel könnte das
                  sein: <name typ=“vorname“>Franz</name>.
                  Dabei ist ‘typ’ das Attribut, das mit seinem Wert ‘vorname’ angibt, dass es sich
                  bei dem getaggten Namen nicht um einen Nachnamen, sondern um einen Vornamen
                  handelt. \\
            
        XML-Notation unterscheidet auch zwischen Groß- und Kleinschreibung:
                     <name> hat als Tag eine andere Bedeutung als
                     <Name>. Weitere Regeln, denen XML unterliegt, sind: leere
                  Tags können von der Schreibung <name></name> zu der
                  gleichwertigen Schreibung <name/> verkürzt werden; Tags können
                  innerhalb der spitzen Klammern Buchstaben, Zahlen und Striche enthalten,
                  Leerzeichen jedoch sind nur für die Trennung von einem oder mehreren Attributen
                  erlaubt.\\
            
        XML folgt einer hierarchischen Struktur. Das bedeutet, dass ein Element andere
                  Elemente beinhalten kann, sich Elemente jedoch nicht überlappen können. \\
            
        Richtig ist die Auszeichnungsvariante:
                     <name><vorname>Franz</vorname></name>,
                  falsch wäre:
                     <name><vorname>Franz</name></vorname>.
                  Zusätzlich müssen alle Elemente in einem Wurzelelement enthalten sein, um eine
                  Baumstruktur zu bilden. Dieses Wurzelelement bildet ein Elternelement für alle
                  darin enthaltenen Elemente, die Kindelemente oder \emph{child-elements} genannt werden. Unter diesen Kindelementen kann es
                  Geschwisterelemente geben, damit sind Elemente gemeint, die auf derselben Ebene
                  liegen und nicht ineinander verschachtelt sind. Jedes Element unter dem
                  Wurzelelement kann weitere Kindelemente einschließen. Wenn alle diese Regeln
                  eingehalten werden, wird von einem wohlgeformten Dokument gesprochen.\\
            
        Damit bietet XML unendlich viele Möglichkeiten, um Daten hierarchisch zu
                  modellieren und zu annotieren. Daher gibt es unterschiedliche XML-Dialekte, um
                  Standards für den Datenaustausch zu schaffen, wie z. B. XHTML, MathML, \href{http://gams.uni-graz.at/o:konde.125}{SVG} oder \href{http://gams.uni-graz.at/o:konde.178}{TEI}.\\
            
        \subsection*{Literatur:}\begin{itemize}\item XML. URL: \url{https://www.w3.org/XML/}\item Vogeler, Georg; Sahle, Patrick: XML. In: Digital Humanities. Eine Einführung. Stuttgart: 2017, S. 128–148.\end{itemize}\subsection*{Verweise:}\href{https://gams.uni-graz.at/o:konde.126}{Markup}, \href{https://gams.uni-graz.at/o:konde.178}{TEI}, \href{https://gams.uni-graz.at/o:konde.137}{Modellierung}, \href{https://gams.uni-graz.at/o:konde.195}{Textmodellierung}, \href{https://gams.uni-graz.at/o:konde.17}{Annotation}, \href{https://gams.uni-graz.at/o:konde.29}{Annotationsstandards}, \href{https://gams.uni-graz.at/o:konde.79}{Einführung XML/TEI}, \href{https://gams.uni-graz.at/o:konde.120}{Office-Formate}\subsection*{Themen:}Einführung, Archivierung, Digitale Editionswissenschaft\subsection*{Lexika}\begin{itemize}\item \href{https://edlex.de/index.php?title=Extensible_Markup_Language_(XML)}{Edlex: Editionslexikon}\item \href{https://lexiconse.uantwerpen.be/index.php/lexicon/xml/}{Lexicon of Scholarly Editing}\end{itemize}\subsection*{Zitiervorschlag:}Raunig, Elisabeth. 2021. XML. In: KONDE Weißbuch. Hrsg. v. Helmut W. Klug unter Mitarbeit von Selina Galka und Elisabeth Steiner im HRSM Projekt "Kompetenznetzwerk Digitale Edition". URL: https://gams.uni-graz.at/o:konde.215\newpage\section*{XSLT} \emph{Galka, Selina; selina.galka@uni-graz.at }\\
        
    Bei XSLT (Teil der \emph{Extensible Stylesheet Language}, seit
                  1999 ein W3C-Standard) handelt sich um eine \emph{turing}-vollständige Sprache zur Transformation von \href{http://gams.uni-graz.at/o:konde.215}{XML}-Dokumenten. XSLT-Programme, sogenannte
                  XSLT-Stylesheets, werden von XSLT-Prozessoren eingelesen, welche die in den
                  Stylesheets enthaltenen Anweisungen anwenden. (Tidwell 2008, S.
                     27ff.)\\
            
        XSLT-Dokumente basieren, genauso wie XML-Dokumente, auf Baumstrukturen. XSLT
                  benutzt die Abfragesprache \emph{XPath}, um durch XML-Dokumente zu
                  navigieren und Teile davon zu adressieren und zu manipulieren. XSLT eröffnet nun
                  die Möglichkeit, XML-Eingabedokumente in eine andere Struktur und ein anderes
                  Ausgabeformat zu bringen (XML, HTML, PDF, Text etc.).\\
            
        Ein XSLT-Dokument besteht aus Template-Regeln, die bei einem @match-Attribut gewisse Anweisungen ausführen. Ein Beispiel: \\
            
        \begin{verbatim}<xsl:template match=”div”>
    <p>
        <xsl:apply-templates></xsl:apply-templates>
    <p>
</xsl:template>\end{verbatim}Dieses Template adressiert zunächst ein <div>-Element eines
                  XML-Eingabedokuments. Im Ausgabedokument wird ein <p>-Element
                  erzeugt und der Inhalt des <div>-Elements weiterverarbeitet.
                  <xsl:apply-templates> weist an, dass weitere Templates ausgeführt werden
                  sollen.\\
            
        XSLT kann also dem Verwalten von Dokumenten (Aufräumen und Anreichern von Daten,
                  Zusammenführen von Dokumenten, Aufsplitten von Dokumenten), der Aufbereitung von
                  Daten für den Datenaustausch und zur Aufbereitung der Daten für eine etwaige
                  Darstellung und Publikation dienen. \\
            
        \subsection*{Literatur:}\begin{itemize}\item Clark, James; others: Xsl transformations (xslt). In: World Wide Web Consortium (W3C). URL http://www. w3.
                              org/TR/xslt: 1999.\item Bia, Alejandro; Sanchez-Quero, Manuel; Deau, Regis: Multilingual markup of digital library texts using XML,
                              TEI and XSLT. In: Europe XML conference: 2003.\item Mangano, Sal: XSLT Kochbuch. Lösungen für XML- und
                              XSLT-Entwickler. Köln: 2006.\item Kay, Michael: XSLT programmer's reference. Birmingham: 2002.\item W3C Recommendation: XSL Transformations (XSLT) Version
                              3.0. URL: \url{https://www.w3.org/TR/2017/REC-xslt-30-20170608/}\item Tennison, Jeni: Beginning XSLT 2.0: from novice to professional Beginning XSLT 2.0. Berkeley, CA : New York: 2005, URL: \url{http://it-ebooks.info/book/1789/}.\item Tidwell, Doug: XSLT: XML-Dokumente transformieren. Köln: 2002.\item Wadler, Philip: A formal semantics of patterns in XSLT. In: Markup technologies 99: 1999.\item XSL Transformation. URL: \url{https://de.wikipedia.org/wiki/XSL_Transformation}\end{itemize}\subsection*{Software:}\href{http://oxygenxml.com/}{Oxygen}, \href{https://github.com/acdh-oeaw/xsl-tokenizer}{xsl-tokenizer}, \href{https://github.com/KONDE-AT}{Wippets -
                           XSLT/HTML/JS-Fragmente für GAMS}\subsection*{Verweise:}\href{https://gams.uni-graz.at/o:konde.215}{XML}\subsection*{Themen:}Einführung, Datenanalyse, Software und Softwareentwicklung\subsection*{Lexika}\begin{itemize}\item \href{https://edlex.de/index.php?title=Extensible_Stylesheet_Language_(XSL)}{Edlex: Editionslexikon}\end{itemize}\subsection*{Zitiervorschlag:}Galka, Selina. 2021. XSLT. In: KONDE Weißbuch. Hrsg. v. Helmut W. Klug unter Mitarbeit von Selina Galka und Elisabeth Steiner im HRSM Projekt "Kompetenznetzwerk Digitale Edition". URL: https://gams.uni-graz.at/o:konde.86\newpage\section*{Zentrum für Informationsmodellierung / Universität Graz} \emph{Stigler, Johannes; johannes.stigler@uni-graz.at }\\
        
     Das Zentrum für Informationsmodellierung in den Geisteswissenschaften (ZIM) wurde
                  2008 mit der Intention eingerichtet, methodologische Kompetenzen im IT-Bereich zu
                  bündeln und eine nachhaltige Forschungsinfrastruktur aufzubauen, die
                  IT-unterstützte, geisteswissenschaftliche Forschung ermöglicht. Die Universität
                  Graz war damit eine Vorreiterin in der Etablierung der Digital Humanities in
                  Österreich. 2019 schließlich fand die Entwicklung des ZIM durch die Umwandlung des
                  ursprünglichen Forschungszentrums in ein Institut und damit der
                  Institutionalisierung des Fachbereiches ‘Digital Humanites’ an der
                  Geisteswissenschaftlichen Fakultät der Universität Graz seinen vorläufigen
                  Höhepunkt. \\
            
        \emph{\href{http://gams.uni-graz.at/o:konde.59}{Digitale Edition}} ist derzeit einer der zentralen Forschungbereiche des Institutes (neben
                  digitaler Museologie und digitalen Archiven). Die Forschungstätigkeit schlägt sich
                  in einer Vielzahl von nationalen und internationalen Forschungskooperationen
                  nieder. Auch die aktive Beteiligung an den facheinschlägigen Prozessen der
                  \emph{European Research Infrastructure Consortia} (ERIC) (a) \emph{Digital Research
                  Infrastructure for the Arts and Humanities} (DARIAH) und (b) \emph{Common Language
                  Resources and Technology Infrastructure} (CLARIN) verweist auf die Bedeutung des
                  Institutes in der facheinschlägigen Community. Daneben ist das Institut auch an
                  der Universität Graz Kooperationspartner in geisteswissenschaftlichen
                  Forschungsvorhaben aller Disziplinen, in denen basierend auf Ergebnissen
                  angewandter Forschung moderne IT-Strukturen entwickelt und betreut werden. \\
            
         Das Institut betreibt seit 2003 eine Infrastruktur zur Verwaltung digitaler
                  Inhalte: Das \href{http://gams.uni-graz.at/o:konde.11}{OAIS}-kompatible
                  \emph{Geisteswissenschaftliche Asset-Management-System} (\href{http://gams.uni-graz.at/o:konde.70}{GAMS}) wurde und wird im Rahmen einer Vielzahl von
                  Kooperationsprojekten mit inner- und außeruniversitären Partnerinnen und Partnern
                  und in Auseinandersetzung mit konkreten Erfordernissen geisteswissenschaftlicher
                  Forschung genutzt und laufend weiterentwickelt. Basierend auf standardisierten
                  Datenmodellen und Annotationssprachen unterstützt GAMS bei der semantischen
                  Erschließung und nachhaltigen \href{http://gams.uni-graz.at/o:konde.6}{Langzeitarchivierung} wissenschaftlicher Inhalte. Eine Vielzahl von
                  Digitalen Editionen verwendet GAMS. \\
            
        Seit dem Studienjahr 2017/18 können Studierende an der Universität Graz als erstem
                  und derzeit einzigem Standort in Österreich ein Masterstudium ‘Digitale
                  Geisteswissenschaften’ absolvieren, in dem die Möglichkeit besteht, ein Profil im
                  Bereich der IT-gestützten Erschließung und Verarbeitung geisteswissenschaftlicher
                  Forschungsinhalte zu entwickeln, das auf dem Arbeitsmarkt im Bereich der Bildungs-
                  und Kulturerbeinstitutionen zur Zeit begehrt und nachgefragt ist. \\
            
        \textbf{Links:} \\
            
        https://informationsmodellierung.uni-graz.at/\\
            
        https://dariah.eu/\\
            
        https://clarin.eu/\\
            
        https://gams.uni-graz.at/\\
            
        http://www.oais.info/\\
            
        \subsection*{Software:}\href{http://gams.uni-graz.at/archive/objects/o:gams.doku/methods/sdef:TEI/get?locale=de}{GAMS}, \href{https://github.com/KONDE-AT}{Wippets -
                           XSLT/HTML/JS-Fragmente für GAMS}\subsection*{Projekte:}\href{https://informationsmodellierung.uni-graz.at/}{ZIM-ACDH}, \href{https://www.dariah.eu}{DARIAH}, \href{https://www.clarin.eu}{CLARIN}, \href{http://www.oais.info/}{OAIS}\subsection*{Verweise:}\href{https://gams.uni-graz.at/o:konde.70}{GAMS}, \href{https://gams.uni-graz.at/o:konde.11}{OAIS}, \href{https://gams.uni-graz.at/o:konde.6}{Digitale Nachhaltigkeit}, \href{https://gams.uni-graz.at/o:konde.59}{Digitale Edition}\subsection*{Themen:}Institutionen\subsection*{Zitiervorschlag:}Stigler, Johannes. 2021. Zentrum für Informationsmodellierung / Universität Graz. In: KONDE Weißbuch. Hrsg. v. Helmut W. Klug unter Mitarbeit von Selina Galka und Elisabeth Steiner im HRSM Projekt "Kompetenznetzwerk Digitale Edition". URL: https://gams.uni-graz.at/o:konde.217\newpage\section*{Zielgruppen digitaler Editionen} \emph{Klug,Helmut W.; helmut.klug@uni-graz.at }\\
        
    Eine bestimmte Zielgruppe für \href{http://gams.uni-graz.at/o:konde.59}{Digitale Editionen} zu definieren, gestaltet sich als schwierig. Einerseits sprechen bereits die einzelnen \href{http://gams.uni-graz.at/o:konde.76}{Editionstypen} unterschiedliche, eingeschränkte Rezipientengruppen an, andererseits öffnet die Publikationsplattform Internet eine Edition für ein breitest mögliches Zielpublikum. Als prototypische Nutzergruppen Digitaler Editionen können aber eigentlich nur (fach)wissenschaftliche Peers, Studierende und bis zu einem sehr eingeschränkten Grad auch eine thematisch interessierte Öffentlichkeit ausgemacht werden. Ist der Anspruch, mit  einer Digitalen Edition eine interessierte Öffentlichkeit erreichen zu wollen, schon grenzwertig, kann eine breite Öffentlichkeit nur mit speziell für die Wissenschaftskommunikation aufbereiteten Inhalten erreicht werden.\\
            
        Wie bisher müssen Editorinnen und Editoren daher von vorneherein richtungsweisende Entscheidungen treffen und bereits vorab eine Zielgruppe für ihr jeweiliges Editionsprojekt definieren. Für eine glaubwürdige Umsetzung sollten diese Entscheidungen dann in allen Bereichen der digitalen Edition konsequent verfolgt werden. Einen maßgeblichen Anteil hat dabei natürlich auch die Präsentation der Editionsdaten nach außen.\\
            
        \subsection*{Literatur:}\begin{itemize}\item Beavan, Ian; Arnott, Michael; McLaren, Colin: Text and Illustration: The Digitisation of a Medieval Manuscript. In: Computers and the Humanities 31: 1997, S. 61–71.\item Henzel, Katrin: Digitale genetische Editionen aus der Nutzerperspektive. In: Textgenese in der digitalen Edition. Berlin/Munich/Boston: 2019, S. 66–80.\item Hofmeister, Wernfried; Stigler, Johannes: Die Edition als Interface. Möglichkeiten der Semantisierung und Kontextualisierung von domänenspezifischem Fachwissen in einem Digitalen Archiv am Beispiel der XML-basierten Augenfassung zur Hugo von Montfort-Edition. In: editio 24: 2010, S. 80–95.\item Runow, Holger: Wem nützt was? Mediävistische Editionen (auch) vom Nutzer aus gedacht. In: editio. Internationales Jahrbuch für Editionswissenschaft 28: 2014, S. 50–57.\item Sahle, Patrick: Digitale Editionsformen. Zum Umgang mit der Überlieferung unter den Bedingungen des Medienwandels. Teil 2: Befunde, Theorie und Methodik. Norderstedt: 2013.\item Vogeler, Georg; Sahle, Patrick: Urkundenforschung und Urkundenedition im digitalen Zeitalter. In: Geschichte und Neue Medien in Forschung, Archiven, Bibliotheken und Museen Tagungsband .hist 2003: 2005, S. 333–378.\item Steding, Sören Alexander: Benutzerorientierte Digitale Editionen. Eine empirische Annäherung. In: www.germanistik2001.de. Vorträge des Erlanger Germanistentags 2001. Hg. von H. Kugler. Band 2: 2003, S. 729–742.\end{itemize}\subsection*{Verweise:}\href{https://gams.uni-graz.at/o:konde.76}{Editionstypen}, \href{https://gams.uni-graz.at/o:konde.148}{BenutzerInnen Digitaler Editionen}\subsection*{Themen:}Digitale Editionswissenschaft\subsection*{Zitiervorschlag:}Klug, W. 2021. Zielgruppen digitaler Editionen. In: KONDE Weißbuch. Hrsg. v. Helmut W. Klug unter Mitarbeit von Selina Galka und Elisabeth Steiner im HRSM Projekt "Kompetenznetzwerk Digitale Edition". URL: https://gams.uni-graz.at/o:konde.218\newpage\section*{Zitierbarkeit digitaler Ressourcen} \emph{Bleier, Roman; roman.bleier@uni-graz.at }\\
        
    Zur Referenzierung im Internet wird häufig die URL der jeweiligen Webseite, die zitiert werden soll, verwendet. URLs sind jedoch primär Adressen im Internet und nicht fest an die zitierten Inhalte gebunden, die außerdem leicht geändert, gelöscht oder verschoben werden können. Besonders für wissenschaftliche Publikationen im Internet und Ressourcen in \href{http://gams.uni-graz.at/o:konde.59}{Digitalen Editionen} ist daher persistente Zitierbarkeit ein wichtiges Thema.\\
            
        Für die Referenzierung von digitalen Ressourcen sollten stabile, global einzigartige Namen, \emph{\href{http://gams.uni-graz.at/o:konde.12}{Persistent Identifiers}} (PID), oder persistente Adressen, Permalinks, angeboten und bei der Zitation verwendet werden. Wichtig ist dabei auch, dass im Falle von Änderungen der Daten jede Version einer digitalen Ressource auffindbar bleibt und stabil zitiert werden kann. \\
            
        \subsection*{Literatur:}\begin{itemize}\item Arnold, Eckhart; Müller, Stefan: Wie permanent sind Permalinks? In: Informationspraxis 3: 2017.\item Persistent Identifier: eindeutige Bezeichner für digitale Inhalte. URL: \url{http://www.persistent-identifier.de/}\item Klump, Jens; Huber, Robert: 20 Years of Persistent Identifiers – Which Systems are Here to Stay? In: Data Science Journal 16: 2017, S. 1-7.\item Schroeder, Kathrin: 9.4. Persistent Identifier (PI) - ein Überblick. In: nestor Handbuch. Eine keine Enzyklopädie der digitalen Langzeitarchivierung. Version 2.3. Glückstadt: 2009.\item Sompel, Herbert Van de; Sanderson, Robert; Shankar, Harihar; Klein, Martin: Persistent Identifiers for Scholarly Assets and the Web: The Need for an Unambiguous Mapping Persistent Identifiers for Scholarly Assets and the Web. In: International Journal of Digital Curation 9: 2014, S. 331–342.\end{itemize}\subsection*{Verweise:}\href{https://gams.uni-graz.at/o:konde.6}{Digitale Nachhaltigkeit}, \href{https://gams.uni-graz.at/o:konde.8}{Linked Open Data}, \href{https://gams.uni-graz.at/o:konde.12}{Allgemein: Persistent Identifier}, \href{https://gams.uni-graz.at/o:konde.220}{Zitiervorschlag}, \href{https://gams.uni-graz.at/o:konde.14}{Versionierung}\subsection*{Themen:}Einführung, Archivierung, Digitale Editionswissenschaft\subsection*{Zitiervorschlag:}Bleier, Roman. 2021. Zitierbarkeit digitaler Ressourcen. In: KONDE Weißbuch. Hrsg. v. Helmut W. Klug unter Mitarbeit von Selina Galka und Elisabeth Steiner im HRSM Projekt "Kompetenznetzwerk Digitale Edition". URL: https://gams.uni-graz.at/o:konde.219\newpage\section*{Zitiervorschlag} \emph{Bleier, Roman; roman.bleier@uni-graz.at }\\
        
    Onlineressourcen werden zunehmend wichtiger für den wissenschaftlichen Diskurs und
                  manche Publikationen sind nur mehr in digitaler Form online veröffentlicht. Wie
                  bei allen wissenschaftlichen Publikationen ist auch bei Onlineressourcen die
                  korrekte Zitation wichtig. Ein Probleme dabei ist, dass Onlinepublikationen von
                  der Form und den Inhalten her gegenüber Druckpublikationen unterschiedlich sind
                  und daher traditionelle Zitiermethoden nicht immer sinnvoll angewendet werden
                  können. \\
            
        \href{http://gams.uni-graz.at/o:konde.59}{Digitale Editionen} sind darüber
                  hinaus ein Sonderfall, da sie oft unterschiedliche Objekte kombinieren (z. B.
                  Bild, diplomatischer und normalisierter Text, Audio, Daten), welche individuell
                  zitierbar sein sollten. Granularität ist ein weiteres Problem, da z. B. ein
                  Onlinetext nicht in Seiten strukturiert sein muss und daher andere
                  Struktureinheiten (z. B. Absätze) für die Zitation verwendet werden können. Da es
                  für alle diese Fälle keine allgemeinen Regeln und Richtlinien gibt, empfiehlt es
                  sich, dass Anbieterinnen und Anbieter von Onlineressourcen sicherstellen, dass
                  diese zitiert werden können und sollen. Das geschieht üblicherweise mit
                  sogenannten Zitiervorschlägen, die entweder direkt beim Objekt, das zitiert werden
                  soll, oder an einer gut sichtbaren Stelle der Webseite platziert sind. \\
            
        Eine Zitierhilfe (bzw. ein Zitiervorschlag) sollte die wichtigsten
                  bibliographischen Informationen über das zu zitierende Objekt enthalten. Ganz
                  wichtig ist auch ein Permalink oder ein \emph{\href{http://gams.uni-graz.at/o:konde.12}{Persistent Identifier}} (PID), der verwendet werden kann, um das Objekt auch zukünftig wieder
                  aufzurufen. Falls es \href{http://gams.uni-graz.at/o:konde.14}{mehrere Versionen
                     eines digitalen Objektes} gibt, ist es ganz wichtig, dass diese
                  unterschieden werden können und individuell zitierbar sind. Es sollte auch klar
                  ersichtlich sein, welche Untereinheiten persistent sind und für granulare Zitation
                  zur Verfügung stehen. Oft wird auch als Zitiervorschlag ein Beispiel angeboten,
                  welches ein Titelzitat in einem weit verbreiteten Zitierstil (Chicago, MLA etc.)
                  darstellt. Für Beispiel und Überlegungen zur “best practice” siehe Bleier,
                     2021.\\
            
        \subsection*{Literatur:}\begin{itemize}\item Arnold, Eckhart; Müller, Stefan: Wie permanent sind Permalinks? In: Informationspraxis 3: 2017.\item Bleier, Roman: How to cite this edition? In: Digital Humanities Quarterly: 2021
                              .\end{itemize}\subsection*{Verweise:}\href{https://gams.uni-graz.at/o:konde.6}{Digitale Nachhaltigkeit}, \href{https://gams.uni-graz.at/o:konde.12}{Allgemein: Persistent
                           Identifier}, \href{https://gams.uni-graz.at/o:konde.14}{Versionierung}, \href{https://gams.uni-graz.at/o:konde.219}{Zitierbarkeit digitaler
                           Ressourcen}\subsection*{Themen:}Interfaces, Digitale Editionswissenschaft\subsection*{Zitiervorschlag:}Bleier, Roman. 2021. Zitiervorschlag. In: KONDE Weißbuch. Hrsg. v. Helmut W. Klug unter Mitarbeit von Selina Galka und Elisabeth Steiner im HRSM Projekt "Kompetenznetzwerk Digitale Edition". URL: https://gams.uni-graz.at/o:konde.220\newpage\section*{acdh-spacytei} \emph{Andorfer, Peter; peter.andorfer@oeaw.ac.at / Schlögl, Matthias;
                  matthias.schloegl@oeaw.ac.at}\\
        
    \emph{acdh-spacytei} ist ein \emph{Python}-Package, das eine Reihe von Hilfsklassen und -funktionen zur
                  Prozessierung von \href{http://gams.uni-graz.at/o:konde.215}{XML}/\href{http://gams.uni-graz.at/o:konde.178}{TEI}-kodierten Texten mit \emph{\href{http://gams.uni-graz.at/o:konde.170}{spaCy}} zur Verfügung stellt. Wie die meisten \href{http://gams.uni-graz.at/o:konde.145}{NLP}-Tools ist auch \emph{spaCy}
                  dahingehend konzipiert, \emph{plain text}, also nicht annotierte
                  Volltexte, zu verarbeiten. Um also beispielsweise \emph{\href{http://gams.uni-graz.at/o:konde.141}{Named-Entity-Recognition}} (NER) in einer bereits in XML/TEI-kodierten Edition durchführen zu können,
                  müssten die entsprechenden Textknoten extrahiert, in \emph{plain text
                  } konvertiert, mit dem jeweiligen Tool prozessiert und die generierten \href{http://gams.uni-graz.at/o:konde.17}{Annotationen} separat vom
                  XML/TEI-Dokument gespeichert werden, was zu einer gewissen Datenduplizierung
                  führen würde. \\
            
        \emph{acdh-spacytei} hingegen prozessiert XML/TEI-kodierte Texte
                  annotationsbewahrend. Dafür werden die XML/TEI-kodierten Dokumente tokenisiert,
                  die Tokens in <w>-Tags geschrieben und mit
                  IDs versehen. Die Tokenisierung wird dabei jedoch nicht von \emph{acdh-spacytei} direkt vorgenommen, \emph{acdh-spacytei}
                  ruft dafür vielmehr den Webservice \emph{\href{http://gams.uni-graz.at/o:konde.216}{xTokenizer}} auf. Aus der Sequenz der Tokens wird daraufhin ein \emph{spaCy}-Doc-Element erstellt und jedem Token-Element die <w>-Tag-IDs hinzugefügt. Dieses Doc-Element wird
                  dann mit \emph{spaCy} prozessiert. Die von \emph{spaCy} vorgenommenen Anreicherungen (z. B. \href{http://gams.uni-graz.at/o:konde.115}{Lemmatisierung}, \href{http://gams.uni-graz.at/o:konde.156}{POS-Tag}, \href{http://gams.uni-graz.at/o:konde.141}{NER}) werden als TEI-Attribute in die jeweiligen <w>-Tags zurückgeschrieben bzw. \emph{named
                     entities} von typisierten <rs>-Tags
                  umfasst. Dafür nutzt \emph{acdh-spacytei} die in \emph{custom attributes} der \emph{spaCy}-Token-Klasse
                  abgelegten <w>-Tag-IDs.\\
            
        Abgesehen von dem oben beschriebenen Interface zur annotationsbewahrenden
                  Anreicherung stellt \emph{acdh-spacytei} auch Methoden zur
                  Verfügung, um aus bereits semantisch annotierten XML/TEI-Dokumenten Daten für das
                  Training von NER-Modellen generieren zu können. \\
            
        \subsection*{Software:}\href{https://spacy.io/}{spacy }, \href{https://www.nltk.org/}{Natural Language Toolkit
                           (nltk)}, \href{https://github.com/acdh-oeaw/acdh-spacytei}{acdh-spacytei}\subsection*{Verweise:}\href{https://gams.uni-graz.at/o:konde.145}{NLP}, \href{https://gams.uni-graz.at/o:konde.141}{Named Entity Recognition /
                           NER}, \href{https://gams.uni-graz.at/o:konde.156}{Part-of-Speech-Tagging}, \href{https://gams.uni-graz.at/o:konde.170}{spacy}, \href{https://gams.uni-graz.at/o:konde.216}{xTokenizer}, \href{https://gams.uni-graz.at/o:konde.1}{ACDH-CH}\subsection*{Themen:}Natural Language Processing, Annotation und Modellierung, Software und Softwareentwicklung\subsection*{Zitiervorschlag:}Andorfer, Peter; Schlögl, Matthias. 2021. acdh-spacytei. In: KONDE Weißbuch. Hrsg. v. Helmut W. Klug unter Mitarbeit von Selina Galka und Elisabeth Steiner im HRSM Projekt "Kompetenznetzwerk Digitale Edition". URL: https://gams.uni-graz.at/o:konde.2\newpage\section*{critique génétique} \emph{Lenhart, Elmar; elmar.lenhart@aau.at / Bosse, Anke; anke.bosse@aau.at }\\
        
    Die \emph{critique génétique} ist eine literaturwissenschaftliche Methode, die seit den 1970er-Jahren die Gesamtheit der materialen Spuren untersucht, die sich einem literarischen Werk zuordnen lassen.\\
            
        1. Ursprung\\
            
        Ein Artikel Louis Hays in \emph{Le Monde} vom Februar 1967 kann heute als die Geburtsstunde der \emph{critique génétique} gelten. (Bosse 2005) Ihre Entwicklung steht in engem Zusammenhang mit dem Erwerb des Heine-Nachlasses 1968 durch die \emph{Bibliothèque Nationale de France}. Eine deutsch-französische Forschergruppe, die sich mit dem Bestand auseinandersetzte, gründete in der Folge das \emph{Institut des textes et manuscrits modernes }(ITEM). Anhand moderner, d. h. aus dem 18. bis 20. Jahrhundert stammender Schriftstellermanuskripte und -typoskripte entwickelte das ITEM den methodischen Ansatz, den die \emph{critique génétique} bis heute weiterentwickelt. War man zu Beginn dem Strukturalismus verpflichtet (Grésillon 1999, S. 15), emanzipierte man sich davon zunehmend, indem man vor allem dessen statischen, von Material und Medium abstrahierenden Textbegriff ablehnte.\\
            
        2. Perspektive\\
            
        Die \emph{critique génétique} rechnet dem literarischen Text alle materialen Spuren zu, die mit dessen Entstehung im Zusammenhang stehen. Im Fokus steht nicht der literarische Text als das gedruckte Produkt eines Schreibprozesses, sondern dieser selbst. Die Aufgabe der \emph{critique génétique} besteht in der Klassifizierung und Interpretation aller daran beteiligten Schreibzeugnisse. Damit verbunden ist eine „Entheiligung“ und „Entmythisierung“ (Grésillon 1999, S. 17) des definitiven Textes und eine Aufwertung der Rolle des Autors und seiner unmittelbaren schriftlichen Äußerungen. Zwei Metaphern beschreiben die Beziehung der \emph{critique génétique} zu den von ihr untersuchten Schreibzeugnissen: die der Geburt und die der Konstruktion. (vgl. Grésillon 2016, S. 15–21) Beide Vergleiche illustrieren die Orientierung der \emph{critique génétique} am Entstehungsprozess des Kunstwerks.\\
            
        3. Arbeitsweise, Arbeitstechnik und Ziele\\
            
        Im Zentrum des methodischen Ansatzes der \emph{critique génétique} steht die Erstellung eines sogenannten \emph{dossier génétique}, welches alle Schreibzeugnisse, die für die jeweilige Forschungsfrage Relevanz haben, aufnimmt, chronologisch nach ihrer Entstehungszeit ordnet und damit den „\emph{généticiens}“ (Grésillon 2016, S. 21) die Möglichkeit eröffnet, den Prozess des Schreibens kritisch-interpretierend nachzuzeichnen. Sowohl in der Zusammenstellung des Dossiers wie in dessen Lektüre liegt das kritische Potenzial der \emph{critique génétique}. Die Gestalt des \emph{dossier génétique }hängt wesentlich von der Reichhaltigkeit der Überlieferung ab.\\
            
        Die produktionsästhetische Perspektive, die die Autorin/den Autor und deren/dessen Handeln ins Zentrum des Interesses rückt, führt zur Beschreibung von sogenannten Schreibweisen (\emph{écritures}), die sich aus den Spezifika der Schreibzeugnisse ablesen und sich auf einer Skala zwischen programmorientiert und prozessorientiert kategorisieren lassen. Entscheidend ist für die \emph{critique génétique} die Frage, welche Wege Autorinnen und Autoren im Zuge ihrer Arbeit gegangen sind und auf welche Weise diese Wege darstellbar sind. Als besondere Herausforderung erweist sich die Nutzung moderner, digitaler Schreibmedien, die den Begriff der Schreibspur neu definieren und damit wohl auch in Zukunft die Gestalt des \emph{dossier génétique} entscheidend verändern.\\
            
        4. Edition\\
            
        Mit der \href{http://gams.uni-graz.at/o:konde.90}{textgenetischen Edition}, wie sie insbesondere in den deutschsprachigen Ländern praktiziert wird, verbindet die \emph{critique génétique} die starke Berücksichtigung und die Darstellung des \emph{avant-textes}, jedoch ist jene auf den Text am Ende einer Genese fokussiert und damit weiterhin einem Textbegriff verpflichtet, den die \emph{critique génétique} zugunsten des prozessualen Textbegriffs und der Darstellung von Schreibprozessen grundsätzlich ablehnt.\\
            
        \subsection*{Literatur:}\begin{itemize}\item Bosse, Anke: Rezension zu: Louis Hay: La littérature des écrivains. Questions de critique génétique. Paris: José Corti 2002. In: editio. Internationales Jahrbuch für Editionswissenschaft 19: 2005, S. 207–209.\item Grésillon, Almuth: Literarische Handschriften: Einführung in die "critique génétique" Literarische Handschriften. Bern: 1999.\item Grésillon, Almuth: Éléments de critique génétique: lire les manuscrits modernes Éléments de critique génétique. Paris: 1994.\item Critique génétique. URL: \url{https://edlex.de/index.php?title=Critique_génétique}\item Grésillon, Almuth: Über die allmähliche Verfertigung von Texten beim Schreiben. In: Schreiben als Kulturtechnik: Grundlagentexte: 2015, S. 152–186.\item La Naissance du texte. Hrsg. von  und Louis Hay. Paris: 1989.\end{itemize}\subsection*{Verweise:}\href{https://gams.uni-graz.at/o:konde.197}{Transkription}, \href{https://gams.uni-graz.at/o:konde.90}{genetische Edition}, \href{https://gams.uni-graz.at/o:konde.93}{historisch-kritische Edition}\subsection*{Themen:}Digitale Editionswissenschaft\subsection*{Lexika}\begin{itemize}\item \href{https://edlex.de/index.php?title=Critique_g%C3%A9n%C3%A9tique}{Edlex: Editionslexikon}\item \href{https://lexiconse.uantwerpen.be/index.php/lexicon/genetic-criticism/}{Lexicon of Scholarly Editing}\end{itemize}\subsection*{Zitiervorschlag:}Lenhart, Elmar; Bosse, Anke. 2021. critique génétique. In: KONDE Weißbuch. Hrsg. v. Helmut W. Klug unter Mitarbeit von Selina Galka und Elisabeth Steiner im HRSM Projekt "Kompetenznetzwerk Digitale Edition". URL: https://gams.uni-graz.at/o:konde.46\newpage\section*{mobile first} \emph{Galka, Selina; selina.galka@uni-graz.at }\\
        
    \emph{mobile first} ist ein Konzept für das \href{http://gams.uni-graz.at/o:konde.56}{Design} von Webseiten, wobei man davon ausgeht, dass diese zunächst an allererster Stelle für mobile Endgeräte wie Smartphone oder Tablet konzipiert werden.\\
            
        Grundsätzlich ist es seit ca. 2010 Usus, Webseiten und Tools sowohl für größere Displays (z. B. Desktop) als auch für kleinere (z. B. Smartphone) zu optimieren. Dafür wurde eine dynamische Designmöglichkeit entwickelt: das \emph{\href{http://gams.uni-graz.at/o:konde.164}{Responsive Design}}. Oft wird zuerst ausgehend von der Desktop-Version die Webseite entwickelt und danach eine abgespeckte Version für mobile Endgeräte angeboten. Beim Prinzip \emph{mobile first} geht es aber eben darum, den Fokus auf die mobilen Endgeräte zu lenken. Ob diese Umsetzungsmöglichkeit Sinn macht, hängt von der Nutzergruppe und dem Zweck des Tools/der Webseite ab – denkbar wäre z. B. ein Szenario, in dem Schülerinnen und Schüler das Zielpublikum darstellen.\\
            
        Es wurde beobachtet, dass Nutzerinnen und Nutzer von DH-Tools/Webseiten oft auch unterschiedliche Ansprüche an die verschiedenen Geräte stellen – von einer Version für kleinere Displays werden eher reduzierte Funktionalitäten und mögliche Interaktionen erwartet, während die zugehörige Desktop-Variante alle Funktionalitäten bieten soll. (Thoden 2017, S. 6)\\
            
        \subsection*{Literatur:}\begin{itemize}\item Thoden, Klaus; Stiller, Juliane; Bulatovic, Natasa; Meiners, Hanna-Lena; Boukhelifa, Nadia: User-Centered Design Practices in Digital Humanities – Experiences from DARIAH and CENDARI. In: ABI Technik 37: 2017, S. 2-11.\end{itemize}\subsection*{Verweise:}\href{https://gams.uni-graz.at/o:konde.148}{Benutzergruppen für Digitale Editionen}, \href{https://gams.uni-graz.at/o:konde.164}{Responsive Design}, \href{https://gams.uni-graz.at/o:konde.207}{User-centered design}, \href{https://gams.uni-graz.at/o:konde.205}{Usability}, \href{https://gams.uni-graz.at/o:konde.56}{Design}\subsection*{Themen:}Interfaces\subsection*{Zitiervorschlag:}Galka, Selina. 2021. mobile first. In: KONDE Weißbuch. Hrsg. v. Helmut W. Klug unter Mitarbeit von Selina Galka und Elisabeth Steiner im HRSM Projekt "Kompetenznetzwerk Digitale Edition". URL: https://gams.uni-graz.at/o:konde.134\newpage\section*{nestor-Kompetenznetzwerk} \emph{Stigler, Johannes; johannes.stigler@uni-graz.at }\\
        
    \emph{nestor} vernetzt unterschiedliche Institutionen aus den Bereichen Forschung und Gedächtnisinstitutionen, die sich mit Fragen der \href{http://gams.uni-graz.at/o:konde.6}{Langzeitarchivierung} digitaler (Forschungs-)Daten befassen. Als deutsches Kompetenznetzwerk kooperiert \emph{nestor} mit europäischen Partnern. Standardisierungsaktivitäten werden gebündelt und in die jeweiligen Communities hineingetragen. Gemeinsam mit Hochschulen entwickelt \emph{nestor} Aus-, Fort- und Weiterbildungsangebote im Bereich der digitalen Langzeitarchivierung. Die Schwerpunkte und Aktivitäten werden durch Workshops und andere Veranstaltungen sowie durch Publikationen, wie beispielsweise das \emph{nestor}-Handbuch, die \emph{nestor}-Expertisen und die \emph{nestor}-Ratgeber, vermittelt. \\
            
        \emph{nestor} bietet Arbeitsgruppen zu Teilbereichen wie \href{http://gams.uni-graz.at/o:konde.12}{dauerhafte Zitierbarkeit}, digitale Bestandserhaltung, Dokumentation der digitalen Langzeitarchivierung, Formaterkennung, \href{http://gams.uni-graz.at/o:konde.13}{Zertifizierung digitaler Repositorien} u. v. m.\\
            
        \subsection*{Literatur:}\begin{itemize}\item nestor Handbuch. Eine keine Enzyklopädie der digitalen Langzeitarchivierung. Hrsg. von Heike Neuroth, Achim Oßwald, Regine Scheffel, Stefan Strathmann und Mathias Jehn. Glückstadt: 2016, URL: \url{urn:nbn:de:0008-2010071949}.\end{itemize}\subsection*{Verweise:}\href{https://gams.uni-graz.at/o:konde.6}{Digitale Nachhaltigkeit}, \href{https://gams.uni-graz.at/o:konde.13}{Zertifizierung}, \href{https://gams.uni-graz.at/o:konde.12}{Persistent Identifier}\subsection*{Projekte:}\href{https://www.langzeitarchivierung.de/Webs/nestor/DE/Home/home_node.htm}{Nestor}\subsection*{Themen:}Archivierung\subsection*{Zitiervorschlag:}Stigler, Johannes. 2021. nestor-Kompetenznetzwerk. In: KONDE Weißbuch. Hrsg. v. Helmut W. Klug unter Mitarbeit von Selina Galka und Elisabeth Steiner im HRSM Projekt "Kompetenznetzwerk Digitale Edition". URL: https://gams.uni-graz.at/o:konde.4\newpage\section*{spaCy} \emph{Andorfer, Peter; peter.andorfer@oeaw.ac.at / Schlögl, Matthias; matthias.schloegl@oeaw.ac.at }\\
        
    \emph{SpaCy} ist eine in \emph{Python (Cython)} geschriebene Programmbibliothek für natürliche Sprachverarbeitung (\href{http://gams.uni-graz.at/o:konde.145}{NLP}). Im Gegensatz zu dem ebenfalls in \emph{Python} implementierten NLP-Framework NLTK, das auf Forschung und Lehre fokussiert, will \emph{spaCy} Lösungen für die Industrie bereitstellen. Dafür wird eine saubere und einfach zu verwendende \href{http://gams.uni-graz.at/o:konde.31}{API} bereitgestellt, die sich auf die performante Erledigung von Standard-NLP-Aufgaben konzentriert. \emph{SpaCy} stellt momentan Sprachmodelle für zehn Sprachen in verschiedenen Ausbaustufen zur Verfügung und erlaubt – je nach Modell – \emph{tokenizing}, \emph{sentence splitting}, \emph{tagging}, \emph{parsing}, \emph{\href{http://gams.uni-graz.at/o:konde.141}{named entity recognition}} und \emph{word similarity calculations}. Zudem erlaubt \emph{spaCy} die relativ einfache Erweiterung der Kernfunktionen. So können etwa in der \emph{spaCy}-Pipeline auch externe Komponenten aufgerufen oder in der Tokenklasse \emph{custom attributes} registriert werden (eine Anwendungsmöglichkeit, die z. B. \emph{\href{http://gams.uni-graz.at/o:konde.2}{acdh-spacytei}} nutzt).\\
            
        Ein Großteil der von \emph{spaCy} zur Verfügung gestellten Modelle, Klassen und Funktionen basiert auf \emph{Deep Learning}-Technologien. Neben der \emph{Python}-Bibliothek stellt \emph{spaCy} auch Shell-Skripte zur Verfügung, mit deren Hilfe neue Modelle auf Basis eigener Trainingsdaten erstellt sowie bestehende Modelle weiter trainiert werden können.\\
            
        \subsection*{Software:}\href{https://spacy.io/}{spacy }, \href{https://www.nltk.org/}{Natural Language Toolkit (nltk)}, \href{https://github.com/acdh-oeaw/acdh-spacytei}{acdh-spacytei}\subsection*{Verweise:}\href{https://gams.uni-graz.at/o:konde.145}{NLP}, \href{https://gams.uni-graz.at/o:konde.141}{Named Entity Recognition / NER}, \href{https://gams.uni-graz.at/o:konde.156}{Part-of-Speech-Tagging}, \href{https://gams.uni-graz.at/o:konde.2}{acdh-spacytei}, \href{https://gams.uni-graz.at/o:konde.176}{Tagger}\subsection*{Themen:}Natural Language Processing, Software und Softwareentwicklung\subsection*{Zitiervorschlag:}Andorfer, Peter; Schlögl, Matthias. 2021. spaCy. In: KONDE Weißbuch. Hrsg. v. Helmut W. Klug unter Mitarbeit von Selina Galka und Elisabeth Steiner im HRSM Projekt "Kompetenznetzwerk Digitale Edition". URL: https://gams.uni-graz.at/o:konde.170\newpage\section*{xsl-tokenizer} \emph{Schopper, Daniel; daniel.schopper@oeaw.ac.at }\\
        
    Unter Tokenisierung versteht man die Zerlegung eines Fließtextes in Einzelsegmente
                  (Tokens), in aller Regel in Wörter, aber auch kleinere (Zeichen) oder größere
                  Einheiten (\emph{Multi Word Items}). Tokenisierung stellt den
                  ersten Verarbeitungsschritt zur (semi-)automatischen linguistischen (\href{http://gams.uni-graz.at/o:konde.156}{Part-of-Speech-Tagging} oder \href{http://gams.uni-graz.at/o:konde.115}{Lemmatisierung}) oder semantischen
                     \href{http://gams.uni-graz.at/o:konde.17}{Annotation} (\emph{\href{http://gams.uni-graz.at/o:konde.151}{Named Entity Recognition/NER}}) dar; gleichzeitig ist sie auch Teil des Indizierungsprozesses für die \href{http://gams.uni-graz.at/o:konde.211}{Volltextsuche}.\\
            
        In \href{http://gams.uni-graz.at/o:konde.59}{Digitalen Editionen} häufig
                  notwendige komplexe \href{http://gams.uni-graz.at/o:konde.126}{Markup}-Strukturen, insbesondere voneinander unabhängige Textflüsse in einem
                  Dokument (wie z. B. in den Haupttext eingebettete Fußnoten oder ein textkritischer
                  Variantenapparat), stellen eine Herausforderung für die Tokenisierung dar.\\
            
        Der \emph{xsl-tokenizer} ist eine auf \href{http://gams.uni-graz.at/o:konde.86}{XSLT} 2.0 aufbauende Softwarelösung, die es
                  ermöglicht, \href{http://gams.uni-graz.at/o:konde.215}{XML}-Instanzen
                  regelbasiert zu tokenisieren und dabei bestehende Dokumentstrukturen zu erhalten.
                  Er ist vollständig parametrierbar und kann somit für unterschiedliche
                  XML-Schemata, \href{http://gams.uni-graz.at/o:konde.177}{Tagsets} und
                  Annotationsrichtlinien verwendet werden. Das Ergebnis der Prozessierung wird im
                  Quelldokument mit den \href{http://gams.uni-graz.at/o:konde.178}{TEI}-Elementen <w> bzw. <pc> kodiert. Wo durch die Tokenisierung
                  überlappende XML-Hierarchien (TEI Guidelines, Kapitel 20: Non-hierarchical
                     Structures) entstünden, wird ein Token in mehreren Elemente abgebildet,
                  die mit @part markiert und durch @prev bzw. @next verbunden sind. Weiters besteht die Option, eine verflachte
                  Tokenliste mit vereinfachter Dokumentstruktur auszugeben, die von einem \href{http://gams.uni-graz.at/o:konde.176}{Tagger} angereichert und
                  anschließend wieder in das Quelldokument integriert werden kann.\\
            
        \subsection*{Literatur:}\begin{itemize}\item 16 Linking, Segmentation, and Alignment. URL: \url{https://tei-c.org/release/doc/tei-p5-doc/en/html/SA.html}\end{itemize}\subsection*{Software:}\href{https://github.com/acdh-oeaw/xsl-tokenizer}{xsl-tokenizer}, \href{https://github.com/acdh-oeaw/acdh-spacytei}{acdh-spacytei}, \href{https://nlp.fi.muni.cz/trac/noske}{No Sketch
                           Engine }\subsection*{Verweise:}\href{https://gams.uni-graz.at/o:konde.145}{NLP}, \href{https://gams.uni-graz.at/o:konde.17}{Textannotation}, \href{https://gams.uni-graz.at/o:konde.115}{Lemmatisierung}, \href{https://gams.uni-graz.at/o:konde.141}{Named Entity Recognition /
                           NER}, \href{https://gams.uni-graz.at/o:konde.156}{POS-Tagging}, \href{https://gams.uni-graz.at/o:konde.176}{Tagger}, \href{https://gams.uni-graz.at/o:konde.177}{Tagsets}, \href{https://gams.uni-graz.at/o:konde.86}{XSLT}\subsection*{Projekte:}\href{https://www.corpusthomisticum.org/wintrode.html}{Corpus
                           Thomasticum}, \href{http://mhdbdb.sbg.ac.at/}{Mittelhochdeutsche
                           Begriffsdatenbank (MHDBDB)}\subsection*{Themen:}Natural Language Processing, Software und Softwareentwicklung\subsection*{Zitiervorschlag:}Schopper, Daniel. 2021. xsl-tokenizer. In: KONDE Weißbuch. Hrsg. v. Helmut W. Klug unter Mitarbeit von Selina Galka und Elisabeth Steiner im HRSM Projekt "Kompetenznetzwerk Digitale Edition". URL: https://gams.uni-graz.at/o:konde.216\newpage\section*{Österreichische Nationalbibliothek} \emph{Fritze, Christiane; christiane.fritze@onb.ac.at / Steindl, Christoph;
                  christoph.steindl@onb.ac.at }\\
        
    An der Österreichischen
                     Nationalbibliothek (ÖNB), einer der bedeutendsten Gedächtnisinstitutionen
                  des Landes, wird eine übergreifende, nachhaltige Infrastruktur für \href{http://gams.uni-graz.at/o:konde.59}{Digitale Editionen} aufgebaut. Alle
                  künftigen Digitalen Editionen, die auf Quellenmaterial der Österreichischen
                  Nationalbibliothek basieren, werden darin realisiert. Die Infrastruktur wird als
                  Teil der Digital Humanities-Strategie zur Erfüllung der Vision 2025 der Österreichischen Nationalbibliothek
                  entwickelt.\\
            
        Die Editionsinfrastruktur verbindet Komponenten zur Erstellung von Digitalen
                  Editionen mit der \emph{Repository}-Komponente zur Bereitstellung
                  der online-Präsentation der Edition. Dabei wurde ein modularer Aufbau gewählt,
                  sodass die Basiskonfiguration, die für jede Edition bereitgestellt wird,
                  projektspezifisch adaptiert werden kann.\\
            
        Bei der Erstellungskomponente kommt der \emph{oXygen}-XML-Editor
                  zum Einsatz, in dem eine ausgearbeitete Projektstruktur aufgesetzt wird. Das
                  Quellmaterial wird als \href{http://gams.uni-graz.at/o:konde.178}{TEI-XML}
                  aufbereitet und in einem \emph{GitLab-Repository} für die
                  Bearbeitung des Editionsteams versioniert abgelegt. Den XML-Dateien liegt ein an
                  der ÖNB entwickeltes Basisschema zugrunde, das mit projektspezifischen Elementen
                  erweitert werden kann.\\
            
        Technische Grundlage der \emph{Repository}-Komponente bildet eine
                  eigenständige Instanz des \emph{Geisteswissenschaftlichen Asset
                     Management Systems} (\href{http://gams.uni-graz.at/o:konde.70}{GAMS}),
                  das am \href{http://gams.uni-graz.at/o:konde.217}{Zentrum für Informationsmodellierung – Austrian Centre for Digital
                     Humanities} der Karl-Franzens-Universität Graz entwickelt wurde. Das
                  edierte Quellmaterial in XML wird via \href{http://gams.uni-graz.at/o:konde.86}{XSLT-Transformationen} zur Anzeige gebracht. Digitalisate an der
                  Österreichischen Nationalbibliothek werden sowohl inhouse als auch mit externen
                  Partnerunternehmen erstellt und in der Regel via \href{http://gams.uni-graz.at/o:konde.123}{IIIF} zur Verfügung gestellt und so in die Digitale
                  Edition eingebunden.\\
            
        Das Design für die Editionsprojekte wurde dahingehend entwickelt, dass es für neue
                  Editionsprojekte wiederverwendet werden kann. Als Unterscheidungsmerkmal zwischen
                  den Digitalen Editionen dient hier das Farbschema, das von einer Primär- und einer
                  Sekundärfarbe geprägt ist. Das Design wurde mit \emph{Bootstrap}
                  umgesetzt und ist auch für mobile Endgeräte bzw. Geräte mit kleinerem Display
                  aufbereitet.\\
            
        \subsection*{Literatur:}\begin{itemize}\item Fritze, Christiane: Wohin mit der digitalen Edition? Ein Beitrag aus der
                              Perspektive der Österreichischen Nationalbibliothek Wohin mit der digitalen Edition?. In: Bibliothek - Forschung und Praxis 43 (3): 2019, S. 432–440.\item Digitale Editionen an der Österreichischen
                              Nationalbibliothek. URL: \url{https://edition.onb.ac.at/start/o:ode.home/methods/sdef:TEI/get}\item Fritze, Christiane; Steindl, Christoph: Nachhaltige Infrastruktur für digitale Editionen an der
                              Österreichischen Nationalbibliothek. Berlin: 2018, URL: \url{urn:nbn:de:0290-opus4-35161}.\end{itemize}\subsection*{Software:}\href{https://getbootstrap.com/}{Bootstrap}, \href{http://gams.uni-graz.at/archive/objects/o:gams.doku/methods/sdef:TEI/get?locale=de}{GAMS}, \href{http://gitlab.com/}{Gitlab}, \href{https://iiif.io/}{iiif}, \href{http://oxygenxml.com/}{Oxygen}\subsection*{Verweise:}\href{https://gams.uni-graz.at/o:konde.59}{Digitale Edition}, \href{https://gams.uni-graz.at/o:konde.70}{GAMS}, \href{https://gams.uni-graz.at/o:konde.123}{International Image
                           Interoperability Framework}, \href{https://gams.uni-graz.at/o:konde.178}{TEI}\subsection*{Projekte:}\href{https://edition.onb.ac.at/okopenko/context:okopenko/methods/sdef:Context/get}{Tagebücher Andreas Okopenko}, \href{https://edition.onb.ac.at/sauer-seuffert}{Briefwechsel August Sauer Bernhard Seuffert 1880 bis 1926 digital}, \href{https://edition.onb.ac.at/blotius}{Blotius
                           Digital. Digitale Edition des Inventars der Wiener Hofbibliothek
                           (1576)}, \href{https://edition.onb.ac.at/}{Infrastruktur für
                           Digitale Editionen an der Österreichischen Nationalbibliothek}, \href{https://www.onb.ac.at/ueber-uns/vision-und-strategie}{Vision 2025
                           der Österreichischen Nationalbibliothek}\subsection*{Themen:}Institutionen, Software und Softwareentwicklung\subsection*{Zitiervorschlag:}Fritze, Christiane; Steindl, Christoph. 2021. Österreichische Nationalbibliothek. In: KONDE Weißbuch. Hrsg. v. Helmut W. Klug unter Mitarbeit von Selina Galka und Elisabeth Steiner im HRSM Projekt "Kompetenznetzwerk Digitale Edition". URL: https://gams.uni-graz.at/o:konde.153\newpage\end{document}